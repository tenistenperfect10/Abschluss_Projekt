% !TeX encoding = UTF-8
\documentclass{article}
\usepackage[T1]{fontenc}
\usepackage[utf8]{inputenc}
\DeclareUnicodeCharacter{202F}{\,}
\usepackage{graphicx}
\usepackage[ngerman]{babel}
\usepackage{hyperref}
\usepackage{enumitem}
\hypersetup{colorlinks=true, linkcolor=black}
\setlist[itemize]{topsep=-5pt}
\begin{document}
\title{Plenarprotokoll 19/159
			}
\date{}
\maketitle
\tableofcontents
\newpage
\section{Zusatzpunkt 1}
\subsection{Ostendorff}
\noindent\textbf{Texts:} Sehr geehrter Herr Präsident! Liebe Kolleginnen und Kollegen! Die Coronaexplosion im Westfleisch-Schlachthof im Kreis Coesfeld führt uns doch vor Augen, was viele lange Zeit nicht sehen wollten: Unser billiges Fleisch wird nicht nur auf dem Rücken der Tiere, sondern auch auf dem Rücken vieler europäischer Arbeitnehmerinnen und Arbeitnehmer, vor allen Dingen aus Rumänien und Bulgarien, produziert. Unter oft menschenunwürdigen Bedingungen verrichten sie die harte Arbeit in vielen deutschen Schlachthöfen, aber auch auf den Feldern der Gemüse-, Spargel- und Erdbeerbauern. Jetzt werden die Arbeiterinnen und Arbeiter auch noch der akuten, heftigen Gefahr der Ansteckung mit dem Covid-19-Virus ausgesetzt. Die Zahlen der letzten Tage sprechen doch für sich: 270 Infizierte bei Westfleisch in Coesfeld, 25 Prozent der Mitarbeiter, Hunderte bei Müller in Birkenfeld, über 100 auch bei VION in Bad Bramstedt, viele bei Westfleisch in Oer-Erkenschwick und bei Boeser in Schöppingen. Drohen uns Zustände wie in den USA, Zustände, wie sie schon Upton Sinclair 1905 im Roman „Der Dschungel“ problematisierte: schonungslose Ausbeutung von Mensch und Vieh? Viele Schlachtbetriebe entwickeln sich zu Coronazentren. Seit Jahren unternimmt aber die Bundesregierung nichts gegen das Billigfleischsystem, das die Weltmärkte überschwemmt, nichts gegen die menschenunwürdige Arbeits- und Wohnsituation, nichts gegen die unzureichende Entlohnung, nichts gegen oft über 60 Wochenstunden harter Arbeit,  aber vor allen Dingen, liebe Kolleginnen und Kollegen, nichts gegen das Werkvertragsunwesen mit oft sehr zwielichtigen Subunternehmern. Seit Jahren liegen der Politik Berichte über die unhaltbaren Zustände in der Schlachtindustrie vor – es wird weggesehen und verdrängt –: schlecht bezahlte Akkordarbeit am Fließband, dicht an dicht zwischen toten Tieren, das oft über mehr als zehn Stunden, oft sechs Tage die Woche, über Werkverträge zum Mindestlohn bei Subunternehmern angestellt – so können Schlachthofbetreiber sehr schön jegliche Verantwortung von sich weisen. Sie betonen ja auch immer wieder gerne, dass sie von den Arbeitsverhältnissen in ihren Betrieben gar nichts wissen. Für die Menschen heißt das: Sammelunterkünfte in oft maroden Bruchbuden, drei bis fünf Menschen zusammen in einem Zimmer, primitive Gemeinschaftsküchen, marode sanitäre Anlagen ohne wirksame Kontrollen der Kommunen, die hier zuständig sind. Vor der Krise war das schon eine Zumutung, jetzt werden die Unterkünfte aber auch noch zum Seuchenhotspot.  Zu allem Übel, liebe Kolleginnen und Kollegen, werden die oft überteuerten Mieten auch noch zusätzlich von dem schmalen Lohn abgezogen. Unentschuldbare, menschenverachtende Profitgier – das muss sofort verboten werden, meine Damen und Herren.  Kolleginnen und Kollegen von der Koalition, es liegt doch an Ihnen, diese Zustände sofort zu beenden.  Jetzt ist schnelles Handeln in den Unterkünften und in den Betrieben gefordert. Die Zeit des Wegsehens, die Zeit des Wegduckens, des Verdrängens muss endlich beendet werden.  Es kann doch nicht sein, was wir erlebt haben – Kollege Henrichmann ist hier, er hat es auch mitbekommen –: dass der Landrat von Coesfeld uns öffentlich vorschlug, dass Westfleisch selber entscheiden soll, ob sie den Betrieb schließen oder nicht. Soweit wir das bisher verstanden haben, ist es doch seine Aufgabe als Landrat, hier zu handeln. Er ist derjenige, der es machen muss.  Erst das Land NRW zwang ihn zur Umkehr, zwang ihn zum Handeln. Doch was macht Westfleisch? Westfleisch geht zu Gericht und beantragt die Aufhebung der Schließung,  statt sich um die unhaltbaren Zustände zu kümmern, Max Straubinger.  Das wäre vielleicht dran gewesen;  das wäre die erste Aufgabe gewesen. Vielleicht ist es in Bayern anders, aber ich sage: Wir müssen diese Zustände anfassen – nichts anderes.  Herr Bundesarbeitsminister, Ihre Empfehlungen und Ihre immer wieder vorgetragene Haltung sind sehr ehrenwert, aber sie müssen sich endlich in klaren Regeln mit Strafbewehrung niederschlagen. Sonst sind sie wirkungslos.  Die betroffenen Unternehmen müssen so lange geschlossen bleiben, bis die Einzelunterbringung gewährleistet ist. Bei der Arbeit muss ein Mindestabstand von 1,50 Metern eingehalten werden. Das ausbeuterische Billigfleischsystem, das europäische Mitbürger schamlos ausnutzt, muss sofort beendet werden.  Stoppen aber auch Sie von der CDU\/CSU den Klöckner’schen Spargelhelferpakt mit dem Bundesinnenminister, der alle Schutzmaßnahmen grob missachtet. Den Betrieben sei gesagt: Stellen Sie die Menschen endlich zu angemessenen Tariflöhnen menschenwürdig ein. Das ist die Aufgabe – nichts anderes.  

\noindent\textbf{Comment:}
\begin{itemize}
    \setlength\itemsep{-3pt}
    \item (Zuruf des Abg. Dr. Martin Rosemann [SPD])
    \setlength\itemsep{-3pt}
    \item (Beifall beim BÜNDNIS 90/DIE GRÜNEN)
    \setlength\itemsep{-3pt}
    \item (Beifall beim BÜNDNIS 90/DIE GRÜNEN sowie des Abg. Pascal Meiser [DIE LINKE])
    \setlength\itemsep{-3pt}
    \item (Max Straubinger [CDU/CSU]: Das ist Rechtsstaat!)
    \setlength\itemsep{-3pt}
    \item (Steffi Lemke [BÜNDNIS 90/DIE GRÜNEN]: Da lacht er auch noch!)
    \setlength\itemsep{-3pt}
    \item (Beifall beim BÜNDNIS 90/DIE GRÜNEN sowie bei Abgeordneten der LINKEN)
    \setlength\itemsep{-3pt}
    \item (Beifall bei der CDU/CSU)
    \setlength\itemsep{-3pt}
    \item (Beifall beim BÜNDNIS 90/DIE GRÜNEN sowie bei Abgeordneten der LINKEN – Stephan Protschka [AfD]: In Bayern ist die Welt in Ordnung!)
    \setlength\itemsep{-3pt}
    \item (Beifall beim BÜNDNIS 90/DIE GRÜNEN und bei der LINKEN sowie bei Abgeordneten der SPD)
    \setlength\itemsep{-3pt}
    \item (Beifall beim BÜNDNIS 90/DIE GRÜNEN und bei der LINKEN)
    \setlength\itemsep{-3pt}
    \item (Beifall beim BÜNDNIS 90/DIE GRÜNEN sowie bei Abgeordneten der LINKEN und der Abg. Susanne Mittag [SPD])
    \setlength\itemsep{-3pt}
    \item (Beifall beim BÜNDNIS 90/DIE GRÜNEN, bei der SPD und der LINKEN)
\end{itemize}
\subsection{Schummer}
\noindent\textbf{Texts:} Verehrtes Präsidium! Meine lieben Kolleginnen! Liebe Kollegen! Es ist in der Tat ein prekärer Arbeitsbereich, über den wir heute miteinander diskutieren. Deshalb hat ja auch die Bundesregierung beispielsweise mit der Verschärfung der Generalunternehmerhaftung 2017 gehandelt. Wir haben beim Zoll das Personal sowie insgesamt die Kontrollmechanismen verstärkt. Es gibt hier natürlich auch den Föderalismus, und ich halte den Föderalismus auch für hilfreich, auch in der Frage der Kontrolldichte, die wir weiter erhöhen müssen. Wenn nämlich die kommunale Ebene beispielsweise für den Gesundheitsschutz, für die Unterkünfte zuständig ist und die Landesebene – der Zoll, die Landesämter – für den Arbeitsschutz, dann kann die eine Ebene die andere Ebene, wenn sie nicht funktioniert, antreiben, sodass ein Wettbewerb um die Kontrollmechanismen stattfindet und nicht ein Wettbewerb im Wegsehen. Das ist der Vorteil von Föderalismus, und den sollten wir auch nutzen.  Aber Voraussetzung ist natürlich, dass diese Kontrollebenen zusammengeführt werden und wie aus einer Hand arbeiten, sodass es auch möglich ist, den Datenausgleich so zu organisieren, dass jeder von jedem weiß, wo es Missstände gibt und wo entsprechende Auffälligkeiten festgestellt wurden. Die Pandemie ist wie ein Brennglas: Man schaut stärker hin. Es gab aber auch bereits im Oktober letzten Jahres eine Schwerpunktkontrolle in der Fleischwirtschaft in Nordrhein-Westfalen, wo bei 30 Schlachthöfen mit 90 Werkvertragsfirmen und 17 000 Beschäftigten 9 000 Verstöße festgestellt wurden. Ein Drittel dieser Verstöße waren Arbeitszeitverstöße. Ich finde, wenn wir über Digitalisierung in allen Bereichen reden, dann müssen wir beispielsweise auch über digitale Zeiterfassung sowohl bei den Paketdienstleistern als auch in der Fleischwirtschaft reden.  Bei Schlachtbetrieben – das hat diese Kontrolle in Nordrhein-Westfalen durch Karl-Josef Laumann auch gezeigt –, die mit festangestellten Arbeitnehmern und ohne Subunternehmen arbeiten – auch die gibt es –, gab es keine Beanstandungen. Beanstandungen gab es dort, wo Werkverträge außer Kontrolle geraten. Von daher ist „Mehr Mittelstand wagen“ auch ein Stück weit eine gute Botschaft für die Fleischwirtschaft. Mit diesem „Geiz ist geil“-Motiv aus der Werbung zahlen wir einen hohen Preis – die Verbraucher, die Arbeitnehmer, aber letztendlich auch die Landwirte, die damit unter Druck gesetzt werden. Die Selbstverpflichtung der Fleischwirtschaft von 2015, mit weniger Subunternehmen, weniger Werkverträgen und mehr Stammbeschäftigten zu arbeiten, hat bis heute keine Wirkung gezeigt. Deshalb ist es gut, dass wir in den nächsten Wochen miteinander über die Umsetzung der europäischen Entsenderichtlinie diskutieren und sie hier auch eins zu eins verabschieden wollen – ohne Abstriche.  Der Grundsatz der europäischen Entsenderichtlinie lautet: gleicher Lohn für gleiche Arbeit und gleiche Arbeitskonditionen für die jeweilige gleiche Arbeit.  Es werden beispielsweise auch Unterkünfte für Wanderbeschäftigte, für Saisonarbeiter entsprechend miteinbezogen. Das sind Unterkünfte für die Arbeitsaufnahme. Es sind keine Ferienwohnungen. Zwischen den Unterkünften, die privat vermietet werden, und denen, die Werkswohnungen sind, sehe ich auch keine großen Unterschiede. Da müssen wir gemeinsam die Hygienestandards überprüfen, da müssen die Kontrollmechanismen funktionieren. Auch das ist Voraussetzung und auch Konsequenz dieser europäischen Entsenderichtlinie. Es geht um den Schutz der Arbeitnehmer. Es geht in der Zeit der Pandemie auch um den Schutz der Bevölkerung, um Infektionsketten nicht neu entstehen zu lassen. Aber es geht letztendlich auch um den Schutz der Menschen in den Herkunftsländern. Wenn die Wanderbeschäftigten bzw. die Saisonarbeiter mit Corona infiziert nach Rumänien oder nach Bulgarien zurückkehren und dort auf ein marodes Gesundheitssystem treffen, dann kann dies für viele tödlich sein. Deshalb müssen wir konsequent und zeitnah handeln. Die Entsenderichtlinie gibt hierzu ein wichtiges Instrumentarium. Karl-Josef Laumann und die Landesregierung in Nordrhein-Westfalen haben schnell gehandelt und die sofortige Schließung des Schlachthofes in Coesfeld angeordnet. Es gab die klare Anweisung, die Beschäftigten aller Schlachthöfe in Nordrhein-Westfalen zu testen. Das sind über 20 000 Menschen. Es war somit die größte Testaktion gegen Covid-19 in ganz Deutschland und, ich denke, auch ein Vorbild für andere Bundesländer. Bund und Länder müssen dieses prekäre Arbeitsfeld ordnen. Wir wollen Kontrollen wie aus einer Hand sicherstellen und die europäische Entsenderichtlinie konsequent umsetzen.  

\noindent\textbf{Comment:}
\begin{itemize}
    \setlength\itemsep{-3pt}
    \item (Beifall bei der AfD)
    \setlength\itemsep{-3pt}
    \item (Beifall bei der CDU/CSU sowie des Abg. Dr. Christoph Hoffmann [FDP])
    \setlength\itemsep{-3pt}
    \item (Friedrich Ostendorff [BÜNDNIS 90/DIE GRÜNEN]: Keine Märchenstunde!)
    \setlength\itemsep{-3pt}
    \item (Beifall des Abg. Hermann Gröhe [CDU/CSU])
    \setlength\itemsep{-3pt}
    \item (Beifall bei der CDU/CSU)
    \setlength\itemsep{-3pt}
    \item (Jutta Krellmann [DIE LINKE]: Da sind wir aber gespannt!)
\end{itemize}
\subsection{Protschka}
\noindent\textbf{Texts:} Habe die Ehre, Herr Präsident! Servus, liebe Kolleginnen und Kollegen! Gott zum Gruße, liebe Gäste zu Hause vor dem Fernseher! Ja, wir alle haben mitbekommen, dass die Zahl der Coronainfizierten in den Schlachthöfen steigt. An dieser Geschichte sind aber nicht nur die Schlachthöfe schuld – die Bösen, so wie es uns die linke Seite im Plenum weismachen will –, sondern die Verantwortung dafür liegt in erster Linie bei der Bundesregierung. Warum? Wir erinnern uns: Am 27. März forderte die Bundeslandwirtschaftsministerin in einem Schreiben an den Kanzleramtschef, dass die Gesundheitsämter in Betrieben der Lebensmittelverarbeitung andere Hygiene- und Quarantäneregeln anwenden sollen. Dadurch sollten bei Coronaausbruch Betriebsschließungen vermieden werden, damit die Lebensmittelversorgung aufrechterhalten werden kann. Sich jetzt darüber zu wundern, dass die Schlachthöfe sich an diese Vorgabe gehalten haben, ist einfach nur scheinheilig, liebe Grüninnen und Grünen.  Das ausgerechnet die Grünen diese Aktuelle Stunde beantragt haben, ist sehr verwunderlich. Die Grünen bejubeln ja bei jeder Gelegenheit die illegale Wirtschaftsmigration.  Aber verteufeln Sie jetzt die legale Arbeitsmigration? Das mag jetzt verstehen, wer will. Sind Sie jetzt plötzlich nicht mehr für die vielgepriesene Arbeitnehmerfreizügigkeit der EU, die der Herr Ostendorff gerade angesprochen hat? Aber falls Sie es noch nicht mitbekommen haben, liebe Grüninnen und Grünen: Es gibt auch Schlachthöfe im grün-regierten Baden-Württemberg. Kümmern Sie sich lieber darum, und halten Sie hier etwas die Füße still! Warum werden denn überhaupt günstige osteuropäische Arbeitskräfte in Schlachthöfen, als Erntehelfer, als Pflegekräfte, als Bauarbeiter usw. usf. bei uns beschäftigt? Glauben Sie es mir: Ich würde mir wünschen, dass es nicht so sein müsste. Aber das sind doch die Früchte Ihrer Politik der letzten Jahrzehnte. Und da nehme ich alle ins Boot. Das eigentliche Problem ist, dass unsere Tierhalter und auch die Schlachthöfe dazu gezwungen werden, mit dem niedrigen Weltmarktpreis zu konkurrieren. Sie haben gar keine andere Wahl, als ihre Produktion zu erhöhen und dadurch ihre Stückkosten zu senken, um wettbewerbsfähig und wirtschaftlich zu bleiben.  Die Betriebe müssen also immer größer werden und immer kostengünstiger produzieren. „Wachse oder weiche“ heißt die Parole. Verstärkt wird dieser Konzentrationsprozess durch immer neue Freihandelsabkommen für Lebensmittelbilligimporte wie beispielsweise das Mercosur-Abkommen. Dazu verabschieden die Superbürokraten aus Brüssel in immer kürzeren Abständen stetig steigende Auflagen für die Betriebe. Kleine Betriebe haben bei diesen Kostenzwängen ja überhaupt gar keine Chance mehr. Sie, meine Damen und Herren der Altparteien, sind verantwortlich für diese Fehlentwicklung.  Wenn Sie jetzt wieder mit neuen Verboten und Auflagen gegensteuern wollen, dann werden Sie damit nur erreichen, dass die Tierhaltung, die Schlachtung und die Arbeitnehmer ins Ausland abwandern. Dort gelten deutlich niedrigere Standards für Arbeitnehmer und bei Umwelt- und Tierschutz als hier bei uns im Land. Eine höhere Importabhängigkeit kann nicht der richtige Weg sein, vor allem in Zeiten von Corona. Wir von der Alternative für Deutschland stehen fest hinter dem deutschen Mittelstand.  Wir wollen den bäuerlichen Familienbetrieben und Schlachtereien eine wirtschaftliche Perspektive bieten und werden die dafür nötigen Rahmenbedingungen setzen. Wenn wir eins aus der Lockdown-Krise, die Sie verursacht haben, gelernt haben, dann ist es, dass die Regionalität wichtiger ist denn je. Deshalb haben wir bereits letztes Jahr gefordert, die regionale Landwirtschaft und Direktvermarktung zu stärken. Diesen Antrag haben aber leider alle Blockparteien hier im Haus abgelehnt.  Darin fordern wir unter anderem auch die Förderung der mobilen Schlachtung. Außerdem haben wir den Abbau von strengen Zulassungsvorschriften und von hohem Bürokratieaufwand für kleinere Schlachtereien usw. gefordert. Denn es sind ja gerade die kleineren Schlachtereien, die die Wertschöpfung und die Arbeitsplätze, die dann auch mit Deutschen besetzt werden, in der Region schaffen. Auch aus Tierschutzsicht sind die kleineren Schlachtereien zu begrüßen, weil die Tiere beispielsweise viel kürzere Strecken zum Schlachter zurücklegen müssen. Liebe Bundesregierung, setzen Sie sich endlich für heimische Familienbetriebe ein, für Tierschutz ein, für Umweltschutz ein! Es besteht dringender Handlungsbedarf. Nur mit der AfD sind Umweltschutz, Tierschutz und Landwirtschaft möglich, meine Damen und Herren.  Ich wünsche einen schönen Tag. Habe die Ehre!  

\noindent\textbf{Comment:}
\begin{itemize}
    \setlength\itemsep{-3pt}
    \item (Lachen beim BÜNDNIS 90/DIE GRÜNEN sowie bei Abgeordneten der SPD – Ulli Nissen [SPD]: Oh Gott! Wie schön, dass das jetzt wieder kommt!)
    \setlength\itemsep{-3pt}
    \item (Beifall bei der AfD)
    \setlength\itemsep{-3pt}
    \item (Ulli Nissen [SPD]: Zu Recht!)
    \setlength\itemsep{-3pt}
    \item (Pascal Meiser [DIE LINKE]: Fest hinter der Arbeitsausbeutung!)
    \setlength\itemsep{-3pt}
    \item (Lachen bei Abgeordneten der SPD, der LINKEN und des BÜNDNISSES 90/DIE GRÜNEN)
    \setlength\itemsep{-3pt}
    \item (Beifall bei der SPD – Renate Künast [BÜNDNIS 90/DIE GRÜNEN]: Die Hütte brennt!)
    \setlength\itemsep{-3pt}
    \item (Dr. Matthias Zimmer [CDU/CSU]: Ach Gott! Die Ärmsten!)
\end{itemize}
\subsection{Heil}
\noindent\textbf{Texts:} Herr Präsident! Liebe Kolleginnen und Kollegen! Die Nachrichten von massenhaften Infektionen, die uns aus Birkenfeld bei Pforzheim, die uns aus Bad Bramstedt in Schleswig-Holstein, die uns jetzt aus Coesfeld und aus Oer-Erkenschwick in Nordrhein-Westfalen erreichen, sind entsetzlich, sie sind beschämend, und sie sind nicht zu tolerieren.  Ich finde, dass wir in einer solchen Situation über zwei Dinge reden müssen: Wir müssen über Klartext und über Verantwortung reden. Ich will auch über die Verantwortung von Unternehmen reden, über die Verantwortung der Bundespolitik und der Landespolitik, damit wir in diesem Bereich wirklich aufräumen. Eines ist richtig, liebe Kolleginnen und Kollegen: Es hat in den letzten Jahren im Bereich der Fleischindustrie immer wieder Gesetzgebung gegeben, auch dieses Hauses – das letzte Mal 2017 –, um Missständen, was Arbeits- und Sozialbedingungen betrifft, entgegenzuwirken. Es ist das Arbeitnehmer-Entsendegesetz – mit der Generalunternehmerhaftung für die Einhaltung des Mindestlohns und die Abführung der Sozialversicherungsbeiträge – und vieles andere verschärft worden. Sigmar Gabriel und Andrea Nahles haben vielfache Initiativen gestartet. Wir haben dann aber immer zwei Dinge erlebt: Zum einen gab es mit Teilen der Branche ein Katz-und-Maus-Spiel – wenn Regelungen getroffen wurden, hat man sich an der einen oder anderen Stelle Umgehungsmöglichkeiten organisiert –, und zum anderen haben wir in parlamentarischen Prozessen immer wieder erlebt – lassen Sie uns Klartext reden –, dass Interessengruppen versucht haben, klare Regeln, sagen wir mal, zu soften, zu verwässern. Deshalb müssen wir in dieser Situation ernsthaft über Folgendes reden: Zum einen dürfen wir als Gesellschaft nicht weiter zugucken, wie Menschen aus Mittel- und Osteuropa in dieser Gesellschaft ausgebeutet werden.  Die Arbeitnehmerinnen und Arbeitnehmer, ob sie sich heute „Beschäftigte“, „Werksvertragler“ oder demnächst „Praktikanten“ nennen, sind arbeitende Menschen, die ein Recht darauf haben, Arbeitsschutz und Gesundheitsschutz zu erfahren, wie alle anderen Menschen in dieser Gesellschaft auch, egal wo sie herkommen. Das ist nicht verhandelbar, und das darf nicht weiter verhandelbar sein. Zum anderen will ich über die gesellschaftliche Dimension reden. Man muss sich die Situation in Coesfeld einmal vergegenwärtigen: Die Menschen, die Betriebe, die Mittelständler in Coesfeld haben sich nach den Beschlüssen der Ministerpräsidenten und den neuen Verordnungen auf Lockerungen des Lockdowns gefreut, um ihr Sozial- und Wirtschaftsleben wieder in Gang zu bringen. Und dann gibt es offensichtlich einige, die sich nicht an Regeln halten und damit einen ganzen Landkreis, die ganze Wirtschaft und die Gesellschaft in Geiselhaft nehmen. – Auch das ist für unsere Gesellschaft nicht akzeptabel, meine Damen und Herren.  Nun will ich sagen, dass es anständige Unternehmer in der Fleischwirtschaft gibt. Aber es gibt viel zu viele schwarze Schafe. Deshalb ist die erste Verantwortung, aus der wir als Staat die Unternehmen nicht entlassen dürfen, die unternehmerische Verantwortung für alle Beschäftigten.  Ich sage an dieser Stelle: Es gibt bestehende Arbeitsschutzregeln. Die sind verbindlich. Die sind einzuhalten. Aber die schärfsten Regeln nützen nichts, wenn ihre Einhaltung nicht kontrolliert wird. Das ist die Erfahrung, die wir machen.  Wir haben auch erlebt, dass die Arbeitsschutzbehörden – seien wir offen und ehrlich, reden wir Klartext – für viele Bundesländer quasi eine Sparkasse waren, genauso wie, leider Gottes, die Gesundheitsämter für die Kommunen, wo auch viel zu sehr gespart wurde. Das heißt, die Kontrolldichte ist in den letzten Jahren runtergefahren worden. Ich habe bereits im Dezember, vor Corona, mit den Arbeits- und Sozialministern der Länder vereinbart, dass wir bundesweit zu verbindlichen Kontrollquoten im Arbeitsschutz kommen. Das heißt: mehr Personal und schärfere Kontrollen im Arbeitsschutz. Das werden wir gesetzgeberisch umzusetzen haben, meine Damen und Herren.  Aber wir müssen weitergehen. Darüber werden wir in der Bundesregierung in den nächsten Tagen, bis Montag, zu reden haben. Wir haben vereinbart – die Kanzlerin hat das vorhin in der Fragestunde deutlich gemacht –, dass wir am Montag im sogenannten Coronakabinett tatsächlich zu Maßnahmen kommen werden, die über das Bestehende hinausgehen, untergesetzlich, aber durchaus auch gesetzlich. Herr Kollege Schummer, Sie haben es angesprochen – ich bin da ganz bei Ihnen; Sie sind dazu mit den Ländern im Gespräch, auch mit dem Kollegen Laumann –: Diese Art von Subsubsubunternehmertum ist die Wurzel des Übels, weil Verantwortung abgewälzt wird, weil Arbeitnehmerrechte geschleift werden, weil Löhne gedrückt werden. Das führt dazu, dass wir in vielen Bereichen Probleme haben, beispielsweise wenn es darum geht, Unterkünfte zu kontrollieren. Werksunterkünfte auf dem Werksgelände kann man kontrollieren, auch noch größere andere Unterkünfte, aber wenn irgendwelche, von dubiosen Subsubsubunternehmern angemietete Privatwohnungen und Garagen – das ist übrigens ein profitables Geschäft, das auch mit krimineller Energie betrieben wird – zu kontrollieren sind, dann stößt man an gesetzliche Grenzen. Deshalb sage ich in Frageform: Sollten wir uns nicht überlegen, wie wir mit dieser Art der sogenannten Werksvertragskonstruktion in der fleischverarbeitenden Industrie umgehen, meine Damen und Herren?  Ich finde, diese Diskussion müssen wir miteinander führen. Wir werden – sehr sorgfältig – Verantwortung übernehmen. Das betrifft aber alle Ebenen. Das betrifft die unternehmerische Ebene. Wir wollen eine fleischverarbeitende Industrie in Deutschland, in der anständige Hygienebedingungen herrschen. Das ist auch im Interesse des Tierwohls, aber solche Arbeitsbedingungen sind vor allen Dingen im Interesse der Menschen. Wir wollen nicht, dass sie um ihr Leben, um ihre Gesundheit fürchten müssen. Herr Ostendorff, Sie haben aus Upton Sinclairs Buch „Der Dschungel“ zitiert.  Auch ich habe das Buch in meiner Jugend gelesen. Dieses Buch hat mich sehr geprägt. Dieser sozialkritische Roman über die fleischverarbeitende Industrie in den USA vor 90, vor 100 Jahren führte zu einer Riesendebatte über diesen Ekel. Er hat übrigens auch politische Konsequenzen im Bereich des Arbeitsschutzes begründet.  Ich sage: Wir dürfen jetzt nicht bei Empörung stehen bleiben,  sondern wir müssen handeln. Das betrifft das unternehmerische Handeln, damit die gesamte Branche nicht in Verruf kommt. Das betrifft das Handeln des Bundes, der den gesetzgeberischen Rahmen nachzuschärfen hat, und zwar dem Grunde nach. Und das betrifft die Verantwortung der Länder, die dafür zu sorgen haben, dass bestehende Regeln von den Arbeitsschutzbehörden effektiv überwacht und kontrolliert werden.  Jetzt ist Zeit für Verantwortung auf allen Ebenen. Das, meine Damen und Herren, werden wir miteinander zu besprechen haben. Meine Bitte an dieses Haus, an alle Fraktionen, an fast alle Fraktionen  lautet, es nicht bei der Empörung in dieser Debatte zu belassen, sich nicht nur gegenseitig zu bezichtigen, sondern gemeinsam Verantwortung zu übernehmen. Ich bitte darum, nicht nur Regeln vorzuschlagen, sondern auch dafür zu sorgen, dass diese Regeln im parlamentarischen Verfahren, wenn wir in die Gesetzgebung gehen, nicht verwässert werden, sondern umgesetzt werden. Und ich bitte darum, dafür zu sorgen, dass in den Arbeitsschutzbehörden der Länder genug Personal vorhanden ist, damit die bestehenden Regeln durchgesetzt werden können. Ich sage es noch einmal: Wir haben viel zu verlieren. Die bulgarische und die rumänische Regierung haben sich an die Bundesrepublik Deutschland gewandt, die EU-Kommission auch. Wir sind ein Land von hoher sozialstaatlicher Qualität. Wir sind ein Land, auf das wir gemeinsam in vielerlei Hinsicht stolz sein können. Aber wir müssen aufhören, das, was in dieser Gesellschaft nicht in Ordnung ist, schönzureden. Ich sage es noch mal an dieser Stelle: Es ist Zeit für Klartext und Verantwortung. Wir müssen und wir werden mit diesen Verhältnissen aufräumen. Ich bitte Sie als Arbeitsminister parteiübergreifend um Unterstützung, weil ich mir lebhaft vorstellen kann, was jetzt passiert. Alle rufen: „Die waren es! Die waren es! Die waren es!“, und alle sagen: Da müssen wir etwas tun. Dann gibt es Vorschläge, und dann kommen die Interessengruppen. – Ich finde, an dieser Stelle geht das Gemeinwohl vor. Das ist unsere Aufgabe, meine Damen und Herren.  

\noindent\textbf{Comment:}
\begin{itemize}
    \setlength\itemsep{-3pt}
    \item (Beifall bei der SPD)
    \setlength\itemsep{-3pt}
    \item (Beifall bei der FDP)
    \setlength\itemsep{-3pt}
    \item (Beifall bei der SPD, der CDU/CSU, der LINKEN und dem BÜNDNIS 90/DIE GRÜNEN sowie des Abg. Dr. Christoph Hoffmann [FDP])
    \setlength\itemsep{-3pt}
    \item (Beifall bei Abgeordneten der SPD und der LINKEN und des Abg. Michael Theurer [FDP])
    \setlength\itemsep{-3pt}
    \item (Heiterkeit bei Abgeordneten der SPD und des BÜNDNISSES 90/DIE GRÜNEN)
    \setlength\itemsep{-3pt}
    \item (Beifall bei der SPD sowie bei Abgeordneten der LINKEN)
    \setlength\itemsep{-3pt}
    \item (Beifall bei der SPD, der LINKEN und dem BÜNDNIS 90/DIE GRÜNEN)
    \setlength\itemsep{-3pt}
    \item (Beifall bei der SPD, der LINKEN und dem BÜNDNIS 90/DIE GRÜNEN sowie bei Abgeordneten der CDU/CSU und der FDP)
    \setlength\itemsep{-3pt}
    \item (Beifall bei Abgeordneten der SPD und des BÜNDNISSES 90/DIE GRÜNEN)
    \setlength\itemsep{-3pt}
    \item (Beifall bei der SPD, der CDU/CSU und dem BÜNDNIS 90/DIE GRÜNEN sowie bei Abgeordneten der LINKEN und des Abg. Michael Theurer [FDP])
    \setlength\itemsep{-3pt}
    \item (Beifall bei der SPD, der CDU/CSU, der LINKEN und dem BÜNDNIS 90/DIE GRÜNEN)
    \setlength\itemsep{-3pt}
    \item (Claudia Roth [Augsburg] [BÜNDNIS 90/DIE GRÜNEN]: Gutes Buch!)
    \setlength\itemsep{-3pt}
    \item (Friedrich Ostendorff [BÜNDNIS 90/DIE GRÜNEN]: Ja!)
\end{itemize}
\subsection{Cronenberg}
\noindent\textbf{Texts:} Herr Präsident! Liebe Kolleginnen und Kollegen! Jetzt haben wir Coronainfektionsgeschehnisse von der Ostsee bis zum Schwarzwald, als ob wir nicht schon genug Probleme in der Fleischindustrie gehabt hätten. Da, wo viel Mist ist, kommt meistens noch mehr Mist dazu. Das hat oft damit zu tun, dass die, die den Mist nicht haben wollen, wegschauen und die, die den Mist verzapfen, genau das spüren. Schwache Rechtsdurchsetzung oder gar stillschweigende Akzeptanz führt zu Zuständen wie jetzt in der Fleischindustrie. Ja, diese Zustände sind unhaltbar.  Aber denen, die sich jetzt laut empören, sage ich auch: Sie haben es seit Jahren gewusst. – So hat die Einführung der Nachunternehmerhaftung für Sozialversicherungsbeiträge aus dem Jahr 2017 wenig Verbesserung gebracht, und wenn, dann eher für die Sozialversicherungsträger als für die Beschäftigten.  Deshalb ist es jetzt umso wichtiger, dass wir uns nicht nur en détail mit den einzelnen Ausprägungen des Skandals befassen – es ist angesprochen worden: mangelnde Hygiene, Verstöße gegen Arbeitszeit- und Arbeitsschutzregelungen sowie  gegen den Mindestlohn, missbräuchliche Lohnkürzungen usw. usf. –, sondern eben auch akzeptieren, dass der Staat offensichtlich nur unzureichend oder überhaupt nicht in der Lage ist, dieser Zustände mit den gegebenen Instrumenten Herr zu werden.  Um es deutlich zu sagen: Wir haben kein Rechtssetzungsproblem, liebe Kolleginnen und Kollegen, wir haben ein Rechtsdurchsetzungsproblem.  Es ist richtig, dass wir uns hier und heute mit den Arbeitsbedingungen in der Fleischindustrie befassen. Falsch wäre es, jetzt nur nach neuen Gesetzen und Verboten zu rufen. Die Pressestatements der vergangenen Tage lassen allerdings anderes befürchten. Für die einen ist allein der Werkvertragsunternehmer schuld. Also weg mit Werkverträgen! Für die anderen sind die Löhne zu niedrig. Also hoch mit dem Mindestlohn!  Für die Nächsten ist die mangelnde Kontrolle schuld. Also her mit 5 000 neuen Stellen beim Zoll! Für wen die Ausländer schuld sein werden, muss ich, glaube ich, niemandem verraten. Liebe Kolleginnen und Kollegen, ich finde, alle diese pauschalen Forderungen werden nicht dadurch richtiger, dass sie lauter oder öfter vorgetragen werden. Vielmehr geht es jetzt darum, zielgenaue Lösungsvorschläge in die Debatte einzubringen. So hat bereits im November 2019, also lange vor Corona, auf Initiative des schleswig-holsteinischen Sozialministers Heiner Garg von der FDP der Bundesrat die Bundesregierung aufgefordert, bestehende Regelungslücken zu schließen. Diese Punkte rufe ich gerne in Erinnerung, lieber Minister Heil: Erstens. Erweitern Sie den Geltungsbereich für den betrieblichen Arbeitsschutz auf alle Beschäftigten in den Schlacht- und Zerlegungsbetrieben, also auch auf die Werkvertragsnehmer!  Zweitens. Sorgen Sie dafür, dass privat angemietete Unterkünfte für Beschäftigte den Anforderungen der Arbeitsstättenrichtlinie unterworfen werden! Das können Sie herbeiführen.  Drittens. Führen Sie eine verpflichtende digitale Zeiterfassung in der Fleischindustrie ein! Einfacher geht es gar nicht. Nehmen Sie § 6 GSA Fleisch, fügen Sie das Wort „digital“ vor „aufzuzeichnen“ ein, und fertig ist die Laube!  Ohne Kontrollen wird es trotzdem nicht gehen. Bevor Sie aber Gott und die Welt in Bewegung setzen: Bringen Sie Ordnung in das Zuständigkeitswirrwarr! Arbeitsschutz ist Ländersache, für privat gemietete Unterkünfte sind die kommunalen Ordnungsämter zuständig, um die Arbeitssicherheit kümmert sich die Berufsgenossenschaft, und zu guter Letzt achtet der Zoll mit seiner Finanzkontrolle Schwarzarbeit auf die Einhaltung des Mindestlohns.  Mit Verlaub: Mit den aktuellen Strukturen werden Ihnen die schwarzen Schafe in der Branche immer einen Schritt voraus sein. Gerade der Föderalismus, lieber Kollege Schummer, braucht ein koordiniertes Vorgehen. Bilden Sie also eine Taskforce Fleisch! Lassen Sie sich die Daten der Werkvertragsunternehmer und ihrer Beschäftigten übermitteln! Nutzen Sie Künstliche-Intelligenz-Systeme zur Analyse von Daten, um Missbrauchsmuster zu erkennen! So schaffen Sie eine wirksame Kontrolle.  Die Branche steht – auch das sei gesagt – unter einem enormen Preis- und Kostendruck. Ich muss sagen: Es geht leider oft nur um Arbeitskosten pro Schnitzel. Viele Endverbraucher – nach meinem persönlichen Geschmack sogar zu viele – sind sehr, sehr preissensibel. Von einer vollständigen Verlagerung von Schlachtung und Weiterverarbeitung ins Ausland – Sie haben es angesprochen – rate ich allerdings auch ab. Das liegt weder im Interesse der landwirtschaftlichen Betriebe, noch wird es dem Verbraucherinteresse an Angebotsvielfalt gerecht. Wir stehen zu fairem Wettbewerb zum Vorteil der Verbraucher. Deshalb plädieren wir für Augenmaß und Konsequenz statt Vorgehen mit der Brechstange. Alle Beschäftigten in der Fleischbranche – ohne Ausnahme – haben Anspruch auf faire Löhne und Arbeitsbedingungen. Genauso haben die anständigen Betriebe – auch das ist angesprochen worden – und das Fleischerhandwerk Anspruch darauf, dass Recht und Gesetz uneingeschränkt für alle gelten. Es darf nicht sein, dass am Ende die Ehrlichen die Dummen sind. Vielen Dank.  

\noindent\textbf{Comment:}
\begin{itemize}
    \setlength\itemsep{-3pt}
    \item (Beifall bei der FDP)
    \setlength\itemsep{-3pt}
    \item (Beifall bei Abgeordneten der FDP)
    \setlength\itemsep{-3pt}
    \item (Beifall des Abg. Michael Theurer [FDP])
    \setlength\itemsep{-3pt}
    \item (Beifall bei der FDP sowie der Abg. Beate Müller-Gemmeke [BÜNDNIS 90/DIE GRÜNEN])
    \setlength\itemsep{-3pt}
    \item (Renate Künast [BÜNDNIS 90/DIE GRÜNEN]: Besser nicht regieren als falsch regieren, was?)
    \setlength\itemsep{-3pt}
    \item (Beifall des Abg. Matthias W. Birkwald [DIE LINKE])
    \setlength\itemsep{-3pt}
    \item (Beifall bei der LINKEN)
\end{itemize}
\subsection{Krellmann}
\noindent\textbf{Texts:} Sehr geehrter Herr Präsident! Liebe Kolleginnen und Kollegen! Als Gewerkschaftssekretärin kenne ich natürlich viele Betriebe. In der Fleischindustrie waren die Arbeitsbedingungen schon immer sehr, sehr hart und oftmals an der Grenze des Zulässigen. Covid-19 hat die Zustände wie mit einem Brennglas verschärft. Der Skandal ist jetzt über die Medien wieder nach oben gepuscht worden, und ich hoffe, dass er im Bewusstsein der Menschen bleibt.  Was sind die Ursachen? In der Fleischindustrie gibt es einen gnadenlosen Preiskampf, der auf dem Rücken der Beschäftigten und der Tiere ausgetragen wird. Hier lebt der Kapitalismus seine Profitgier in vollen Zügen richtig aus. Die Arbeit im Schlachthof ist körperlich schwer und oftmals monoton. Die Löhne sind extrem niedrig. Es sind hauptsächlich entsandte Arbeitnehmer aus Rumänien oder Bulgarien mit ausländischem Arbeitsvertrag, die, um überhaupt Geld zum Leben für ihre Familien zu verdienen, unter diesen schlimmen Bedingungen arbeiten und leben müssen. Sie werden mit Werkverträgen und niedrigen Löhnen abgespeist. Sogenannte Subunternehmer bieten Leiharbeiter an, und so sind diese miserablen Arbeits- und Entlohnungsbedingungen die Regel. Circa 85 Prozent der Beschäftigten in den Schlachthöfen arbeiten auf Werkvertragsbasis. Das alles hat mit Menschenwürde so viel zu tun wie Edelstahl mit Diebstahl, nämlich nichts.  Das Entscheidende ist, dass diese Zustände auch schon vor dieser Pandemie bekannt waren, und diese Regierung tat leider nichts, lieber Hubertus Heil. Ans Tageslicht kommt nun, dass Menschen oftmals zusammengepfercht mit 20 Leuten in einem Raum in miserablen Massenunterkünften wohnen und dafür teilweise auch noch zahlen müssen. Vor fünf Jahren wurde in der Branche eine Selbstverpflichtung aller Unternehmen über Löhne und Arbeitsbedingungen vereinbart. Aus heutiger Sicht war das eine echte Nullnummer.  Arbeitszeitverstöße und Lohnbetrug finden einfach trotzdem statt. Zur Unterbringung ist in dieser Selbstverpflichtung überhaupt nichts geregelt, und genau das rächt sich heute. Sie müssen mir sagen, wie die Einhaltung von Hygieneregeln und Abstandsregeln dort überhaupt möglich sein soll. Mit diesen skandalösen Bedingungen muss endlich Schluss sein.  Die Sofortmaßnahmen der Länder und der Bundesregierung waren okay. Die Probleme in der Fleischwirtschaft sind aber grundsätzlicher. Verdeutlichen möchte ich dies an zwei Beispielen: Erstens. Während die Finanzkontrolle Schwarzarbeit 2008 826 Arbeitgeber in der Fleischindustrie überprüft hat, waren es 2018 nur noch 332. Zweitens. Arbeitsschutzkontrollen in Betrieben finden durchschnittlich nur noch alle 25 Jahre statt.  Das ist ein Freibrief für schwarze Schafe. Als Gesetzgeber nicht zu handeln, obwohl die Probleme bekannt sind, grenzt an Körperverletzung. Die Linke schließt sich den Forderungen der zuständigen Gewerkschaft NGG an. Fünf Sofortmaßnahmen sind nötig: erstens das Verbot von Werkverträgen im Kernbereich der unternehmerischen Tätigkeit,  zweitens Schwerpunktstaatsanwaltschaften für Arbeits- und Gesundheitsschutz,  drittens klare Regeln für Unterkünfte, viertens eine Begrenzung der Unterkunftskosten und fünftens ein brancheneinheitlicher Mindestlohn.  Arbeits- und Gesundheitsschutz können nicht alleine den Unternehmen überlassen bleiben. Lieber Hubertus Heil, wir werden uns genau anschauen, was die Regierung vorhat. Es gab schon so viele Ankündigungen. Deswegen werden wir als Opposition unsere Möglichkeiten nutzen, das scharf zu überwachen, damit das Kind nicht wieder auf Kosten der Arbeitnehmerinnen und Arbeitnehmer in den Brunnen fällt. Vielen Dank.  

\noindent\textbf{Comment:}
\begin{itemize}
    \setlength\itemsep{-3pt}
    \item (Matthias W. Birkwald [DIE LINKE]: Unfassbar!)
    \setlength\itemsep{-3pt}
    \item (Beifall bei der CDU/CSU)
    \setlength\itemsep{-3pt}
    \item (Beifall bei der LINKEN)
\end{itemize}
\subsection{Zimmer}
\noindent\textbf{Texts:} Herr Präsident! Meine Damen und Herren! Man muss in diesen Tagen nicht an Upton Sinclairs Beschreibung der Schlachthöfe von Chicago denken, damit einem der Appetit vergeht. In Hessen ist vor einigen Monaten wegen unhaltbarer hygienischer Zustände ein Unternehmen der Fleischindustrie geschlossen worden, in NRW nun das: Bedingt durch Arbeits- und Unterbringungsbedingungen ist ein Coronahotspot entstanden. Die Liste der Sauereien in dieser Industrie lässt sich fortsetzen. Manchmal habe ich den natürlich völlig unzutreffenden Eindruck: Die Schweine sind nicht nur die, die geschlachtet werden.  Wem hier, meine Damen und Herren, der Appetit nicht vergeht, dem dürfte auch sonst alles egal sein. Ich bin kein Vegetarier und weiß sehr wohl, dass Wurst und der Braten nicht dadurch entstehen, dass die Tiere zu Tode gestreichelt werden. Ich weiß aber auch: Wenn ich mich zu Hause mit Metzgermeistern unterhalte, mit den anständigen Vertretern des Fleischerhandwerkes, dann wird in ihrer Arbeit ein Ethos deutlich, das ich bei den schwarzen Schafen in der Branche schmerzlich vermisse. In meiner Heimatstadt Frankfurt gab es über 600 Jahre lang einen kommunalen Schlachthof. Er wurde 1989 geschlossen, weil es damals als Ausweis gesellschaftlichen Fortschritts galt. Schließlich kann der Markt fast alles besser. Mittlerweile gibt es eine leise Nostalgie. Waren nicht im alten Schlachthof die Arbeitsbedingungen besser, auch die hygienischen Bedingungen? Vor allem aber – so fragt der Metzgermeister meines Vertrauens – ist es nicht besser, regionale Strukturen zu haben? Entspricht es nicht eher dem heutigen Denken, die Schnittstelle von Erzeugung und Verbrauch möglichst wohnortnah zu organisieren und nicht geografisch zu strecken? Ist denn – so möchte man fortfahren – das, was wir jetzt vorfinden, mit der Idee des staatlichen Schutzauftrags für Tiere noch vereinbar, mit der Idee der christlichen Mitgeschöpflichkeit? Sicherlich, Gottes Auftrag aus der Bibel lautet: „Macht euch die Erde untertan.“ Aber war da nicht die treuhänderische, fürsorgliche Dimension mit eingeschlossen, die so wenig vereinbar scheint mit der industriellen Erzeugung und Vernichtung tierischen Lebens? Was hätten die biblischen Künder zu diesem Umgang mit dem tierischen Leben gesagt, angefangen vom unsäglichen Kükenschreddern bis zu dem Prozess der Fleischerzeugung unter fragwürdigsten Bedingungen? Ich gestehe: Je mehr ich darüber nachdenke, desto größer wird die Versuchung, aus diesem System – konkret: aus dem Schweinesystem – durch Verweigerung auszusteigen. Das wäre eine sehr individuelle Entscheidung. Aber wir sind hier als Gesetzgeber gefragt. Das waren wir bereits vor einigen Jahren, als wir das Gesetz zur Sicherung von Arbeitnehmerrechten in der Fleischindustrie verabschiedet haben. Wir sehen heute: Die Branche hat nichts verstanden. Man solle die Branche nicht unter Generalverdacht stellen, hat dieser Tage einer der Vertreter dieser Unternehmen gesagt. Ich denke aber, doch. Wir sollten die Hoffnung nicht über die Erfahrung stellen.  Ich persönlich glaube, dass wir noch einmal gesetzgeberisch tätig werden sollten.  – Ja, in der Tat, tätig werden müssen. – Es kann nicht angehen, dass das Instrument des Werkvertrags hier so missbräuchlich eingesetzt wird.  Wenn festangestellte Mitarbeiter nur noch in Randbereichen der geschäftlichen Aktivität zu finden sind, ist etwas faul. Arbeit in Schlachthöfen darf es nur noch im Rahmen einer Festanstellung geben.  Das System mit osteuropäischen Arbeitskräften, die über Werkverträge angestellt sind, stinkt zum Himmel und muss beendet werden.  Meine Damen und Herren, wir müssen auch erheblich dichter kontrollieren. Fleisch darf nicht krank machen, weder über die Produktionsbedingungen noch über die Arbeitsbedingungen und auch nicht über die Wohnbedingungen. Wir sind nicht im Manchesterkapitalismus.  Ich füge hinzu: Ich bin dem Kollegen Cronenberg für seine sehr nachdenklich stimmende Rede und die vielen Vorschläge, die er gemacht hat, ausgesprochen dankbar. Ich freue mich auf die Diskussion dazu im Ausschuss. Ich finde, das war ein sehr wertvoller Beitrag.  Meine Damen und Herren, ich wünsche mir, dass wir auch hier im Deutschen Bundestag die notwendigen Mehrheiten finden. Denn es ist ja richtig: Kluge Menschen suchen für ein Problem eine Lösung, weniger kluge suchen einen Schuldigen, gelle, Herr Protschka?  Als persönliche Konsequenz, meine Damen und Herren, werde ich jedenfalls heute einmal etwas ausprobieren. Man sagt nämlich, vegetarisches Mett sei gar nicht so schlecht.  

\noindent\textbf{Comment:}
\begin{itemize}
    \setlength\itemsep{-3pt}
    \item (Beifall bei Abgeordneten der CDU/CSU, der SPD, der LINKEN und des BÜNDNISSES 90/DIE GRÜNEN)
    \setlength\itemsep{-3pt}
    \item (Beifall bei der CDU/CSU sowie bei Abgeordneten der SPD und der FDP – Stephan Protschka [AfD]: Eine Lösung hatte ich am Schluss! Ihr müsst aufpassen!)
    \setlength\itemsep{-3pt}
    \item (Beifall bei der FDP sowie bei Abgeordneten der CDU/CSU)
    \setlength\itemsep{-3pt}
    \item (Beifall bei Abgeordneten der SPD, der LINKEN und des BÜNDNISSES 90/DIE GRÜNEN)
    \setlength\itemsep{-3pt}
    \item (Beifall bei der AfD – Stephan Protschka [AfD]: Guter Mann!)
    \setlength\itemsep{-3pt}
    \item (Steffi Lemke [BÜNDNIS 90/DIE GRÜNEN]: Nicht nur einmal! – Ulli Nissen [SPD]: Tätig werden müssen!)
    \setlength\itemsep{-3pt}
    \item (Beifall bei der LINKEN und dem BÜNDNIS 90/DIE GRÜNEN sowie bei Abgeordneten der CDU/CSU und der SPD und des Abg. Dr. Christoph Hoffmann [FDP])
    \setlength\itemsep{-3pt}
    \item (Beifall bei Abgeordneten der SPD und der LINKEN)
    \setlength\itemsep{-3pt}
    \item (Beifall bei Abgeordneten der CDU/CSU, der SPD und der FDP)
\end{itemize}
\subsection{Springer}
\noindent\textbf{Texts:} Herr Präsident! Meine sehr verehrten Damen und Herren! In einigen Schlachtbetrieben in Niedersachsen sind bis zu 50 Prozent der Mitarbeiter erkrankt, aber nicht an Covid‑19, sondern an Tuberkulose, und auch nicht in den letzten Wochen, sondern schon 2018. Das Problem gesundheitsgefährdender Arbeitsbedingungen in der Fleischindustrie ist also kein neues Problem, und der Bundesregierung ist es seit Jahren bekannt. Bei allem Respekt, Herr Minister Heil, da hilft dann hier vorne auch kein betroffenes Gesicht mehr aus der Patsche.  In den vergangenen Jahren haben sich sogar unsere Nachbarländer Belgien, Frankreich und Dänemark kritisch zu unseren, in Deutschland vorherrschenden Lohn- und Arbeitsbedingungen in der Schlachtbranche geäußert. Das sagt nicht irgendwer; das sagt das Bundesarbeitsministerium in seiner Antwort auf eine Kleine Anfrage aus dem Jahr 2017. Der aktuelle Coronaausbruch in Betrieben der fleischverarbeitenden Industrie ist daher nur die Folge eines permanenten Versagens dieser Bundesregierung.  Seit Jahren werden Arbeitskräfte besonders aus dem EU-Ausland in deutschen Schlachtbetrieben ausgebeutet. Sie müssen unter teilweise menschenunwürdigen und gesundheitsgefährdenden Umständen leben und arbeiten. Das ist – so hat es ein Pastor kürzlich bei einer Demo in Coesfeld zum Ausdruck gebracht – nichts anderes als moderne Sklaverei.  Abertausende Solo-Selbstständige – vorwiegend aus Osteuropa – werden mit fragwürdigen Werkverträgen zu Billiglöhnen beschäftigt. Damit werden dem Staat nicht nur Millionen an Sozialabgaben entzogen, sondern auch Arbeitskräfte aus dem In- und Ausland entlang der Lohngrenze gegeneinander ausgespielt. Arbeitnehmer in der Fleischindustrie verdienen heute rund 36 Prozent weniger als in der Gesamtwirtschaft. Stellen Sie sich vor, Sie hätten ein Drittel weniger in der Tasche. Woher kommen die niedrigen Löhne? Vielleicht hängt es damit zusammen, dass die Zahl der deutschen Arbeitskräfte in der Fleischverarbeitung in den letzten zehn Jahren um 20 Prozent gesunken, während die Zahl der ausländischen Arbeitskräfte um über 270 Prozent gestiegen ist. Für diese zumeist osteuropäischen Arbeitskräfte rechnet sich ein ausbeuterisches Arbeitsverhältnis leider häufig, nicht nur weil sie Miete sparen, indem sie sich in Sammelunterkünften einpferchen lassen, sondern auch weil sie in Deutschland Anspruch auf Kindergeld haben  und aufstockende Leistungen, Hartz IV. In der Summe ergibt es ein Einkommen, das weit über dem liegt, das die Menschen in ihrem Heimatland zur Verfügung hätten. Durch die Freizügigkeit innerhalb der Europäischen Union und das konsequente Wegschauen der Bundesregierung wird diese Form der Selbstausbeutung überhaupt erst ermöglicht und werden die Löhne noch weiter gedrückt. Und der Steuerzahler – und das ist die Idiotie an der ganzen Sache – subventioniert diese neue Form der Sklaverei auch noch mit Sozialleistungen.  Das muss dieses soziale Europa sein, von dem Sie immer reden, auch Sie, Herr Heil. Sie reden von einem sozialen Europa, meinen aber die Auflösung nationalstaatlicher Souveränität in einem EU-Superstaat auf dem Rücken der Beschäftigten. Dafür dulden Sie, Herr Minister, dafür duldet die Bundesregierung, auch der überwiegende Teil hier im Parlament, alles, was billig ist, nicht nur in der Fleischindustrie. Schon seit Jahren sehen wir die gleiche Praxis mieser Arbeitsbedingungen und mieser Löhne in der Landwirtschaft, im Sicherheitsgewerbe, im Baugewerbe und auch in der Paketbranche. Wir alle erinnern uns noch gut an die Debatte, die wir zu Weihnachten dazu geführt haben. Dabei liegen die Lösungen, die im Interesse aller sind, doch auf der Hand. Zunächst mal darf es keine Billigimporte, kein Billigfleisch aus dem Ausland geben.  Wir müssen über die sinnvolle Gestaltung der Globalisierung durch Begrenzung der EU-Freizügigkeit nachdenken. Wir müssen die Landwirtschaft regionalisieren, und wir brauchen stärkere und engmaschigere Kontrollen der Betriebe durch staatliche Behörden nicht nur auf Bundesebene, sondern auch dort, wo die Verantwortlichkeiten liegen, bei Ländern und Kommunen. Wenn 80 Prozent der Beschäftigten in der Fleischindustrie einen Werkvertrag haben, also selbstständig sind, dann muss man sich fragen, ob dieses Instrument in dieser Branche nicht verboten gehört. Machen wir uns nichts vor: Schlussendlich wird es faire Arbeitsbedingungen und faire Löhne auch nur dann geben, wenn die Produkte einen gerechten Preis haben. Das ist eine Sache, bei der sich jeder an seine eigene Nase packen muss. Lohndumping und schlechte Arbeitsbedingungen lassen sich verhindern, nicht mit Worten, sondern mit Taten. Herr Heil, oft wurden Taten angekündigt, diesmal wieder. Räumen Sie endlich auf! Danke schön.  

\noindent\textbf{Comment:}
\begin{itemize}
    \setlength\itemsep{-3pt}
    \item (Beifall bei der SPD)
    \setlength\itemsep{-3pt}
    \item (Beifall bei der AfD)
    \setlength\itemsep{-3pt}
    \item (Ulli Nissen [SPD]: Das musste ja wieder kommen! – Matthias W. Birkwald [DIE LINKE]: Unterirdisch!)
\end{itemize}
\subsection{Mast}
\noindent\textbf{Texts:} Herr Präsident! Meine sehr verehrten Kolleginnen und Kollegen! 800 Coronainfizierte gibt es in der Fleischindustrie in Deutschland. Ich habe die aktuellen Zahlen noch mal recherchiert: ungefähr 400, davon einige schon wieder geheilt, in Birkenfeld bei Müller Fleisch, in meinem Wahlkreis in Pforzheim, weitere noch in Bad Bramstedt in Schleswig-Holstein, in Coesfeld in Nordrhein-Westfalen. Und jeder einzelne und jede einzelne Infizierte ist einer bzw. eine zu viel.  Die Coronapandemie macht eines deutlich: Sie richtet den Scheinwerfer darauf und ermöglicht diese Debatte. Ich bin den Grünen ausdrücklich dankbar für die Beantragung dieser Aktuellen Stunde, weil sie auch ermöglicht, dass wir uns gemeinsam vergewissern, wo wir stehen und wo wir in dieser Debatte vielleicht auch nicht stehen.  – Ja, und wo wir hinmüssen; das gehört ja am Ende des Tages dazu. Mir ist das wichtig, was Minister Heil als Verantwortung bezeichnet hat; ich bezeichne es immer als Geschäftsmodell. Das Geschäftsmodell der Fleischindustrie, und zwar das der gesamten Industrie – deshalb haben wir nämlich schon drei solche Pandemiefälle in Deutschland; es sind nicht einzelne Akteure; es ist ein System –, beinhaltet: wenig eigene Mitarbeiterinnen und Mitarbeiter und einen Riesenlöwenanteil von Menschen aus Südosteuropa, die unter extremen Bedingungen, extremen Unterkunftsbedingungen und Transportbedingungen, dort täglich schuften. Deshalb geht es mir auch in dieser Debatte um die Frage: Wer hat hier eigentlich im Sinne des Gemeinwohls, im Sinne eines Arbeitsschutzes für Arbeitnehmerinnen und Arbeitnehmer und Beschäftigte die Verantwortung? Die Verantwortung liegt bei der Fleischindustrie, Kolleginnen und Kollegen, als Allerallerallererstes,  bei den Unternehmerinnen und Unternehmern, und zwar für die ganze Kette der Beschäftigten. Was hier stattfindet, ist, dass man auf Subunternehmen verschiebt, auf Subsubunternehmen, auf Vermieterinnen und Vermieter, auf die Behörden, die angeblich nicht ordentlich kontrollieren, und alles Mögliche. Es geht darum: Die Verantwortung für diese Infektionsfälle liegt bei den Unternehmerinnen und Unternehmern, und es ist mir ganz wichtig, dass wir da gemeinsam klar haben, dass es darum in dieser Debatte geht.  Natürlich ist es unerträglich – und ich glaube, für jeden Menschen, der das Herz am richtigen Fleck hat, muss es unerträglich sein –, dass mitten unter uns Menschen leben, die sich Matratzen in Unterkünften teilen, dafür Geld abgezogen bekommen, Arbeitszeiten haben, die weit über einen Zwölfstundentag hinausgehen, und oft sechs oder sieben Tage die Woche arbeiten. Das ist nicht überall der Fall. Aber es gibt doch verdammt viele Berichte darüber, die sich in diesem Bereich häufen, und deshalb müssen wir handeln. Jetzt ist mir aber als sozialdemokratische Bundestagsabgeordnete wichtig, dass hier nicht so der Slang reinkommt, wir hätten noch nie was getan. Wir haben mit unserer Arbeitsministerin Andrea Nahles den Mindestlohn durchgesetzt, übrigens mit vielen Stimmen hier aus dem Haus, die damals nicht mit in der Regierung waren. Wir haben Werkverträge und Leiharbeit reguliert, wenn auch, wie wir immer wieder merken, leider nicht genug. Die Mehrheiten waren nun mal so, wie sie waren. Aber wir haben etwas getan. Weil ich weiß, dass er auch zuschaut, will ich ausdrücklich Karl Schiewerling für das gemeinsame Engagement danken, aufgrund dessen wir 2017 gemeinsam in der Koalition die Voraussetzungen für die Fleischwirtschaft massiv verschärft haben. Wir haben das Messergeld abgeschafft,  wir haben die Arbeitszeitdokumentation verschärft, und wir haben die Generalunternehmerhaftung bewirkt, und jetzt wissen wir: Wir müssen noch mehr tun. Mir ist aber auch Folgendes wichtig, und deshalb will ich noch mal meine Kolleginnen und Kollegen von den Grünen anschauen – nach mir spricht ja auch Beate Müller-Gemmeke aus dem gleichen Bundesland wie ich –: Ich bin in Pforzheim Abgeordnete, und ich beschäftige mich – seit Ostern täglich mehrere Stunden, auch davor übrigens schon, viele Jahre – mit den Arbeitsbedingungen in der Fleischwirtschaft. 2013 waren bei mir im Wahlkreis ungarische Werkverträgler, die kein Geld gekriegt haben, die vor der Firma kampiert haben. Ich bin wirklich bewegt, betroffen, dass ich kein einziges Wort zu diesem Thema von einem Mitglied der Landesregierung in Baden-Württemberg höre.  Das heißt nicht, dass ich Schwarze Peter hin- und herschieben will. Ich finde aber, zur Wahrheit in der Debatte gehört auch, dass wir eine Haltungs- und eine Zukunftsdebatte nicht nur im Deutschen Bundestag brauchen, sondern wir brauchen sie auch in allen Landtagen. Überall, wo wir Verantwortung haben, müssen wir uns gemeinsam hinstellen und das auch einfordern, egal wo wir aktiv sind.  Jetzt noch mal zu den Punkten, worum es mir geht, was verändert werden muss. Ein Projekt ist noch gar nicht erwähnt worden, nämlich das Projekt „Faire Mobilität“. Wir müssen dringend in der Koalition daran arbeiten, dass die Leute, die die Menschen beraten, die da zu schäbigen Bedingungen arbeiten, auch eine gute, dauerhafte Beratung bieten können – weg von der Projektförderung, hin in die dauerhafte Förderung. Ansonsten sind viele Dinge gesagt worden: Kontrollen intensivieren, Arbeitszeitdokumentation, Arbeitsschutz ausbauen, auch gerne die Strafen ausbauen, die Justiz ermächtigen, Werkverträge in Kernbereichen der Produktion nicht zulassen. Es gibt viel gemeinsam zu tun. Ich nehme auch wahr, dass nicht nur die Regierungsfraktionen, sondern viele von der Opposition dabei mit uns unterwegs sind. Lassen Sie uns die Kraft gemeinsam nutzen! Die Menschen in der Fleischindustrie brauchen uns.  

\noindent\textbf{Comment:}
\begin{itemize}
    \setlength\itemsep{-3pt}
    \item (Beifall bei der SPD sowie bei Abgeordneten der CDU/CSU)
    \setlength\itemsep{-3pt}
    \item (Beifall bei der SPD)
    \setlength\itemsep{-3pt}
    \item (Ulli Nissen [SPD]: Hört! Hört!)
    \setlength\itemsep{-3pt}
    \item (Beifall beim BÜNDNIS 90/DIE GRÜNEN)
    \setlength\itemsep{-3pt}
    \item (Beifall bei der SPD und der CDU/CSU)
    \setlength\itemsep{-3pt}
    \item (Steffi Lemke [BÜNDNIS 90/DIE GRÜNEN]: Und wo wir hinmüssen vor allem!)
\end{itemize}
\subsection{Müller-Gemmeke}
\noindent\textbf{Texts:} Sehr geehrter Herr Präsident! Kolleginnen und Kollegen! Die schlechten Arbeitsbedingungen sind seit Jahren bekannt. Die Ausbeutung in der Fleischbranche ist immer wieder Thema und gerät dann doch immer wieder in Vergessenheit. Ich hoffe, dass das dieses Mal nicht passiert, nachdem die Fleischbranche zum Hotspot der Coronakrise geworden ist. Liebe Katja Mast, natürlich reagiert auch die Landesregierung in Baden-Württemberg; Manne Lucha reagiert natürlich darauf. Ich finde es schon komisch: Jetzt, heute auf einmal, haben es alle verstanden, wirklich alle. Wie oft haben wir hier schon das Thema diskutiert! Ich kann Anträge erwähnen, die wir gestellt haben,  wo wir zum Beispiel die Werkverträge verändern wollten. Da habe ich von der SPD nichts gehört, aber auch gar nichts. Also, von daher: Ein bisschen Ehrlichkeit in der Debatte ist schon wichtig.  Wir müssen jetzt natürlich die Beschäftigten schützen, die Infektionen stoppen. Es geht aber nicht nur um das Coronavirus jetzt. Es geht vor allem um die Menschen, die ausgebeutet werden, und deshalb muss ganz grundsätzlich das System, das Geschäftsmodell Fleischbranche, kritisiert werden. Genau das ist schon lange überfällig.  Das System funktioniert mit Werkvertragsbeschäftigten, Scheinselbstständigen und Leiharbeitskräften. Die Beschäftigten, meist aus Osteuropa, arbeiten hart für billige Löhne, zehn und mehr Stunden, sechs Tage in der Woche. Sie leben in engen Unterkünften – zum Teil mehrere Personen in einem Zimmer. Sie werden in vollgestopften Bussen zu den Schichten gefahren. Sie werden ausgebeutet und halten so das System am Laufen. Diese Arbeitsbedingungen sind menschenunwürdig, und in Zeiten von Corona sind das auch Risikofaktoren. Das zeigt: Das System der Fleischbranche ist einfach nicht akzeptabel.  In dieser Branche wird per Werkvertrag Verantwortung verlagert – genau das ist das Geschäftsmodell –, und deshalb ist Schlachten in Deutschland so billig. Ich zitiere: „Die Problematik liegt im System. Und für das System verantwortlich sind die Betreiber der Schlachthöfe.“ Das war ein Zitat von Karl-Josef Laumann, dem zuständigen CDU-Minister aus NRW. Mit diesem System muss endlich Schluss sein. Damit spreche ich vor allem die Union an,  die sich bisher da immer gesperrt hat. Alle Beschäftigten haben ein Recht auf gute Arbeitsbedingungen und auf effektiven Arbeits- und Gesundheitsschutz.  Fünf Maßnahmen und Überlegungen: Erstens. Die Beschäftigten dürfen natürlich jetzt weder am Arbeitsplatz noch in ihrer Unterkunft einem Infektionsrisiko ausgesetzt werden. Die Arbeitgeber müssen regelmäßige Tests sicherstellen. Wichtig ist auch, dass die Beschäftigten Information und Beratung bekommen, und zwar in einer Sprache, die sie auch verstehen.  Zweitens. Im Jahr 2017 – das wurde schon angesprochen – wurde in einer Nacht-und-Nebel-Aktion durch den damaligen, von mir sehr geschätzten CDU-Kollegen Karl Schiewerling das Fleischgesetz auf den Weg gebracht. Seither gibt es in der Fleischbranche die Generalunternehmerhaftung bei den Sozialversicherungsbeiträgen. Genau das brauchen wir jetzt wieder: eine Generalunternehmerhaftung, und zwar für den Arbeitsschutz; denn kein Fleischkonzern darf sich mit Subunternehmen beim Arbeitsschutz aus der Verantwortung stehlen.  Drittens. Es geht aber nicht nur um den Schlachthof als Betrieb, sondern auch um die Wohnsituation der Beschäftigten. Es muss der Zugang zu ordentlichem und bezahlbarem Wohnraum garantiert werden. Auch das darf nicht auf die Subunternehmen verlagert werden. Es sind die Betreiber der Schlachthöfe, die dafür Verantwortung tragen müssen.  Viertens. Arbeitsplatz und Unterkunft müssen dann natürlich auch effektiv kontrolliert werden. Wir, Bund und Land, sind momentan gemeinsam in der Verantwortung. Ich würde gerne die Finanzkontrolle Schwarzarbeit zu einer Arbeitsinspektion weiterentwickeln. Wenn die Kontrollen von Gesundheitsschutz, von Arbeitsschutz, von Arbeitszeit und dann noch von korrekter Entlohnung an einer Stelle gebündelt werden, dann wären die Kontrollen zielgenau und auch effektiv.  Und fünftens. Wir müssen uns auch ganz grundsätzlich mit den Werkverträgen beschäftigen. Wenn zwei Drittel der Beschäftigten bei Subunternehmen angestellt sind, aber genau die Tätigkeiten machen, die einen Schlachthof ausmachen, Schlachten und Zerlegen, dann stimmt da etwas nicht, dann sind das zweifelhafte Werkvertragskonstruktionen, mit denen sich die großen Fleischkonzerne bei allen arbeitsrechtlichen Aspekten aus der Verantwortung stehlen. Das geht nicht. Die Rahmenbedingungen müssen wirklich so verändert werden, dass sie die Beschäftigten selber anstellen müssen; und dann muss der Druck erhöht werden, dass sie auch noch einen Tarifvertrag abschließen. Herr Minister, dann nehmen wir Sie wirklich beim Wort. – Jetzt hat er es nicht gehört, aber es ist egal. Wir werden ihn immer und immer wieder daran erinnern.  Mein Fazit: In der Fleischbranche sind die Arbeitsbedingungen katastrophal, nicht nur jetzt, in Zeiten von Corona. Freiwillig wird hier nichts passieren. Deshalb müssen Sie, die Regierungsfraktionen, endlich handeln. Vielen Dank.  

\noindent\textbf{Comment:}
\begin{itemize}
    \setlength\itemsep{-3pt}
    \item (Jutta Krellmann [DIE LINKE]: Wir auch!)
    \setlength\itemsep{-3pt}
    \item (Beifall beim BÜNDNIS 90/DIE GRÜNEN)
    \setlength\itemsep{-3pt}
    \item (Beifall beim BÜNDNIS 90/DIE GRÜNEN sowie bei Abgeordneten der LINKEN)
    \setlength\itemsep{-3pt}
    \item (Beifall bei der CDU/CSU)
    \setlength\itemsep{-3pt}
    \item (Beifall beim BÜNDNIS 90/DIE GRÜNEN und bei der LINKEN)
\end{itemize}
\subsection{Schimke}
\noindent\textbf{Texts:} Herr Präsident! Meine Damen und Herren! Corona ist ein ernstes Thema, ein Thema von historischer Tragweite für unser Land. Es hat uns seit Wochen lahmgelegt, fährt unsere Wirtschaft praktisch an die Wand, und wir wissen noch nicht, wie die Zukunft aussieht. Umso mehr finde ich es erschreckend, dass die Situation in den Schlachtbetrieben, die hohen Infektionsraten, die wir dort haben, jetzt von der Opposition als Mittel zum Zweck verwendet werden,  die übliche Klaviatur politischer Forderungen hier erneut anzubringen,  wie wir sie schon seit Langem kennen.  – Sie können sich gerne aufregen,  aber, meine Damen und Herren, was haben denn bitte Werkverträge oder Entlohnungsfragen mit dem Coronavirus zu tun? Nichts.  Meine Damen und Herren, wenn wir in der Coronapandemie die Infektionen verhindern wollen, wenn wir die Infektionen in Deutschland eindämmen wollen, dann müssen wir über Hygiene reden,  dann müssen wir über Abstand reden, aber nicht über Vertragsformen. Das eine hat mit dem anderen doch nichts zu tun.  An dieser Stelle, finde ich, geht es um das Prinzip unserer politischen Debatte, und es geht mir persönlich – da möchte ich bitte auch nicht falsch verstanden werden – nicht um die Bevorzugung irgendeiner Branche, sondern es geht schlichtweg darum, politische Klarheit an den Tag zu legen und auch glaubwürdig in seinen Forderungen aufzutreten.  Ich finde es wirklich schlimm, dass Sie sich hier über alles Mögliche unterhalten, nur nicht über Lösungen, wie man die Menschen vor diesem Virus schützen kann, meine Damen und Herren.  Mal etwas zur Klarheit: Was wissen wir eigentlich zum heutigen Tag? Wir wissen, dass es am 27. April den ersten Coronafall bei Westfleisch gab. Wir wissen auch, dass das BMAS am 16. April seinen sogenannten Arbeitsschutzstandard mit Bezug auf Corona herausgebracht hat, gut zwei Wochen vorher. Am 23. April folgten dann die ersten Handlungsempfehlungen des Landes Nordrhein-Westfalen; weitere folgten im Mai, und sie werden regelmäßig angepasst. Was will ich damit sagen? Die politischen Standards, die wir vorgeben, natürlich auch in Beratung mit vielen Wissenschaftlern und Forschern, kommen, gemessen an dem Infektionsgeschehen, doch recht spät. Was glauben Sie denn, wie schnell so ein Unternehmen reagieren kann, die Standards, die wir politisch formulieren, am Ende auch in die betriebliche Praxis umzusetzen?  Was darf man auch an dieser Stelle sagen? Das, worüber wir hier reden, wer die Schuld hat, woran das alles liegt, meine Damen und Herren, sind alles Vermutungen zum Infektionsgeschehen. Das kann niemand ganz genau sagen; selbst die Behörden vermuten nur.  Fest steht: Die Masken wurden wohl nicht richtig im Dienst getragen. Zum Transport gibt es keine validen Aussagen. Wohl hat das Unternehmen darauf geachtet, dass die Abstände eingehalten wurden; das ist aber nicht verifiziert. Es ist im Allgemeinen auch recht wenig bekannt über die Unterkünfte bei Westfleisch. Auch der Landrat des Landkreises hat gesagt, dass es überwiegend keine gravierenden Mängel gegeben hat. Also, meine Damen und Herren, ich frage gerne noch einmal: Worüber reden wir hier eigentlich? Westfleisch hat sehr wohl auch über ein Hygienekonzept verfügt. Das war sogar behördlich abgestimmt. Der Vorwurf, den man hier machen kann – da gebe ich Ihnen sehr wohl recht –, ist der – –   Vielen Dank, Herr Präsident; vielen Dank. – Der Vorwurf, den man dem Unternehmen vielleicht machen kann, ist, dass es dieses Hygienekonzept, das es sich selbst auferlegt hat, möglicherweise nicht ordnungsgemäß geprüft und eingehalten hat. Aber das, was hier heute in der Debatte bisher zum Besten gegeben wurde, hat weder mit dem Infektionsgeschehen in Deutschland noch mit einem verbesserten Schutz der Arbeitnehmerinnen und Arbeitnehmer in dieser Branche zu tun. Jetzt ist es so: Endlich beginnen die Massentests; jetzt beginnen die Tests, die wir eigentlich schon seit Wochen in Deutschland bräuchten. Ich wäre doch mal sehr gespannt, zu erfahren, wie solche Tests in anderen Einrichtungen ausfallen würden, beispielsweise in Flüchtlingsunterkünften oder in Studentenwohnheimen.  Also: Was ist unser Wissensstand bei der Frage „Wo sind Infektionsherde, wie sieht es im Moment überhaupt aus, wer ist alles erkrankt, wer war schon erkrankt?“? Das wissen wir nicht.  Mir ist immer sehr an Sachlichkeit in so einer Debatte gelegen.  Ich finde, unsere Aufgabe ist es auch, unser politisches Handeln und unsere politischen Forderungen an der Wahrheit auszurichten. Alles andere ist schädlich und wird der Würde dieses Hauses nicht gerecht. In diesem Sinne, meine Damen und Herren: Ich wünsche uns weiterhin gute Beratungen. Vielen Dank.  

\noindent\textbf{Comment:}
\begin{itemize}
    \setlength\itemsep{-3pt}
    \item (Beifall bei der CDU/CSU – Matthias W. Birkwald [DIE LINKE]: O si tacuisses, philosophus mansisses! – Steffi Lemke [BÜNDNIS 90/DIE GRÜNEN]: Die Rede da drüben, die holen wir im Wahlkampf wieder raus!)
    \setlength\itemsep{-3pt}
    \item (Beifall bei der SPD)
    \setlength\itemsep{-3pt}
    \item (Steffi Lemke [BÜNDNIS 90/DIE GRÜNEN]: So eine Unverschämtheit! Schämen Sie sich!)
    \setlength\itemsep{-3pt}
    \item (Zurufe von der LINKEN)
    \setlength\itemsep{-3pt}
    \item (Zurufe von der LINKEN und dem BÜNDNIS 90/DIE GRÜNEN)
    \setlength\itemsep{-3pt}
    \item (Steffi Lemke [BÜNDNIS 90/DIE GRÜNEN]: Eine schäbige Rede ist das, eine sehr schäbige Rede!)
    \setlength\itemsep{-3pt}
    \item (Ulli Nissen [SPD]: Das ist ja superpeinlich!)
    \setlength\itemsep{-3pt}
    \item (Beifall bei Abgeordneten der CDU/CSU)
    \setlength\itemsep{-3pt}
    \item (Steffi Lemke [BÜNDNIS 90/DIE GRÜNEN]: Nicht zugehört bisher!)
    \setlength\itemsep{-3pt}
    \item (Steffi Lemke [BÜNDNIS 90/DIE GRÜNEN]: Frechheit!)
    \setlength\itemsep{-3pt}
    \item (Widerspruch bei der LINKEN und dem BÜNDNIS 90/DIE GRÜNEN – Steffi Lemke [BÜNDNIS 90/DIE GRÜNEN]: Einfach wegreden das Problem – so kommen Sie nicht durch!)
    \setlength\itemsep{-3pt}
    \item (Widerspruch beim BÜNDNIS 90/DIE GRÜNEN)
    \setlength\itemsep{-3pt}
    \item (Steffi Lemke [BÜNDNIS 90/DIE GRÜNEN]: Nichts verstanden!)
    \setlength\itemsep{-3pt}
    \item (Steffi Lemke [BÜNDNIS 90/DIE GRÜNEN]: Ich rege mich nicht auf!)
    \setlength\itemsep{-3pt}
    \item (Widerspruch bei der LINKEN und dem BÜNDNIS 90/DIE GRÜNEN – Klaus Ernst [DIE LINKE]: Peinlich!)
    \setlength\itemsep{-3pt}
    \item (Lachen bei der LINKEN und dem BÜNDNIS 90/DIE GRÜNEN – Matthias W. Birkwald [DIE LINKE]: Der war gut! Der war gut!)
    \setlength\itemsep{-3pt}
    \item (Widerspruch bei der LINKEN)
    \setlength\itemsep{-3pt}
    \item (Steffi Lemke [BÜNDNIS 90/DIE GRÜNEN]: In Schlachthöfen unter anderem!)
\end{itemize}
\subsection{Gerdes}
\noindent\textbf{Texts:} Herr Präsident! Meine Damen und Herren! Die Coronapandemie hat vieles in unserem Alltag verändert, aber eben nicht alles. Manches kommt jetzt noch deutlicher zum Vorschein als sonst. Darunter fallen auch die Arbeitsbedingungen in der Fleischindustrie, die an vielen Stellen mangelhaft sind, leider nicht erst seit gestern. Durch Corona kommen die Sauereien – im Sinne des Wortes – wieder ans Licht. Schon länger kämpfen wir gegen fragwürdige Geschäftsmodelle mit Subunternehmen und Missachtung von Schutzstandards. Bereits im Juni 2017 – wir haben es vorhin schon gehört – haben wir im Bundestag das Gesetz zur Sicherung von Arbeitnehmerrechten in der Fleischwirtschaft beschlossen. Hierbei ging es nicht nur um die konkrete und korrekte Zahlung von Sozialversicherungsbeiträgen, sondern auch um die Bereitstellung von aus Hygienegründen oder Gründen der Arbeitssicherheit vorgeschriebener besonderer Arbeitskleidung und persönlicher Schutzausrüstung durch den Arbeitgeber. Doch wer der eigentliche Arbeitgeber ist, das wird durch windige Werkverträge verschleiert. Wir kennen Berichte von unerlaubten Überstunden, fehlenden Pausen, unzureichender Hygiene, nicht eingehaltenem Mindestabstand, überfüllten Sammelunterkünften und schlechten Löhnen. Auch wenn es die Branchenvertreter abstreiten, für mich steht es außer Frage: Es hat auch mit Corona zu tun. Und dass ausgerechnet in den Schlachthöfen die Anzahl der Coronainfizierten so hochgegangen ist, ist Fakt und ist auch eine Folge von menschenunwürdigen Arbeits- und Lebensbedingungen, denen die Beschäftigten, zumeist Leiharbeiter aus Osteuropa, ausgesetzt sind. Bis zu 80 Prozent der Beschäftigten mancher Betriebe arbeiten über Werkverträge. Wir müssen uns also damit befassen, ob gegebenenfalls die Einführung einer Quote, also eines Höchstmaßes an Werkverträgen, sinnvoll wäre. Es gibt aber auch Forderungen, die Werkverträge im Kernbereich der Fleischindustrie komplett zu verbieten. Ich glaube, das ist auch einer Überlegung wert.  Liebe Kolleginnen und Kollegen, ich hoffe sehr, dass wir mit Blick auf diese Corona-Hotspots keine Tragödie erleben und die Infizierten schnell gesunden und vor allem keine schweren Krankheitsverläufe durchleben müssen. Wir reden aber leider nicht von Einzelfällen. Betroffen sind Beschäftigte verschiedener Fleischbetriebe, und das in mehreren Bundesländern. Was können wir besser machen? Die Berichte über die Zustände in den Fleischfabriken und die Zwänge, denen die Mitarbeiter ausgesetzt sind, widersprechen oftmals allen Regeln, die der Arbeits-, Gesundheits- und Infektionsschutz vorgibt. Teilweise – das haben wir hier gerade schon gehört, und ich stehe dazu – stoßen wir auf mafiöse Strukturen. Die Drahtzieher sitzen oftmals im osteuropäischen Ausland und sind für uns schwer haftbar zu machen.  Minister Heil hat es bereits gesagt, und auch Vorredner haben darauf hingewiesen: Wir haben weniger ein Regelproblem, sondern vor allem ein Verantwortungsproblem. Um es klar zu sagen: Arbeitgeber sind mit oder ohne Coronagefahr dazu verpflichtet, dafür zu sorgen, dass es eine Gefährdungsbeurteilung gibt, dass sie den Gesundheits- und Arbeitsschutz ihrer Mitarbeiter sicherstellen. Es gibt klar definierte Standards, technische Anweisungen und viele Hilfestellungen seitens der Berufsgenossenschaften, wie der Arbeitsschutz durchzuführen ist. In der Arbeitsstättenverordnung beispielsweise lässt sich sogar nachlesen, wie hoch die Anzahl der Quadratmeter pro Bewohner in den Unterkünften zu sein hat. Nützt aber alles nichts, wenn Verantwortlichkeiten umgangen werden und niemand kontrolliert. Hier sehe ich leider ein Defizit vor allem bei den Ländern. Die Kontrollen im Bereich des Arbeits- und Gesundheitsschutzes müssen dringend hochgefahren werden. Auf Vertrauen und Selbstverpflichtung alleine können wir an dieser Stelle anscheinend nicht setzen.  Hier werden wir handeln und bestehende Gesetze anwenden, zum Teil auch erweitern müssen. Gute Arbeit und Fairness in den Großbetrieben der Fleischbranche sehen anders aus.  Wir müssen überlegen, wie wir das Umgehen von Gesetzen und Standards durch Subunternehmen und Werkverträge verhindern und wie wir die Auftraggeber, sprich: die Betreiber der Schlachtbetriebe, stärker in die Verantwortung nehmen, und zwar, wenn es notwendig ist, auch mit höheren Strafen, die dann allerdings auch wehtun müssen.  Wir müssen auch mal darüber nachdenken – wir haben hier schon über die Notwendigkeit der Tarifbindung gesprochen –, ob wir das Betriebsverfassungsgesetz nicht an der einen oder anderen Stelle nachschärfen sollten, damit Betriebsräte in den Schlachtbetrieben beispielsweise nicht nur ein Informations-, sondern auch ein Mitbestimmungsrecht erhalten.  Nicht nur die Gesundheit der ausländischen Beschäftigten in den Schlachthöfen steht auf dem Spiel; es geht auch um das Tierwohl, gesunde Ernährung, faire Arbeitsbedingungen und gute Lebensbedingungen. Herzlichen Dank. Glück auf!  

\noindent\textbf{Comment:}
\begin{itemize}
    \setlength\itemsep{-3pt}
    \item (Beifall bei Abgeordneten der SPD und des Abg. Artur Auernhammer [CDU/CSU])
    \setlength\itemsep{-3pt}
    \item (Beifall bei der SPD sowie des Abg. Dr. Matthias Zimmer [CDU/CSU])
    \setlength\itemsep{-3pt}
    \item (Zuruf des Abg. Pascal Meiser [DIE LINKE])
    \setlength\itemsep{-3pt}
    \item (Beifall bei Abgeordneten der SPD)
    \setlength\itemsep{-3pt}
    \item (Beifall bei Abgeordneten der SPD und der LINKEN)
    \setlength\itemsep{-3pt}
    \item (Beifall bei der SPD und der LINKEN sowie der Abg. Beate Müller-Gemmeke [BÜNDNIS 90/DIE GRÜNEN])
    \setlength\itemsep{-3pt}
    \item (Beifall bei der CDU/CSU sowie der Abg. Dagmar Ziegler [SPD])
    \setlength\itemsep{-3pt}
    \item (Beifall der Abg. Ulli Nissen [SPD], Dr. Matthias Zimmer [CDU/CSU] und Friedrich Ostendorff [BÜNDNIS 90/DIE GRÜNEN])
\end{itemize}
\subsection{Straubinger}
\noindent\textbf{Texts:} Herr Präsident! Werte Kolleginnen und Kollegen! Am Ende dieser Debatte, die die Grünen aufgrund der Häufungen von Coronafällen in Schlachtbetrieben in Nordrhein-Westfalen bzw. in Baden-Württemberg beantragt haben, gilt es, ein Resümee zu ziehen. Eines ist schon deutlich geworden: Ich glaube, dass Corona jetzt nicht allein auf die Schlachtbetriebe zurückzuführen ist.  Vielmehr kämpfen wir seit Wochen mit der Pandemie. Wir stellen auch in anderen Betriebseinrichtungen ein verstärktes Auftreten fest, nicht nur in Schlachthöfen. Wir haben erleben müssen – leider –, dass in Altenheimen verstärkt Corona festgestellt worden ist, dass in Krankenhäusern verstärkt Corona festgestellt worden ist und der Betrieb eingestellt werden musste.  Das jetzt sozusagen als symptomatisch für die Schlachtbetriebe hinzustellen, wie es die Grünen versuchen,  und daraus dann neue Konstruktionen herbeizuführen, wie es Frau Kollegin Schimke auch schon dargestellt hat, das ist meines Erachtens zu wenig und auch nicht zielführend.  Es ist natürlich das entscheidende Element beim Arbeitseinsatz insgesamt, dass Arbeitsschutz, dass Gesundheitsschutz, dass medizinische Versorgung für die Arbeitnehmerinnen und Arbeitnehmer gewährleistet ist; das ist das höchste Gut. Da sind zuerst die Unternehmen selbst in der Verantwortung, unabhängig davon, um was für ein Unternehmen es sich handelt. Da sind zuerst die Unternehmen gefordert, meine Damen und Herren.  Wir haben als Gesetzgeber den nötigen Rahmen zu setzen, und ich glaube, dass wir in den vergangenen Jahren verstärkt gegen Missstände in der Schlachtindustrie vorgegangen sind. Katja Mast hat ja bereits Karl Schiewerling angesprochen, und wir alle, die wir schon länger in der Arbeitsgruppe Arbeit und Soziales und in der Sozialpolitik tätig sind, waren ja mit dabei, als wir hier 2017 vor allen Dingen mit der Einführung der Generalunternehmerhaftung entscheidende Wegmarken dafür gesetzt haben, dass die Sozialversicherungsbeiträge – auch zum Schutz der Arbeitnehmerinnen und Arbeitnehmer – richtig und ordentlich abgeführt werden, womit auch Ordnung in diesem Bereich des Arbeitsmarktes eingetreten ist. Wir haben gemeinsam in der Koalition auch mit dem Arbeitnehmer-Entsendegesetz wichtige Wegmarken geschaffen. Deshalb: Es gibt hier eher ein Vollzugsproblem mit unseren Gesetzen.  Anstatt nach neuen Gesetzen zu rufen, sollten Sie daran mitarbeiten, dieses Problem zu beheben. Ich danke an dieser Stelle ausdrücklich dem Landesminister Karl-Josef Laumann für seine umsichtige Arbeit; Frau Kollegin Schimke hat es dargelegt. Am 26. April wurde ein Coronafall in dem Betrieb erstmals festgestellt. Was ich bemerkenswert finde, ist, dass es dann doch so lange gedauert hat bzw. dass die Infektionswege nicht so stark eingeschränkt werden konnten, sodass letztendlich über 260 Mitarbeiter mit Corona konfrontiert wurden. Das ist bedauerlich – darüber brauchen wir gar nicht zu reden –, und das dürfte auch nicht sein.  Auf der anderen Seite ist aber vielleicht auch mit ein Grund, dass sich nicht alle Mitarbeiter in diesen Dingen ganz bewusst verhalten.  Manch einer, der hustet, geht trotzdem zur Arbeit, obwohl er eigentlich vielleicht nicht mehr arbeiten und zu Hause bleiben sollte. Auch daran kann es liegen.  Es kann auch sein, dass mancher einfach lieber arbeitet, als sich krankzumelden, und deshalb die Infektionen weitergetragen worden sind. Die Missstände müssen aufgeklärt werden; aber es gilt auch: eigenverantwortliches Handeln. Ich bin auch dankbar, dass im Dezember der Bericht zur Überwachung in der Fleischindustrie zum Abschluss gebracht worden ist. Wenn bei 8 700 Verstößen, die festgestellt worden sind, 86 Bußgeldverfahren eingeleitet worden sind, aber wohl keine staatsanwaltschaftlichen Ermittlungen – vielleicht gab es dafür auch keine Grundlage –, so zeigt das sehr deutlich, dass staatliches Handeln es schon möglich macht, Druck auszuüben, damit Arbeitsschutz, Gesundheitsschutz und alle Dinge, die für die Arbeitnehmerinnen und Arbeitnehmer wichtig sind, auch zur Umsetzung gebracht werden.  Ich weiß: Das ist ein schwerer Weg. Da wünsche ich mir, dass wir gemeinsam daran arbeiten, hier vernünftige Lösungen zu finden. Aber ich glaube nicht, dass es entscheidend ist, irgendwelche Vertragsformen zu verbieten; denn für alle Vertragsarten – ob Einzelunternehmer, ob juristisch organisiertes Unternehmen oder Werkvertrag –  gelten die gleichen Arbeitsbedingungen, gelten die gleichen medizinischen Vorschriften, und die müssen zum Einsatz gebracht werden. Herzlichen Dank für die Aufmerksamkeit.  

\noindent\textbf{Comment:}
\begin{itemize}
    \setlength\itemsep{-3pt}
    \item (Beifall bei Abgeordneten der CDU/CSU – Beate Müller-Gemmeke [BÜNDNIS 90/DIE GRÜNEN]: Genau deswegen muss man die Unternehmen kritisieren! Das kann auch die CSU mal machen!)
    \setlength\itemsep{-3pt}
    \item (Zuruf der Abg. Beate Müller-Gemmeke [BÜNDNIS 90/DIE GRÜNEN])
    \setlength\itemsep{-3pt}
    \item (Beifall bei der SPD)
    \setlength\itemsep{-3pt}
    \item (Zurufe der Abg. Katja Mast [SPD] und Beate Müller-Gemmeke [BÜNDNIS 90/DIE GRÜNEN])
    \setlength\itemsep{-3pt}
    \item (Beifall bei Abgeordneten der CDU/CSU – Steffi Lemke [BÜNDNIS 90/DIE GRÜNEN]: Sie sollen in allen Bereichen handeln!)
    \setlength\itemsep{-3pt}
    \item (Steffi Lemke [BÜNDNIS 90/DIE GRÜNEN]: Da sollten Sie in der Tat auch handeln! Dazu haben wir Sie auch schon aufgefordert! – Gegenruf des Abg. Manfred Grund [CDU/CSU]: Das Genöle!)
    \setlength\itemsep{-3pt}
    \item (Steffi Lemke [BÜNDNIS 90/DIE GRÜNEN]: Das hat ja auch kein Mensch behauptet! – Beate Müller-Gemmeke [BÜNDNIS 90/DIE GRÜNEN]: Das hat niemand gesagt!)
    \setlength\itemsep{-3pt}
    \item (Beate Müller-Gemmeke [BÜNDNIS 90/DIE GRÜNEN]: Das ist wirklich nicht die beste Rede von Herrn Straubinger, muss ich jetzt mal feststellen!)
    \setlength\itemsep{-3pt}
    \item (Beate Müller-Gemmeke [BÜNDNIS 90/DIE GRÜNEN]: Nein! Stimmt nicht!)
    \setlength\itemsep{-3pt}
    \item (Beate Müller-Gemmeke [BÜNDNIS 90/DIE GRÜNEN]: Hat niemand gesagt! Das stimmt nicht!)
    \setlength\itemsep{-3pt}
    \item (Beifall bei der CDU/CSU – Zuruf des Abg. Matthias W. Birkwald [DIE LINKE])
    \setlength\itemsep{-3pt}
    \item (Friedrich Ostendorff [BÜNDNIS 90/DIE GRÜNEN]: Hast du schon jemals eine Unterkunft gesehen? Guck dir mal eine Unterkunft an! Vielleicht siehst du es dann anders!)
    \setlength\itemsep{-3pt}
    \item (Beate Müller-Gemmeke [BÜNDNIS 90/DIE GRÜNEN]: Vielleicht lag es an der Unterkunft!)
\end{itemize}
\subsection{Spiering}
\noindent\textbf{Texts:} Kolleginnen und Kollegen! Sehr geehrte Damen und Herren! Schwere Kost heute. – Ich fange mit einem Zitat aus meiner heimischen Zeitung an: Wenn die Bedingungen jetzt für schlimm erachtet werden, warum hat sich in der Vergangenheit kaum jemand dafür interessiert? Etwa weil es sich um Ausländer handelt, die still ihre Arbeit zum Wohle und Wohlstand – ich wiederhole: Wohlstand – der Mehrheit verrichten? Das hat mich schon sehr nachdenklich gestimmt – und es ist keine linksliberale Zeitung. Zum Kollegen Straubinger, der ja Karl Laumann zitiert hat. Von ihm kommt jüngst der Spruch: Die zu oft verletzte Würde des Menschen muss uns zu konsequentem Handeln anleiten.  Dann zu Frau Schimke, die ja die Frage des Zusammenhangs zwischen dem Coronavirus und Arbeitsplatzbedingungen gestellt hat. Ich will Ihnen das kurz aufzeigen – das ist gar nicht schwer zu verstehen –: Folgen von Niedriglohn sind Niedrigeinkommen, nicht in der Lage sein, Wohnungen zu bezahlen, Massenunterkünfte, Infizierung durch das Coronavirus. – Ganz einfache Rechnung.  Wir haben mehrere Möglichkeiten der Gesetzgebung angesprochen; ich glaube, das ist beim Arbeits- und Sozialminister auch in guten Händen. Ich will im Übrigen auch konzedieren, dass die Arbeits- und Sozialminister der Bundesrepublik Deutschland sich in diesem Bereich immer viel Mühe gegeben haben; aber sie treffen auf ein organisiertes System. Wenn wir über die Vorfälle in der deutschen Fleischwirtschaft sprechen, dann hat das einen Grund: weil die deutsche Fleischindustrie der Niedriglohnsektor schlechthin in Deutschland ist. Damit haben wir übrigens große Schlachtereien in benachbarten ausländischen Ländern plattgemacht. Wir werden uns diesem System mit aller Konsequenz widmen müssen, und zwar nicht nur dem. Wenn man sich dem, was dort im Lohnsektor und im Sozialsektor passiert, zuwendet, dann stellt man fest, dass so, wie dort gearbeitet wird, Sozialversicherung nicht bezahlt werden kann. Überlegen Sie, was passiert, wenn das Schule macht, wie die Damen und Herren agieren, und schauen Sie sich unser Krankenhaussystem an, das im Moment gut finanziert ist. Stellen Sie sich einmal vor, all unsere tollen Mittelständler, alle Handwerker würden in die Sozialversicherung einzahlen wie die deutsche Schlachtindustrie. Würden Sie dann in ein deutsches Krankenhaus gehen wollen? Nein. – Deswegen gehört denen ordentlich auf die Finger gekloppt.  Ich möchte es aber nicht bei der Schlachtwirtschaft belassen. Mit einem großen Teil des Niedriglohnsektors sind wir jetzt sehr stark konfrontiert: Das sind die Saisonarbeitskräfte – 300 000 Saisonarbeitskräfte pro Jahr, jenseits der deutschen Sozialversicherung.  – Staatlich legalisiert. Heute Mittag hat es eine ganz interessante Frage in der Fragestunde gegeben, nämlich nach der Organisation: Wie kommen die Kolleginnen und Kollegen in dieses Land? Wie wird aufgezeichnet, dass sie ins Land kommen? Wie stark sind unsere Bemühungen, das zu kontrollieren? – Jetzt passiert ein Novum: Eine hoheitliche Aufgabe wird an einen großen deutschen Arbeitgeberverband, nämlich an den Bauernverband, gegeben.  Ich habe die Bundeskanzlerin gefragt: Frau Bundeskanzlerin, wie sichern Sie den Datentransfer, dass wir wissen, wo sie oder er geblieben ist? – Wenn Sie heute Mittag aufmerksam zugehört haben, dann werden Sie feststellen: Das ist die einzige Frage, die sie nicht beantwortet hat.  Denn natürlich kann auch sie nicht beantworten, wie es gehen kann, dass man einer nichtstaatlichen Organisation hoheitliche Aufgaben übergibt. Ich finde es schlecht, dass das so gemacht wird.  Folgen Sie jetzt bitte einmal einem älteren Herrn gedanklich.  – Kann passieren. Ist Ihnen schon passiert, nicht?  – Alles gut. Folgendes Beispiel: Sie kommen hier an, sind zwischen 25 und 50 Jahre alt. Sie haben kein Geld in der Tasche. Sie werden irgendwo hingebracht, und Sie fangen an, dort zu arbeiten – 60 Stunden die Woche –, und Sie sind abends platt. Mit Essen und Trinken ist auch nicht viel, weil Sie ja ein geringes Einkommen haben. Stellen Sie sich jetzt bitte einmal die Situation vor: Sie kommen in dem Alter in so einen Schlafsaal, in so eine Kasematte. Das stelle sich bitte jede und jeder für sich vor. – Jeglicher Verlust einer Intimsphäre. Stellen Sie sich alleine – das sage ich Ihnen jetzt ganz deutlich – die Geräusch- und Geruchskulisse vor. Stellen Sie sich das einfach vor! Würden Sie das für sich wollen? Ich will das für mich, meine Angehörigen und die Menschen in diesem Land nicht.  Deswegen will ich den Kollegen Heil – da es ein anderes Ministerium offensichtlich nicht tut –  darum bitten, klare Standards für den Mindestwohnraum eines Arbeitnehmers in diesem Land – egal wo er herkommt – festzulegen, und darum bitten, dass dieses Haus sich darauf verständigt, den Forderungen der NGG zu folgen. Herzlichen Dank fürs Zuhören. 

\noindent\textbf{Comment:}
\begin{itemize}
    \setlength\itemsep{-3pt}
    \item (Michael Grosse-Brömer [CDU/CSU]: Ja, das ist das zuständige Ministerium!)
    \setlength\itemsep{-3pt}
    \item (Max Straubinger [CDU/CSU]: Richtig! – Zuruf von der SPD: Ah ja!)
    \setlength\itemsep{-3pt}
    \item (Beifall bei der SPD, der LINKEN und dem BÜNDNIS 90/DIE GRÜNEN sowie bei Abgeordneten der CDU/CSU)
    \setlength\itemsep{-3pt}
    \item (Beifall bei der SPD, der LINKEN und dem BÜNDNIS 90/DIE GRÜNEN sowie des Abg. Stephan Protschka [AfD])
    \setlength\itemsep{-3pt}
    \item (Heiterkeit beim BÜNDNIS 90/DIE GRÜNEN – Michael Grosse-Brömer [CDU/CSU]: Manchmal verläuft man sich dabei ja!)
    \setlength\itemsep{-3pt}
    \item (Lachen bei der LINKEN)
    \setlength\itemsep{-3pt}
    \item (Beifall bei der SPD und der LINKEN sowie bei Abgeordneten des BÜNDNISSES 90/DIE GRÜNEN – Kersten Steinke [DIE LINKE]: Versteht die aber nicht!)
    \setlength\itemsep{-3pt}
    \item (Beifall bei der SPD, der LINKEN und dem BÜNDNIS 90/DIE GRÜNEN)
    \setlength\itemsep{-3pt}
    \item (Stephan Protschka [AfD]: Richtig!)
    \setlength\itemsep{-3pt}
    \item (Friedrich Ostendorff [BÜNDNIS 90/DIE GRÜNEN]: Staatlich legalisiert!)
    \setlength\itemsep{-3pt}
    \item (Michael Grosse-Brömer [CDU/CSU]: Nein! Ich würde es ja machen!)
\end{itemize}
\section{Tagesordnungspunkt 6}
\subsection{Klein-Schmeink}
\noindent\textbf{Texts:} Danke schön. – Herr Präsident! Meine lieben Kolleginnen und Kollegen hier im Hause! Ich werde die Einbringung unseren Antrags „Professionelle Pflegekräfte wertschätzen und entlasten – Nicht nur in der Corona-Krise“ eröffnen mit einem Einstieg von meiner Kollegin Kordula Schulz-Asche, die erkrankt ist, aber diesen Antrag maßgeblich für uns erarbeitet hat. Sie sagt: Es reicht nicht, der Pflege für ihre großartige Arbeit zu danken – wie das gestern ja vielfach geschehen ist –, sondern Politik ist gefordert, zu handeln. – Wir wollen mit unserem Antrag erreichen, dass die professionelle Pflege in Deutschland endlich in ihrer Bedeutung für die Gesundheitsversorgung der Bevölkerung in Wert gesetzt wird.  Genau darum geht es. Wir wissen alle: Der Pflegebereich ist ein großer Bereich. 1,6 Millionen Menschen arbeiten im Bereich der Pflege: im Krankenhaus, in der Kranken- und Gesundheitspflege, in der Altenpflege. Wir wissen auch, dass 57 Prozent des gesamten Gesundheitspersonals in Europa in der Pflege tätig ist. Es gibt keine Krankenbehandlung, die ohne Pflege auskommt. Es gibt kein Leben im Alter – zumindest für die allermeisten –, das gänzlich ohne Pflege auskommt. Und es gibt sehr, sehr viele Menschen in Einrichtungen der Behindertenhilfe, die Pflege brauchen. Wenn wir uns diese Bedeutung vor Augen führen – oft wird ja davon gesprochen, dass es systemrelevant sei, in der Pflege zu arbeiten –, dann müssen wir leider feststellen, dass die Bedingungen für die Arbeit in der Pflege diesen Stellenwert in keinster Weise widerspiegeln. Genau daran müssen wir arbeiten, nicht zuletzt deshalb, weil wir es mit einem veritablen Pflegenotstand zu tun haben.  Wir sehen natürlich jetzt, in der Coronakrise, dass wir über solche Selbstverständlichkeiten reden müssen wie den Zugang zu Schutzmaterial und zu Testungen oder die Schaffung von Arbeitsbedingungen, die die Bewältigung der Coronakrise überhaupt erst möglich machen. Da müssen wir leider sagen: Auch in der Coronakrise kam die Pflege wieder als Letztes dran, und das muss sich ändern.  Deshalb haben wir in unseren Antrag einen ganzen Block an Forderungen aufgenommen, die sich genau darauf beziehen. Auch da muss man wieder sagen: Es ging nicht nur um den Zugang zu Schutzmaterialien, sondern es ging überhaupt darum, ob Pflege einbezogen worden ist in die Konzepte der Betreuung von zum Beispiel Pflegebedürftigen in Krankenhäusern oder Altenpflegeeinrichtungen, ob Pflege gefragt worden ist, welchen Beitrag sie an dieser Stelle leisten kann. Es ist wenig darüber geredet worden. Das muss sich ändern.  Dann reden wir seit Langem über überfällige Reformen, was die Arbeitsbedingungen angeht, was die Bezahlung angeht, was insgesamt die Vereinbarkeit von Familie und Beruf angeht. Es ist beschämend, dass wir das zu dieser Zeit immer wieder einfordern müssen und noch keine konkreten Pläne existieren, wie sich das ändern soll.  Als Letztes geht es um die Mitspracherechte der Pflege. Wo wird Pflege systematisch in die Entscheidungsstrukturen im Gesundheitswesen einbezogen? Da haben wir Reformbedarf ohne Ende. Die Einrichtung einer Pflegekammer kann nur ein großer und wichtiger Schritt sein. Wir müssen die Pflege vielmehr auf allen Ebenen der Gesundheitsversorgung, in allen relevanten Gremien systematisch einbeziehen. Das sind die Aufgaben, die sich stellen, und ich wünsche mir, dass Sie vonseiten der Koalition mit unserem Antrag konstruktiv umgehen und tatsächlich mit uns gemeinsam nach Wegen suchen, wie wir der Pflege zu wirklicher Wertschätzung verhelfen können. Danke schön.  

\noindent\textbf{Comment:}
\begin{itemize}
    \setlength\itemsep{-3pt}
    \item (Beifall beim BÜNDNIS 90/DIE GRÜNEN sowie der Abg. Nicole Westig [FDP])
    \setlength\itemsep{-3pt}
    \item (Beifall beim BÜNDNIS 90/DIE GRÜNEN)
    \setlength\itemsep{-3pt}
    \item (Beifall bei der CDU/CSU)
    \setlength\itemsep{-3pt}
    \item (Beifall beim BÜNDNIS 90/DIE GRÜNEN sowie des Abg. Dr. Andrew Ullmann [FDP])
\end{itemize}
\subsection{Riebsamen}
\noindent\textbf{Texts:} Sehr geehrter Herr Präsident! Verehrte Kolleginnen und Kollegen! Gute und verlässliche Pflege in der stationären und ambulanten Langzeitpflege und in den Krankenhäusern mit professionellen Kräften sicherzustellen – darauf zielt ja insbesondere auch Ihr Antrag ab –, ist eines der wichtigsten Anliegen dieser Bundesregierung in der Gesundheitspolitik in dieser Legislaturperiode und war es auch schon in der letzten Legislaturperiode. Ich erinnere an die Pflegestärkungsgesetze I, II und III, ich erinnere an die Neuaufstellung der Ausbildung in der Pflege in der letzten Legislaturperiode, und ich komme noch darauf zu sprechen, was wir in dieser Legislaturperiode bereits gemacht haben. Die gute Botschaft ist, dass wir aber bei all den Problemen, die wir nach wie vor haben und die sich jetzt, in der Coronakrise, auch wieder zeigen – das wird niemand bestreiten –, auch von diesen Gesetzgebungen ausgehende Wirkungen sehen. Auch darauf komme ich noch zu sprechen, liebe Kolleginnen und Kollegen. Jetzt hätte man ganz sicher nicht die Coronakrise gebraucht, um noch mal erneut ein Schlaglicht auf diese Problematik zu werfen. Darauf haben wir nicht gewartet, aber wir haben gesehen, dass neben der demografischen Entwicklung, die bekanntlich eine Herausforderung für die Pflege ist, auch noch ganz andere Dinge auf uns zukommen können – eben die Coronakrise. Sie haben recht, Frau Klein-Schmeink, wenn Sie ausführen, wie schwierig die Situation gerade in dieser Krise für die Pflege vor allem in den Alten- und Pflegeheimen ist. Ich erlebe das hautnah: Meine Schwägerin ist in einer solchen Einrichtung tätig; sie ruft mich oft an, und das ist auch richtig so. Sie ist im Heim mit einem Teil ihrer Kolleginnen und Kollegen und einem Teil der Bewohner selber an Corona erkrankt, hat die Infektion nach Hause getragen und meine Schwiegermutter angesteckt, die dann wiederum ins Krankenhaus musste. Sie hat sich Vorwürfe gemacht: Bin ich möglicherweise schuld, wenn das nicht gut ausgeht? – Es ist gut ausgegangen, Gott sei Dank. Und die ganze Familie war vier Wochen in Quarantäne. Daran wird klar, vor welchen Situationen die Familien stehen. Deswegen war und ist es richtig, dass wir – um mal diesen einen Punkt, weil er so aktuell ist, herauszugreifen – eine Prämie nicht nur an die Pflegerinnen und Pfleger in den Alten- und Pflegeheimen, sondern an alle Berufsgruppen auszahlen – das wird der Bund machen –, und zwar in einer Höhe von 1 000 Euro. Es steht den Ländern an – und ich hoffe, sie werden es tun; einige Länder haben signalisiert, dass sie es tun werden –, diesen Betrag um weitere 500 Euro zu ergänzen. Da kann man sagen: Das ist zu wenig. – Natürlich, es kann immer mehr sein; aber ich bin froh, dass wir diesen Beschluss haben und dass wir zumindest an dieser Stelle jetzt handeln.  Völlig unabhängig von Coronakrise und Koalitionsvertrag haben wir – ich habe es gesagt – langfristig Verbesserungen in der Pflege eingeleitet, zum Teil sogar in Form eines Paradigmenwechsels. Sie wissen: Wir haben im Krankenhausbereich nicht mehr die Fallpauschalen in der Pflege, wir haben dort jetzt Pflegebudgets, die für jedes Krankenhaus an den dort vorhandenen Selbstkosten individuell ausgerichtet sind. Jede zusätzliche Kraft wird bezahlt, jede Tariferhöhung wird bezahlt.  Wenn es überhaupt noch Einschränkungen gibt – ja, auch das müssen wir einräumen –, dann die, dass es schwer ist, Pflegekräfte zu bekommen.  Das gilt auch für die 13 000 Stellen, die wir für die Altenpflege geschaffen haben. Deswegen haben wir mit der Konzertierten Aktion Pflege genau diesen Punkt, diese Herausforderung in den Mittelpunkt gestellt: Mittel und Wege zu suchen, um Pflegekräfte zu finden. Natürlich spielt an der Stelle auch das Geld eine große Rolle, um den Beruf attraktiv zu machen. Deswegen haben wir bereits am 24. Oktober – das sind jetzt die Dinge, von denen ich sage, dass sie wirken, Dinge, die beschlossen sind und die wir umsetzen – ein Gesetz zur besseren Bezahlung, das Gesetz für bessere Löhne in der Pflege, beschlossen. Tagesaktueller könnte es kaum sein: Diese besseren Löhne sind zum 1. Mai 2020 in Kraft getreten. Es geht an der Stelle zunächst einmal um Mindestlöhne. Bitte schön. Sie heben ja jetzt auf die Mindestlöhne ab. Da reden wir ja perspektivisch über 15,40 Euro ab übernächstem Jahr, also sehr, sehr wenig. Glauben Sie, dass das das entsprechende Signal an junge Menschen gibt, den Pflegeberuf zu lernen und ihn auszuüben, und dass die besondere Kompetenz, die gerade Pflege in die Versorgung einbringen kann, damit auch nur annähernd abgegolten werden kann? Braucht es nicht ein ganz anderes Signal, wenn wir es gemeinschaftlich hinbekommen wollen, dass der Stellenwert von Pflege in dieser Gesellschaft auch wirklich gespiegelt wird? Können wir es uns da leisten, mit einem Mindestlohn von 15 Euro für eine ausgebildete Fachkraft um die Ecke zu kommen?  In der Tat, Frau Klein-Schmeink, das ist zu wenig. Aber auch Sie haben in Ihrem Antrag ausgeführt, dass das, was wir an der Stelle jetzt erreichen, ein wichtiger Punkt ist. Es sind 2 678 Euro brutto, bei einer dreijährigen Ausbildung. Das könnte mehr sein. Aber, wie gesagt, auch Sie haben in Ihrem Antrag ausgeführt, dass dies ein wichtiger Anfang ist und es wünschenswert wäre – dem schließen wir uns in vollem Umfang an –, dass wir nicht mit Mindestlöhnen arbeiten müssen,  sondern dass wir zu Tarifverträgen für die gesamte Pflegebranche kommen. An der Stelle schließe ich mich vollkommen dem Appell, den Sie in Ihrem Antrag an die Tarifpartner formulieren, an, zu Tarifverträgen zu kommen, die wir dann allgemeinverbindlich machen können. Das ist das Ziel.  Wir müssen aber unterscheiden zwischen dem, was der Gesetzgeber machen kann – das haben wir getan; deswegen ist es ein gutes Signal –, und dem, was wir den Tarifpartnern überlassen müssen.  Da müssen wir zukünftig – das fordern wir auch – den entsprechenden Druck ausüben. Der zweite Punkt ist die Personalbemessung. Auch das geht bereits auf die letzte Legislaturperiode zurück und wird demnächst wirken. Wir haben das Gutachten längst vorliegen – Stichwort: Professor Rothgang –, jetzt geht es um die Umsetzung, also die Personalbemessung in den Alten- und Pflegeheimen neu zu regeln, indem wir sie auf ein Fundament stellen, das wissenschaftlich fundiert ist. Es liegt jetzt auch an den Ländern, genau dies bei sich umzusetzen. Letzter Punkt – – ein Punkt noch –: Digitalisierung haben Sie in Ihrem Antrag gar nicht ausgeführt. Ich bin der Auffassung, dass zur Entlastung der Pflege in den Alten- und Pflegeheimen sowie den Krankenhäusern durch Digitalisierung noch deutlich mehr gemacht werden kann, als es bisher der Fall ist. Herzlichen Dank.  

\noindent\textbf{Comment:}
\begin{itemize}
    \setlength\itemsep{-3pt}
    \item (Beifall bei der AfD)
    \setlength\itemsep{-3pt}
    \item (Beifall bei Abgeordneten der CDU/CSU und der SPD)
    \setlength\itemsep{-3pt}
    \item (Beifall beim BÜNDNIS 90/DIE GRÜNEN)
    \setlength\itemsep{-3pt}
    \item (Maria Klein-Schmeink [BÜNDNIS 90/DIE GRÜNEN]: Ja, das hat einen Grund!)
    \setlength\itemsep{-3pt}
    \item (Beifall bei Abgeordneten der CDU/CSU)
    \setlength\itemsep{-3pt}
    \item (Beifall bei der CDU/CSU)
    \setlength\itemsep{-3pt}
    \item (Beifall bei der CDU/CSU sowie bei Abgeordneten der SPD)
    \setlength\itemsep{-3pt}
    \item (Beifall der Abg. Karin Maag [CDU/CSU])
\end{itemize}
\subsection{Sichert}
\noindent\textbf{Texts:} Herr Präsident! Meine Damen und Herren! Die Grünen fordern eine Coronaprämie aus Steuermitteln für Beschäftigte im Gesundheitswesen, und sie begründen das mit geringen Einkommen und den besonderen Risiken einer Infektion. Als gerechtigkeitsliebendem Menschen erschließt sich mir das nicht. Auch Verkäufer im Lebensmitteleinzelhandel oder Mitarbeiter von Sicherheitsdiensten waren wochenlang ohne ausreichende Schutzmaßnahmen einem besonderen Risiko ausgesetzt und haben sehr geringe Einkommen. Warum soll nun aus deren Steuern eine Sonderprämie im Gesundheitswesen bezahlt werden? Natürlich kann sich jetzt jede Partei gegenseitig mit Sonderzahlungen für bestimmte Gruppen überbieten und versuchen, sich in dieser Coronakrise zu profilieren und neue Wähler zu sichern. Das aber ist, wenn man das weiterdenkt, eine Form der Scheckbuchdemokratie, die höchst unseriös ist und die in eklatantem Widerspruch zum Auftrag der Abgeordneten als Vertreter aller Bürger steht.  Reden wir doch mal über die Erkenntnisse, die wir in den letzten Wochen gewonnen haben.  Wir alle haben gelernt, dass das Outsourcen betreuungsbedürftiger Familienangehöriger von den Familien an den Staat höchst problematisch ist. Die Eltern jammern, weil Kitas geschlossen sind und sich die Eltern nun, wie es früher selbstverständlich war, ganztägig um die Kinder kümmern müssen. In der Betreuung der Alten und Kranken sehen wir, dass der Trend, pflegebedürftige Familienmitglieder ins Heim abzugeben, in Krisenzeiten zur Katastrophe werden kann; denn durch Besuchsverbote vereinsamen die Pflegebedürftigen, und das Ansteckungsrisiko in den Heimen ist hoch und führt zu vielen Toten. Wenn uns die letzten Wochen eines gezeigt haben, dann, dass Deutschland eine Politik der Stärkung der Familie als Keimzelle der Gesellschaft braucht. Je mehr Kinder von Familienangehörigen betreut werden und je mehr Pflegebedürftige im Kreis ihrer Lieben umsorgt werden, umso besser kommt eine Gesellschaft durch eine Pandemie.  Deswegen verfolgen wir als AfD einen anderen Ansatz. Wir wollen keine einmaligen Sonderprämien auf Kosten der Steuerzahler, deren Wirkung schnell verpufft; denn wegen einer einmaligen Zahlung wird der Beruf nicht attraktiver. Wir wollen Gelder lieber nachhaltig einsetzen, um häusliche Pflege stärker zu honorieren. Es ist niemandem erklärbar, warum die Pflege durch Familienangehörige weit geringer honoriert wird als die Pflege durch Fremde.  Aktuell wird häusliche Pflege mit weniger als der Hälfte des Betrags der ambulanten Pflege vergütet, und das, obwohl pflegende Angehörige oft zusätzlich die extrem wichtige liebende Fürsorge leisten, die vielen professionellen Pflegekräften wegen Zeitmangels  und oft auch fehlender Deutschkenntnisse gar nicht möglich ist.  Wenn ein pflegender Angehöriger mal krankheitsbedingt für einen Monat ausfällt, wird in diesem Monat für ambulante Pflege mehr als das Doppelte gezahlt. Jeder Angehörige, der diese Geringschätzung erlebt, fühlt sich zu Recht ungerecht behandelt. Die Lenkungswirkung der bisherigen Politik hat weg von häuslicher, hin zu ambulanter und stationärer Pflege geführt und damit den aktuellen Pflegenotstand verursacht. Lassen Sie uns also die Ursachen bekämpfen und die häusliche Pflege stärken! So können wir professionelle Pflegekräfte am besten entlasten. Denn je mehr Menschen zu Hause von Angehörigen oder Bekannten gepflegt werden, umso weniger müssen ambulant oder stationär von professionellen Kräften gepflegt werden und umso weniger Arbeit fällt für diese professionellen Kräfte an. Auch für kommende Generationen ist es ein wichtiger Aspekt für die Ausbildung emotionaler Intelligenz, wenn wir zeigen, dass wir alles in unserer Macht Stehende tun, damit Pflegebedürftige so lange wie irgend möglich im Kreis ihrer Angehörigen umsorgt werden. Daher haben wir als AfD beantragt, häusliche Pflege künftig genauso hoch zu vergüten wie ambulante Pflege. Investieren wir das Geld lieber in die Honorierung der außergewöhnlichen Leistung all der pflegenden Angehörigen und in die Stärkung der Familie als Keimzelle der Gesellschaft,  anstatt – wie in diesem Antrag der Grünen – mit Einmalzahlungen zu versuchen, irgendwelche Berufsgruppen als Wähler zu kaufen.  Vielen Dank.  

\noindent\textbf{Comment:}
\begin{itemize}
    \setlength\itemsep{-3pt}
    \item (Beifall bei der SPD)
    \setlength\itemsep{-3pt}
    \item (Beifall bei der AfD)
    \setlength\itemsep{-3pt}
    \item (Claudia Moll [SPD]: Ach! Das auch noch!)
    \setlength\itemsep{-3pt}
    \item (Ulli Nissen [SPD]: Sie? Erkenntnisse? Das ist ein Widerspruch!)
    \setlength\itemsep{-3pt}
    \item (Beifall bei der AfD – Zurufe von der SPD und der LINKEN)
    \setlength\itemsep{-3pt}
    \item (Claudia Moll [SPD]: Das stimmt nicht!)
    \setlength\itemsep{-3pt}
    \item (Maria Klein-Schmeink [BÜNDNIS 90/DIE GRÜNEN]: Das war einer von zwölf Punkten! Lesen würde helfen! – Zuruf der Abg. Ulli Nissen [SPD])
    \setlength\itemsep{-3pt}
    \item (Matthias W. Birkwald [DIE LINKE]: Veraltetes Gesellschaftsbild!)
    \setlength\itemsep{-3pt}
    \item (Maria Klein-Schmeink [BÜNDNIS 90/DIE GRÜNEN]: Sie sollten den ganzen Antrag mal lesen!)
\end{itemize}
\subsection{Moll}
\noindent\textbf{Texts:} Ach ja!  Sehr geehrter Herr Präsident! Liebe Kolleginnen und Kollegen! Die Coronakrise zeigt eindrücklich, wie groß die Gefahr ist, dass ein Gesundheitssystem an seine Grenzen stoßen kann. Es war richtig, dass wir zügig eine ganze Reihe von Maßnahmen ergriffen haben, die einen Kollaps verhindern. Das hat viel Kraft und auch Geld gekostet. Diese Kraft haben ganz besonders die Menschen in unserem Gesundheitswesen aufgebracht. Mit Aufopferung und Hingabe bis an die Belastungsgrenze haben Pflegerinnen und Pfleger Außergewöhnliches erreicht.  Ich glaube, viele wissen gar nicht, wie komplex dieser Beruf eigentlich ist. Manche glauben, das hat nur etwas mit Körperflüssigkeiten zu tun. Aber es wird gepflegt, betreut, aktiviert, validiert, gefördert, medizinisch versorgt, geplant, beobachtet, dokumentiert, und es werden Menschen auf dem letzten Weg begleitet. Es ist ein unfassbar vielfältiger Job, den die Pflegekräfte leisten. Das bedarf einer großen Anerkennung.  Aber vergessen möchte ich nicht die vielen anderen Menschen, die einen unschätzbaren Beitrag leisten. An diese muss auch über die Coronakrise hinaus immer gedacht werden; denn auch dieses Personal gibt abseits von jeder Krise immer 100 Prozent. Denken Sie an die Arbeitnehmerinnen und Arbeitnehmer in der Haustechnik, in der Küche, in der Gebäudereinigung, die Servicekräfte im Empfangsbereich, in der Wäscherei, aber auch in der Verwaltung. Jeder in der Kette ist wichtig und darf nicht vergessen werden. Ein Beispiel aus dem Krankenhaus: Stellen Sie sich einfach nur mal vor, ein OP-Saal wird nicht richtig gereinigt. Dann kann der Chefarzt nicht operieren. – Und das gilt in der Altenpflege, in Krankenhäusern und in Einrichtungen für Menschen mit Behinderungen. Tagtäglich leistet dieses Personal in Deutschland Außergewöhnliches: Schichtdienst, Wochenenddienst sind keine Seltenheit, ständig für Kollegen einspringen müssen, weil es gar nicht anders geht. Hier können wir gar nicht genug Wertschätzung aufbringen. Trotz all dieser Umstände ist ein Beruf in der Pflege ein ganz wunderbarer Beruf. Wir alle können dazu beitragen, dass das auch in der Außendarstellung so wahrgenommen wird; denn es geht eben nicht nur um Toilettengänge und hat nicht nur mit Körperflüssigkeiten zu tun.  Wertschätzung ist sicher das eine. Und jede Altenpflegerin und jeder Altenpfleger freut sich über den Bonus, den sie oder er ausgezahlt bekommen soll. Was wir aber brauchen, sind dauerhaft gerechte Löhne und gute Arbeitsbedingungen in der Pflege.  Bonuszahlungen sind da nur ein kleines Geschenk. Diesen Bonus, liebe Antragsteller von den Grünen, bekommen die Arbeitnehmerinnen und Arbeitnehmer  aber nicht für die Risiken, denen sie im Moment ausgesetzt sind, sondern für die aufrichtige und überaus wichtige Arbeit, die ständig geleistet wird.  – Nein, das machen wir gleich unter vier Augen. Die Pflegekräfte wissen, wie man mit Risiken durch Krankheiten umzugehen hat. Ständig ist man verschiedenen Gefahren ausgesetzt. Deshalb ist es zwar richtig, dass Sie hier mehr Schulungen fordern; Sie fordern aber kontinuierliche Schulungen speziell für Covid‑19.  – Lesen Sie Ihren Antrag.  Damit entzieht man doch erst mal nur Personal von Stationen und sorgt für eine noch höhere Belastung, weil die Arbeitskraft fehlt. Es stimmt aber auch, dass die Situation in der Intensivpflege normalerweise anders aussieht; aber auch dort kennt sich das Personal bestens mit Risiken aus. Dann fordern Sie Tarifverträge. Gesetzgeberisch haben wir da doch längst gehandelt. Jetzt sind die Tarifpartner gefragt, gute Tarifverträge zu vereinbaren. Glauben Sie mir, dass unser Arbeitsminister Hubertus Heil den Dialog zwischen Arbeitnehmern und Arbeitgebern selbstverständlich aktiv fördert, damit uns hier der Durchbruch gelingen kann.  Als Sozialdemokraten sagen wir ganz klar: Diese Tarifverträge müssen allgemeinverbindlich werden.  Wir zeigen hier, wie gute Regierungsarbeit geht; davon können sich Ihre Kolleginnen und Kollegen von den Grünen in Baden-Württemberg oder in Hessen ja mal etwas abschauen.  Liebe Kolleginnen und Kollegen, den Pflegenotstand gibt es schon länger; wir reden ständig darüber, und es wird ihn auch über die Krise hinaus geben.  Hier wünsche ich mir einfach, dass wir viel pragmatischer handeln – was hat er gesagt? –, um den Entwicklungen in der Pflege entgegenzuwirken. Tarifverträge und gute Bezahlung sind das eine, gute Arbeitsbedingungen, flexible Arbeitszeiten das andere. Nur so können wir erreichen, dass ausgebildete Pflegekräfte in ihrem Beruf bleiben und neue Kräfte gewonnen werden können;  denn wir brauchen Entlastungen, die mehr bringen als Geld. Der Arbeitsalltag in der professionellen Pflege muss vereinfacht werden. Genau da müssen wir hin.  Gute Bezahlung und vor allem dauerhaft gute Arbeitsbedingungen: Das ist die Wertschätzung für die Pflege, die wir langfristig brauchen. Vielen Dank.   

\noindent\textbf{Comment:}
\begin{itemize}
    \setlength\itemsep{-3pt}
    \item (Beifall bei der SPD sowie bei Abgeordneten der CDU/CSU)
    \setlength\itemsep{-3pt}
    \item (Beifall bei der SPD)
    \setlength\itemsep{-3pt}
    \item (Beifall bei der FDP)
    \setlength\itemsep{-3pt}
    \item (Beifall bei der SPD sowie bei Abgeordneten der CDU/CSU und der FDP)
    \setlength\itemsep{-3pt}
    \item (Beifall bei der SPD sowie bei Abgeordneten der CDU/CSU – Maria Klein-Schmeink [BÜNDNIS 90/DIE GRÜNEN]: Wir reden doch vom Allgemeinverbindlichen!)
    \setlength\itemsep{-3pt}
    \item (Tino Chrupalla [AfD]: Seid bereit!)
    \setlength\itemsep{-3pt}
    \item (Maria Klein-Schmeink [BÜNDNIS 90/DIE GRÜNEN]: Lesen Sie doch mal weiter!)
    \setlength\itemsep{-3pt}
    \item (Harald Ebner [BÜNDNIS 90/DIE GRÜNEN]: Wie lange regiert ihr jetzt schon?)
    \setlength\itemsep{-3pt}
    \item (Maria Klein-Schmeink [BÜNDNIS 90/DIE GRÜNEN]: Das ist einer von zwölf Punkten!)
    \setlength\itemsep{-3pt}
    \item (Maria Klein-Schmeink [BÜNDNIS 90/DIE GRÜNEN]: Nicht bei der Hälfte stehen bleiben!)
    \setlength\itemsep{-3pt}
    \item (Beifall bei der SPD sowie des Abg. Matthias W. Birkwald [DIE LINKE])
    \setlength\itemsep{-3pt}
    \item (Beifall bei der SPD sowie bei Abgeordneten der CDU/CSU, der FDP und des BÜNDNISSES 90/DIE GRÜNEN und des Abg. Matthias W. Birkwald [DIE LINKE])
    \setlength\itemsep{-3pt}
    \item (Heiterkeit bei Abgeordneten der FDP)
    \setlength\itemsep{-3pt}
    \item (Heiterkeit bei der SPD und der CDU/CSU)
    \setlength\itemsep{-3pt}
    \item (Beifall bei der SPD – Maria Klein-Schmeink [BÜNDNIS 90/DIE GRÜNEN]: Schließt sich aber nicht aus!)
\end{itemize}
\subsection{Westig}
\noindent\textbf{Texts:} Herr Präsident! Meine Damen und Herren! Zunächst einmal von hier aus ganz herzliche Genesungswünsche an die geschätzte Kollegin Kordula Schulz-Asche!  Liebe Kolleginnen und Kollegen von den Grünen, Sie legen hier einen guten Antrag zu einem drängenden Thema vor. Viele Punkte sind aus Sicht der Freien Demokraten zustimmungswürdig. Auch wir fordern ausreichend Schutzkleidung und Testkapazitäten für die Beschäftigten im Gesundheitswesen. Auch wir halten es nicht für gerecht, mit der Coronaprämie nur die Beschäftigten in der Altenpflege zu bedenken, während die Pflegenden in den Kliniken mit demselben Einsatz und großem Infektionsrisiko leer ausgehen sollen. Aber die Sonderprämie entbindet die Politik nicht von ihrer Pflicht, grundsätzlich für angemessene Vergütung und bessere Arbeitsbedingungen in der Pflege zu sorgen.  Deswegen unterstützen wir auch Ihre Forderung nach höheren Ausbildungszahlen. Doch woher sollen die Auszubildenden kommen? Wir müssen die Ausbildung insgesamt verbessern. Praxisanleitung muss zuverlässig stattfinden, Auszubildende dürfen nicht länger auf den Personalschlüssel angerechnet werden. Für evidenzbasierte Personalbemessung zu sein, heißt, anzuerkennen, dass wir mehr Pflegeassistenzkräfte benötigen. Hierfür brauchen wir endlich bundeseinheitliche Ausbildungsstandards.  Wir Freie Demokraten wollen Aufstiegsmöglichkeiten für jeden, auch in der Pflege. Dafür muss es aber ein durchlässiges System und die Chance auf lebenslanges Lernen geben. Allerdings halten wir die von Ihnen geforderte Übertragung von heilkundlichen Tätigkeiten so pauschal für problematisch. Die Übertragung von Kompetenzen muss klar und präzise definiert werden. Dies wäre dann ein mögliches Feld für die akademisch ausgebildeten Pflegekräfte. Als Freie Demokraten fordern wir zudem, der Digitalisierung mehr Raum zu geben. Gerade in den Bereichen der Dokumentation und der Logistik liegt ein enormes Potenzial, um Pflegekräfte zu entlasten und mehr Zeit für Zuwendung zu gewinnen.  Ja, auch ich bin persönlich der Meinung, dass es eine starke Selbstverwaltung in der Pflege braucht. Pflegende sind sehr wohl in der Lage, das Heft selbst in die Hand zu nehmen. Sie gehören selbst in die Gremien, in denen über ihren Beruf entschieden wird. Das ist auch eine Frage des Respekts. Dabei sollten sie auch selbst entscheiden, ob sie eine Kammer wollen und entsprechend befragt werden. Ich freue mich jedenfalls auf die weitere Diskussion dieses Antrags. Vielen Dank.  

\noindent\textbf{Comment:}
\begin{itemize}
    \setlength\itemsep{-3pt}
    \item (Beifall bei der FDP)
    \setlength\itemsep{-3pt}
    \item (Beifall bei der FDP, der CDU/CSU, der SPD, der LINKEN und dem BÜNDNIS 90/DIE GRÜNEN)
    \setlength\itemsep{-3pt}
    \item (Beifall bei der LINKEN)
\end{itemize}
\subsection{Zimmermann}
\noindent\textbf{Texts:} Herr Präsident! Liebe Kolleginnen und Kollegen! Pflege ist mehr wert, und deshalb begrüße ich ausdrücklich den Antrag von Bündnis 90\/Die Grünen,  auch wenn ich anmerken möchte, dass diese Misere in der Pflege ja nicht erst seit Corona und nicht erst seit letztem Jahr existiert, sondern schon ungefähr seit 30 Jahren in unterschiedlichen Regierungsbeteiligungen. Ich möchte auch anmerken, dass ich für die Ausgestaltung noch ein paar Ideen haben, wie wir es wirklich solide, solidarisch und allgemeinverbindlich hinbekommen können. Wir wollen, dass Menschen mit Pflegebedarf nicht erneut zur Kasse gebeten werden. Die Bundesregierung steht für die Coronaprämie in der Pflicht und muss die vollständige Refinanzierung über Steuermittel zweckgebunden zusichern; denn das Gezerre um die Finanzierung dieser Prämie ist vor allem eins: peinlich und unwürdig.  Es geht um die Menschen, die dafür sorgen, dass das System in Gesundheit und Pflege nicht zusammenbricht, und zwar mit aller Kraft und schon lange und auch über ihre Kräfte hinaus. Und die Bundesregierung bekommt es nicht hin, ihnen kurzfristig zumindest eine symbolische Anerkennung zukommen zu lassen. Das Geld ist doch da! Es wird nur nicht dafür ausgegeben. Schauen Sie doch in den Rüstungshaushalt; schauen Sie doch auf die Finanzspritzen für Großkonzerne und Banken. Wenn man möchte, kann man das Geld genau dahin bringen, wo es hingehört.  Wir müssen endlich dafür sorgen, dass sich die Situation der Pflegekräfte sofort spürbar, aber auch grundlegend und nachhaltig verbessert. Das sind wir ihnen längst schuldig. Und wenn wir es nicht tun, machen wir uns als Parlament unglaubwürdig;  denn das Wissen, als Pflegekraft systemrelevant zu sein, zahlt keine Miete, und abendlicher Applaus ist zwar nett gemeint, zahlt aber auch keine Miete. Und seien wir doch mal ehrlich: Eine einmalige Prämie für Beschäftigte in der Altenpflege zahlt auch keine Miete.  Wir brauchen in Gesundheit und Pflege endlich attraktive Löhne und vernünftige Arbeitszeiten. Und dafür brauchen wir einen Paradigmenwechsel in der Finanzierung von Gesundheit und Pflege; das verbessert die Situation von Pflegekräften spürbar, sofort und dauerhaft.  Das Geld für diese Veränderung ist ja da. Es ist eine politische Entscheidung, wofür wir das Geld ausgeben. Und es kommt ja auch noch mehr Geld rein, wenn endlich alle, also auch Beamte, Selbstständige und Bundestagsabgeordnete, gemäß ihren finanziellen Möglichkeiten in eine solidarische Gesundheits- und Pflegeversicherung einzahlen,  wenn die Beitragsbemessungsgrenze nicht mehr die Spitzenverdiener schont, wenn endlich auch Einkünfte aus Kapitalerträgen herangezogen werden. Diese Krise darf nicht mehr auf den Schultern der Pflegekräfte ausgetragen werden, nicht auf den Schultern der pflegenden Angehörigen. Die Menschen mit Pflegebedarf und die Patientinnen und Patienten sollen das nicht ausbaden. Das ist kein soziales System. Wir brauchen eine Finanzierung, die solidarisch und solide zugleich ist. Deshalb brauchen wir einen Systemwechsel in der Finanzierung, damit „systemrelevant“ kein leeres Wort bleibt. Wir müssen das System verändern. Ich frage Sie: Wann, wenn nicht jetzt? Herzlichen Dank.  

\noindent\textbf{Comment:}
\begin{itemize}
    \setlength\itemsep{-3pt}
    \item (Beifall bei der LINKEN sowie des Abg. Dr. Wolfgang Strengmann-Kuhn [BÜNDNIS 90/DIE GRÜNEN])
    \setlength\itemsep{-3pt}
    \item (Beifall bei der LINKEN sowie bei Abgeordneten des BÜNDNISSES 90/DIE GRÜNEN)
    \setlength\itemsep{-3pt}
    \item (Beifall bei der CDU/CSU)
    \setlength\itemsep{-3pt}
    \item (Beifall bei der LINKEN – Zuruf des Abg. Erich Irlstorfer [CDU/CSU])
    \setlength\itemsep{-3pt}
    \item (Beifall bei der LINKEN)
\end{itemize}
\subsection{Irlstorfer}
\noindent\textbf{Texts:} Herr Präsident! Verehrte Kolleginnen und Kollegen! Frau Zimmermann, ich wollte meine Rede eigentlich ganz anders anfangen, aber wenn ich das, was Sie hier von sich geben, höre, muss ich sagen: Das ist wieder der Beweis, dass man Gesundheitspolitik wie Politik generell mit dem Kopf und nicht mit dem Kehlkopf macht.  Ich möchte Ihnen in aller Deutlichkeit sagen: Die Ansätze, die Sie hier vorbringen, sind kontraproduktiv. Der Antrag der Grünen, der hier beraten wird, hat eine Basis. Er hat ein Fundament, von dem ich sage: Ja, da sind wir uns in diesem Haus – so vermute ich zumindest – ziemlich einig. In diesem Antrag sind Punkte aufgeführt, die in meinen Augen Selbstverständlichkeiten sein müssten. „Versorgung mit Schutzausrüstung … verbessern“: Ja, selbstverständlich; das ist doch völlig klar. Für Pflegekräfte, die direkt am Patienten arbeiten, ist das die Basis. Auch das ist eine Lehre dieser Pandemie. Ich glaube, dafür zu sorgen, dass regelmäßige Testungen durchgeführt werden müssen und das Personal geschützt wird, ist eine Kern- und Grundaufgabe von Politik; das müssen wir sicherstellen. „Bundeseinheitliche Standards zur Krisenintervention in gesundheitlichen und pflegerischen Einrichtungen … zu entwickeln“, gehört selbstverständlich zu den Aufgaben von Politik. Ich glaube, dass es hierbei diesen Austausch von Bund und Ländern gibt und dass das alles auch machbar ist. Wir müssen auch dafür sorgen, dass die Beschäftigten in der Pflege „Zugang zu kontinuierlichen Schulungen“ und zu Weiterbildung haben. Wir sollten den Umgang mit Pandemien – nicht nur mit Corona – vielleicht auch in die Ausbildung aufnehmen. All diese Dinge bilden ein Fundament, für das wir hier eine breite Zustimmung haben. Meine sehr geehrten Damen und Herren, es ist aber, glaube ich, auch wichtig, dass wir hier im Hause, wenn es um die Pflegeszene geht, von diesen Worthülsen wegkommen. Es nervt mich ein bisschen, wenn ich hier immer wieder die Schlagworte „Wertschätzung“, „Respekt“, „Anerkennung“ höre. Ich glaube, wir müssen das nicht ständig wiederholen; denn das ist doch die Basis für den Umgang mit unseren Pflegekräften. Das ist eine Selbstverständlichkeit; das setze ich voraus. Deshalb möchte ich auch klarstellen: Auf Veranstaltungen, aber auch hier in der Diskussion wird das Ganze oft weinerlich oder verniedlichend dargestellt. Da wird von den „Engelchen am Bett“ oder den „kleinen Heldinnen und Helden des Alltags“ gesprochen. Nein, meine sehr geehrten Damen und Herren, die Pflegeszene ist eine hochprofessionalisierte Szene, in der gut ausgebildete Menschen tätig sind. Menschen, die in diesem Bereich arbeiten, sind hochqualifizierte Profis, die Menschenleben retten,  die Teilhabe organisieren und die natürlich auch Leiden lindern. Deshalb brauchen sie ordentliche Bedingungen und nicht Mitleid, meine sehr geehrten Damen und Herren.  – Hören Sie lieber zu.  Meine sehr geehrten Damen und Herren, ich möchte hier auch klarstellen, dass für uns als Regierungsfraktionen klar ist, dass die tarifliche Bindung notwendig ist. Da können wir noch so lange diskutieren; das ist die Basis. Wir müssen die einzelnen Anbieter natürlich mit sanftem Druck darauf hinweisen – das tun wir auch –, aber wir müssen auch den Einheiten, die Pflege anbieten, Hilfen an die Hand geben, um die Bedingungen zu verbessern. Ich denke hier an das Thema Leiharbeit und dergleichen. Diese Dinge müssen wir klar ansprechen und dann natürlich auch verbessern. Stichwort „Aus- und Weiterbildung“: Wir investieren nicht nur in Straßen, in Gebäude, in Bauwerke, in Technik; wir investieren in Menschen, weil Menschen in der Pflege als Pflegekräfte, als Arbeitnehmerinnen und Arbeitnehmer wichtig sind, weil sie diejenigen sind, die wirkliche Verbesserungen schaffen – am Bett. Deshalb nehmen wir hier auch viel Geld in die Hand. Ich finde es nicht angemessen, hier immer so zu tun, als würde das Ganze nur ein bisschen dahinplätschern. Nein, seit Jahren schaffen wir Verbesserungen. Wir haben mit den Pflegestärkungsgesetzen angefangen, mit denen wir für deutliche Verbesserungen für die zu pflegenden Personen gesorgt haben. Wir haben natürlich die Einrichtungen und auch die Angehörigen gestärkt, denen ich hier auch danken möchte. Jetzt haben wir den großen Block der Beschäftigten vor uns, für die wir die Rahmenbedingungen und Schritt für Schritt auch das Arbeitsumfeld verbessern werden. Das ist unsere Aufgabe. Dazu sind wir bereit. Ich sehe, auch die Grünen sind so wie die Regierungsfraktionen dazu bereit. Das, was ich hier von der AfD und auch von der Linken gehört habe, ist in meinen Augen doch relativ dünn gewesen. Herzlichen Dank. 

\noindent\textbf{Comment:}
\begin{itemize}
    \setlength\itemsep{-3pt}
    \item (Zuruf des Abg. Matthias W. Birkwald [DIE LINKE])
    \setlength\itemsep{-3pt}
    \item (Beifall bei der CDU/CSU – Matthias W. Birkwald [DIE LINKE]: Und mindestens 3 000 Euro brutto!)
    \setlength\itemsep{-3pt}
    \item (Beifall bei der CDU/CSU sowie bei Abgeordneten der SPD)
    \setlength\itemsep{-3pt}
    \item (Matthias W. Birkwald [DIE LINKE]: Das tue ich!)
    \setlength\itemsep{-3pt}
    \item (Pia Zimmermann [DIE LINKE]: Das sagen Sie immer!)
\end{itemize}
\section{Tagesordnungspunkt 7}
\subsection{Maas}
\noindent\textbf{Texts:} Herr Präsident! Liebe Kolleginnen und Kollegen! Wenn die letzten Wochen uns eines deutlich gemacht haben, dann das: Es gibt einen Zusammenhang zwischen langem Atem und Erfolg. Das gilt für die Eindämmung einer Pandemie; das gilt aber genauso für nachhaltigen Frieden. Nach inzwischen gut 21 Jahren Bundeswehreinsatz auf dem Balkan kann man durchaus von einem sehr langen Atem sprechen, einem langen Atem, der sich, wie wir finden, aber auch ausgezahlt hat. Die Sicherheitslage auf dem Balkan ist weitestgehend stabil. Das hat uns erlaubt, das Engagement der Bundeswehr in den letzten Jahren Schritt für Schritt zurückzufahren. Während zu Beginn noch über 6 000 deutsche Soldatinnen und Soldaten vor Ort waren und auch vor vier Jahren noch über 1 300 Soldatinnen und Soldaten dafür vorgesehen waren, sind es heute weniger als 70, die vor Ort sind. Es spricht auch für sich, dass KFOR in den letzten Jahren nicht mehr, und zwar kein einziges Mal, in sicherheitsgefährdende Situationen eingreifen musste. Auch der Charakter dieses Einsatzes hat sich in den letzten zwei Jahrzehnten stark verändert. Die Bundeswehr ist heute nur noch in Pristina präsent. Der Schwerpunkt unseres Engagements liegt inzwischen auf der Beratung der kosovarischen Sicherheitskräfte. Um im Jargon unserer Zeit zu sprechen: Das alles sind Lockerungen, aber solche mit Augenmaß. Wir reduzieren und verändern unsere Präsenz, bewahren KFOR aber gleichzeitig die nötige Flexibilität, um auf unerwartete, aber eben auch nicht ganz auszuschließende Verschlechterungen der Sicherheitslage nach wie vor schnell reagieren zu können. Dass die kosovarische Bevölkerung Vertrauen gefasst hat in die neu aufgebauten multiethnischen Polizei- und Sicherheitskräfte, das ist auch ein Verdienst der Soldatinnen und Soldaten der Bundeswehr, die dazu in den letzten 21 Jahren Enormes beigetragen haben. Dafür gilt ihnen unser großes Dankeschön, meine sehr verehrten Damen und Herren.  Aber wir wollen auch das nicht verschweigen: Politisch haben sich trotz aller Fortschritte die Beziehungen zwischen dem Kosovo und Serbien gerade in den letzten 16 Monaten deutlich verschlechtert. Der bilaterale Dialog beider Staaten, der von der EU moderiert wird, ist praktisch zum Erliegen gekommen. Das schürt Unruhe, und zwar immer wieder auch aufgrund kleinster Ereignisse auf beiden Seiten. Das hat aber auch Auswirkungen auf den ganzen Kontinent: vom Stocken des EU-Annäherungsprozesses bis hin zu Spannungsschwankungen im europäischen Stromnetz. Die Spannungen auf dem Balkan bekommen wir europaweit zu spüren. Der Balkan liegt eben nicht nur geografisch, sondern auch politisch und historisch im Herzen Europas. Meine Damen und Herren, ja, auch das spielt in dem Zusammenhang eine Rolle: Wie nah wir uns sind, das zeigt gerade im Moment ausgerechnet ein globales Virus. Anders nämlich als auch von manchem der Verantwortlichen vor Ort dargestellt, hat die Europäische Union 3,3 Milliarden Euro mobilisiert, um den Westbalkanstaaten in der Coronakrise auf medizinischem und wirtschaftlichem Gebiet zu helfen. Außerdem können die Länder des westlichen Balkans sich jetzt auch an der gemeinsamen europäischen Beschaffung von medizinischen Gütern beteiligen; denn auch das haben wir innerhalb der Europäischen Union auf den Weg gebracht. Bilateral haben wir dort unterstützt, wo der Bedarf am größten gewesen ist, etwa in den Roma-Gemeinden. Wir sind auch bereit, noch mehr zu tun, und sind dazu bereits in guten Gesprächen auf der Grundlage der Entscheidungen, die hier im Bundestag, insbesondere im Haushaltsausschuss, zu treffen sind. Denn, meine Damen und Herren, gerade jetzt ist es wichtig, zu zeigen, dass wir verlässlich zu unseren Partnern auf dem westlichen Balkan stehen. Dabei spielt auch dieses Mandat eine wichtige Rolle. Wir werden während unserer EU-Präsidentschaft hier einen Schwerpunkt für den westlichen Balkan setzen und auch für die komplette Östliche Partnerschaft im Kampf gegen die Folgen der Coronapandemie. Parallel arbeiten wir darauf hin, den politischen Dialog zwischen dem Kosovo und Serbien wieder anzuschieben; denn das ist bitter nötig. Ein erster Schritt ist es ja auch gewesen, dass Deutschland und Frankreich den Gipfel aufgesetzt haben im Rahmen des sogenannten Berliner Prozesses. Wenn es hier Fortschritte gibt, werden wir zusammen mit Frankreich ein Follow-up vorbereiten. Auch die Ernennung von Miroslav Lajcak zum EU-Sonderbeauftragten für den Dialog ist eine gute Voraussetzung, um ein Momentum zu schaffen, in dem dieser Dialog wieder in Gang gesetzt wird. Letztlich geht es darum, wegzukommen von Scheinlösungen wie Grenzänderungen, die ja nicht nur von Betroffenen, sondern auch von anderen, die dort Einfluss nehmen wollen, mit auf den Tisch gelegt worden sind. Dies ist allerdings nur ein vermeintlicher Fortschritt, der damit erreicht würde, wenn das, was damit erreicht würde, bei der Geschichte, die es dort gibt, überhaupt als Fortschritt zu bezeichnen wäre. Stattdessen verschieben solche Gedankenspiele die Lösung komplexer Probleme eher in die Zukunft, und sie führen in nichts anderes als eine Sackgasse.  Meine Damen und Herren, der Weg nach vorne führt – das ist unsere feste Überzeugung – einzig und allein über einen strukturierten, über einen transparenten und auch inklusiven Verhandlungsprozess, so wie ihn die Europäische Union vorgeschlagen hat. Daran werden wir auch festhalten. Am Ende muss nichts anderes als ein umfassendes Abkommen zwischen Belgrad und Pristina stehen, das in beiden Ländern politisch tragfähig ist, das zur regionalen Stabilität beiträgt – nicht nur in diesen beiden Ländern; denn von dem, was dort entschieden wird, sind auch andere betroffen – und damit diesen beiden Ländern letztlich auch, wenn man ganz weit nach vorne schaut, den Weg in Richtung EU-Mitgliedschaft ebnet. Bis dahin wird KFOR als Sicherheitsgarant und als Stabilitätsfaktor in der Region weiter gebraucht werden. Dass KFOR dabei die Unterstützung beider Seiten genießt, die im Moment gerade in nur noch wirklich wenigen Themen übereinstimmen, ist Gold wert, auch wenn es um Deeskalation und Vermittlung bei vielen anderen politischen Themen geht. Insofern, meine Damen und Herren, bleibt unser weiteres Engagement an die Lage vor Ort gekoppelt. Der Erfolg bedingt einen langen Atem. Deshalb bitte ich Sie um die Zustimmung zur Verlängerung dieses Mandates. Herzlichen Dank.  

\noindent\textbf{Comment:}
\begin{itemize}
    \setlength\itemsep{-3pt}
    \item (Beifall bei der AfD)
    \setlength\itemsep{-3pt}
    \item (Beifall bei der SPD und der CDU/CSU – Dr. Alexander S. Neu [DIE LINKE]: Niemals!)
    \setlength\itemsep{-3pt}
    \item (Dr. Alexander S. Neu [DIE LINKE]: Heuchelei!)
    \setlength\itemsep{-3pt}
    \item (Beifall bei der SPD und der CDU/CSU sowie bei Abgeordneten der FDP)
\end{itemize}
\subsection{Friesen}
\noindent\textbf{Texts:} Sehr geehrter Herr Präsident! Sehr geehrte Damen und Herren Abgeordnete! Liebe Bürger! Es begann mit einer Lüge: Die Rechtfertigung für den Kosovokrieg und damit auch für den längsten Bundeswehreinsatz, über den wir hier beraten, war nämlich Fake News des damaligen Bundesverteidigungsministers Scharping. Insofern tritt Innenminister Seehofer eine würdige Nachfolge an, was den Titel „Größter Fake-News-Produzent“ angeht.  Den serbischen Hufeisenplan, eine genozidale Wahnvorstellung mit serbischen KZs und ausgemergelten Kosovo-Albanern, hat es nämlich nie gegeben. Der Kosovokrieg war daher genauso wenig notwendig, wie es heute die grundrechtseinschränkenden und ebenfalls Menschenleben gefährdenden Coronazwangsmaßnahmen sind.  Während beides auf Fake News beruht, dauern die Fake News im Fall des Kosovo schon 21 Jahre an.  Das Märchen vom friedlichen multikulturellen, demokratischen Kosovo glauben wohl nicht einmal diejenigen, die das hier verbreiten.  Das Kosovo ist heute ein faktisch monoethnischer Staat, regiert von Kriegsverbrechern, geprägt von organisierter Kriminalität, von Menschen-, Organ- und Drogenhandel. Die Bürger Kosovos sind die zweitärmsten Europas. Die offizielle Arbeitslosigkeitsrate lag im letzten Jahr bei fast 26 Prozent, die Jugendarbeitslosigkeit gar bei 50 Prozent. Derzeit macht dieses NATO-EU-Protektorat vor allem durch eine Regierungskrise von sich reden. Multiethnisch und multikulturell, so sollte dieser neugeborene Pseudostaat sein. Das ist wohl der Traum der Brüsseler und Berliner Machteliten. Die Realität sieht so aus, dass seit 1999 rund 150 Kirchen, Klöster, andere Heiligtümer zerstört wurden. Mehr als 10 000 Ikonen wurden geschändet und gestohlen, und die serbisch-orthodoxe Kirche muss in ihrer historischen Heimat um ihre Existenz kämpfen.  21 Jahre nach dem Beginn des Kosovo-Einsatzes hat der deutsche Steuerzahler mehr als 464 Millionen Euro für einen gescheiterten Mafiastaat ausgegeben, und das ohne die Kosten des Verteidigungsministeriums, die geheim gehalten werden. Anstatt diesen Lebenslügen nachzuhängen, sollte Berlin darauf hinwirken, dass ein Gebietsaustausch zwischen Kosovo und Serbien stattfindet, sodass die Serben im heutigen Nordkosovo innerhalb des serbischen Staates geschützt werden und einige instabile albanisch geprägte Gebiete in Südserbien durch ihre Eingliederung in das Kosovo stabilisiert werden.  Fake News und Märchen helfen hier nicht weiter. Die Wahrheit ist unseren Bürgern zuzumuten. Deshalb wollen wir diesen Einsatz sofort beenden. Vielen Dank.  

\noindent\textbf{Comment:}
\begin{itemize}
    \setlength\itemsep{-3pt}
    \item (Beifall bei der AfD)
    \setlength\itemsep{-3pt}
    \item (Alexander Graf Lambsdorff [FDP]: Gerassimow wäre stolz auf Sie!)
    \setlength\itemsep{-3pt}
    \item (Alexander Graf Lambsdorff [FDP]: Der Patriarch ist auch stolz auf Sie!)
    \setlength\itemsep{-3pt}
    \item (Alexander Graf Lambsdorff [FDP]: Russia Today im Plenum!)
    \setlength\itemsep{-3pt}
    \item (Peter Beyer [CDU/CSU]: Das wäre das Dümmste, was man machen kann!)
    \setlength\itemsep{-3pt}
    \item (Beifall bei der CDU/CSU)
    \setlength\itemsep{-3pt}
    \item (Ursula Groden-Kranich [CDU/CSU]: Ausgerechnet!)
    \setlength\itemsep{-3pt}
    \item (Peter Beyer [CDU/CSU]: Wahnsinn! – Alexander Graf Lambsdorff [FDP]: Originalton Kreml!)
\end{itemize}
\subsection{Tauber}
\noindent\textbf{Texts:} Herr Präsident! Meine sehr verehrten Damen! Meine Herren! Es war ein wirklich weiter Weg von der Rückgabe des deutschen Feldlagers vor einiger Zeit durch den Parlamentarischen Staatssekretär Thomas Silberhorn hin zu einem deutsch-kosovarischen Innovations- und Trainingspark. Das beschreibt schon ganz gut, vor welchen Herausforderungen dieses junge europäische Land immer noch steht. Wir bleiben präsent – nicht wir als Deutsche alleine, sondern gemeinsam mit Freunden und Verbündeten –, um diesem Land weiter zu helfen, Stabilität und Staatlichkeit zu entwickeln. Es war aber nicht nur ein weiter Weg für die Menschen dort, für die Bundeswehr in ihrem Einsatz, sondern auch für die deutsche Außen- und Sicherheitspolitik insgesamt. Mit Erlaubnis des Präsidenten zitiere ich den Grünenpolitiker Winfried Nachtwei, der gesagt hat: „Wir schlugen auf dem Betonboden der Realpolitik auf.“ Ich glaube, das gilt nicht nur für die damalige rot-grüne Bundesregierung, sondern das gilt auch für den gesamten Deutschen Bundestag mit Blick auf den Einsatz unserer Streitkräfte im Kosovo. Das gilt auch für die Bundeswehr selbst. Und ja, wir sagen hier immer: Es ist eine Parlamentsarmee. – Am Ende ist es die Armee dieses Landes; aber sie hat sicherlich auch im Vergleich zu anderen ein besonderes Verhältnis zu ihrem Parlament. In einem Buch, das ich manchem sehr ans Herz lege – Lesen kann bekanntlich bilden –, von Hans-Peter Kriemann, „Der Kosovokrieg 1999“, beschreibt der Autor sehr gut, wie die Soldaten selbst diesen Einsatz, in den sie damals gegangen sind, empfunden haben. Er schreibt – ich zitiere mit Erlaubnis des Präsidenten –: Viele der deutschen Soldaten waren fest überzeugt, dass sie durch ihren Einsatz einen wichtigen Beitrag zur Beendigung von schweren Menschenrechtsverletzungen, von Mord und Vertreibung im gemeinsamen multinationalen Rahmen der Allianz leisten würden. … Ganz wesentlich für die Soldaten war … das Vertrauen in die deutsche Politik und das Gefühl, mit dem Rückhalt der deutschen Gesellschaft das moralisch Richtige zu tun. Damit sind wir bei einem wichtigen Punkt. Der Außenminister hat darauf hingewiesen, wie lange wir uns schon im Kosovo engagieren. Und das kann man in Zahlen ausdrücken: Fast 130 000 deutsche Soldaten waren dort im Einsatz; 29 Tote, 131 Verwundete hat das die Bundeswehr gekostet. Und für uns als Streitkräfte kann ich sagen, dass wir viel gelernt haben in diesem Einsatz, wenn es um seelische Schäden und um körperliche Verwundungen unserer Soldatinnen und Soldaten geht. Deswegen, glaube ich, ist es an der Zeit, an diesem Pult einmal Danke zu sagen, nicht nur denen, die dort jetzt im Einsatz sind, sondern auch den Kosovoveteranen der Bundeswehr.  Auch deswegen: Man kann ja mal darüber diskutieren: Ist es jetzt nicht Zeit, diesen Einsatz zu beenden angesichts der Struktur, bei der wir als Bundeswehr sozusagen nur noch in der dritten Linie, wenn wirklich dramatische sicherheitspolitische Veränderungen im Kosovo eintreten würden, noch gefragt wären? Ist es dann nicht an der Zeit, zu gehen – oder bleiben wir noch ein bisschen, um es etwas salopp zu formulieren? – Ich glaube, dass es gut ist, dass wir noch ein bisschen bleiben,  auch, um am Ende diesen Einsatz erfolgreich zu beenden. Ich habe aus diesem sehr lesenswerten Buch zitiert, und ich will auch mit einem Zitat des Historikers Hans-Peter Kriemann enden, der in der Frage, wie dieser Einsatz zu bewerten ist, zu folgendem Ergebnis kommt: 20 Jahre sind seit dem Luftkrieg und dem Einmarsch der KFOR inzwischen vergangen. Wo also steht das Kosovo heute? Die Provinz wurde inzwischen weitgehend stabilisiert und ist seit 2008 der jüngste europäische Staat. Vieles spricht dafür, dass vor diesem Hintergrund von einem erfolgreichen Einsatz gesprochen werden kann … Deswegen bitten wir Sie um eine weitere Verlängerung des Mandats. Herzlichen Dank.  

\noindent\textbf{Comment:}
\begin{itemize}
    \setlength\itemsep{-3pt}
    \item (Beifall bei der FDP)
    \setlength\itemsep{-3pt}
    \item (Peter Beyer [CDU/CSU]: Sehr richtig!)
    \setlength\itemsep{-3pt}
    \item (Beifall bei der CDU/CSU sowie bei Abgeordneten der SPD)
    \setlength\itemsep{-3pt}
    \item (Beifall bei der CDU/CSU, der SPD, der FDP und dem BÜNDNIS 90/DIE GRÜNEN)
\end{itemize}
\subsection{Alt}
\noindent\textbf{Texts:} Sehr geehrter Herr Präsident! Meine sehr geehrten Damen und Herren! Als ich im vergangenen Jahr die Brücke über den Fluss Ibar in Mitrovica gegangen bin, überkam mich ein mulmiges Gefühl. Wie keine andere Stadt steht Mitrovica für den Konflikt in Kosovo. Seit jeher trennt die Brücke die Serben im Norden von den Albanern im Süden. Kaum jemand traut sich auf die andere Seite. Daran hat sich seit 1998 kaum etwas geändert. Die vielen Hassbotschaften an den Fassaden verdeutlichen den immer noch vorhandenen ethnischen Konflikt. Der Frieden ist vor allem im Norden des Landes äußerst labil. Die Präsenz von KFOR war und ist der Stabilitätsanker auf dem Westbalkan.  Erstens. Die Spannungen zwischen Serbien und Kosovo haben zugenommen. Debatten – die immer wiederkehren – über mögliche Grenzverschiebungen haben zusätzlich für eine neue Nervosität gesorgt. Der Streit über den richtigen Kurs gegenüber Serbien führte zum Sturz der Regierung von Premierminister Albin Kurti. Seitdem kämpft das Land mit einer ausgewachsenen Regierungs- und Verfassungskrise. Dazu kommen jetzt noch die Herausforderungen bei der Bekämpfung des Coronavirus. Wissen wir, welche Folgen eine anhaltende politische Instabilität für die Lage vor Ort haben kann? Wissen wir das? Nein, das wissen wir nicht! Die unübersichtlichen Verhältnisse steigern das Risiko einer Eskalation. Zweitens. Der Westbalkan ist noch immer die Achillesferse Europas. Russland, China und die Türkei versuchen zunehmend, ihren Einfluss in der Region auszuweiten. KFOR sendet deshalb das wichtige politische Signal, dass die NATO weiterhin für die Stabilität in der Region eintritt.  Beim Westbalkangipfel vorige Woche hat die Europäische Union unmissverständlich klargemacht, dass sie sich zur europäischen Perspektive für die Region voll und ganz bekennt. Ich appelliere deshalb an die Länder des Westbalkans: Wendet euch nicht ab! Beschreitet auch weiterhin den Weg der Reformen und damit auch den Weg in die EU! Ihr gehört zu Europa, ihr seid Europa! Unser Ziel muss sein: ein dauerhafter Frieden und Stabilität auf dem Westbalkan. Solange aber beides in Gefahr ist, bleibt KFOR für dauerhaften Frieden auf dem Balkan und in Europa unverzichtbar. Deshalb werden wir der Mandatsverlängerung zustimmen.   

\noindent\textbf{Comment:}
\begin{itemize}
    \setlength\itemsep{-3pt}
    \item (Manuel Sarrazin [BÜNDNIS 90/DIE GRÜNEN]: Kollege Neu, jetzt überraschen Sie mich mal! – Gegenruf des Abg. Alexander Graf Lambsdorff [FDP]: Ja, mal was Neues!)
    \setlength\itemsep{-3pt}
    \item (Beifall bei der LINKEN – Peter Beyer [CDU/CSU]: Der sagt nichts Neues!)
    \setlength\itemsep{-3pt}
    \item (Beifall bei der FDP sowie bei Abgeordneten der CDU/CSU)
    \setlength\itemsep{-3pt}
    \item (Beifall bei der FDP, der CDU/CSU und der SPD)
\end{itemize}
\subsection{Neu}
\noindent\textbf{Texts:} Sehr geehrter Herr Präsident! Sehr geehrte Damen und Herren! Was ist KFOR? KFOR bedeutet 21 Jahre lang militärische Besetzung der südserbischen Provinz Kosovo  aufgrund eines Missbrauchs der UN-Sicherheitsratsresolution 1244.  Warum? KFOR ist der Nachfolger der Operation Allied Force, also des US-geführten NATO-Krieges gegen Jugoslawien 1999, seinerzeit unter Beteiligung von Rot-Grün.  Ziel der NATO war von Anfang an die Unterstützung der kosovo-albanischen Nationalisten, wie schon die Scheinverhandlungen von Rambouillet offenbarten. Über Nacht wechselte dann die NATO den Hut von der Kriegspartei an der Seite der UCK-Terroristen zur UN-mandatierten KFOR-„Friedenstruppe“. Faktisch blieb KFOR natürlich als überlackierte NATO an der Seite der UCK und sicherte die illegale Sezession im Folgenden 2008 militärisch ab.  Die deutsche Politik hegte und pflegte ja nicht nur die Separatisten der serbischen Provinz Kosovo. Nein, die deutsche Politik nutzte die innerstaatlichen und wirtschaftlichen Probleme Jugoslawiens geschickt aus, unterstützte alle nationalistischen Kräfte in Jugoslawien, sofern sie an der Zerschlagung Jugoslawiens arbeiteten. Bisweilen erhielten sie sogar Asyl in Deutschland, vorzugsweise in München nahe Pullach.  Wie skrupellos auch die Regierung Kohl da mitagierte, zeigt, dass sie seinerzeit genau die Kräfte im damaligen Jugoslawien unterstützte, die Hitler-Deutschland schon im Zweiten Weltkrieg unterstützte. Die Parallelen sollten zu denken geben.  Sehr geehrte Damen und Herren, die Einmischung in die inneren Angelegenheiten von Staaten ist eine große Konfliktquelle und daher völkerrechtlich verboten. Diese Gefahr erkannte bereits der große Philosoph Immanuel Kant. In seinem berühmten Werk „Zum ewigen Frieden“ formulierte er den fünften Präliminarartikel. Darin sagt er – ich zitiere –: „Kein Staat soll sich in die Verfassung und Regierung eines andern Staats … einmischen.“ Aber was macht die deutsche Politik? Sie macht das Gegenteil. Seit über 30 Jahren tut sie das. Aktuelles Beispiel: Es gibt erste Debatten zwischen der serbischen Regierung und dem kosovo-albanischen Präsidenten Thaci über einen Gebietsaustausch; wurde gerade angesprochen. Serben wollen einfach nicht unter albanischer Verwaltung leben, Albaner nicht unter serbischer Verwaltung. So weit, so gut. Ein Gebietsaustausch könnte ein Schritt zur friedlichen Koexistenz in der Region sein.  Russland ist dafür, selbst die USA sind offen; auch in der Europäischen Union gibt es Staaten, die dafür offen sind. Nur: Wer ist dagegen? Die deutsche Regierung! Danach.  Die deutsche Regierung erklärt dreist: Bitte keine Grenzveränderungen! Diese könnten ja Instabilitäten auf dem Balkan hervorrufen. – Da fasst man sich ja an den Kopf. Unfassbar! Nachdem die deutschen Regierungen seit rund 30 Jahren nichts anderes getan haben, als die Grenzen in Südosteuropa selbstherrlich zu verschieben und damit das Völkerrecht mit den Füßen zu treten, entdeckt nun die deutsche Regierung die Rechtsnorm der territorialen Integrität wieder. Das ist doch lächerlich.  Sehr geehrte Damen und Herren, ist das Kosovo nun souverän und Staat? Ja oder nein? Kurzum: Wenn das Kosovo ein eigenständiger Staat und souverän wäre, könnten das Kosovo und Serbien natürlich einvernehmlich einen Gebietsaustausch vornehmen. Aber: Wenn die Bundesregierung dagegen ist, ist das ein indirektes Eingeständnis – eigentlich ein direktes Eingeständnis –, dass das Kosovo nichts anderes ist als eine deutsche Kolonie. Vielen Dank.  Jetzt bin ich auf Ihre Fragestellung gespannt. Okay.  

\noindent\textbf{Comment:}
\begin{itemize}
    \setlength\itemsep{-3pt}
    \item (Beifall beim BÜNDNIS 90/DIE GRÜNEN)
    \setlength\itemsep{-3pt}
    \item (Beifall bei der LINKEN – Dr. Johann David Wadephul [CDU/CSU]: Jetzt wird es langsam abenteuerlich! – Alexander Graf Lambsdorff [FDP]: Und dazwischen war nichts? Zwischen 1945 und 2008 ist gar nichts passiert! – Gegenruf des Abg. Dr. Johann David Wadephul [CDU/CSU]: Jetzt wechselt der zur AfD!)
    \setlength\itemsep{-3pt}
    \item (Manuel Sarrazin [BÜNDNIS 90/DIE GRÜNEN]: Die Terroristen in Serbien wurden nur von der Linkspartei unterstützt!)
    \setlength\itemsep{-3pt}
    \item (Dr. Tobias Lindner [BÜNDNIS 90/DIE GRÜNEN]: Und warum? – Alexander Graf Lambsdorff [FDP]: Davor war gar nichts, oder?)
    \setlength\itemsep{-3pt}
    \item (Dr. Johann David Wadephul [CDU/CSU]: Das ist die alte Platte! – Peter Beyer [CDU/CSU]: Bringen Sie mal was Neues!)
    \setlength\itemsep{-3pt}
    \item (Alexander Graf Lambsdorff [FDP]: Das hat die AfD auch gesagt! – Gegenruf des Abg. Dr. Alexander Gauland [AfD]: Wo er recht hat, hat er recht!)
    \setlength\itemsep{-3pt}
    \item (Beifall bei der LINKEN sowie des Abg. Dr. Anton Friesen [AfD])
    \setlength\itemsep{-3pt}
    \item (Lachen bei Abgeordneten der CDU/CSU – Peter Beyer [CDU/CSU]: Oh Gott!)
    \setlength\itemsep{-3pt}
    \item (Heiterkeit bei der SPD und dem BÜNDNIS 90/DIE GRÜNEN)
    \setlength\itemsep{-3pt}
    \item (Dr. Johann David Wadephul [CDU/CSU]: Das hat die AfD auch so gesagt!)
    \setlength\itemsep{-3pt}
    \item (Beifall bei der LINKEN)
\end{itemize}
\subsection{Sarrazin}
\noindent\textbf{Texts:} Herr Präsident, ich glaube, die Endfrage hat sich erledigt. – Ich möchte zwei Kommentare machen: einen zum Kollegen Friesen und einen zum Kollegen Neu. Ich glaube, Herr Kollege Friesen, das, was Sie hier gezeigt haben – diese Verachtung, dieser strotzende Orientalismus, dieser Chauvinismus gegenüber den Menschen im Kosovo – ist die Essenz, aus der früher wie heute Kriege gemischt werden. Schämen Sie sich dafür.  Kollege Neu, ich möchte Sie ansprechen, bewusst nicht gleichsetzend mit dem Kollegen, bewusst mit einem Unterschied: Ihr Verschweigen der Opfer, der Gräueltaten und der Kriegsverbrechen  trägt in der Region auch dazu bei, dass sich alle Seiten nicht gegenseitig als Opfer anerkennen. Das ist auch eine Essenz, aus der leider Kriege gemacht werden. Das bedaure ich sehr.  Vieles, was von den Kollegen gesagt wurde, ist richtig. Wir erleben im Kosovo gerade eine schwere politische Krise. Diese politische Instabilität ist nicht nur einfach schlecht, sondern gerade in der Situation der Pandemie hochgefährlich. Es ist fatal, dass wir in genau dieser Situation erleben, dass das politische Verantwortungsgefühl in Pristina, in Belgrad und in Washington offensichtlich nicht ausreicht, um im Interesse einer gemeinsamen Zukunft der Region zu handeln; das müssen wir so klar sagen. Das Kosovo wird von Herrn Trump behandelt wie eine Kolonie; nicht von der Europäischen Union, von Herrn Trump.  Das sollten wir ablehnen, in dem wir sagen: Einen „Land Swap“ wird es nicht geben,  weil die Menschen im Kosovo und in Serbien dagegen sind. Das müssen Sie sich wirklich hinter die Ohren schreiben.  Wir erleben im Kosovo einen Präsidenten und eine Koalitionspartei, LDK, die um den eigenen Machterhalt mit der Zukunft des Landes spielt. Wir als Opposition hatten in den letzten Wochen kaum die Gelegenheit, die Bundesregierung oder die CDU\/CSU dafür zu kritisieren, weil sogar sie die richtige Position dort eingenommen haben. Sie haben versucht, Albin Kurti und seine Regierung zu stützen. Sie haben versucht, Einfluss zu nehmen, dass das US-amerikanische Spiel in dem Land nicht gelingt.  Aber wir müssen festhalten: Die Europäer haben diesen Machtkampf genauso verloren, wie die progressiven Kräfte im Kosovo und in der LDK ihn verloren haben. Das ist ein ganz bitteres Zeichen für den – positiven – Einfluss der Europäischen Union in der Region. Das ist einer der Gründe, warum wir meiner Ansicht nach jetzt in dieser gefährlichen Lage auch vor Ort präsent bleiben sollten. Das heißt nicht, dass wir alle Teile des Mandats richtig finden. Die kritische Betrachtung, die in den Ausschüssen vorgenommen wurde und die in der zweiten Lesung vorgenommen werden wird, ist richtig. Aber die Präsenz ist und bleibt richtig, nicht trotz, sondern wegen der friedlichen Lage, die wir heute im Kosovo im Alltag erleben. Danke sehr.  

\noindent\textbf{Comment:}
\begin{itemize}
    \setlength\itemsep{-3pt}
    \item (Zurufe von der LINKEN)
    \setlength\itemsep{-3pt}
    \item (Beifall beim BÜNDNIS 90/DIE GRÜNEN)
    \setlength\itemsep{-3pt}
    \item (Beifall beim BÜNDNIS 90/DIE GRÜNEN, bei der CDU/CSU, der SPD und der FDP sowie bei Abgeordneten der LINKEN – Karsten Hilse [AfD]: Völkerrecht gebrochen hat der Minister! Unglaublich!)
    \setlength\itemsep{-3pt}
    \item (Beifall bei der CDU/CSU)
    \setlength\itemsep{-3pt}
    \item (Beifall beim BÜNDNIS 90/DIE GRÜNEN und bei der CDU/CSU sowie bei Abgeordneten der SPD)
    \setlength\itemsep{-3pt}
    \item (Beifall bei Abgeordneten)
    \setlength\itemsep{-3pt}
    \item (Beifall beim BÜNDNIS 90/DIE GRÜNEN, bei der CDU/CSU, der SPD und der FDP – Dr. Alexander S. Neu [DIE LINKE]: Sie vermeiden den Begriff „Genozid“! – Weitere Zurufe von der LINKEN)
    \setlength\itemsep{-3pt}
    \item (Gerold Otten [AfD]: Kriegstreiberminister und hält hier solche Reden!)
    \setlength\itemsep{-3pt}
    \item (Dr. Alexander S. Neu [DIE LINKE]: Die übelsten Chauvinisten, genau!)
\end{itemize}
\subsection{Schmidt}
\noindent\textbf{Texts:} Herr Präsident! Meine sehr verehrten Kolleginnen und Kollegen! Ich bleibe gerne im Bild der Pandemie, das der Außenminister bezüglich der Covidkrise gezogen hat. Nach dem wirren Bild, das ich von Herrn Friesen gehört habe, habe ich mir fast noch die Frage gestellt: Steht der Verursacher vielleicht in der Liste der Schlimmen? Haben Sie den üblicherweise von Ihnen zitierten Bill Gates vergessen? Den hätten Sie auch noch nennen können. Nichts entspricht der Wahrheit. Nein, die pandemische Situation ist: Wenn es einen Infizierten null gegeben hat, dann war das Milosevic, als er am Veitstag 1989 auf dem Amselfeld ein Groß-Serbien ausgerufen hat, was die eigentliche ideologische Grundlage gewesen ist.  – Was ist los?  – Ich kenne die Rede sehr wohl, mein Lieber. Da müssen Sie mit Ihrer Propaganda früher aufstehen. Sie wissen offensichtlich nicht, was in der Region, in diesem Land war.  Klaus Kinkel hat damals die Einladung des Westbalkan nach Europa ausgesprochen, aus der klugen Erkenntnis, aus den Trümmern Jugoslawiens, das leider – oder Gott sei Dank – Tito nicht überlebt hat, weswegen sich die Serben dann in die Richtung von Milosevic entwickelt hatten, zu einer friedvollen Lösung zu finden. Nein, die Pandemie war leider ausgebrochen und hat viele Todesopfer gefordert. Wenn Sie genau wissen wollen, wieso sich Helmut Kohl in diesem Bereich engagiert hat, dann hören Sie zu: Das lag natürlich am Balkan insgesamt – Srebrenica und andere –; denn der Umgang mit den Menschen wäre in Europa in der Weise nicht zu vertreten gewesen. Wir haben uns der Verantwortung gestellt und im positiven Sinne interveniert.  Die Resolution 1244 des Sicherheitsrates der Vereinten Nationen gilt heute immer noch.  Das ist eine Grundlage. Deswegen ist KFOR da. KFOR muss dableiben. Deswegen kann ich nur, Kollege Sarrazin, sagen: Lasst uns bitte auch in Washington klarmachen, dass es nicht darum geht: „Take our boys home!“ Lasst die wenigen, die noch da sind, helfen, dass die Stabilisierung im Kosovo weiterhin funktioniert.  Diejenigen, mich eingeschlossen, die die Möglichkeit hatten, mit Ibrahim Rugova zu sprechen und zu hören, wie die Unterdrückung der Albaner im Kosovo in einer gewissen Phase des auseinanderbrechenden Jugoslawiens war, verstehen auch, dass es humanitäre Notwendigkeiten gab. Man sieht auch heute – unbestritten –, dass die Entwicklung im Kosovo, im neuen Staatsgebilde Kosovo, die letztendlich auf dem Vorschlag von Ahtisaari fußt, noch sehr viele Defizite hat: Defizite im Bereich der Korruptionsbekämpfung sowie in vielen anderen Fragestellungen, auch Fehler, die daraus entstanden sind, weil man das Zusammenleben in der Region – die Einladung an den Westbalkan, um Klaus Kinkel noch einmal zu zitieren – nie richtig umgesetzt hat. Das erfordert nicht nur von den serbischen Ministerpräsidenten und Präsidenten, das erfordert nicht nur von den kosovarischen Präsidenten und Ministerpräsidenten Handlungsbereitschaft, das nimmt auch uns mit in die Pflicht. Ja, das Kosovo ist ein Kind Europas. Wenn es ein Kind Europas ist, muss man sich darum kümmern. Man muss auch durchaus vermitteln, man darf aber nicht nur vermitteln. Deswegen bleibt eines ganz klar: Wer meint, „Land Swaps“, Gebietstausch, seien die Lösung in einer Region, in der man keine Strecke länger als 10 Kilometer laufen kann, ohne in das Gebiet einer ethnischen Minderheit zu kommen, der möge mir sagen, was er damit vorhat. Nein, so wie es ist, muss es bleiben. Wir müssen mit unserem Engagement dazu beitragen,  dass im Kosovo, auch mit KFOR, die Stabilität wächst. Man sieht und spürt, wie wichtig es ist, dass die Vernünftigen die Richtung beibehalten.  

\noindent\textbf{Comment:}
\begin{itemize}
    \setlength\itemsep{-3pt}
    \item (Zaklin Nastic [DIE LINKE]: Was steht da drin? Haben Sie die mal gelesen?)
    \setlength\itemsep{-3pt}
    \item (Zurufe von der LINKEN)
    \setlength\itemsep{-3pt}
    \item (Dr. Alexander S. Neu [DIE LINKE]: Sie kennen die Rede gar nicht!)
    \setlength\itemsep{-3pt}
    \item (Beifall bei der CDU/CSU und der FDP – Dr. Alexander S. Neu [DIE LINKE]: Hören Sie mit dem Märchen auf! Gott sei Dank ist es vorbei!)
    \setlength\itemsep{-3pt}
    \item (Dr. Alexander S. Neu [DIE LINKE]: Diese Verbrechen sind nicht vom Himmel gefallen!)
    \setlength\itemsep{-3pt}
    \item (Beifall bei der CDU/CSU – Zaklin Nastic [DIE LINKE]: Als Besatzungsmacht! – Weiterer Zuruf des Abg. Dr. Alexander S. Neu [DIE LINKE])
    \setlength\itemsep{-3pt}
    \item (Beifall bei der CDU/CSU)
    \setlength\itemsep{-3pt}
    \item (Dr. Alexander S. Neu [DIE LINKE]: Sie haben die Rede nie gelesen! Sie kennen die Rede gar nicht!)
    \setlength\itemsep{-3pt}
    \item (Beifall bei der CDU/CSU -Zuruf des Abg. Dr. Alexander S. Neu [DIE LINKE])
\end{itemize}
\subsection{Manderla}
\noindent\textbf{Texts:} Herr Präsident! Liebe Kolleginnen und Kollegen! Wenn man jungen Leuten erklären sollte, was Geschichtsverdrehung ist, dann sollte man die heutige Debatte um dieses Thema nehmen. Sowohl von rechts als auch von links hat man heute erlebt,  wie Geschichte vollkommen falsch dargestellt wird.  Ihre menschenverachtenden Eindrücke und Einbringungen, meine Damen und Herren, können wir nur zutiefst verachten.  Nein, gestatte ich nicht.  Liebe Kollegen und Kolleginnen, worum geht es heute? Wir diskutieren allein in dieser Woche vier Bundeswehrmandate. Das zeigt doch, wie labil die Situation weltweit ist.  Wie labil sie in Afrika ist, hat unsere Verteidigungsministerin eben dargestellt. Sie hat auch gesagt: Es geht uns bei unseren Einsätzen um das Leben jedes Einzelnen und jeder Einzelnen. Wir wissen doch, wie es vor 21 Jahren im ehemaligen Jugoslawien war, wie die Menschen gelitten haben, wie sie sich gegenseitig ermordet haben, wie besonders die Frauen unter der Situation gelitten haben. Deshalb ist es wirklich gut, dass wir diesen Einsatz der Bundeswehr seit 21 Jahren haben. Und ich sage Ihnen noch eines, liebe Kolleginnen und Kollegen von den Linken, da Sie immer wieder die Rüstungsausgaben mit anderen Themen in Zusammenhang bringen: Wenn wir unsere Soldaten und Soldatinnen nicht mit vernünftiger Rüstung ausstatten können, dann können wir nicht weltweit Leben retten. Und darum geht es uns hier.  Wenn es vor 21 Jahren darum ging, Frieden zu schaffen, dann geht es jetzt darum, Frieden zu erhalten. Und wir werden unserem Auftrag, den wir von der NATO und von den Vereinten Nationen bekommen haben, selbstverständlich weiterhin gerecht werden. Heute geht es darum, den Frieden zu erhalten. Wie schon gesagt wurde, ist die Situation zurzeit relativ stabil. Aber – und das wurde auch schon gesagt – die Situation im Kosovo ist angespannt. Es geht heute darum, dass wir die politischen Akteure in den betreffenden Ländern – im Kosovo, in Serbien – auffordern, miteinander zu reden und miteinander die wirtschaftlichen und sozialen Probleme zu lösen. Worum geht es, liebe Kollegen und Kolleginnen? Es geht darum, dass diese Länder Mitglied der EU werden sollen. Das wollen wir auch befürworten. Aber wenn dies geschehen soll, dann muss sich dort einiges ändern, und das geht nicht, wenn man nicht miteinander redet. Es geht auch um die Anerkennung völkerrechtlich anerkannter Grenzen. Das müssen wir, denke ich mal, klarmachen. Der Staatsaufbau im Kosovo war und ist ein transatlantisches Projekt.  – Ich denke, es ist wichtig, das zu sagen. Vielen Dank, Herr Dr. Neu, dass Sie mir einmal recht gegeben haben.  Wir als CDU\/CSU-Fraktion sind für den weiteren Einsatz im Kosovo. Herzlichen Dank. 

\noindent\textbf{Comment:}
\begin{itemize}
    \setlength\itemsep{-3pt}
    \item (Beifall bei der CDU/CSU und der FDP)
    \setlength\itemsep{-3pt}
    \item (Beifall bei der CDU/CSU – Dr. Alexander S. Neu [DIE LINKE]: Die Märchenstunde nimmt kein Ende!)
    \setlength\itemsep{-3pt}
    \item (Karsten Hilse [AfD]: Nein, das zeigen sie nicht!)
    \setlength\itemsep{-3pt}
    \item (Beifall bei Abgeordneten der CDU/CSU)
    \setlength\itemsep{-3pt}
    \item (Beifall bei der CDU/CSU)
    \setlength\itemsep{-3pt}
    \item (Tobias Pflüger [DIE LINKE]: Nur draufhauen, aber sich nicht der Frage stellen!)
    \setlength\itemsep{-3pt}
    \item (Tobias Pflüger [DIE LINKE]: Unverschämtheit! – Dr. Alexander S. Neu [DIE LINKE]: Unglaublich! – Alexander Graf Lambsdorff [FDP]: Richtig!)
    \setlength\itemsep{-3pt}
    \item (Dr. Alexander S. Neu [DIE LINKE]: Das ist wahr! Da haben Sie recht!)
    \setlength\itemsep{-3pt}
    \item (Dr. Alexander S. Neu [DIE LINKE]: Danke für die Aussage!)
\end{itemize}
\subsection{Nastic}
\noindent\textbf{Texts:} Vielen Dank, Herr Präsident. – Ich möchte zur Aufklärung zwei ergänzende Dinge zum Beitrag meiner Vorrednerin sagen. Erstens. Wenn Sie uns Linke als menschenverachtend bezeichnen, nur weil wir nicht Ihrer politischen Meinung sind – und Sie haben ja selbst gesagt, dass es ein Staatsvertragsprojekt war, das Sie da quasi am lebenden Objekt, an den Menschen vor Ort, in Jugoslawien, im Kosovo, in Metochien und Serbien, verübt haben –, kann ich Ihnen eines sagen: Ich spreche Serbisch, ich bin seit mehr als 20 Jahren ständig dort, meine Kinder sind unter anderem serbische Staatsbürger. Und zu sagen, wir seien Menschenverächter, weil wir die Interessen der Menschen von dort vertreten – und ich habe sehr gute Kontakte und bekomme fast täglich Informationen eben von den Bürgern, von denen Sie reden, die es in ihrem Alltagsleben tagtäglich betrifft –, das ist menschenverachtend. Sie interessiert nicht, was diese Menschen bewegt und unter welchen Umständen sie leben, sondern Sie reden darüber, wie Sie über sie bestimmen, auch im Rahmen eines Staatsvertragsprojektes. Zweitens. Dazu, dass Sie hier links und rechts gleichsetzen, möchte ich Ihnen sagen: Sowohl von deutschen als auch von albanischen Nationalisten bekomme ich regelmäßig Drohungen. Und da reihen Sie uns ein? Vielen Dank, liebe Kollegin.

\subsection{Manderla}
\noindent\textbf{Texts:} Ich will jetzt die Sitzung nicht in die Länge ziehen; aber das, was ich gesagt habe, wurde hier vollkommen verdreht. Selbstverständlich ist das, was da passiert, schlimm, und ich bedauere das auch sehr. Das wissen Sie auch ganz genau. Aber ich habe nicht Sie und auch nicht die serbischen Mitbürger als Menschenverachtende bezeichnet,  sondern davon gesprochen, dass hier Geschichte verdreht worden ist,  und dabei bleibe ich auch. 

\noindent\textbf{Comment:}
\begin{itemize}
    \setlength\itemsep{-3pt}
    \item (Zaklin Nastic [DIE LINKE]: Von Ihnen!)
    \setlength\itemsep{-3pt}
    \item (Beifall bei der CDU/CSU sowie des Abg. Ulrich Lechte [FDP])
    \setlength\itemsep{-3pt}
    \item (Tobias Pflüger [DIE LINKE]: Rechts und links gleichgesetzt!)
\end{itemize}
\section{Tagesordnungspunkt 4}
\subsection{Lay}
\noindent\textbf{Texts:} Frau Präsidentin! Meine sehr verehrten Damen und Herren! Viele Mieterinnen und Mieter wissen nicht, wie sie ihre Miete bezahlen können. 74 Prozent der Deutschen haben Angst, ihre Wohnung zu verlieren – und das galt schon vor Corona. Jetzt, während der Pandemie, haben viele ihre Jobs verloren, mussten ihre Läden, ihre Kneipen, ihre Kosmetikstudios schließen, haben kein Einkommen mehr, müssen aber weiter die Mietkosten bezahlen. Und andere erhalten 60 Prozent Kurzarbeitslohn, müssen aber 100 Prozent Miete zahlen. Das passt offensichtlich nicht zusammen. Wir müssen etwas tun; denn die Coronakrise darf die Mietenkrise nicht verschärfen.  Schon jetzt geben 1,6 Millionen Haushalte an, dass sie ihre Miete wegen Corona nicht mehr bezahlen können. Ohne ein Soforthilfeprogramm für den Wohnungsmarkt werden viele aus ihrer Wohnung fliegen. Ohne ein solches Programm droht die nächste Verdrängungswelle und wird dafür sorgen, dass noch mehr Normalverdienende aus den Wohnungen, aus den Innenstädten fliegen. Das müssen wir verhindern.  Das, was die Regierung vor ein paar Wochen vorgelegt hat, reicht bei Weitem nicht aus. Drei Monate lang darf wegen coronabedingten Mietausfällen nicht gekündigt werden. Das ist schön und gut, aber: Was kommt danach? Sie sagen doch selbst: Wir stehen erst am Anfang der Pandemie. – Dann muss der Kündigungsschutz auch bis zum Ende der Pandemie verlängert werden. Das sollte doch hier der kleinste gemeinsame Nenner sein.  Außerdem schützt die Regierung nur vor coronabedingten Mietausfällen. Alles andere bleibt weiter erlaubt: ordentliche Kündigungen, Eigenbedarfskündigungen, selbst Zwangsräumungen. Wir Linke finden: Niemand darf in der Coronakrise seine Wohnung verlieren. In dieser Krise gehören alle Kündigungen verboten.  Betroffene Mieterinnen und Mieter, aber auch Kleinvermieter, selbstnutzende Eigentümer brauchen finanzielle Hilfen, um diese Krise zu überstehen. Wir schlagen unter anderem eine Ausweitung und eine Vereinfachung des Wohngeldes vor. Der Deutsche Mieterbund fordert einen Fonds, die Grünen haben einen dritten Vorschlag eingebracht. Aber was überhaupt nicht geht, ist, dass von der Koalition überhaupt nichts kommt, um die Betroffenen zu unterstützen. Das ist schlichtweg nicht hinnehmbar.  Wir Linke finden: In der Krise darf die Miete nicht erhöht werden. Das fordert auch der Mieterbund. Keine Mieterhöhung in der Coronakrise. Das sollte eigentlich selbstverständlich sein.  Wir finden auch, dass Mieterinnen und Mieter nicht alleine auf den Mietschulden sitzen bleiben dürfen. Diese innerhalb von zwei Jahren zurückzuzahlen, wie es die Regierung vorsieht, ist für viele Leute völlig realitätsfern, die jetzt schon 50 Prozent des Einkommens für das Wohnen ausgeben. Die Grünen fordern, dass sie innerhalb von zehn Jahren zurückgezahlt werden sollen. Das wäre schon besser, aber das ist aus meiner Sicht nicht gerecht; denn niemand hat dieses Virus verursacht. Deswegen müssen die Kosten der Krise auch gerecht verteilt werden und nicht allein auf den Rücken der Mieterinnen und Mieter abgewälzt werden.  Deswegen muss es für Betroffene auch die Möglichkeit der Mietsenkung geben. Wir haben einen Vorschlag gemacht, der neben Mietern den Staat, aber auch die Vermieterinnen und Vermieter in die Pflicht nimmt. Keine Sorge: Oma Erna, die vielleicht eine Wohnung vermietet, um ihre kleine Rente aufzupäppeln, wird anders behandelt als Vonovia. Sie bekäme Unterstützung durch unseren Härtefallfonds „Soziales Wohnen“. Aber Deutsche Wohnen, Vonovia und Co können die Mietsenkungen auch ohne staatliche Hilfen stemmen; denn wer während der Coronakrise milliardenschwere Dividenden und Boni ausschüttet, darf keine Staatshilfen bekommen. Wir wollen einen fairen Lastenausgleich. Zu guter Letzt: Sammelunterkünfte in der Krise auflösen; denn auch Obdachlose und Geflüchtete dürfen nicht die Verlierer dieser Krise sein. Das Recht auf Wohnen muss für alle gelten, auch und gerade während der Coronakrise. Vielen Dank.  

\noindent\textbf{Comment:}
\begin{itemize}
    \setlength\itemsep{-3pt}
    \item (Beifall bei der LINKEN und der AfD – Michael Grosse-Brömer [CDU/CSU]: Das ist schlichtweg falsch!)
    \setlength\itemsep{-3pt}
    \item (Beifall bei der CDU/CSU)
    \setlength\itemsep{-3pt}
    \item (Beifall bei der LINKEN und dem BÜNDNIS 90/DIE GRÜNEN)
    \setlength\itemsep{-3pt}
    \item (Beifall bei der LINKEN)
\end{itemize}
\subsection{Zeulner}
\noindent\textbf{Texts:} Sehr geehrte Frau Präsidentin! Liebe Kolleginnen und Kollegen! Keine Mieterhöhungen in der Coronakrise – das macht beispielsweise Vonovia. Aber natürlich passt der, der es tut, den Linken nicht. Deswegen ist es manchmal schwierig, das bei der Argumentation positiv zu erwähnen. So gibt es ganz viele Beispiele – auch bei mir in der Umgebung –, bei denen Vermieter beispielsweise kleinen Kneipen die Miete erlassen. Man hilft sich gegenseitig, man unterstützt sich. Deswegen: Zu Beginn einen herzlichen Dank an alle in unserem Land, die in dieser Krise zusammenhalten.  Vorneweg möchte ich grundsätzlich sagen, dass ich es sehr schätze, wenn sich Kollegen in die aktuellen Debatten produktiv und konstruktiv einbringen, zum Beispiel die Grünen.  – Ich habe ein Beispiel genannt. – Es ist für mich manchmal ganz spannend, dass im Gesundheitsausschuss zum Beispiel die AfD-Kollegen zu Beginn der Diskussion und der Krise den Gesundheitsminister noch positiv gelobt haben. Mit der Zeit, als man gemerkt hat: „Das trifft nicht auf Wohlgefallen der Parteispitze“, wurde schnell die Bremse gezogen. Das finde ich sehr schade. Passend dazu fällt einem natürlich ein Satz ein: Intelligente suchen in Krisenzeiten nach Lösungen, während die Unwissenden nach Schuldigen suchen. Deswegen bleibt es dabei: Danke für diesen Zusammenhalt.  Nun zu den beiden Anträgen der Linken und Grünen, in denen sie verschiedene Maßnahmen vorschlagen, um Mieter und private Kleinvermieter vor Mietausfällen während der Krise zu schützen. Natürlich kann man viel darüber diskutieren, aber eigentlich sind wir schon lange einen Schritt weiter und haben als Große Koalition auch gehandelt und viele Maßnahmen unternommen. Ich möchte nicht unerwähnt lassen, dass bei den Linken und Grünen mittlerweile auch erkannt wurde, dass es wichtig ist, einen funktionierenden Wohnungsmarkt mit privaten Kleinvermietern zu haben. Auch diese Vermieter haben unsere Unterstützung verdient. Es sind immerhin tatsächlich zwei Drittel der Kleinvermieter, die Wohnungen zur Verfügung stellen. 40 Prozent davon sind Rentnerinnen und Rentner. Das ist ein Teil ihrer Altersvorsorge. Wir haben mit den Maßnahmen, die wir verfolgt haben, das Ziel gehabt, die bestehenden sozialen Sicherungssysteme zu stärken, entsprechend auszubauen und anzupassen, um die Auswirkungen abzufedern und Mietausfälle zu verhindern. Deswegen möchte ich einige Beispiele aufzählen: Erstens. Wir haben die Voraussetzungen für das Kurzarbeitergeld heruntergesetzt und werden in dieser Woche noch eine Erhöhung beschließen. Zweitens. Wir haben den Zugang zur Grundsicherung erleichtert, und für alle, die zwischen dem 1. März und dem 30. Juni einen Neuantrag stellen, werden für sechs Monate die Kosten der Unterkunft, also die Miete inklusive Heiz- und Nebenkosten, übernommen.  – Vielen Dank, liebe Kollegin Nissen, für den Zwischenapplaus. Ich weiß, dass du uns dabei immer sehr unterstützt und immer mit vorne kämpfst. Drittens. Die Auszahlung des Arbeitslosengeldes wird für alle, deren Anspruch zwischen dem 1. Mai und 31. Dezember enden würde, um drei Monate verlängert. Viertens. Das Bundesinnenministerium hat die Antragstellung und Prüfung für das Wohngeld vereinfacht, damit Betroffene hier schneller Hilfe erhalten. Fünftens. Für unsere Solo-Selbstständigen und Kleinunternehmer haben wir ein Soforthilfeprogramm von bis zu 50 Milliarden Euro aufgelegt. Dieses Programm wird von den Bundesländern ergänzt und aufgestockt, beispielsweise sind in Bayern schon über 1,5 Milliarden Euro ausgezahlt worden. Sechstens. Für die Unternehmen haben wir das Angebot an KfW-Krediten verbessert und unter anderem auch einen KfW-Schnellkredit eingeführt, der für Unternehmen mit mindestens zehn Mitarbeitern gilt und bei dem die KfW zu 100 Prozent das Kreditausfallrisiko übernimmt. Mit all diesen Maßnahmen, die ich jetzt aufgezählt habe, soll – also von wegen Untätigkeit – verhindert werden, dass gewerbliche und private Mieter ihre Miete nicht mehr zahlen können. Kommt es dennoch zu finanziellen Engpässen, so haben wir gleich zu Beginn der Krise gesetzlich geregelt, dass die Miete für die Monate April, Mai und Juni gestundet werden kann und die Rückzahlung erst bis Juni 2022 erfolgen muss. Ebenso können die Zahlungen für Strom und Wasser gestundet werden sowie Darlehenszahlungen, was vor allem den privaten Kleinvermietern zugutekommt. Diese Regelungen können gegebenenfalls bis September verlängert werden. Das ist bereits in diesem Gesetz angelegt.  Deshalb brauchen wir kein Sicher-Wohnen-Programm, wie es die Grünen vorschlagen, oder Fonds, wie sie die Linken vorschlagen. Aktuelle Umfragen geben uns natürlich auch recht; denn tatsächlich ist es bisher kaum zu Mietausfällen gekommen. So spricht der Bundesverband deutscher Wohnungs- und Immobilienunternehmen von Ausfällen in Höhe von circa 1 Prozent. Wir haben schon zu Beginn der Krise schnell reagiert, sodass es nicht nötig ist, Doppelstrukturen einzurichten. Aber Sie können sich sicher sein: Wir sind wachsam; denn wir haben ein Interesse daran, dass wir gut aus dieser Krise wieder herauskommen. In diesem Sinne freue ich mich auf die Beratungen. Vielen herzlichen Dank.  

\noindent\textbf{Comment:}
\begin{itemize}
    \setlength\itemsep{-3pt}
    \item (Beifall bei der AfD)
    \setlength\itemsep{-3pt}
    \item (Beifall bei der CDU/CSU)
    \setlength\itemsep{-3pt}
    \item (Beifall bei der CDU/CSU sowie bei Abgeordneten der SPD)
    \setlength\itemsep{-3pt}
    \item (Heiterkeit bei Abgeordneten der LINKEN – Matthias W. Birkwald [DIE LINKE]: Wir auch!)
    \setlength\itemsep{-3pt}
    \item (Beifall des Abg. Klaus Mindrup [SPD])
    \setlength\itemsep{-3pt}
    \item (Beifall der Abg. Klaus Mindrup [SPD] und Ulli Nissen [SPD])
    \setlength\itemsep{-3pt}
    \item (Beifall bei der CDU/CSU und der FDP sowie des Abg. Klaus Mindrup [SPD])
\end{itemize}
\subsection{Hemmelgarn}
\noindent\textbf{Texts:} Sehr geehrte Frau Präsidentin! Sehr geehrte Kolleginnen und Kollegen! Seltsame Zeiten führen zu seltsamen Ideen, und seltsame Ideen führen zu seltsamen Anträgen. Der Antrag der Grünen fordert überraschend eine Lösung, die auch den privaten Kleinvermietern nutzen würde. So viel ökonomischen Sachverstand hätte man von den Grünen nicht erwartet. Wahrscheinlich hat die Parteibasis hier Druck gemacht.  Lieber Herr Kühn, über einige dieser Punkte würden wir sogar mit Ihnen reden. Die Linken hingegen wollen allen Ernstes die Nettokaltmiete um 30 Prozent absenken, wenn der Mieter coronabedingt von erheblichen Einkommensverlusten betroffen ist. Zum Ausgleich soll es auch zugunsten von Privatvermietern ein Moratorium für Kreditverpflichtungen geben. Die eingetretenen Schäden würden damit von den Vermietern in den Bankensektor verlagert. Offensichtlich hassen die Linken die Banken noch mehr als den privaten Vermieter.  Wir haben schon häufig darauf hingewiesen: Ein Großteil der privaten Vermieter – das sind mehr als 60 Prozent der Vermieter – macht mit seinen Wohnungen keinen oder nur einen geringen Gewinn. Die geforderte Absenkung der Nettokaltmiete würde für viele den sicheren Ruin bedeuten. Aber die Linken haben auch dafür eine Lösung: Es soll ein Härtefallfonds eingerichtet werden, der auch Privatvermietern Hilfe gewährt, wenn sie aufgrund der Coronakrise in Not geraten.  Man will den Vermieter erst ruinieren, um ihm dann die huldvolle Hilfe des Staates angedeihen zu lassen. Das Ganze wird abgerundet durch bekannte Maßnahmen linker Ideologen wie einem Moratorium für Mieterhöhungen, dem Verbot von Zwangsräumungen und schließlich dem Verbot jeder Kündigung von Mietern. Fairerweise muss man sagen, dass die Linken das meiste davon auf die Dauer der Coronapandemie beschränken wollen. Aber keine Angst, liebe Linke! Das reicht locker aus, um die meisten Kleinvermieter nachhaltig zu schädigen. Man kann den Linken an dieser Stelle kaum einen Vorwurf machen. Ihr Antrag ist nur ein wenig chaotischer als das Krisenmanagement der Bundesregierung. Am Ende fragt man sich: Wozu das Ganze überhaupt? Um eine ökonomische Krise zu bekämpfen, die in erster Linie durch den Shutdown dieser Bundesregierung hervorgerufen wurde?  Wie gesagt: Wir leben in seltsamen Zeiten. Kritiker der Bundesregierung in dieser Krise werden wahlweise als Spinner, Verschwörungstheoretiker oder Coronaleugner diffamiert, ungeachtet der Tatsache, dass Schweden einen anderen, besseren Weg gegangen ist.  Aber um es klar zu sagen: Niemand leugnet die Existenz des Coronavirus.  Der Vorwurf, den wir der Bundesregierung machen, ist, dass sie mit völlig überzogenen Maßnahmen diese Pandemie zu eigenen Zwecken ausnutzt, um ihre katastrophale Politik der letzten Jahre zu verdecken.  Euro-Krise, Migrationskrise, der Wahnsinn der Energiewende: alles schön überdeckt von Corona. Nebenbei nutzt man das Ganze, um die Bevölkerung auf Linie zu bringen und die Autorität des Staates oder, besser gesagt, die eigene Autorität zu stärken. Hier noch eine weitere Kuriosität in dieser Krise: Während der Großteil der Bundesbürger auch jetzt noch nicht an Nord- und Ostsee reisen darf, können Asylbewerber und Flüchtlinge aller Art  immer noch ohne Quarantäne in das Bundesgebiet einreisen. Minister Seehofer sieht hier keinen Handlungsbedarf. Hier stellt sich wirklich die Frage: Innerdeutsche Grenzen können wir sichern, unsere Außengrenzen nicht?  Gleichzeitig wird ein Beamter des Innenministeriums von seinen Pflichten entbunden, weil er den hektisch herbeigeführten Shutdown als Fehlalarm bezeichnete.  Zu Recht wies er darauf hin, dass adäquate Instrumente zur Gefahrenanalyse und ‑bewertung fehlten und auch heute noch fehlen und damit kaum Chancen bestehen, sachlich richtige Entscheidungen zu treffen. Sie müssen mal zuhören, Herr Möring.  Da kann ich insgesamt nur sagen: ein Innenminister in Topform. Meine Damen und Herren, es ist richtig, Lösungen zu finden, um die Folgen der Coronamaßnahmen abzumildern. Sehr viel wichtiger ist es allerdings, die immer noch bestehenden, unnötigen Beschränkungen sofort aufzugeben.  Den übrigen Oppositionsparteien in diesem Parlament empfehle ich, sich sehr genau zu überlegen, ob sie weiter die Statisten in dieser Regierungsinszenierung sein wollen. Vielen Dank.  

\noindent\textbf{Comment:}
\begin{itemize}
    \setlength\itemsep{-3pt}
    \item (Beifall bei der SPD)
    \setlength\itemsep{-3pt}
    \item (Beifall bei der AfD)
    \setlength\itemsep{-3pt}
    \item (Matthias W. Birkwald [DIE LINKE]: Sie sollten jetzt aufhören! Reicht! Setzen! Ist gut jetzt! – Zurufe vom BÜNDNIS 90/DIE GRÜNEN)
    \setlength\itemsep{-3pt}
    \item (Karsten Möring [CDU/CSU]: Völlig falsch, Herr Hemmelgarn! Weil er den Briefkopf verwandt hat!)
    \setlength\itemsep{-3pt}
    \item (Zuruf von der SPD: Na, immerhin!)
    \setlength\itemsep{-3pt}
    \item (Beifall bei der AfD – Ulli Nissen [SPD]: Da redet gerade der mit dem größten Sachverstand! – Steffi Lemke [BÜNDNIS 90/DIE GRÜNEN]: Sagen Sie doch mal was zu Ihrem politischen Programm!)
    \setlength\itemsep{-3pt}
    \item (Beifall bei der AfD – Ulli Nissen [SPD]: Unsinn! Was für ein Glück, dass diese ätzende Rede vorbei ist!)
    \setlength\itemsep{-3pt}
    \item (Beifall bei Abgeordneten der AfD – Matthias W. Birkwald [DIE LINKE]: Wir hassen weder noch! Wir kritisieren höchstens! Hass ist Ihre Abteilung! – Caren Lay [DIE LINKE]: Mit Hass kennen Sie sich doch am besten aus! -Daniela Wagner [BÜNDNIS 90/DIE GRÜNEN]: Die Einzigen, die hassen, sind doch Sie!)
    \setlength\itemsep{-3pt}
    \item (Beifall bei Abgeordneten der AfD)
    \setlength\itemsep{-3pt}
    \item (Ulli Nissen [SPD]: Zuhören ist schwierig bei Ihnen!)
    \setlength\itemsep{-3pt}
    \item (Caren Lay [DIE LINKE]: Ist doch gut!)
    \setlength\itemsep{-3pt}
    \item (Ulli Nissen [SPD]: Die Todesrate ist da viel höher als bei uns!)
    \setlength\itemsep{-3pt}
    \item (Beifall bei der AfD – Zuruf der Abg. Ulli Nissen [SPD])
\end{itemize}
\subsection{Mindrup}
\noindent\textbf{Texts:} Sehr geehrte Frau Präsidentin! Liebe Kolleginnen und Kollegen! Herr Hemmelgarn, normalerweise sind Ihre Helden ja Trump, Putin und Boris Johnson.  Diese Großmäuler und Versager in der Coronakrise haben Sie ja wohl bewusst nicht erwähnt.  Auch und gerade in der Pandemie haben Mieterinnen und Mieter das Recht, gesichert zu wohnen. Deswegen ist es gut, dass wir hier heute diesen Antrag debattieren. Und es ist klar – manchmal wundere ich mich, wie schnell die Amnesie einsetzt –, dass hier die Bundesregierung und auch der Bundestag unglaublich schnell gehandelt haben.  Wir haben den Schutz vor Kündigungen wegen Mietschulden bis zum 30. Juni eingeführt, mit Verlängerungsoption; Kollegin Zeulner hat schon darauf hingewiesen. Wir haben den Zugang zu Wohngeld vereinfacht. Wir haben die Leistungen zur Sicherung des Lebensunterhalts verbessert. Wir haben die Stundung von Verbraucherkrediten ermöglicht. Wir haben das Kurzarbeitergeld deutlich verbessert und werden es weiter verbessern.  Wir haben einen Schutzschirm für Solo-Selbstständige und KMU eingeführt. Zuschüsse für Unternehmen mit bis zu zehn Mitarbeitern leisten wir als Bund, die Länder leisten Zuschüsse für Unternehmen mit bis zu 250 Mitarbeitern. Wir haben ein umfassendes KfW-Schnellkreditprogramm auf den Weg gebracht. Wir haben Hilfen für Großunternehmen auf den Weg gebracht; denn auch sie beschäftigen Menschen, die Mieterinnen und Mieter sind. Und wir haben die erleichterte Möglichkeit von Steuerstundungen auf den Weg gebracht. Was kaum beachtet wird und auch wichtig für Vermieter ist: Wir haben die Bankenregularien verändert, sodass Tilgungsaussetzungen heute leichter möglich sind, ohne dass die Schuldner anschließend in Schwierigkeiten kommen. Das hilft sowohl den Hoteleigentümern als auch den Vermietern als auch den Taxiunternehmen. Wir sind also schon sehr stark tätig gewesen.  Aber, liebe Kolleginnen und Kollegen, wie es immer so ist: Man muss natürlich seine Maßnahmen ständig evaluieren, erst recht in einer solchen Krise wie dieser Pandemie. Deswegen sind wir ja auch in der neuen Phase der Pandemiebekämpfung, was die Maßnahmen angeht, wie in der Strategie „Hammer and Dance“ beschrieben. Der große Hammer, die vielen kleinen Hämmer kommen jetzt nicht mehr. Wir müssen das wirtschaftliche Leben ja wieder ankurbeln. Aber wir müssen natürlich auch schauen: Was passiert jetzt in unserem Land? Welche Maßnahmen wirken, welche nicht? Und eins muss man klar sagen: Das Virus ist noch da, und wir haben keinen Impfstoff, wir haben kein wirksames Medikament. Wir sind also noch mitten in der Krise, liebe Kolleginnen und Kollegen.  An dieser Stelle muss ich Kollegin Lay recht geben: Wir haben zwar unterschiedliche Auffassungen zu Zwangsräumungen außerhalb einer Pandemie, aber in einer Pandemie sind Zwangsräumungen schon aus Gründen des Gesundheitsschutzes zu vermeiden.  Liebe Kolleginnen und Kollegen, die Grünen fordern ein KfW-Sonderprogramm. Das überzeugt mich nicht. Erstens wäre das wieder ein Kredit, und Kredite unterliegen den entsprechenden gesetzlichen Regelungen. Und ich glaube auch nicht daran, dass man heute die Tilgungsverzichte oder den Kreditverzicht am Ende gleich vereinbaren kann. Außerdem sollten wir kein Programm machen, das die KfW lahmlegt. Wenn es ein solches Programm geben sollte, dann wäre es eher eines für die Förderbanken der Länder, weil sie für die eher kleineren Einheiten zuständig sind. Aber der Grundansatz – nach dem Motto „Wir müssen etwas für Vermieter und Mieter gleichermaßen tun“ – ist schon richtig. Ich schlage etwas anderes vor. Die Coronakrise kommt an den Wohnungs- und Gewerbeimmobilienmärkten verspätet an; das wissen wir alle. Das heißt, wir müssen in den nächsten Wochen beobachten: Funktionieren die Maßnahmen, die wir gemeinsam ergriffen haben – die hat ja nicht nur die Koalition beschlossen, sondern bei vielen hat auch die Opposition mitgestimmt –, und müssen wir nachschärfen? Eine Frage der Nachschärfung ist etwa die Verlängerung des Kündigungsschutzes. Ich habe aber auch zur Kenntnis genommen, dass sowohl die Wohnungswirtschaft als auch der Deutsche Mieterbund über den Sicher-Wohnen-Fonds reden. Im Augenblick gibt es noch keine ausreichenden Indizien, dass wir so etwas brauchen. Aber wir müssen genau nachschauen: Ist das nicht doch notwendig? An einer Stelle – das ist klar – gibt es große Probleme: bei den Gewerbemietern. Die Gewerbemieter sind oftmals in extrem schwierigen Lagen. Eines ist klar: Der große Gewinner der Pandemiekrise ist der Internethandel. Das geht natürlich zulasten von vielen kleinen Geschäften in unseren Innenstädten. Die Lage der Kneipen ist ja ebenfalls schon beleuchtet worden. Viele Vereine sind in Problemen. Und wir werden daran arbeiten, denen zu helfen. Aber wir müssen auch die Mietsituation betrachten. Es geht um lebendige, gemischte Städte. Liebe Kolleginnen und Kollegen, lassen Sie uns in zwei bis vier Wochen schauen, ob wir da nicht noch einmal nachlegen müssen; denn das wäre absolut vernünftig.  Der Antrag der Linken – um noch dazu zu kommen – sagt zu Recht, dass wir etwas gegen Wohnungslosigkeit tun müssen. Gegen Wohnungslosigkeit zu kämpfen, ist vor allen Dingen eine Aufgabe der Städte. Viele Städte haben bereits gehandelt. Hamburg ist vorbildlich. Frankfurt macht sehr viel, wie mir die Kollegin Nissen gesagt hat. In München gab es ein sehr spannendes Beispiel: die Überbauung des Dantebades. Dort hat man einen Holzhybridbau auf einen Parkplatz gesetzt – innerhalb eines Jahres, also extrem schnell, ökologisch gebaut – und damit wirklich ein gemischtes Wohnquartier geschaffen. Aber selbst eine Stadt wie München hat bei den sinkenden Einnahmen aus Steuern und aus dem öffentlichen Nahverkehr große Probleme, so etwas langfristig zu finanzieren. Deswegen ist es notwendig, dass wir den Kommunen unter die Arme greifen. Denn ansonsten droht der Teufelskreislauf: Sie haben kein Geld, um zu investieren, die Transferausgaben steigen, und die Situation wird immer schlimmer, so wie wir das heute schon aus den überschuldeten Kommunen kennen. Deswegen ist es so wichtig – die SPD fordert das energisch –, dass wir einen kommunalen Schutzschirm schaffen; denn wir müssen die Kommunen als die Basis unserer Demokratie unbedingt am Leben erhalten.  Wenn wir dann in einem solchen Schutzschirm Projekte zur Bekämpfung der Wohnungslosigkeit unterbringen können wie die Bebauung des Dantebades, dann ist das hilfreich; denn so etwas ist billiger als die sogenannten Läuse-Pensionen. Es ist besser für den Gesundheitsschutz, es ist für den Klimaschutz gut, wenn wir mit Holz bauen, und es fördert die lokale Wirtschaft. Denn wir wollen ja alle, dass unser Land wieder vorankommt. Danke für die Aufmerksamkeit. Auf eine gute Beratung in den Fachausschüssen!  

\noindent\textbf{Comment:}
\begin{itemize}
    \setlength\itemsep{-3pt}
    \item (Daniela Wagner [BÜNDNIS 90/DIE GRÜNEN]: Genau! – Christian Kühn [Tübingen] [BÜNDNIS 90/DIE GRÜNEN]: Sie haben Assad vergessen! Assad auch noch!)
    \setlength\itemsep{-3pt}
    \item (Beifall bei der SPD sowie bei Abgeordneten der CDU/CSU)
    \setlength\itemsep{-3pt}
    \item (Beifall bei der SPD)
    \setlength\itemsep{-3pt}
    \item (Beifall bei der FDP)
    \setlength\itemsep{-3pt}
    \item (Beifall bei der SPD, der LINKEN und dem BÜNDNIS 90/DIE GRÜNEN sowie bei Abgeordneten der CDU/CSU)
    \setlength\itemsep{-3pt}
    \item (Ulli Nissen [SPD]: Wir sind echt super! Genau!)
    \setlength\itemsep{-3pt}
    \item (Beifall bei der SPD und der Abg. Sabine Leidig [DIE LINKE])
    \setlength\itemsep{-3pt}
    \item (Beifall bei Abgeordneten der SPD)
    \setlength\itemsep{-3pt}
    \item (Beifall bei Abgeordneten der SPD und der Abg. Emmi Zeulner [CDU/CSU])
    \setlength\itemsep{-3pt}
    \item (Beifall bei Abgeordneten der SPD und der CDU/CSU – Lachen bei Abgeordneten der AfD)
    \setlength\itemsep{-3pt}
    \item (Beifall bei Abgeordneten der SPD – Ulli Nissen [SPD]: Leider ja!)
\end{itemize}
\subsection{Reinhold}
\noindent\textbf{Texts:} Zuerst einmal schönen Dank für das ordentlich desinfizierte Rednerpult und für die vielen fleißigen Hände, die es uns ermöglichen, hier überhaupt Sitzungen durchzuführen.  Ich glaube, das hat es mal verdient. Denn heute ist das Parlament viel wichtiger denn je. Sehr geehrte Präsidentin! Meine sehr geehrten Damen und Herren! Frau Zeulner, auch wenn der GdW vielleicht noch von 1 Prozent redet: Haus \& Grund redet schon von 6 bis 7 Prozent, und wir sind gerade erst am Anfang dieser Krise. Wie sehr das auf den Mietmarkt durchschlägt, wie sehr das bei Vermietern, bei Mieterinnen und Mietern ankommt, das werden wir in nächster Zeit noch sehen. Das wird uns noch mehr beschäftigen, als uns heute lieb ist. Deshalb ist es richtig und wichtig, dass wir uns heute damit befassen. Wir kommen zum Antrag der Linken und stellen fest – es waren einige Sachen dabei, die mir schon im Wortlaut sehr gefallen haben –: Das Recht auf gesunden und angemessenen Wohnraum wird also infrage gestellt. Wer stellt denn hier Recht infrage, frage ich mich. Ganz im Gegenteil: Ich habe eher mitbekommen, dass die Mieterinnen und Mieter in dieser Krise mehr Rechte bekommen haben. Sie können die Miete aussetzen. Also, Recht stellt bestimmt kein Mensch infrage. Aber solch ein Geist geht von vorne bis hinten durch Ihren Antrag. Ein Moratorium zum Beispiel kann ich auf zwei Arten definieren: Es ist entweder eine Übereinkunft zwischen Gläubiger und Schuldner, den Schuldendienst vorübergehend zu unterlassen, oder eben ein gesetzlich angeordneter Aufschub. Für diesen entscheiden Sie sich, für die Keule. Nichts anderes als Misstrauen ist das denen gegenüber, die wir jetzt in der Krise brauchen.  Sie versuchen unter Punkt 5, uns still und heimlich einen kleinen Mietendeckel unterzuschieben, und sagen: 30 Prozent weniger Miete, und nach der Krise behalten wir das gleich bei und senken auf die ortsübliche Vergleichsmiete ab. – Das ist durchschaubar. Hier haben Sie „Corona“ vor Anträge gesetzt, die Sie schon oft eingereicht haben und mit denen Sie immer wieder gescheitert sind.  Sie sagen, die Krisenkosten würden nur den Mietern aufgebürdet werden – tatsächlich –, und wollen, dass die Vermieter endlich beteiligt sind. Im nächsten Punkt nehmen Sie es zurück und sagen, welche Vermieter davon ausgeschlossen sind: Das sind die Privaten und die genossenschaftlichen und kommunalen Wohnungsgesellschaften. Da sieht man Ihren Klassenfeind, der wieder durchkommt: Das sind nämlich die lediglich 13 Prozent kapitalgebundenen Vermieter. Auf die wollen Sie sich stürzen. Auf die wollen Sie alles abwälzen. Das wird in dieser Krise nicht reichen. Das sind keine Lösungen, die wir heutzutage brauchen; tut mir leid.  Das einzig Gute ist die dezentrale Unterbringung; da bin ich ganz bei Ihnen. Ich finde auch gut, dass Sie fehlende Schutzräume für Schutzbedürftige ansprechen, nämlich vor Gewalt zu schützende Personen. Aber das trifft viele Familienmitglieder und nicht nur einzelne. Ich hätte mir gewünscht, Sie hätten dabei jeden aufgezählt; das wäre gut gewesen. Ich komme zum Antrag der Grünen. Richtig erkannt: Wir dürfen keinen Dominostein zu Fall bringen. Wenn wir anfangen, die Mieter davon abzuhalten, Miete zu zahlen, kommen die Vermieter, kommen die Banken. Dann haben wir eine nächste Krise, die wir nicht brauchen. Und in Coronazeiten noch eine Bankenkrise wäre das Schlimmste, was uns passieren kann. Deshalb geht der Antrag in die richtige Richtung. Die KfW davor zu bewahren, noch eine Million Anträge zusätzlich zu kriegen, ist, glaube ich, richtig und wichtig. Deshalb müssen wir im Ausschuss darüber reden, wie man das behandeln kann. Aber richtig erkannt ist: Wir müssen dafür sorgen, dass überhaupt kein Dominostein fällt. Das ist das Einzige, was zählt. Deshalb schwebt uns als FDP auch immer wieder vor, sowohl das zu nutzen, was diesen Sozialstaat stark gemacht hat, als auch die Möglichkeiten, die wir haben. Wir haben zum Beispiel recht schnell gefordert, das Wohngeld anzuheben, also dass es nicht einfach nur digital und schnell beantragt werden kann, sondern dass zusätzliches Wohngeld ausgeschüttet wird, damit weiter Miete gezahlt werden kann.  Das sind richtige und wichtige Maßnahmen. Denn so kann es gelingen. Das Sonderwohngeld könnte dafür sorgen, dass Miete weiter gezahlt wird und Vermieter erst gar nicht in Schwierigkeiten kommen. Solche Lösungen braucht es heute, und das wäre der richtige Ansatz. Wir könnten zum Beispiel genauso Vermieter mit einem Obolus bei der AfA in die Lage versetzen, mehr abzuschreiben, wenn sie freiwillig auf Miete verzichten.  Was machen wir zurzeit aber? Wir haben freiwilligen Mietverzicht bei den Vermietern. Es gibt nämlich viele Vermieter, die sagen: Mein Gott, du bist in Schwierigkeiten; ich verzichte auf die Miete. – Dagegen spricht zurzeit, dass man gar nicht weniger als 66 Prozent der ortsüblichen Vergleichsmiete nehmen darf. Man darf die Werbekosten gar nicht ansetzen. Wir bestrafen noch die Guten in diesem Land. Ich hoffe, da gibt es schnell ein BMF-Schreiben, das die ganze Sache verhindert. Dann wären wir damit durch. Ich habe das Rednerpult noch nicht schmutzig.  Ich danke recht herzlich für die Aufmerksamkeit und freue mich auf die Beratungen im Ausschuss. Recht herzlichen Dank.  

\noindent\textbf{Comment:}
\begin{itemize}
    \setlength\itemsep{-3pt}
    \item (Beifall bei der FDP sowie bei Abgeordneten der CDU/CSU, der SPD, der AfD, der LINKEN und des BÜNDNISSES 90/DIE GRÜNEN)
    \setlength\itemsep{-3pt}
    \item (Beifall bei der FDP)
    \setlength\itemsep{-3pt}
    \item (Beifall beim BÜNDNIS 90/DIE GRÜNEN)
    \setlength\itemsep{-3pt}
    \item (Heiterkeit bei der FDP sowie bei Abgeordneten der CDU/CSU und des BÜNDNISSES 90/DIE GRÜNEN)
    \setlength\itemsep{-3pt}
    \item (Widerspruch bei der LINKEN)
\end{itemize}
\subsection{Kühn}
\noindent\textbf{Texts:} Sehr geehrte Frau Präsidentin! Werte Kolleginnen und Kollegen! Ausgelöst durch die Coronapandemie erleben wir gerade die schwerste globale ökonomische Krise seit dem Zweiten Weltkrieg. Diese Krise ist, anders als die AfD vorher behauptet hat, nicht dadurch ausgelöst, dass wir in Deutschland den Shutdown haben, sondern dadurch, dass dieses Coronavirus weltweit die Ökonomien bedroht, die Lieferketten unterbrochen sind und deshalb die globale Wirtschaft nicht funktioniert. Das ist die Wahrheit. Diese Krise stellt die Ölkrisen von 1973 und 1979, die New-Economy-Blase, die 2000 geplatzt ist, und die letzte Finanzkrise von 2008 und 2009 in den Schatten. Wer glaubt, dass die deutschen Wohnungsmärkte diesen Druck nicht abkriegen werden, der lebt in einem Märchenschloss, der hat nichts mit der Realität zu tun. Diese Krise wird verzögert auf den Wohnungsmärkten ankommen. Sie ist noch nicht da; aber sie wird diesen Sommer im Juni, Juli, August oder September wirklich einschlagen, und zwar deswegen, weil dann die Menschen die wenigen Ersparnisse, die sie noch haben, aufgebraucht haben werden und dann auch die Mieten nicht mehr zahlen können. Ich kann deswegen, Frau Zeulner, den Lobesmodus in Bezug auf Ihre Maßnahmen, in den Sie hier verfallen sind, irgendwie überhaupt nicht nachvollziehen. Denn eines ist ganz klar: Viele Menschen werden ihre Miete nicht mehr zahlen können. Und klar muss sein: In der Coronakrise darf es nicht dazu kommen, dass jemand nicht mehr in der Lage ist, seine Miete zu zahlen, und seine Wohnung verliert. Das muss das Ziel sein.  Aber es muss auch das Ziel sein, dass niemand danach infolge der Coronakrise unverschuldet seine Wohnung verliert. Wohnen ist ein soziales Gut, und deswegen haben wir Grünen – das haben Sie vorhin ja auch lobend erwähnt – ein KfW-Programm vorgeschlagen, um Menschen zu ermöglichen, weiterhin ihre Miete zu zahlen. Das hilft den Mieterinnen und Mietern, und das wird den Vermietern in Deutschland helfen. Wir machen das deswegen, weil wir sehen, dass Menschen in prekären Situationen sind. Ich komme familiär aus der Gastronomie. Mein Vater ist Hotelkaufmann gewesen. Der Gastronomie geht es richtig dreckig. Wenn man sich jetzt anschaut, wie es beispielsweise dem Hotelfachmann in Deutschland im Augenblick geht, sieht man: Es ist schwierig. Er verdient im Durchschnitt 2 000 Euro. Davon 60 Prozent Kurzarbeitergelt: Das sind netto 860 Euro.  Wenn wir das jetzt erhöhen, sind wir bei 1 150 Euro. Davon wird der Hotelfachmann in den Ballungsräumen wie Stuttgart, München oder Berlin niemals dauerhaft die Miete zahlen können. Er wird auch nichts sparen können, um die Mietschulden in Zukunft zu zahlen. Diesen Menschen muss geholfen werden, und da reicht das Wohngeld eben nicht aus. Deswegen braucht es ein KfW-Programm, das die Mieten absichert, damit die Menschen nicht in zwei Jahren vor der Kündigung ihrer Wohnung stehen.  Ich kann nicht verstehen, dass Sie als Regierung da auf irgendein Wunder warten, das vom Himmel kommt.  Wir müssen da jetzt handeln. Wir müssen jetzt den Menschen auf den Wohnungsmärkten Sicherheit geben.  Wir müssen klarmachen, dass wir auch den kleinen Vermietern und ihren Mietern helfen und nicht nur den Konzernen. Denn die Konzerne können im Augenblick Liquiditätshilfen beantragen. Die Vermieter können es nicht. Deswegen ist es dringend nötig, dieses KfW-Programm aufzulegen. Das bringt Sicherheit. Das ist sozial, das ist gerecht, und das ist am Ende auch Mieterschutz. Danke schön.  

\noindent\textbf{Comment:}
\begin{itemize}
    \setlength\itemsep{-3pt}
    \item (Emmi Zeulner [CDU/CSU]: Machen wir doch gar nicht!)
    \setlength\itemsep{-3pt}
    \item (Beifall beim BÜNDNIS 90/DIE GRÜNEN)
    \setlength\itemsep{-3pt}
    \item (Emmi Zeulner [CDU/CSU]: Machen wir doch! Genau das machen wir: Wir geben ihnen Sicherheit!)
    \setlength\itemsep{-3pt}
    \item (Beifall bei der CDU/CSU)
    \setlength\itemsep{-3pt}
    \item (Emmi Zeulner [CDU/CSU]: Deswegen erhöhen wir das Kurzarbeitergeld!)
\end{itemize}
\subsection{Möring}
\noindent\textbf{Texts:} Frau Präsidentin! Liebe Kolleginnen und Kollegen! Dem Kollegen Kühn stimme ich in Bezug auf das, was er am Anfang gesagt hat, zu. Wir haben in der Tat eine extrem schwere Krise. Eigentlich ist es bewundernswert, mit welcher Disziplin, mit welchem Verantwortungsbewusstsein und wie solidarisch ein großer Teil der Bevölkerung mit dieser Krise umgeht. Ohne dieses Verhalten wären wir heute nicht bei den Zahlen, bei denen wir sind; das muss man mal ausdrücklich feststellen. Dafür gebührt allen, die sich verantwortungsvoll verhalten haben, großer Dank.  Die große Herausforderung, vor der wir stehen, können wir meistern, weil wir dafür gut gerüstet sind. Wir sind unter anderem deshalb gut gerüstet, weil wir über Jahre hinweg eine sehr solide Haushaltsführung gemacht haben, die es uns jetzt erlaubt, mit großen Beträgen Hilfen zu leisten, in der ganzen Bandbreite derjenigen, die es nötig haben. Verglichen mit anderen Ländern mobilisieren wir Beträge in einer ganz anderen Größenordnung. Es kommt natürlich darauf an, sie auch gezielt und richtig einzusetzen. Wir setzen sie ein zur Verstärkung des Gesundheitssystems, wir setzen sie ein für Unternehmen, die Liquiditätsprobleme haben oder denen ein Teil ihres Marktes wegbricht, wir setzen sie ein für Selbstständige und für abhängig Beschäftigte – auf ganz verschiedene Weise und mit ganz verschiedenen Methoden. Die heutige Debatte freut mich, nicht wegen der Anträge, sondern weil sie uns Gelegenheit gibt, einmal klar darzustellen, was wir hier inzwischen erreicht haben. Daran waren Sie zum Teil ja selber beteiligt; es ist gut, dass wir das in großer Übereinstimmung gemacht haben.  Aber diese Debatte wird auch zeigen, dass Ihre Anträge, die Sie heute stellen, überflüssig sind. Wenn wir damit heute anfangen würden, dann wären wir zu spät. Apropos spät: Als die Anträge gestern sehr spät eingingen, hatte ich etwas Schwierigkeiten, sie zu lesen; das habe ich erst heute Morgen geschafft, aber ich hatte ja zum Glück schon Ihre Pressemeldungen und die Berichterstattung in der Zeitung gelesen.  – Ja, das war ein netter Service, den Sie da geleistet haben.  – Ja, ist in Ordnung. Das können wir ja machen. Die Zusammenfassung ist dann ja schon mal hilfreich. Spaß beiseite! Liebe Grüne, als ich die Überschrift Ihres Antrags gelesen habe, war ich etwas irritiert. In Berlin haben die Grünen Enteignungsfantasien,  und hier in Berlin liegt uns heute ein Antrag zur Eigentumssicherung vor. Liebe Grüne, Volkspartei zu sein, verlangt zwar eine gewisse Bandbreite,  aber ich glaube, diese Spreizung ist nicht geeignet.  Sie sollten sich schon entscheiden, worauf Sie Ihren Schwerpunkt legen wollen. Volkspartei geht anders!  – Ja, wir werden mal sehen. Dann schreiben Sie: „sichern in Zeiten der Krise“. Lieber Herr Kollege Kühn, wir sichern vor der Krise, in der Krise und nach der Krise. Unsere politische Grundhaltung ist von demselben Motiv getrieben. Natürlich wollen wir Wohnraum schaffen, Wohnraum sichern. Und wenn es dafür in der Krise andere Methoden braucht als davor oder danach, dann machen wir das. Das ist doch der entscheidende Punkt. Wir kümmern uns darum; denn sicher Wohnen ist für uns ein Thema, das immer gilt, und nicht nur in der Krise.  Sie schlagen jetzt ein zinsloses KfW-Darlehen vor. Gleichzeitig möchten Sie das unkompliziert haben: unkomplizierte Antragstellung und eine schnelle Bewilligung. Und dann machen Sie aus der KfW ein echtes Bürokratiemonster. Was sollen die leisten? Die sollen die monatliche Mietzahlung für Betroffene übernehmen, die sollen für Vermieter die monatlichen Hypothekenzinsen und Tilgungsraten leisten, und das für eine große Zahl von Betroffenen, die Sie unterstellen; Sie sagen ja, dass das ein sehr großes Problem ist. – So funktioniert das nicht. Aus der KfW ein Bürokratiemonster zu machen, und das in einer Zeit, in der sie sowieso genug Aufgaben hat, das sollten wir uns ersparen.  – Wieso? Das steht in Ihrem Antrag.  – Ah, okay, gut. Aber ich werde, glaube ich, nicht in die Situation kommen. Wir werden darüber im Ausschuss noch mal diskutieren; aber ich glaube nicht, dass wir die KfW in diese Verlegenheit bringen werden. Wir setzen auf andere Lösungen, wir setzen auch auf partnerschaftliche Lösungen. Wenn Mieten nicht gezahlt werden können, dann gibt es mehrere Möglichkeiten. Wir haben die Möglichkeit, Wohngeld zu beziehen, unter Coronabedingungen erleichtert, auch rückwirkend. Wir haben das deutlich erleichtert, um Ausfälle der Mieten, die darauf zurückzuführen sind, auf diese Weise aufzufangen. Das ist ein wichtiger Punkt. Wir setzen aber auch darauf, dass sich ein Mieter, der nicht zahlen kann, mit seinem Vermieter darüber unterhält, wie man diese Situation beseitigen kann. Ich kenne aus meinem Kölner Umfeld eine ganze Reihe solcher Beispiele, nicht nur bei Privatvermietern, sondern beispielsweise auch bei Wohnungsunternehmen. Ja, danke. Noch 22 Sekunden. Ich wollte eigentlich früher aufhören.  Ach, Entschuldigung. Ach so. Herr Gysi hat doch auch immer einen Zuschlag gekriegt.  Noch ein Satz. – Wir gehen sorgfältig mit dem Geld um, zielgerichtet, und wenn es nötig ist, nachzusteuern, dann tun wir das auch. Das muss nicht heute sein. Wir beobachten die Situation und machen es dann, wenn es so weit ist. Vielen Dank für Ihre Geduld – auch für Ihre, Frau Präsidentin. 

\noindent\textbf{Comment:}
\begin{itemize}
    \setlength\itemsep{-3pt}
    \item (Christian Kühn [Tübingen] [BÜNDNIS 90/DIE GRÜNEN]: Rufen Sie einfach mal die KfW an! Dann kriegen Sie die Information, wie man so was machen kann!)
    \setlength\itemsep{-3pt}
    \item (Heiterkeit bei der CDU/CSU – Matthias W. Birkwald [DIE LINKE]: Da kennt ihr euch ja aus!)
    \setlength\itemsep{-3pt}
    \item (Beifall der Abg. Ulli Nissen [SPD])
    \setlength\itemsep{-3pt}
    \item (Beifall bei Abgeordneten der CDU/CSU und der SPD – Christian Kühn [Tübingen] [BÜNDNIS 90/DIE GRÜNEN]: Das sind doch Ammenmärchen!)
    \setlength\itemsep{-3pt}
    \item (Matthias W. Birkwald [DIE LINKE]: Der war ja auch Fraktionsvorsitzender!)
    \setlength\itemsep{-3pt}
    \item (Beifall bei der CDU/CSU und der FDP)
    \setlength\itemsep{-3pt}
    \item (Christian Kühn [Tübingen] [BÜNDNIS 90/DIE GRÜNEN]: Sie sind nicht der Erste, der diesen Witz gebracht hat!)
    \setlength\itemsep{-3pt}
    \item (Beifall bei der CDU/CSU)
    \setlength\itemsep{-3pt}
    \item (Jürgen Trittin [BÜNDNIS 90/DIE GRÜNEN]: Wieso Fantasien?)
    \setlength\itemsep{-3pt}
    \item (Beifall bei der CDU/CSU sowie bei Abgeordneten der SPD)
    \setlength\itemsep{-3pt}
    \item (Christian Kühn [Tübingen] [BÜNDNIS 90/DIE GRÜNEN]: Sonst geht es uns immer so, Herr Möring!)
    \setlength\itemsep{-3pt}
    \item (Beifall bei der CDU/CSU sowie bei Abgeordneten der SPD, der FDP, der LINKEN und des BÜNDNISSES 90/DIE GRÜNEN)
    \setlength\itemsep{-3pt}
    \item (Heiterkeit bei Abgeordneten der CDU/CSU)
    \setlength\itemsep{-3pt}
    \item (Christian Kühn [Tübingen] [BÜNDNIS 90/DIE GRÜNEN]: Das ist doch mal ein Service!)
\end{itemize}
\section{Tagesordnungspunkt 5}
\subsection{Kramp-Karrenbauer}
\noindent\textbf{Texts:} Sehr geehrte Frau Präsidentin! Verehrte Damen und Herren Abgeordnete! Gestatten Sie mir, aus aktuellem Anlass eine Vorbemerkung zu machen. Sie haben sicherlich in den Medien verfolgt – die Obleute sind darüber unterrichtet worden –, dass heute Morgen die Generalstaatsanwaltschaft und das LKA Dresden nach Hinweisen durch den MAD auf dem Privatgelände eines Bundeswehrangehörigen des KSK eine Durchsuchung durchgeführt haben. Dabei wurden nach jetzigem Stand Waffen, Sprengstoff und Munition sichergestellt. Dieser Bundeswehrsoldat, der schon länger im Fokus des MAD stand, ist, wie gesagt, nach den Informationen heute dieser Durchsuchung unterzogen worden. Ich begrüße diesen Ermittlungserfolg des MAD und der Behörden in Sachsen außerordentlich. Das ist eine enge Zusammenarbeit, die sich bewährt hat und die wir bei der weiteren Aufklärung des Vorfalles auch fortsetzen werden. Ich will an dieser Stelle auch für alle Angehörigen der Bundeswehr ganz klar sagen: Niemand, der in radikaler Art und Weise in unseren Streitkräften auffällt, hat in der Bundeswehr Platz.  Deswegen haben wir heute unmittelbar die disziplinaren Ermittlungen eingeleitet. Dieser Soldat wird keine Uniform mehr tragen und auch keine Liegenschaft der Bundeswehr mehr betreten dürfen. Ich möchte das an dieser Stelle noch einmal ganz ausdrücklich betonen.  Meine sehr geehrten Damen und Herren, solche Soldaten stellen auch die Leistungen und die Werte der Männer und Frauen infrage, die für uns nicht nur hier in Deutschland, sondern an vielen Orten der Welt – auch in der Sahelzone – Leib und Leben riskieren. Die Sahelzone – das ist vorhin bei der Debatte zu EUTM schon deutlich geworden – ist eine schwierige Zone. Es ist eine gefährliche Zone; es ist eine unsichere Zone. Sie ist zunehmend eine Drehscheibe für Terroristen und für organisierte Kriminalität, und sie ist auch eine Bedrohung für unsere Sicherheit hier in Europa und hier in Deutschland. Es ist ein gefährlicher Einsatz, in den wir unsere Soldatinnen und Soldaten schicken, und zwar unabhängig davon, ob es EUTM Mali, MINUSMA oder die bilaterale Mission Gazelle ist, die wir zusammen mit der Armee in Niger durchführen. Deswegen muss es gut überlegt sein, ob wir unsere Soldatinnen und Soldaten diesem Risiko aussetzen. Meine sehr geehrten Damen und Herren, schauen Sie sich diese Region an. Wenn man mit dem Verteidigungsminister von Niger spricht, dann sagt er einem, dass dieses Land, zurzeit eines der ärmsten in der Welt, zeitgleich an fünf seiner Außengrenzen von fünf unterschiedlichen, grenzüberschreitend tätigen und sehr miteinander vernetzten terroristischen Gruppen attackiert wird, was nichts anderes bedeutet, als dass Soldatinnen und Soldaten der eigenen Armee und der internationalen Kräfte angegriffen werden; darüber hinaus wird die Zivilbevölkerung – vor allen Dingen auch Frauen und Mädchen – massakriert. Viele Kinder werden daran gehindert, zur Schule gehen zu können. Dann kann ich nur sagen: Alles, was wir in dieser Region tun, jeder Terroranschlag, den wir durch unseren Einsatz verhindern, und jedes Leben eines Mädchens, das wir retten, ist es wert, dass wir uns dort einsetzen; denn es macht das Leben der Menschen vor Ort lebenswerter, schwächt den Terrorismus und macht damit auch unsere Region hier sicherer.  Meine sehr geehrten Damen und Herren, wir tun dies mit Blick auf EUTM Mali mit folgendem Ansatz: Die Bundeswehr wird die Ausbildung für die malische Armee mit dem neuen Mandat jetzt in einer veränderten Form durchführen, in einer eigens dafür aufzubauenden Militärschule und durch ein sehr viel stärkeres Begleiten und Beraten als bisher. Das ist genau die Methode, die wir im Rahmen der Mission Gazelle im bilateralen Verhältnis zu Niger angewandt haben und die dort dazu geführt hat – der Generalinspekteur war erst vor Kurzem in Niger und hat sich die Ergebnisse angeschaut –, dass es ein erheblich gesteigertes Maß an Resilienz der nigrischen Armee gibt, die nun in der Lage ist, auch Terroristen, bewaffneten Aufständischen und anderen Gruppen die Stirn zu bieten. Wir ergänzen dies durch den Ansatz von MINUSMA, einem Mandat, mit dem wir den Friedensprozess bzw. das Friedensabkommen absichern, indem wir vor allen Dingen Kräfte für boden- und luftgestützte Aufklärung einsetzen und durch den Lufttransportstützpunkt in Niamey in Niger den Lufttransport von Patienten und damit die logistische Unterstützung auch für unsere Partner sichern. Deutschland ist bei MINUSMA eine Anlehnnation für Soldatinnen und Soldaten aus Belgien, aus Dänemark, aus Estland, aus Irland, aus Litauen, aus den Niederlanden und aus der Schweiz. Sie sind in das deutsche Kontingent integriert, und deswegen ist die Forderung, diesen Einsatz zu beenden, eine ganz klare Absage an die Verantwortung, die Deutschland in der Welt trägt – gerade im Moment auch durch unsere Position im UN-Sicherheitsrat –, und eine klare Absage an andere Nationen, die sich bei diesem Einsatz auf uns verlassen. Deswegen ist es wichtig, dass wir diesen Einsatz fortsetzen.  Mit diesen drei Missionen werden wir die Möglichkeit haben, einen Raum zu definieren, den wir so absichern, dass er zumindest die Grundvoraussetzung dafür bietet, dass ziviles Leben wieder entstehen kann,  dass Kinder in die Schule gehen können und dass wieder ein Mindestmaß an wirtschaftlichem Leben entstehen kann. Dadurch schaffen wir die beste Resilienz einer Zivilgesellschaft, die man erreichen kann, um den Anwerbeversuchen von Terroristen zu begegnen. Zum Terrorkampf gehört mehr als nur das militärische Agieren. Dazu gehört auch, dass man den Terroristen den Nährboden entzieht, und genau das machen wir mit dem militärischen, aber auch mit dem zivilen Ansatz in der Sahelzone. Deswegen wollen wir ihn mit fortsetzen, auch wenn wir wissen: Das ist ein gefährlicher Einsatz, der uns noch eine ganz geraume Zeit fordern wird und unseren Soldatinnen und Soldaten vieles abverlangt. Zurzeit herrschen in Gao Temperaturen von 46 Grad. Dort ist Quarantäne einzuhalten, damit wir den Covid-19-Virus nicht in ein Land mit einem schwachen Gesundheitssystem, wie dem in Mali, einschleppen. Dass wir diese Interessen unserer Soldatinnen und Soldaten immer auch im Blick haben und hatten, ist auch das Verdienst der Wehrbeauftragten, insbesondere des jetzigen Wehrbeauftragten, Herrn Dr. Bartels, dem ich bei dieser Gelegenheit noch einmal ganz herzlich auch im Namen der Soldatinnen und Soldaten für seine Arbeit und seinen Einsatz danken möchte. Gleichzeitig möchte ich Frau Kollegin Högl – ich sehe sie hier – als seiner Nachfolgerin auch alles Gute im Namen der Soldatinnen und Soldaten und eine gute Hand für unsere Zusammenarbeit wünschen.  Herzlichen Dank.  

\noindent\textbf{Comment:}
\begin{itemize}
    \setlength\itemsep{-3pt}
    \item (Beifall bei der AfD)
    \setlength\itemsep{-3pt}
    \item (Beifall bei der CDU/CSU, der SPD und der FDP sowie bei Abgeordneten des BÜNDNISSES 90/DIE GRÜNEN)
    \setlength\itemsep{-3pt}
    \item (Gerold Otten [AfD]: Das haben Sie bei Afghanistan auch erzählt!)
    \setlength\itemsep{-3pt}
    \item (Beifall bei der CDU/CSU, der SPD und der FDP)
    \setlength\itemsep{-3pt}
    \item (Beifall bei der CDU/CSU sowie bei Abgeordneten der FDP)
    \setlength\itemsep{-3pt}
    \item (Gerold Otten [AfD]: Sie ist nicht hier! Wo ist sie denn? Drei Mandate hier heute, und die zukünftige Wehrbeauftragte ist nicht hier! – Gegenruf der CDU/CSU: Doch!)
    \setlength\itemsep{-3pt}
    \item (Beifall bei der CDU/CSU sowie bei Abgeordneten der SPD und der FDP)
    \setlength\itemsep{-3pt}
    \item (Beifall bei der CDU/CSU, der SPD, der FDP und dem BÜNDNIS 90/DIE GRÜNEN)
\end{itemize}
\subsection{Hampel}
\noindent\textbf{Texts:} Danke schön. – Frau Präsidentin! Meine sehr verehrten Damen und Herren! Zuschauer haben wir ja keine mehr. Sehr verehrter Herr Wehrbeauftragter Bartels! Ich mache die Heuchelei von Ihnen, Frau Ministerin, nicht mit. Sie hätten ja dagegenstimmen können, als es um den Wehrbeauftragten ging. Ich spreche Ihnen aber auch meinen Respekt und meine Anerkennung für Ihre hervorragende Arbeit aus, Herr Wehrbeauftragter.  Mali ist ein Kunstprodukt der Franzosen, entstanden in der Kolonialzeit. Die Bevölkerung lebt in ganz unterschiedlichen Identitäten, die nicht von der Staatsgrenze abhängen; mein Kollege Lothar Maier hat das vorhin bei EUTM Mali schon ausgeführt.  Dort geht es nicht um Nationalitäten, sondern um tribale Identitäten, und dort findet nicht ein Kampf zwischen islamistischen Terroristen und friedliebenden Menschen, sondern zwischen Viehzüchtern und Ackerbauern statt. Der Konflikt ist jahrhundertealt. Früher konnte das durch Kompromisse gelöst werden, heute leider nicht mehr. Das exzessive Bevölkerungswachstum von 4 Millionen auf absehbar 36 Millionen überfordert die natürlichen Ressourcen. Es handelt sich nicht um ein demografisches Problem in der Region, sondern um eine demografische Katastrophe. Das ist die Ausgangslage, die wir vorgefunden haben. Mittlerweile verfolgen wir dort seit sieben Jahren eine Operation, die nicht irgendeinen Erfolg für uns erkennbar oder absehbar gemacht hat. Herr Staatsminister, ich habe im Ausschuss nachgefragt, was denn die wenigen Grundparameter sind, die man abfragen kann, um einen Erfolg zu messen. Ich habe Sie gefragt: Ist denn das Bruttoinlandsprodukt in Mali gestiegen, oder ist es gefallen? Um wie viel ist es gestiegen? Ich brauchte nicht die genaue Prozentzahl; aber ich wollte es gerne ansatzweise wissen. Um ein Viertel? Um die Hälfte? – Sie konnten darauf keine Antwort geben. Auch Ihr Fachmann, der Abteilungsleiter, ist uns die Antwort schuldig geblieben. Ich habe dann nachgefragt: Wie viel des Staatsgebietes kontrollieren die Regierungstruppen denn jetzt, nach sieben Jahren Einsatz? Sind das 10 Prozent oder 50 Prozent oder 80 Prozent? Ungefähr! Ich will keine konkrete Zahl haben. – Selbst darauf fanden Sie keine Antwort. Sie konnten nur sagen: Da gibt es einige Wüstenregionen, wo gar nichts unter Kontrolle ist.  Was für eine Aussage zu einem Konflikt, den wir seit sieben Jahren begleiten! Nicht mal die Frage nach einer verbesserten Infrastruktur oder der Produktivität in dem Lande konnten Sie mit einer ungefähren Zahl oder Prozentangabe beantworten. Wer uns noch nicht einmal die Grundparameter eines solchen Einsatzes vorlegen kann, kann doch von uns nicht erwarten, dass wir weiterhin einen Einsatz gutheißen, bei dem wir es derzeit mit einer Regierung zu tun haben, die gerade mal mit 36 Prozent – wenn ich es richtig in Erinnerung habe – gewählt worden ist. Der Oppositionsführer wurde weggeschnappt, befindet sich irgendwo eingekastelt und wird nicht herausgegeben. Der islamistische Terror ist zudem nicht auf dem Rückzug, sondern auf dem Vormarsch. Ja wo ist denn da ein Erfolg dieser Operation zu verzeichnen, Herr Staatsminister?  Kommen wir zu den europäischen Nationen, die sich hier engagieren. In diesem Zusammenhang kann man einen Aspekt hinzufügen. Man könnte sagen: Ja, das machen wir, um unsere französischen Freunde zu unterstützen. Schließlich wissen wir alle, dass da vornehmlich französische Interessen wichtig sind, nicht unsere, keine deutschen. Dann könnten wir das zu einem politischen Instrument machen. – Wir könnten ja sagen: Wir geben den Franzosen was; wir stellen uns an ihre Seite. Aber dafür wollen wir vielleicht etwas von unseren französischen Freunden in anderen Bereichen haben. – Selbst diese Idee zu formulieren, kommt Ihnen nicht in den Sinn. Ich ahne nur eins: dass das in Afrika nur funktioniert, wenn wir Afrika den Afrikanern zurückgeben, auch die Lösung ihrer Probleme. Wir alle haben so etwas nämlich erlebt – das war vorbildlich – beim Konflikt zwischen Äthiopien und Eritrea. Ich habe das immer wieder erwähnt, weil es in der Tat interessant ist, nachzuvollziehen, wie sich da zwei Staaten ohne jegliche westliche Einmischung auf ein aufgezeichnetes Friedensabkommen haben einigen können. Das ist der Weg. Meine Damen und Herren, überlassen wir Afrika den Afrikanern. Sie werden für ihre Zukunft kämpfen. Ja, sie müssen sie auskämpfen. Aber eine europäische Intervention wird auch weiterhin scheitern. Schicken Sie Soldaten nicht in die Wüste, schicken Sie sie nach Hause, Frau Minister!  

\noindent\textbf{Comment:}
\begin{itemize}
    \setlength\itemsep{-3pt}
    \item (Beifall bei der SPD)
    \setlength\itemsep{-3pt}
    \item (Beifall bei der AfD)
    \setlength\itemsep{-3pt}
    \item (Zuruf vom BÜNDNIS 90/DIE GRÜNEN: Das war da schon falsch!)
    \setlength\itemsep{-3pt}
    \item (Heiterkeit bei Abgeordneten der AfD)
\end{itemize}
\subsection{Matschie}
\noindent\textbf{Texts:} Frau Präsidentin! Werte Kolleginnen und Kollegen! Ich will zu Beginn noch mal klarstellen, warum wir mit der Bundeswehr in Mali sind, warum wir uns in der Sahelregion engagieren: nicht weil wir Frankreich einen Gefallen schulden, sondern weil es unser eigenes Interesse und das Interesse der internationalen Gemeinschaft ist, dass Staaten nicht zerfallen. Zerfallene Staaten sorgen nämlich dafür, dass sich Terror und Kriminalität ausbreiten und die Sicherheit der gesamten Region und letztendlich auch die europäische Sicherheit bedrohen. Das ist der Grund, warum wir uns dort engagieren.  MINUSMA ist als UN-Mission dort eingesetzt worden, um die Umsetzung des Friedensvertrages, der zwischen den Beteiligten vorher verabredet worden war, zu unterstützen, und zwar auf Bitten der malischen Regierung. Wir kommen hier nicht von außen und sagen: „Wir stülpen euch eine Lösung über“, sondern wir sind auf Bitten der malischen Regierung da, um den Friedensprozess in Mali zu unterstützen. Ja, man muss auch offen eingestehen: Trotz jahrelangem Engagement sind die Probleme dort nicht gelöst. In Teilen haben die gewalttätigen Aktionen zugenommen, haben terroristische Aktivitäten zugenommen, und sie weiten sich auf weitere Staaten aus. Wir sind hier in einem echten Dilemma. Angesichts dieses Dilemmas stellt sich die Frage: Wo liegen die Gründe dafür, und ist ein Rausgehen hilfreich, oder richtet es noch mehr Schaden an? Ich bin im Januar in Burkina Faso gewesen; denn dieser Konflikt hat inzwischen längst den Norden Burkina Fasos erreicht. Dort kann man genau beobachten, was sich vollzieht. Der Staat hat sich aus dem Norden zurückgezogen, nachdem er schon jahrelang die Entwicklung dort vernachlässigt hatte. Die Armee hat sich auf wenige Stützpunkte zurückgezogen. Weil keine Sicherheit mehr da war, ist die öffentliche Verwaltung aus dieser Region zum Teil verschwunden. Die Schulen haben zugemacht. In dieses Vakuum stoßen dann militärische Gruppen, terroristische Gruppen, zum Teil organisierte Kriminalität, um die Staatlichkeit dort zu ersetzen und junge Leute für sich zu rekrutieren. Das ist der Teufelskreis, in dem solche Staaten stecken. Diesen Teufelskreis kann man nicht durchbrechen, wenn man das staatliche Gewaltmonopol nicht wiederherstellt. Nun bin ich mir auch im Klaren darüber, dass wir das nicht einfach mit MINUSMA erreichen können, dass man nicht einfach von außen sozusagen die Staatlichkeit ersetzen kann. Wir können das Handeln dieser Staaten nicht ersetzen. Aber wir können sie in ihrem Handeln unterstützen. Darum geht es bei der europäischen Trainingsmission EUTM Mali, über die wir schon vorhin geredet haben, und darum geht es auch bei MINUSMA. Klar ist aber auch: Allein militärisch gibt es keine Lösung. Wir brauchen politische Lösungen, und wir brauchen die Unterstützung von Entwicklungen in diesen Regionen; denn junge Leute schließen sich terroristischen Gruppen an, weil sie keine Perspektiven haben, weil es keine Ausbildung gibt, weil es keine Schulen gibt, weil es keine Jobs gibt, weil sie nicht wissen, wie sie ihr Leben gestalten und wie sie ihren Lebensunterhalt verdienen können. Deshalb ist es wichtig, politische Lösungen zu unterstützen, Verhandlungslösungen dort, wo sie möglich sind, zu erreichen – einiges zeichnet sich ab –, aber auch die Anstrengungen der Entwicklungsarbeit zu verstärken, um insbesondere jungen Menschen eine andere Perspektive zu geben und zu zeigen: Die internationale Gemeinschaft ist nicht nur da, um militärische Sicherheit zu unterstützen, sondern sie ist dafür da, das Leben von Menschen in dieser Region zu verbessern. Deutschland hat hier in den letzten Jahren eine ganze Menge investiert, nicht nur in MINUSMA. In dieser Zeit hat Deutschland weit über 500 Millionen Euro für Entwicklungszusammenarbeit ausgegeben. Diese Anstrengung muss jetzt in den kommenden Jahren verstärkt werden, gemeinsam mit der Suche nach politischen Lösungen. Vorerst aber bleibt MINUSMA wichtig als ein Stabilitätsanker, als eine Unterstützung für die Regierung in der Region. Deshalb bitte ich um Zustimmung zu diesem Mandat. Zum Schluss. Ich will allen Soldatinnen und Soldaten, die dort in einem harten Einsatz sind, ein ganz großes Dankeschön sagen. Bleiben Sie gesund, und kommen Sie heil nach Hause! Herzlichen Dank.  

\noindent\textbf{Comment:}
\begin{itemize}
    \setlength\itemsep{-3pt}
    \item (Beifall bei der SPD sowie bei Abgeordneten der CDU/CSU)
    \setlength\itemsep{-3pt}
    \item (Beifall bei der FDP)
    \setlength\itemsep{-3pt}
    \item (Beifall bei der SPD und der CDU/CSU)
\end{itemize}
\subsection{Beeck}
\noindent\textbf{Texts:} Hochverehrte Frau Präsidentin! Sehr geehrte Kolleginnen und Kollegen im Hohen Hause! Wir wissen, dass die Situation in Mali komplex ist. Wir wissen, dass die letzten sieben Jahre des Einsatzes noch nicht zu den Ergebnissen geführt haben, die wir uns alle wünschen. Die Situation vor Ort ist immer noch geprägt von Terror. Auch die Sicherheitssituation hat sich in den letzten Jahren eher nicht verbessert, sondern bleibt hoch angespannt. Wenn man auf aktuelle Entwicklungen wie die Coronapandemie schaut, dann kann man positiv feststellen, dass in Mali heute 20 Prozent mehr Beatmungsgeräte vorhanden sind als früher, nämlich möglicherweise fünf statt vier – bei einer 19-Millionen-Bevölkerung. Allein dieses kurze Schlaglicht zeigt sehr deutlich, dass dort noch vieles zu tun ist. Es handelt sich um einen der gefährlichsten Einsätze, die unter dem Siegel der UN geführt werden, mit mittlerweile 209 Toten alleine im Rahmen des Mandats MINUSMA. Deswegen, Frau Bundesministerin, sind wir voll bei Ihnen, wenn Sie sagen: Man muss sehr wohl überlegen, ob man einen solchen Einsatz eigentlich weiterführen kann oder nicht.  Wir sind allerdings der Auffassung, dass es richtig ist, diesen Einsatz fortzuführen, weil er sich hehren Zielen widmet, nämlich der Wiederherstellung staatlicher Autorität, der Implementierung des auch mit unserer Hilfe geschaffenen Friedensabkommens aus dem Jahr 2015, und weil er insgesamt zu einer deutlichen Verbesserung der Situation dort geführt hat, was auch in unserem eigenen Interesse liegt. Ich will nicht wiederholen, was alle Redner vorher gesagt haben. Aber es bringt eben nichts, Kolleginnen und Kollegen von der AfD, sich an dieser Stelle seine eigene Wirklichkeit mit strammen Sprüchen wie „Afrika den Afrikanern“ zurechtzulegen  und dabei völlig auszublenden, dass hier eigene europäische, eigene deutsche Interessen im Spiel sind, dass es eben nicht so ist, wie Sie beschreiben, dass alles nur eine Katastrophe ist, dass unsere Kräfte überhaupt nicht reichen. Das hat der Kollege Maier bei der Diskussion vor einer Stunde zu EUTM Mali gesagt, als er von 1 400 Dienstposten gesprochen hat. Allein MINUSMA hat über 15 000 – über 12 100 Soldaten, 1 725 Polizeibeamte, fast 1 800 zivile Experten, die sich dort nicht nur bemühen, die Sicherheitssituation zu verbessern, sondern auch, dafür zu sorgen, dass es wieder einen demokratischen Staat gibt. Herr Hampel, auch das haben Sie völlig ausgeblendet. Natürlich ist Mali irgendwann mal im Zuge des Kolonialismus entstanden. Damals war es europäische Willkür, heute ist es eine schreckliche Situation. Sie blenden alles aus, was davor war, dass nämlich Mali bis 2012 sehr wohl als eine demokratische Hoffnung in Afrika gegolten hat, und Sie erwähnen mit keinem Wort, dass Ende März und am 19. April – ja, mit einer geringen Wahlbeteiligung, aber in dieser Sicherheitssituation und während einer Coronapandemie – noch eine demokratische Wahl in diesem Land stattgefunden hat. Also: Wir wissen um die Gefahren dieses Mandates; aber wir wissen auch, dass es sich lohnt und dass es keineswegs nur negative Entwicklungen gibt. Die Freien Demokraten wünschen sich auch an dieser Stelle – das ist angesprochen worden, auch vom Kollegen Matschie – eine stärkere Vernetzung der Bemühungen um wirtschaftliche Zusammenarbeit. Aber in Abwägung all dieser Argumente werden wir der Überweisung und am Ende dem Mandat MINUSMA zustimmen. Herzlichen Dank.  

\noindent\textbf{Comment:}
\begin{itemize}
    \setlength\itemsep{-3pt}
    \item (Beifall bei der FDP)
    \setlength\itemsep{-3pt}
    \item (Beifall bei der LINKEN)
\end{itemize}
\subsection{Vogler}
\noindent\textbf{Texts:} Vielen Dank. – Verehrte Frau Präsidentin! Meine Damen und Herren! Kolleginnen und Kollegen! Wir stecken gerade in einer gigantischen globalen Krise, die weltweit Leben und Gesundheit von Millionen von Menschen bedroht, und trotzdem vergeht keine Sitzungswoche, in der wir hier nicht über die Verlängerung oder Neueinsetzung von Militäreinsätzen beraten, wie jetzt heute über den Einsatz von MINUSMA in Mali. Meine Damen und Herren von der Bundesregierung, ich finde ja, Sie müssten dringend mal an Ihren Prioritäten arbeiten. Ich könnte mich jetzt hier drei Minuten mit der desaströsen Sicherheitslage für die Zivilbevölkerung in Mali auseinandersetzen, wo seit 2012 die Zahl der bewaffneten Milizionäre von einigen Tausend auf viele Zehntausend, vielleicht auf über hunderttausend angestiegen ist. Und ich könnte dann fragen, warum ein Militäreinsatz, der seine Ziele so wenig erreicht wie MINUSMA, dennoch Jahr um Jahr verlängert und auch ausgeweitet wird. Aber dazu hat meine Kollegin Christine Buchholz schon eine Menge gesagt.  Ich will mal eine ganz andere Frage stellen: Müssten wir nicht jetzt einmal die alten Muster verlassen und überlegen, was eigentlich vor Ort akut am dringendsten gebraucht wird?  Angesichts der Bedrohung durch Covid‑19 rät sogar die Stiftung Wissenschaft und Politik der Bundesregierung mit dem Blick auf Mali: „Das Hauptaugenmerk sollte vorerst auf zivilen und medizinischen Projekten liegen, weniger auf militärischen Fähigkeiten.“ Richtig so!  Wie so viele Konflikte wird ja auch der in Mali ganz entscheidend davon befeuert, dass die Zentralregierung von den Menschen in der Fläche des Landes nicht als Garant ihrer menschlichen Grundbedürfnisse wahrgenommen wird. Eine funktionierende Basisgesundheitsversorgung in allen Teilen des zerrissenen Landes, zu der alle Menschen, gleich welcher Herkunft, welcher Ethnie, welchen Glaubens unbewaffnet den gleichen kostenlosen Zugang hätten, würde Mali vermutlich mehr stabilisieren, als es die massive ausländische Militärpräsenz seit acht Jahren dort tut.  Die Prinzipien einer medizinischen Ethik, diese Prinzipien von Altruismus, Solidarität und Kooperation, die dafür benötigt und dadurch gestärkt werden, könnten soziale Beziehungen stärken und die Menschen zu gewaltfreien Konfliktlösungen befähigen. Die zentrale Frage für mich dabei ist doch: Wie logisch ist es eigentlich, dass zivile und medizinische Projekte von Militärs aufgebaut werden sollen, die doch für ganz andere Aufgaben ausgebildet wurden? Warum machen das nicht Fachleute? Ich finde, wir sollten es Fachleute machen lassen.  Dafür gibt es auch Beispiele. Das arme Kuba zum Beispiel unterstützt mit medizinischem Personal und Ausrüstung gerade 60 Länder in der ganzen Welt bei der Bewältigung der Coronakrise. Warum können wir als eines der reichsten Länder der Welt das eigentlich nicht tun?  In unserem Land fehlen Hunderttausende Pflegekräfte in Kliniken und Senioreneinrichtungen. Aber die Bundeswehr benutzt im Augenblick die Coronakrise, um bei 16-Jährigen für den Dienst an der Waffe zu werben.  Wann eigentlich schreibt die Bundesregierung alle Schulabgängerinnen an, um sie für die Gesundheitsberufe zu begeistern? Das wären doch mal Prioritäten, die wir jetzt brauchen.  Ja, vielen Dank. – Es ist Zeit zum Umdenken. Beenden Sie die Militäreinsätze! 

\noindent\textbf{Comment:}
\begin{itemize}
    \setlength\itemsep{-3pt}
    \item (Beifall bei der LINKEN – Dr. Johann David Wadephul [CDU/CSU]: Das ist unglaublich, was Sie uns zumuten! Alles falsch! Das Mandat der Vereinten Nationen kam überhaupt nicht vor! Unglaublich!)
    \setlength\itemsep{-3pt}
    \item (Alexander Graf Lambsdorff [FDP]: Das war auch schon falsch!)
    \setlength\itemsep{-3pt}
    \item (Alexander Graf Lambsdorff [FDP]: Oh Gott!)
    \setlength\itemsep{-3pt}
    \item (Beifall bei der LINKEN)
    \setlength\itemsep{-3pt}
    \item (Beifall bei der LINKEN – Dr. Johann David Wadephul [CDU/CSU]: Ja! Sicherheit!)
\end{itemize}
\subsection{Brugger}
\noindent\textbf{Texts:} Sehr geehrte Frau Präsidentin! Meine Damen und Herren! Mit Blick auf die Sahelzone gibt es immer mehr Nachrichten, die uns mit tiefer Sorge erfüllen: Die Sicherheitslage verschlechtert sich, Terroranschläge nehmen zu, die Coronapandemie und die Klimakrise wirken als Konfliktverschärfer auf die bereits bestehenden sozialen und politischen Krisen ein. Das Friedensabkommen wird nicht konsequent umgesetzt, weil Regierung und andere Kriegsparteien es ausbremsen und blockieren. Frankreich reagiert auf all dies mit mehr Militär und will neben den bereits bestehenden Missionen mit einer weiteren Koalition der Willigen mit einigen europäischen Staaten den Antiterrorkampf noch weiter ausweiten. Politische Probleme lassen sich aber nicht mit mehr Militär lösen. Im Gegenteil: Offensiveres Vorgehen droht die Krise eher noch weiter zu verschärfen.  Die Bundesregierung unterstützt diesen Strategiewechsel nicht mit Personal, und ich vermute, dies ist so, weil Sie die vielen Bedenken teilen, die die Expertinnen und Experten bei diesem gefährlichen Strategiewechsel haben. Trotzdem haben Sie von der Bundesregierung mehrfach bekräftigt, dass Sie diesen Kurs aber politisch unterstützen. Wir fordern Sie auf: Weichen Sie nicht einfach der sicherlich schwierigen Diskussion mit unseren französischen Partnern aus, und ziehen Sie hier eine klare rote Linie.  „Agree to disagree“ zwischen Frankreich und Deutschland, das kann ganz sicherlich nicht die Grundlage einer gemeinsamen europäischen Strategie in Mali sein. Meine Damen und Herren, bei aller Kritik sollte man die kleinen, aber trotzdem auch wichtigen Fortschritte in Mali sehen und darüber sprechen: Der nationale Dialog wurde durchgeführt und begonnen. Die ehemals feindlichen Kräfte patrouillieren endlich gemeinsam im Norden. – Das alles wäre ohne die Friedensmission der Vereinten Nationen nur schwer vorstellbar, das Friedensabkommen, das vor ein paar Jahren geschlossen wurde, ebenso. Der UN-Generalsekretär hat die Mitgliedstaaten bei MINUSMA um Hilfe gebeten, technisch wie politisch. Es ist deshalb auch richtig, dass die Flugstunden der deutschen Aufklärungsdrohne im Mandat ausgeweitet werden um den Schutz der MINUSMA-Kräfte, aber auch, um die Überwachung des Friedensabkommens zu verbessern. Es bereitet aber schon große Sorge, wenn trotz großer Bitten der Vereinten Nationen ab Oktober die Rettungskette nicht mehr steht, weil es keine Ablösung für die rumänischen Hubschrauber gibt, die das jetzt übernehmen. Das ist auch eins von vielen traurigen Symbolen dafür, wie die Mitgliedstaaten die Vereinten Nationen bei ihren schwierigen Aufgaben viel zu oft im Regen stehen lassen, und das muss sich endlich ändern.  Die Rettungskette muss stehen, die Mitgliedstaaten müssen die Vereinten Nationen bei ihren Aufgaben stärker unterstützen, mit gutem Personal und technischen Fähigkeiten, aber vor allem auch mit mehr politischer Unterstützung. Dafür ist Druck auf die Konfliktparteien in Mali ganz zentral. Es ist ein großes Versäumnis, dass hier so wenig passiert ist. Meine Damen und Herren, ich will aber auch nicht unkommentiert lassen, was die Ministerin am Anfang ihrer Rede gesagt hat. Ich fand es schon sehr bezeichnend, dass bei den Sätzen, auf die wir uns, glaube ich, alle einigen können – dass jeder Fall in der Bundeswehr, bei dem Rechtsextremismus nachgewiesen ist, ein Fall zu viel ist –, nur fünf Fraktionen in diesem Haus geklatscht haben. Ich hoffe, das hat jeder mitbekommen und jeder gesehen. Aber, Frau Ministerin, mich stört es schon: Hitlergrüße, Nazirock, Chatgruppen mit Gewaltfantasien. Das hier ist schon eine neue Qualität. Hier sind bei einem KSK-Soldaten Waffen gefunden worden. Wir erwarten von Ihnen mehr Informationen über alle Verbindungen, über alle Hintergründe, und nicht nur die eine Seite, die Sie uns heute in der Obleuteunterrichtung vorgelegt haben. Sie sprechen von möglichen Netzwerken. Der MAD hat deren Existenz immer wieder bestritten. Sie müssen klare Informationen hierzu dem Parlament unverzüglich auf den Tisch legen und auch darstellen, welche Konsequenzen Sie aus dieser neuen Qualität der Bedrohung ziehen.  

\noindent\textbf{Comment:}
\begin{itemize}
    \setlength\itemsep{-3pt}
    \item (Beifall bei der SPD)
    \setlength\itemsep{-3pt}
    \item (Beifall beim BÜNDNIS 90/DIE GRÜNEN)
    \setlength\itemsep{-3pt}
    \item (Beifall beim BÜNDNIS 90/DIE GRÜNEN und bei der LINKEN)
\end{itemize}
\subsection{Brecht}
\noindent\textbf{Texts:} Frau Präsidentin! Meine sehr verehrten Kolleginnen und Kollegen im Deutschen Bundestag! Wir haben ja heute zwei Debatten zu Mali gehabt, und ich habe festgestellt, dass wir sowohl bei der Ursachenbeschreibung als auch bei den Zielstellungen und schließlich auch bei den Mitteln zur Befriedung des Landes an vielen Stellen auseinandergehen. Ich stelle immer wieder fest, umso mehr ich mich mit Mali befasse, dass eine der wesentlichen Ursachen des Konfliktes die Bevölkerungsexplosion in Mali ist, der Kampf um die Ressourcen in einem Land, das ohnehin schon gebeutelt ist. Deswegen glaube ich, dass die multilateralen Bemühungen der Weltgemeinschaft für Klimaschutz, für Entwicklungshilfe, aber insbesondere für das Weltbevölkerungsprogramm deutlich verstärkt werden müssen. Ich richte diesen Appell eher weniger an die eigene Bundesregierung als vielmehr an jene rechtspopulistischen Staaten, die nationale Interessen vor alle anderen politischen Zielstellungen stellen.  Für uns alle ist die Zustimmung zu der MINUSMA-Operation nicht ganz einfach. Wir haben über die Schwierigkeiten und die fehlenden Fortschritte schon hinlänglich debattiert. Man muss natürlich auch die räumliche und die zeitliche Dimension des Mandates hinzunehmen: Dann wird klar, dass das ein sehr ambitionierter Anspruch ist, dem wir da folgen. Ein Beobachter der Situation hat sie ironisch charakterisiert: Mali is a sick patient with many doctors. – Wenn die Behandlung durch viele Ärzte erfolgreicher sein soll als in der Vergangenheit, dann ist eine bessere Abstimmung zwischen ihnen unverzichtbar. MINUSMA, EUTM, die G‑5-Einsatztruppe, EUCAP, GSVP und Gazelle sollten daher ihr Agieren besser koordinieren. Um im Bild der medizinischen Behandlung zu bleiben: Alle ärztliche Kunst ist ohne Compliance, also die Mitwirkung des Patienten, vergeblich. Der Patient Mali hat mit den überwiegend friedlich verlaufenen Wahlen einen Anfang gemacht. Aber nun erwarten wir von der Regierung in Bamako, dass die Verfassung unter Beteiligung der Bevölkerung zügig verabschiedet und dann auch respektiert wird. Der Sicherheitssektor muss dringend neustrukturiert werden, um die omnipräsente Korruption in Mali effektiv bekämpfen zu können. Schließlich sollte die malische Regierung über das Abkommen von Algier hinaus die inoffiziell stattfindenden Gespräche mit moderaten Aufständischen forcieren, da eine rein militärische Option für einen Frieden in Mali illusorisch ist. Wir haben eben schon über die Schärfung des MINUSMA-Mandates gesprochen. Ich denke, dass wir darüber hinaus auch noch weiteren Veränderungsbedarf haben. Wir erwarten, dass der Anteil gut ausgebildeter und ausreichend bewaffneter Soldaten durch die internationale Gemeinschaft deutlich erhöht wird. Es kann doch nicht sein, – – dass geschätzte 80 Prozent der MINUSMA-Aktivitäten für den Eigenschutz verbraucht werden. Insbesondere muss alles darangesetzt werden, dass die Akzeptanz der MINUSMA in der Bevölkerung zunimmt. Nur 5 Prozent – –  – Das war ein Wechsel, den ich nicht mitbekommen habe.  Ich dachte, ich kriege einen Bonus dadurch. – Nur 5 Prozent der Einwohner Malis bewerten die UN-Operation positiv. Wir müssen an dieser Stelle durch zivile Begleitmaßnahmen die MINUSMA-Operation zu mehr Erfolg führen. Vielen Dank für Ihre Aufmerksamkeit.   

\noindent\textbf{Comment:}
\begin{itemize}
    \setlength\itemsep{-3pt}
    \item (Beifall des Abg. Dr. Fritz Felgentreu [SPD])
    \setlength\itemsep{-3pt}
    \item (Beifall bei der SPD)
    \setlength\itemsep{-3pt}
    \item (Der Redner wendet sich zum Präsidenten)
    \setlength\itemsep{-3pt}
    \item (Heiterkeit)
    \setlength\itemsep{-3pt}
    \item (Beifall bei der CDU/CSU – Dr. Eberhard Brecht [SPD]: Das war unfair! – Tobias Pflüger [DIE LINKE]: Zu mir ist Claudia Roth auch sympathischer!)
\end{itemize}
\subsection{Brandl}
\noindent\textbf{Texts:} Vielen Dank, Herr Präsident; ich freue mich, dass Sie da sind. – Verehrte Damen und Herren! Ich möchte Ihnen jetzt in den drei Minuten, die ich nicht überziehen werde, drei Gründe nennen, warum wir bei MINUSMA sind und warum wir bei MINUSMA bleiben sollten. Erstens. MINUSMA dient unserer eigenen Sicherheit. Es liegt in unserem Interesse, dass in einem Land wie Mali und in der ganzen Sahelzone stabile Verhältnisse herrschen und dass dieses Gebiet nicht zu einer Region wird, in der Terror und Islamismus regieren. Den Wert von Sicherheit und Stabilität erkennt man meistens erst dann, wenn sie nicht mehr vorhanden sind. Wir erleben es ja gerade in Libyen: Wenn sich staatliche Strukturen auflösen, dann entsteht automatisch ein Nährboden für kriminelle und terroristische Gruppierungen. Das kann uns auch in der Sahelregion passieren. Dort gibt es aber noch staatliche Strukturen. Und, meine Damen und Herren, wir müssen alles unternehmen, sie zu schützen und zu stärken; denn wenn die uns mal wegbrechen, dann fehlt uns jeglicher Anker, an den wir mögliche Hilfe ansetzen können. Das ist der erste Grund. Der zweite Grund ist: MINUSMA ist eine der größten Missionen der Vereinten Nationen, und durch unsere Beteiligung daran unterstreichen wir, dass wir an das Konzept der internationalen Zusammenarbeit glauben. Wir wollen Probleme wie in Mali in einer Weltgemeinschaft lösen. Unser Ziel ist, gemeinsam zu agieren, und nicht: Deutschland zuerst, und der Rest der Welt ist egal. – Aber wenn wir in der Welt Ziele erreichen wollen, dann hilft es nicht, nur schlaue Reden in der Politik zu halten – das ist auch manchmal wichtig, aber nicht immer zielführend –, sondern dann muss man auch bereit sein, zu handeln. Wir sind bereit, in Westafrika zu handeln, und zwar in vielfältiger Weise, in erster Linie durch Entwicklungszusammenarbeit, aber auch durch Militäreinsätze, wo es notwendig ist, wie zum Beispiel in Mali. Der dritte Grund ist: Unser Engagement in Mali ist auch ein Engagement zur Stärkung der Rolle Europas in der Welt. Wir wollen ein starkes Europa. Wir wollen ein Europa, das souverän und handlungsfähig ist, ein Europa, das in der Lage ist, auf Sicherheitsprobleme in seiner weiteren Nachbarschaft auch selbst zu reagieren. Wenn nicht wir, wer denn sonst? Wir sind das größte Land innerhalb der europäischen Gemeinschaft. Wir haben auch eine besondere Verantwortung. MINUSMA ist ein Teil unseres europäischen Engagements. Es gibt eine ganze Reihe von weiteren Initiativen europäischer Länder. Die sind komplementär. Man kann sie noch besser abstimmen. Aber in Summe zeigt Europa dadurch, dass es bereit ist, Verantwortung in seiner Nachbarschaft mit zu übernehmen; und das ist richtig. Ich fasse zusammen: Erstens. Der Einsatz dient unserer eigenen Sicherheit. Zweitens. Er hilft den Vereinten Nationen, unterstützt das Konzept der internationalen Zusammenarbeit. Drittens. Er stärkt die Rolle Europas in der Welt. Wir sollten ihn deswegen fortsetzen. Herzlichen Dank. 

\noindent\textbf{Comment:}
\begin{itemize}
    \setlength\itemsep{-3pt}
    \item (Beifall bei der CDU/CSU sowie bei Abgeordneten der SPD)
\end{itemize}
\section{Tagesordnungspunkt 3}
\subsection{Maas}
\noindent\textbf{Texts:} Frau Präsidentin! Meine sehr verehrten Kolleginnen und Kollegen! 4 000 Kilometer trennen Mali und den Sahel von Deutschland. Das reicht schon, dass einige der Auffassung sind, zu glauben, dass wir damit nichts zu tun hätten. Spätestens aber seit terroristische Gruppen im Jahr 2012 – vielleicht erinnert sich der eine oder andere noch daran – Mali zu überrennen drohten, sollte uns, und zwar als Europäern, eines klar geworden sein: Was dort passiert, das gefährdet nicht nur die Stabilität unserer südlichen Nachbarschaft, sondern das wirkt als Brandbeschleuniger für die Ausbreitung von Terrorismus, organisierter Kriminalität und illegaler Migration bis nach Europa. Wir alle haben vor Augen, was droht, wenn Gruppen wie der IS oder al-Qaida im Sahel einen sicheren Rückzugsraum bekommen; denn schließlich sind auch europäische Länder und auch unsere Bürgerinnen und Bürger immer wieder Opfer ihrer Gewalttaten geworden. Dass Mali inzwischen der zweitgrößte Einsatzort der Bundeswehr im Ausland ist, ist auch Ausdruck unserer Sorge darüber, und zwar einer nach wie vor berechtigten Sorge. Aber ich verstehe auch diejenigen, die mit wachsender Ungeduld auf diesen Einsatz blicken; schließlich ist die bisherige Bilanz – auch das soll hier nicht verschwiegen bleiben – durchwachsen. Auf hoffungsvolle Schritte zur Aussöhnung, wie dem Beginn des nationalen Dialoges im Dezember des letzten Jahres, folgten immer wieder herbe Rückschläge. Die Zahl terroristischer Anschläge ist auch zuletzt wieder gestiegen. Teile im Zentrum Malis geraten immer stärker unter Druck, auch gerade aktuell. Erst am Montag hat uns die traurige Nachricht von drei getöteten Blauhelmsoldaten aus dem Tschad erreicht. Meine Damen und Herren, dieser Einsatz ist ein schwieriger, und er wird es auch bleiben. Es ändert aber nichts daran, dass unser Ziel, nämlich Stabilität im Sahel zu schaffen, auch eine der vielen Voraussetzungen ist, um die Sicherheit in Europa zu erhöhen. Anders, als das manchmal gesagt wird, reiht sich Mali eben nicht in die Gruppe der Failed States ein. Die Parlamentswahlen im März waren ein demokratisches Lebenszeichen. Gerade die junge Bevölkerung – jeder, der schon einmal da gewesen ist, wird das mit eigenen Augen gesehen haben – setzt sich spürbar ein für eine friedliche Zukunft ihres Landes. Der Schlüssel dafür heißt Sicherheit; Sicherheit, die immer stärker vor allen Dingen von den Menschen in der Region und auch den Verantwortlichen getragen werden muss. Dafür zu sorgen, dass das möglich ist, ist das Ziel der europäischen Ausbildungs- und Beratungsmission EUTM Mali. Liebe Kolleginnen und Kollegen, wir haben EUTM Mali in den vergangenen Monaten in Brüssel sehr gründlich und auch kritisch überprüft. Dabei ist deutlich geworden, dass wir nachsteuern müssen. Die Stellschrauben heißen dabei: mehr Einsatznähe und größere regionale Flexibilität. Malische Soldatinnen und Soldaten sollen künftig verstärkt dezentral ausgebildet werden, eben näher an ihren Operationsgebieten. Wir wollen so die Ausbildung verbessern und vor allen Dingen auch noch praxisnäher machen. Es ist aber ausdrücklich keine Begleitung in Einsätze geplant. Das wäre auch widersinnig; denn es geht uns gerade darum, dass die malischen Sicherheitskräfte eigenständig agieren. Noch etwas ist in diesem Zusammenhang wichtig, wenn wir über die Entsendung deutscher Soldatinnen und Soldaten sprechen: Dezentrale Ausbildung gibt es nur mit ausreichend Schutzkräften, auch das gehört dazu; denn das erhöht die Sicherheit unserer eigenen Soldatinnen und Soldaten, die für uns immer oberste Priorität hat.  Die zweite Stellschraube, mit der wir die Mission an die Herausforderungen vor Ort anpassen, betrifft die terroristische Bedrohung, die im Sahel eben keine Ländergrenzen kennt. EUTM Mali wird deshalb in Zukunft alle fünf Sahel-Länder beraten, in einzelnen Fällen auf Anfrage auch die nationalen Streitkräfte ausbilden können. Dazu gehört, dass wir unsere bisherigen Ausbildungsmaßnahmen in Niger verstärken und unter dem Dach von EUTM Mali alle Aktivitäten bündeln werden; auch das besonders betroffene Burkina Faso werden wir gezielt unterstützen, nicht durch eine dauerhafte Präsenz, sondern, je nach Bedarf, durch mobile Ausbildungsteams. Liebe Kolleginnen und Kollegen, so notwendig diese Anpassungen sind, sie sind keine Wundermittel. Die nachhaltige Wirkung erzielt unser gesamtes Engagement im Sicherheitsbereich – also die Ausbildung im Rahmen von EUTM und die Absicherung des politischen Prozesses durch die VN-Mission MINUSMA – nur in Verbindung mit Diplomatie, Stabilisierung und Entwicklungszusammenarbeit. Dieses Zusammenspiel charakterisiert unser gesamtes Engagement, so wie es auch im Perspektivbericht der Bundesregierung an den Bundestag im März beschrieben worden ist. Diesen vernetzten Ansatz haben wir inzwischen auch auf europäischer Ebene verankert. Das politische Dach bildet die internationale Partnerschaft für Sicherheit und Stabilität im Sahel, die wir im vergangenen Jahr gemeinsam mit Frankreich aus der Taufe gehoben haben. Ihr Ziel ist, staatliche Strukturen so zu stärken, dass die Sahel-Länder Schritt für Schritt selbst Verantwortung für Stabilität, Sicherheit, aber auch für nachhaltige Entwicklung übernehmen können. Meine Damen und Herren, wir tun gut daran, mit EUTM Mali ganz entschieden zu dieser Entwicklung beizutragen; denn diese Region wird uns auch in Zukunft ein außenpolitischer Nachbar sein, dessen Schicksal uns direkt und unmittelbar betrifft. Deshalb bitte ich Sie um Ihre Unterstützung für dieses Mandat. Herzlichen Dank.  

\noindent\textbf{Comment:}
\begin{itemize}
    \setlength\itemsep{-3pt}
    \item (Beifall bei der AfD)
    \setlength\itemsep{-3pt}
    \item (Beifall bei der SPD und der CDU/CSU)
    \setlength\itemsep{-3pt}
    \item (Beifall bei Abgeordneten der SPD)
\end{itemize}
\subsection{Maier}
\noindent\textbf{Texts:} Frau Präsidentin! Der Einsatz der deutschen Streitkräfte in Mali ist nun im siebten Jahr, und die Nachrichten, die wir bekommen, sind alles andere als erhebend. Sind wir dem Ziel dieser Mission näher gekommen? Offensichtlich nicht. Die Kämpfe sind nicht abgeflaut, sondern sie nehmen fast mit jedem Tag zu. Wir hören Nachrichten von Kämpfen mit zum Teil mehreren Hundert Toten, nicht nur an den Grenzen des malischen Staates, sondern auch auf dem malischen Territorium im Inneren. Dem sollte die Mission EUTM – wie auch die MINUSMA-Mission – entgegenwirken. Aber hat sie das wirklich gekonnt? Wenn wir uns die Zahlen für die Ausbildung anschauen, dann sehen wir, dass EUTM seit Beginn dieser Mission ungefähr ebenso viele malische Soldaten ausgebildet hat, wie die malische Armee insgesamt an Soldaten hat. Also jeder hat eine hochwertige Ausbildung bekommen; aber das hat offensichtlich zu nichts geführt. Woran liegt das? Waren die Ausbildungsinhalte nicht in Ordnung? Dafür gibt es Beispiele aus anderen Missionen: In Afghanistan etwa sind Leute an einer Technik ausgebildet worden, mit der sie nicht umgehen konnten oder die sie hinterher verkauft haben. In Mali kommt dazu, dass sich sehr viele dieser Soldaten mit ihrer von der Regierung geforderten Mission gar nicht richtig identifizieren können. Wir müssen auch erkennen: Mali ist vielleicht ein Staat – der Herr Außenminister sagte: Es ist kein Failed State; das bezweifle ich –, aber es ist keine Nation. Mali besteht aus einer Reihe von Nationen, die sehr unterschiedlich strukturiert sind, die unterschiedlichen Sprachfamilien angehören, unterschiedlichen Stammeskulturen, die in sehr unterschiedlichen, weit voneinander entfernten Regionen leben. Wenn Sie einmal einen Staatsbürger von Mali fragen würden, welcher Nation er angehört, dann wird er kaum sagen: „Ich bin Malier“, sondern er wird sagen: Ich bin Bambara, Dogon, Tuareg, Fulani, Kunta und wie sie alle heißen. Es ist eine Stammesgesellschaft. Wir versuchen hier, einen Konflikt mit militärischen Mitteln zu lösen; aber er ist ein politischer Konflikt. Diesen Staat auf der jetzigen Basis zu regieren, dürfte kaum möglich sein.  Für solche Staaten ist zunächst einmal die sinnvolle Lösung, sie zu regionalisieren, zu föderalisieren. Die malische Regierung – das ist zum Beispiel allgemein bekannt – hat sich um den Norden des Landes überhaupt nicht geschert.  Sie hat keine Entwicklungsanstrengungen unternommen; sie überlässt das den ausländischen Interventionen. So geht das nicht! Hier muss man eben versuchen, die politische Lösung voranzubringen, statt auf der militärischen Lösung, so wie sie jetzt ist, zu beharren. Es ist auch eine Frage etwa der Stärke der hier eingesetzten europäischen und auch der deutschen Kräfte. EUTM, MINUSMA und die französische Operation Barkhane haben zusammen, wenn ich richtig informiert bin, 1 400 Dienstposten. Diese sollen auf einem Territorium von 1,2 Millionen Quadratkilometern Sicherheit herstellen. Allein die Außengrenzen von Mali umfassen mehrere Tausend Kilometer. Sie sind faktisch nicht zu sichern, und Interventionen erfolgen von allen Seiten, von allen benachbarten Staaten aus. So wie diese Mission jetzt angelegt ist und ohne die notwendige politische Beteiligung ist der weitere Einsatz nicht zu verantworten. Er sollte beendet werden. Danke Ihnen.  

\noindent\textbf{Comment:}
\begin{itemize}
    \setlength\itemsep{-3pt}
    \item (Beifall bei der AfD)
    \setlength\itemsep{-3pt}
    \item (Beifall bei der CDU/CSU)
    \setlength\itemsep{-3pt}
    \item (Henning Otte [CDU/CSU]: Vogel-Strauß-Perspektive!)
\end{itemize}
\subsection{Hardt}
\noindent\textbf{Texts:} Frau Präsidentin! Liebe Kolleginnen und Kollegen! Ich finde, der afrikanische Kontinent ist ein Schlüsselkontinent für die Zukunft der ganzen Welt, und er ist auch ein Schicksalskontinent für uns in Europa; denn auf keinem anderen Kontinent der Erde zeigen sich die Chancen, aber auch die Herausforderungen des 21. Jahrhunderts so klar und so drastisch, wie das auf dem afrikanischen Kontinent der Fall ist. Blicken wir auf die Klimaveränderungen, die Herausforderungen von Bevölkerungswachstum, die Dynamik von Nationen mit einer sehr jungen Bevölkerung – die natürlich einerseits große Wachstumschancen und Entwicklungschancen haben, andererseits, wenn die Staaten aber wiederum nicht in der Lage sind, diesen jungen Menschen eine Zukunftsperspektive zu geben, dann doch zu hoher Migrationsdynamik führen –, die Schwäche staatlicher Strukturen in vielen dieser Staaten, die eben sehr häufig junge Demokratien sind, in denen der Demokratiebegriff noch weiterentwickelt werden muss und dadurch – wenn die Regierungen schwach sind – auch Platz entsteht für terroristische Entwicklung. Dazu gehört nicht zuletzt auch der Versuch von Staaten, diese Staaten von außen zu beeinflussen, indem sie mit scheinbar großzügigen Hilfsangeboten, dann jedoch mit Knebelverträgen diese Länder bereits um die Zukunftschancen bringen, bevor sie sie überhaupt ergreifen können, indem sie sich etwa auf lange Sicht Rechte an Rohstoffen sichern zu Konditionen, die für die Länder jeweils unvorteilhaft sind. Deshalb ist es richtig und wichtig, dass sich Deutschland, die Europäische Union und viele andere Nationen der Welt für diesen afrikanischen Kontinent engagieren – im Hinblick auf die wirtschaftliche Zusammenarbeit und auf die Entwicklung dieser Staaten, aber eben auch im Hinblick auf die Entwicklung der Sicherheitsstrukturen in diesen Staaten. Die fünf Staaten der Sahelregion, vor allem die angrenzenden Staaten südlich davon, am Golf von Guinea, sind ein Stück weit Schlüsselstaaten für den Kontinent Afrika, nicht nur für den gesamten Westen, sondern auch für den Norden und für andere Teile Afrikas. Wenn sich dort tatsächlich Terrorismus breitmachen sollte und Failed States entstehen sollten, dann gäbe es Probleme, die weit über die Region hinausreichen. Deswegen ist es richtig und gut, dass Deutschland sich an der EU-Mission EUTM Mali beteiligt. Es war auch richtig und gut, dass wir als mit der Außenpolitik befasste Abgeordnete uns sehr intensiv mit dem Mandatsvorschlag der Bundesregierung auseinandergesetzt haben; denn er fußt auf einem EU-Beschluss, in dem von einem robusteren Ansatz die Rede ist. „Robusterer Ansatz“ bedeutet nach unserer Erfahrung, dass das Militär dabei möglicherweise eine größere Rolle spielt und auch die Gefährdung für unsere Soldatinnen und Soldaten größer ist. Wir kennen das aus unseren Erfahrungen aus Afghanistan. Uns liegt ein gutes EUTM-Mali-Mandat vor. Insbesondere im Begründungstext wird jetzt die eine oder andere Frage, die wir an die Regierung gerichtet haben, beantwortet. Ich glaube, es ist absolut richtig, dass sich dieser Einsatz nicht mehr auf Mali beschränkt, sondern den Niger-Einsatz der Bundeswehr integriert und eben offenlässt, Burkina Faso, Tschad und Mauretanien mit zu unterstützen. Eigentlich hätte die EU den Einsatz auch „EUTM Sahel“ und nicht nur „EUTM Mali“ nennen können. Bei diesem Einsatz wird die Zahl der eingesetzten deutschen Soldaten erhöht. Ich finde es sehr gut – wir haben dazu mit der Verteidigungsministerin und dem Außenminister intensive Gespräche geführt –, dass wir eine ständige Evaluierung dieses Einsatzes vorsehen.  Im Antrag steht ausdrücklich, dass uns die Regierung nach Ablauf von sechs Monaten einen Zwischenbericht vorlegt, sodass wir schauen können, was entsprechend verbessert werden kann bzw. ob wir den Vorschlägen und Schlussfolgerungen der Regierung entsprechen können. Ich glaube, dass es über das, was bereits festgelegt ist, hinausgehend gut wäre, wenn es, ähnlich wie es im zivilen Hilfsbereich bereits der Fall ist, auf der Ebene der Ausbildungsunterstützung und der Stärkung der staatlichen Strukturen dieser Staaten eine noch stärkere internationale Koordination gäbe, und zwar vor allem aus einem Grund: Ich glaube, es ist total wichtig, dass sich alle die, die Hilfe in der Region leisten, mit klaren Worten und mit einer einheitlichen Stimme an die Regierungen dieser Staaten wenden, sodass diese eine klare Botschaft von uns bekommen und wissen, was wir bereit sind zu tun, was wir aber umgekehrt natürlich auch mit Blick auf die zum Beispiel von uns ausgebildeten Streitkräfte von ihnen erwarten. Ich finde es auch gut, dass wir überprüfen werden, ob die Mittel, die wir der Bundeswehr für diesen Einsatz zur Verfügung stellen, geeignet sind und ob sie gegebenenfalls angepasst werden müssen. Ich finde, dass wir eine starke Debatte über Deutschlands und Europas Rolle im Sahel brauchen, und freue mich deswegen, dass die CDU\/CSU-Bundestagsfraktion – letzter Satz – – Frau Präsidentin, ich sage jetzt meinen letzten Satz, und dann kann der Kollege ja, wenn er möchte, intervenieren; eine Frage will ich nicht zulassen. Das entscheiden Sie. Der letzte Satz ist: Ich freue mich, dass die CDU\/CSU-Bundestagsfraktion gestern ein, wie ich finde, sehr fortschrittliches und vorausschauendes Diskussionspapier zum Thema „Entwicklung im Sahel“ vorgelegt hat, in dem formuliert ist, was zu tun ist. Es ist eine Einladung an alle gutgewillten Kräfte in diesem Hause, sich an dieser Diskussion zu beteiligen, damit wir am Ende des Tages etwas Gutes für die Menschen in der Region machen können. Herzlichen Dank. 

\noindent\textbf{Comment:}
\begin{itemize}
    \setlength\itemsep{-3pt}
    \item (Beifall bei der CDU/CSU sowie bei Abgeordneten der SPD)
    \setlength\itemsep{-3pt}
    \item (Dr. Marie-Agnes Strack-Zimmermann [FDP]: Wird Zeit!)
\end{itemize}
\subsection{Hampel}
\noindent\textbf{Texts:} Vielen Dank, Frau Präsidentin. – Herr Kollege, nach Ihren Ausführungen sind Einsätze in Mali und der Sahelzone an sich überhaupt nicht mehr mit irgendwelchen spezifischen messbaren Fragen zu verbinden. Ich habe von Ihnen jetzt kein einziges Mal gehört, was wir in welchem Zeitraum mit welchen Mitteln mit welchem finanziellen Aufwand wo und wie erreichen wollen. Sie wollen in Zukunft den Einsatz über Mali hinaus auf die Sahelzone ausdehnen; damit haben Sie völlig recht. Da sind wir ja schon; der Minister sagte es gerade. In welche Regionen wollen wir denn noch vordringen? Noch einmal: das alles immer, ohne eine Spezifizierung, ohne Kriterien aufzustellen. Was wollen wir in welchem Zeitraum wo durchsetzen?

\subsection{Hardt}
\noindent\textbf{Texts:} Danke schön, Frau Präsidentin. – Herr Kollege Hampel, ich glaube, ich habe deutlich gemacht, dass wir mit dem, was wir da tun, bereits einen wichtigen Beitrag leisten, nämlich Ausbildungsunterstützung für die Streitkräfte in Mali und Niger. Ich habe auch deutlich gemacht, dass wir uns schon vorstellen, dass in der nächsten Zeit eine klare Festlegung – etwa unsere Erwartung an die Regierung in Mali – formuliert wird, so wie wir es vor vielen Jahren auch gegenüber der afghanischen Regierung gemacht haben, als festgelegt wurde: Wie viele Soldaten, wie viele Polizisten braucht man, und was muss man einsetzen, um die zu bekommen? – Das ist in Konturen bereits klar erkennbar; aber wir versprechen uns zum Beispiel von dem Zwischenbericht in sechs Monaten weitere Erkenntnisse. 

\noindent\textbf{Comment:}
\begin{itemize}
    \setlength\itemsep{-3pt}
    \item (Beifall bei der FDP)
\end{itemize}
\subsection{Lechte}
\noindent\textbf{Texts:} Sehr geehrte Frau Präsidentin! Liebe Kolleginnen und Kollegen! Seit 2003 sind wir mit der Bundeswehr in Mali präsent. Wir können einige Erfolge vorweisen, aber leider verschlechtert sich die Sicherheitslage in jüngster Zeit zusehends. Besonders in den Grenzgebieten zu Burkina Faso und Niger sind terroristische Gruppen verstärkt aktiv, und die malischen Streitkräfte müssen schwere Verluste hinnehmen. Aus diesem Grund kommt der Ausbildungsmission EUTM Mali eine besondere Bedeutung zu. Sie ist zwar kleiner als die große UN-Mission MINUSMA, aber sie hat die wichtige Aufgabe, die Sicherheitskräfte vor Ort so zu unterstützen, dass sie selbst für Sicherheit sorgen können. Das ist aber bisher nicht so gut gelungen. Die Fähigkeiten der malischen Sicherheitskräfte lassen zu wünschen übrig. Das haben wir als FDP auch schon mehrfach kritisiert. Deshalb freut es mich, dass die Bundesregierung nun zumindest einige unserer Kritikpunkte aufgenommen und das Mandat nachgebessert hat. Zum einen: die stärkere Betonung der grenzüberschreitenden Zusammenarbeit der G-5-Sahelstaaten. Das ist der richtige Ansatz zur Bekämpfung des grenzüberschreitenden Terrorismus. Deswegen müssen wir auch unsere verschiedenen Maßnahmen in der Sahelregion stärker verzahnen. Es ist daher richtig, dass wir endlich – endlich! – die Ausbildungsmission Gazelle in Niger in das Mandat von EUTM Mali integrieren.  Das hatte übrigens nicht nur die FDP gefordert, sondern auch der bisherige und heute anwesende Wehrbeauftragte Hans-Peter Bartels.  Er hat uns stets kluge Impulse gegeben, die nicht immer auf der Linie der Bundesregierung waren; denn er hat sein Amt wirklich als Wehrbeauftragter des Bundestags und nicht der Bundesregierung ausgeübt.  Das haben wir sehr geschätzt. Wir bedauern es daher sehr, wie die SPD ihn nun in die Wüste schickt. Seine Nachfolgerin Eva Högl tritt in große Fußstapfen. Ich wünsche ihr viel Erfolg, auch wenn sie unserer Debatte heute fernbleibt und damit zeigt, dass sie unsere Fachexpertise offensichtlich nicht benötigt.  Die Bundesregierung lässt in Mali eine klare Strategie vermissen. Hierfür benötigt man nämlich Ziele und Wege zur Erreichung dieser Ziele. Ein Ziel ist uns allen klar: die Stabilisierung der Sahelregion. Aber der Weg dorthin ist noch unklar. Das zeigt sich exemplarisch an der Mentoring-Problematik. Der UN-Sicherheitsrat – da sitzen wir immerhin drin – und der Rat der Europäischen Union – da sitzen wir auch drin – definieren in ihren Beschlüssen zu Mali Mentoring als die Begleitung von auszubildenden Soldaten im Einsatz. Die Bundesregierung spricht in ihrem Mandatstext jedoch nur von einer Begleitung an gesicherten Orten wie beispielsweise in Kasernen. Was ist dann hier die sensationelle Neuerung zu bisher? Anstatt einer klaren Linie zur Zielerreichung sehen wir hier erneut einen faulen Kompromiss – so leid es mir tut – zwischen CDU\/CSU und SPD, der mehr Ungewissheiten als Klarheiten schafft. Dies wird einer der Punkte sein, die wir im Ausschuss nochmals beraten müssen. Daher stimmen wir als FDP-Fraktion gerne der Überweisung zu. Vielen Dank.  

\noindent\textbf{Comment:}
\begin{itemize}
    \setlength\itemsep{-3pt}
    \item (Beifall bei der FDP)
    \setlength\itemsep{-3pt}
    \item (Beifall bei der FDP sowie des Abg. Armin-Paulus Hampel [AfD])
    \setlength\itemsep{-3pt}
    \item (Beifall bei der FDP und der AfD)
    \setlength\itemsep{-3pt}
    \item (Beifall bei der LINKEN)
\end{itemize}
\subsection{Buchholz}
\noindent\textbf{Texts:} Frau Präsidentin! Meine Damen und Herren! Vor mehr als sieben Jahren hat die Bundesregierung deutsche Soldaten nach Mali geschickt. Seitdem wurde kein einziges Versprechen, das mit dem internationalen Militäreinsatz verbunden war, eingelöst. Im Gegenteil: Seitdem gibt es immer mehr blutige Konflikte. Die Bundesregierung lässt die Bundeswehr immer tiefer in einen Krieg hineinschlittern, der lange dauern wird und nicht gewonnen werden kann. Das zeigt: Sie haben aus dem Afghanistan-Desaster nichts gelernt.  Das Scheitern der Ziele hält die Bundesregierung aber nicht davon ab, den Militäreinsatz in Mali Jahr für Jahr immer noch weiter auszudehnen und robuster zu machen. So auch dieses Jahr: Die Kosten für die deutschen Militärausbilder der europäischen Trainingsmission EUTM Mali werden mehr als verdoppelt. Das Einsatzgebiet wird versiebenfacht. Bislang war es auf halb Mali begrenzt. Jetzt soll es ganz Mali sowie die Staatsgebiete Burkina Fasos, Nigers, Mauretaniens und Tschads umfassen. Sie verlieren wirklich jedes Maß, wenn es darum geht, deutsche Militärpräsenz auf dem afrikanischen Kontinent auszuweiten.  Zusammen mit der französischen Regierung zieht die Bundesregierung immer mehr Länder in den Konflikt hinein. Aus EU-Mitteln wird die G‑5-Saheltruppe bezahlt, die als Art Hilfstruppe gegen Aufständische vorgehen soll. Diese Methode schürt die Konflikte. Laut Human Rights Watch haben Soldaten Burkina Fasos in der Stadt Djibo vor einem Monat im Rahmen einer sogenannten Antiterroraktion 31 Menschen verhaftet und ohne Gerichtsverfahren hingerichtet. Und nun sollen mit dem neuen Mandat Bundeswehrausbilder und ‑berater genau diese Truppen beraten. So treiben Sie die Militarisierung der Sahelzone voran. Es ist doch Wahnsinn, zu glauben, dass mit einem robusteren Einsatz die wirtschaftlichen, sozialen und politischen Probleme, die den Konflikten in der Sahelzone zugrunde liegen, erfolgreich gelöst werden können.  Noch etwas: Die Bundesregierung lobt in ihrem Antrag den sogenannten inklusiven nationalen Dialog der malischen Regierung mit zahlreichen politischen Kräften. Nur verschweigt sie dabei einen wichtigen Punkt: Der malische Präsident Keïta hat sich im Zusammenhang mit dem inklusiven Dialog explizit für Verhandlungen mit den aufständischen Truppen ausgesprochen, die bisher nicht in den Friedensprozess einbezogen sind. Der wichtigste Rebellenführer Ag Ghali hat ebenfalls seine Verhandlungsbereitschaft erklärt, unter der Voraussetzung des Abzugs der internationalen Truppen. Darüber reden Sie hier nicht. Offenbar geht es nicht um Mali, sondern um die Präsenz der Bundeswehr in der Region. Die Linke sagt: Ziehen Sie die Bundeswehr endlich ab, aus Mali und der gesamten Sahelzone.  

\noindent\textbf{Comment:}
\begin{itemize}
    \setlength\itemsep{-3pt}
    \item (Beifall beim BÜNDNIS 90/DIE GRÜNEN)
    \setlength\itemsep{-3pt}
    \item (Beifall bei der LINKEN – Henning Otte [CDU/CSU]: Meine Güte!)
    \setlength\itemsep{-3pt}
    \item (Beifall bei der LINKEN)
\end{itemize}
\subsection{Trittin}
\noindent\textbf{Texts:} Frau Präsidentin! Meine Damen und Herren! Anders als Christine Buchholz habe ich, obwohl wir damals schon in der Opposition waren, von Anfang an diese Mali-Mission unterstützt. Wir halten es für richtig, den aus Libyen fortschreitenden Staatszerfall dieser Region aufzuhalten. Dazu bedarf es eines integrierten Ansatzes, und zu diesem integrierten Ansatz gehört auch Militär. Aber gerade wenn man das so sagt, dann darf man sich nicht wegducken vor der Entwicklung der letzten acht Jahre. Und da müssen wir ganz realistisch sehen, dass diese Entwicklung keine gute ist. Die Situation ist schlechter geworden. Der politische Prozess in Mali stockt. Wir loben einen Dialog, während einer der Oppositionsführer entführt wurde und bis heute nicht wiedergefunden wurde. Wir erleben, dass der Konflikt innerhalb Malis sich vom Norden in die Mitte und in den Süden verschoben hat, wir erleben, dass er sich erweitert hat auf andere Länder, auf Burkina Faso, auf Niger, und wir erleben, dass in der gesamten Region das Fortschreiten von terroristischen Anschlägen dazu führt, dass immer mehr Menschen vertrieben werden. Aber es sind nicht nur Vertreibungen aufgrund von Anschlägen. Es finden auch Vertreibungen von Menschen statt, weil es in bestimmten Regionen Free Fire Zones gibt, die verhindern, dass Menschen dort Handel und Ackerbau treiben. Was erleben wir also? Wir erleben etwas, was ich schon einmal erlebt habe in einem Einsatz, nämlich in einem Gebiet in Afghanistan: Wir erleben das Nebeneinanderher von zwei militärischen Doktrinen. Die eine Doktrin, die wir aufgebracht haben, ist: Wir bilden einen Staat aus, wir bilden Armeen aus, damit Staatsbildung stattfinden kann, um Institutionen zu schaffen, die Vertrauen haben. Die andere ist die Logik der Aufstandsbekämpfung, des Terrorismusbekämpfens. Und diese beiden Logiken gehen offensichtlich nicht zusammen.  Wir haben diesen Einsatz mit Frankreich begonnen aus fester Überzeugung, dass es einer europäischen Antwort bedarf. Die Bundesregierung sagt aber selber offen – das ist nicht einmal streitig –, dass sie an dieser Stelle einen Konflikt mit Frankreich hat.  Aber wie löst sie diesen Konflikt? Werden diese unterschiedlichen Einsatzdoktrinen irgendwo thematisiert? Auf der einen Seite eine Operation Barkhane mit einer Koalition der Willigen, zu der auch der Tschad, ein Land einer autokratischen Diktatur, gehört, und auf der anderen Seite die beiden Missionen EUTM und MINUSMA. Mein Eindruck ist: Sie wollen sich diesem Konflikt nicht stellen. Sie wollen dieses Nebeneinander nicht auflösen zugunsten des integrativen Ansatzes, sondern Ihre, wie Sie es nennen, ambitionierte Weiterentwicklung heißt eigentlich: Wir passen uns der Strategie von Barkhane an. Jetzt bilden wir da mehr aus.  Wir schließen nicht einmal aus, Truppen für den Tschad auszubilden. Ich finde, das ist eine schlechte Weiterentwicklung.  Ich finde, was wir brauchen, ist eine Rückkehr zu dem politischen Prozess, zu einem integrierten zivil-militärischen Ansatz. Das vermissen wir in diesem Mandat, und wir würden uns wünschen, dass das in diesem Sinne verändert wird.  

\noindent\textbf{Comment:}
\begin{itemize}
    \setlength\itemsep{-3pt}
    \item (Tobias Pflüger [DIE LINKE]: So ist es!)
    \setlength\itemsep{-3pt}
    \item (Beifall beim BÜNDNIS 90/DIE GRÜNEN – Dr. Marie-Agnes Strack-Zimmermann [FDP]: Erzählen Sie das mal Ihrem Ex-Außenminister!)
    \setlength\itemsep{-3pt}
    \item (Beifall beim BÜNDNIS 90/DIE GRÜNEN)
    \setlength\itemsep{-3pt}
    \item (Gerold Otten [AfD]: Richtig!)
    \setlength\itemsep{-3pt}
    \item (Beifall beim BÜNDNIS 90/DIE GRÜNEN sowie bei Abgeordneten der LINKEN)
    \setlength\itemsep{-3pt}
    \item (Beifall bei der CDU/CSU)
\end{itemize}
\subsection{Nick}
\noindent\textbf{Texts:} Frau Präsidentin! Liebe Kolleginnen und Kollegen! Ja, die Region und Mali beschäftigen uns seit Langem. Die Mission EUTM Mali war bereits Gegenstand meiner allerersten Rede überhaupt in diesem Hause vor etwas mehr als sechs Jahren. Heute müssen wir in aller Nüchternheit miteinander feststellen: Trotz allen internationalen Engagements hat sich die Sicherheitslage in der Sahelregion Jahr für Jahr weiter verschlechtert. Daraus ziehen wir heute Konsequenzen. Wir nehmen künftig über Mali hinaus die gesamte Sahelregion mit den Staaten Burkina Faso, Mauretanien, Niger und Tschad in den Blick. Auch die bislang bilaterale Bundeswehrausbildungsmission Gazelle mit Niger wird in das Mandat integriert. Und wir erweitern das Mandat um das sogenannte Mentoring. Künftig ist damit auch eine Begleitung der regionalen Streitkräfte durch die Bundeswehr bis zur taktischen Ebene möglich, allerdings ohne Exekutivbefugnisse und an gesicherten Orten. Dies sollte uns aber nicht darüber hinwegtäuschen, dass der Einsatz im Sahel einschließlich der Mission MINUSMA nicht nur der faktisch größte, sondern auch der gefährlichste Auslandseinsatz der Bundeswehr ist und bleibt. Deshalb danken wir unseren Soldatinnen und Soldaten ganz herzlich für ihren Einsatz, und wir wünschen gutes Gelingen und vor allem sichere Heimkehr.  Liebe Kolleginnen und Kollegen, EUTM Mali ist eine europäische Mission. Die Stabilisierung der Sahelzone ist eine gemeinsame europäische Priorität, und unser erweitertes Engagement ist daher auch ein wichtiges Signal, vor allem an unsere französischen Freunde, die in der Sahelregion bisher die Hauptlast getragen haben und weiter tragen. Auch wenn die größte Bedrohung der Sicherheit in der Region aktuell von den terroristischen Gruppierungen ausgeht: Das grundlegende Problem für die regionale Stabilität bleiben fragile staatliche Strukturen verbunden mit Armut und Perspektivlosigkeit einer sehr jungen und wachsenden Bevölkerung. Das ist dann auch der Nährboden für Terrorismus. Liebe Kolleginnen und Kollegen, es darf daher nicht allein um die Ertüchtigung der regionalen Sicherheitssektoren gehen, sondern in der ganzen Breite um staatliche Präsenz und Handlungsfähigkeit in allen Dimensionen als unabdingbare Voraussetzung für die Chance auf nachhaltige Entwicklung in der Region.  Ich bin jetzt fast am Ende.  Dann machen wir es lieber als Kurzintervention.  Die Ende April auf den Weg gebrachte internationale Koalition für den Sahel ist deshalb ein wichtiger Schritt. Auch hier im Deutschen Bundestag werden wir diesen Einsatz und die breitere Strategie für den Sahel weiterhin aufmerksam begleiten und im Hinblick auf die beschriebenen Ziele evaluieren und gegebenenfalls weiterentwickeln. Vielen Dank.  

\noindent\textbf{Comment:}
\begin{itemize}
    \setlength\itemsep{-3pt}
    \item (Jürgen Trittin [BÜNDNIS 90/DIE GRÜNEN]: Darüber entscheidet immer noch die Frau Präsidentin!)
    \setlength\itemsep{-3pt}
    \item (Beifall bei der CDU/CSU)
    \setlength\itemsep{-3pt}
    \item (Matthias W. Birkwald [DIE LINKE]: Nur der Redezeit, Herr Nick! Nur der Redezeit!)
    \setlength\itemsep{-3pt}
    \item (Abg. Ulrich Lechte [FDP] meldet sich zu einer Zwischenfrage)
    \setlength\itemsep{-3pt}
    \item (Beifall bei der CDU/CSU sowie bei Abgeordneten der SPD und der FDP)
    \setlength\itemsep{-3pt}
    \item (Heiterkeit bei der CDU/CSU, der SPD, der FDP und dem BÜNDNIS 90/DIE GRÜNEN)
\end{itemize}
\subsection{Lechte}
\noindent\textbf{Texts:} Vielen Dank, Frau Präsidentin, für die Güte der Gestattung einer Kurzintervention. – Mir fehlt heute ein Thema, das in der verkürzten Debatte noch nicht angesprochen wurde, und der Kollege Nick ist mit Sicherheit bereit, mir auf eine Frage dazu vonseiten der CDU Antwort zu geben.  Es geht um das Thema „Sahelregion und humanitäre Hilfe“. Bei aller Liebe für unseren sicherheitspolitischen Einsatz dort unten haben wir im Jahr 2019 bei mittlerweile 6 Millionen auf Hungerhilfe angewiesenen Menschen in Mali 20 Millionen Euro – das steht auch im Mandatstext – ausgegeben; dieses Jahr sind 32 Millionen Euro geplant. Bei einer gesamthumanitären Hilfe von 1,7 Milliarden Euro weltweit bedeutet das, dass wir in diese Region, auf die wir so viel Aufmerksamkeit legen, gerade einmal 1,2 Prozent unserer humanitären Hilfe hineingeben. Es wundert mich nicht, dass die Staaten dort unten aufgrund der Bevölkerungsproblematik und der Bevölkerungsexplosion nicht mehr in der Lage sind, ihrer staatlichen Verantwortung gerecht zu werden, und ich würde dazu gerne einmal die Einschätzung der Unionsfraktion hören.

\noindent\textbf{Comment:}
\begin{itemize}
    \setlength\itemsep{-3pt}
    \item (Michael Grosse-Brömer [CDU/CSU]: Wenn man schon geredet hat, ist man selbst schuld!)
\end{itemize}
\subsection{Nick}
\noindent\textbf{Texts:} Also, ich will jetzt nicht um Nachkommastellen in Haushaltsansätzen diskutieren. Ich glaube, ich habe deutlich gemacht, dass wir einen umfassenden Ansatz für diese Region brauchen und dass natürlich die Herstellung einer fundamentalen Sicherheitslage an vielen Stellen die Voraussetzung ist, um überhaupt mit humanitärer Hilfe an Ort und Stelle zu kommen. Aber was wir im Endeffekt dort erreichen müssen, ist die Chance auf eigenständige, nachhaltige Entwicklung, bei der sowohl die Staaten Verantwortung für ihre Sicherheit übernehmen als auch die Menschen in der Region die Chance bekommen, ihr Leben eigenverantwortlich in die Hand zu nehmen. Dazu tragen wir in der gesamten Breite mit unserem Engagement bei. Dort können wir sicherlich an der einen oder anderen Stelle auch noch einmal nachsteuern; auch das habe ich in meiner Rede angesprochen.  

\noindent\textbf{Comment:}
\begin{itemize}
    \setlength\itemsep{-3pt}
    \item (Beifall bei der CDU/CSU)
\end{itemize}
\subsection{Erndl}
\noindent\textbf{Texts:} Frau Präsidentin! Kolleginnen und Kollegen! Afrika als Kontinent der Chancen – ich darf da an die Worte vom Kollegen Nick anknüpfen –: Das ist das Bild, das viele Initiativen auch in diesem Hause prägen wollen. Wir wollen Afrika so weit entwickeln, dass in Eigenverantwortung Perspektiven für junge Menschen entstehen, dass aus eigener Kraft die Herausforderungen der Zukunft gelöst werden können; das bleibt unsere Vision. Aber natürlich sind wir in der Realität in vielen Regionen davon noch sehr weit entfernt, und es bedarf eines langen Atems, hier auf eine positive Entwicklung hinzuwirken. Ein Blick auf die Sahelregion zeigt die Schwierigkeiten: hohes Bevölkerungswachstum, fehlende staatliche Strukturen, kaum vorhandene Bildungs- oder – jetzt aktuell besonders sichtbar – Gesundheitseinrichtungen – um nur einige Stichpunkte zu nennen. So entstanden Rückzugsräume für terroristische Gruppierungen, ein Zentrum für Menschenhandel, illegale Migrationsrouten und Umschlagplätze für Drogenhandel. Die AfD fragt in ihrem Antrag: „Was geht uns das alles an?“, stellt viele Fragen, bringt aber letztendlich keinen einzigen Lösungsansatz. Wie wollen Sie Terrorismus bekämpfen, Migration verhindern, wirtschaftliche Entwicklung vorantreiben, wenn Sie letztendlich vor dieser Entwicklung die Augen verschließen? Meine Kolleginnen, meine Kollegen, die Unionsfraktion hat in dieser Woche mit einem Positionspapier die Sahelregion in den Blick genommen. Es liegt in unserem unmittelbaren Interesse, dass wir uns hier engagieren. Es liegt im unmittelbaren Interesse Europas, wenn wir hier aktiv sind, damit wir die grundlegenden Sicherheitsstrukturen schaffen können, damit, darauf aufbauend, Entwicklung stattfinden kann, damit Perspektiven entstehen können. Aber die Sicherheitslage – es wurde angesprochen – ist in der Region volatil. Islamistische Terrorgruppen sind nicht nur wegen der zunehmenden Zahl von Anschlägen für die fragile Sicherheitslage verantwortlich; sie profitieren auch umgekehrt von ihr, weil sie sie ausnutzen und die Instabilität und wirtschaftliche Perspektivlosigkeit der jungen Bevölkerung für Rekrutierungszwecke nutzen. Das ist und bleibt eine große Bedrohung. Deswegen ist es wichtig und richtig, dass wir uns in der Mission EUTM Mali weiter engagieren, dass wir einen Beitrag zur Stärkung der Sicherheitsstrukturen leisten und zusammen mit den Ertüchtigungsinitiativen, mit der Materialausstattung eine Hilfe zur Selbsthilfe geben. Wir haben bisher viel geleistet. Wir haben die Armee Malis von wenigen Tausend auf jetzt circa 30 000 Soldatinnen und Soldaten aufgebaut, davon viele ausgebildet. Ich danke allen Soldatinnen und Soldaten aus der Bundeswehr, die hier einen wichtigen Beitrag geleistet haben. Aber eines ist auch klar: Die Mission kann weder alle sozialen, politischen noch wirtschaftlichen Probleme lösen. Die Mission ist nur ein Baustein im Rahmen unseres vernetzten Ansatzes, der viele Facetten umfasst. Aber zusammen mit MINUSMA, EUCAP Sahel Mali und der G 5 Sahel Joint Force ist sie ein wichtiger militärischer Baustein. Deswegen ist es richtig, dass wir das Mandat verlängern, dass wir es an die aktuellen Entwicklungen anpassen und die Erkenntnisse mit aufnehmen. Es ist wichtig, dass wir unser Engagement ausweiten, weil das, glaube ich, deutlich macht, dass wir mehr Verantwortung in der Region übernehmen wollen. Es ist eine wichtige Botschaft – auch mit Hinblick auf die EU-Ratspräsidentschaft –, dass wir da mit gutem Beispiel vorangehen. Herzlichen Dank. 

\noindent\textbf{Comment:}
\begin{itemize}
    \setlength\itemsep{-3pt}
    \item (Beifall bei der CDU/CSU)
\end{itemize}
\section{Tagesordnungspunkt 1}
\subsection{Merkel}
\noindent\textbf{Texts:} Herr Präsident! Liebe Kolleginnen und Kollegen! Ich freue mich, dass wir heute die ursprünglich ja schon für Ende März geplante Regierungsbefragung mit mir nachholen können. Ich hatte ja bereits vor gut drei Wochen, am 23. April, die Gelegenheit, Ihnen in einer Regierungserklärung meinen und unseren Kurs in dieser Pandemie aufzuzeigen. In der Pandemie und mit dem Virus leben wir immer noch, und das wird auch für längere Zeit so bleiben. Denn die grundlegenden Fakten haben sich ja nicht geändert. Auch wenn einzelne Forschungsansätze Hoffnung machen: Noch gibt es keine Medikamente und keinen Impfstoff gegen das Virus. Corona ist und bleibt also eine Gefahr für jede und jeden von uns. Eine Coronainfektion kann in vielen Fällen ganz harmlos verlaufen; sie kann aber auch zu schweren Schäden bis zum Tod führen. Und ich denke an all die Menschen, die wir an dieses Virus verloren haben, und ich denke an ihre Familien. Das sind die unveränderten Fakten. Aber wir stehen natürlich nicht still in dieser Pandemie. In den drei Wochen seit meiner Regierungserklärung ist einiges geschehen, haben wir einiges geschafft, das uns Mut machen kann. Wir waren uns von Anfang an einig: Stoppen können wir die Ausbreitung des Virus derzeit nicht. Aber sie so weit wie möglich verlangsamen, das können und das müssen wir. Und das ist ja auch unser gemeinsames Ziel. Wir können sagen, dass uns in den vergangenen Wochen und Monaten in einer gemeinsamen enormen Anstrengung gelungen ist, dieses Ziel zu erreichen oder ihm näher zu kommen. Es ist gelungen, weil wir alle – Bürger, Wirtschaft, Staat – in einer schweren Zeit und unter schweren Einschränkungen zusammengehalten haben. Die Zahlen der Neuinfektionen, die das Robert-Koch-Institut uns täglich meldet, die liegen jetzt in einem Bereich, mit dem unser Gesundheitssystem zurechtkommen kann. Und es ist wieder einigermaßen möglich, Infektionsketten nachzuverfolgen und durch Tests und Quarantänemaßnahmen diese Ketten dann auch zu durchbrechen. Die derzeit entwickelte Tracing-App wird diese Nachverfolgung hoffentlich bald zusätzlich verbessern können. Wir haben die letzten Wochen genutzt, um unser Gesundheitssystem an vielen Stellen auch zu stärken. Ich habe großen Respekt vor der Leistung der Länder und Kommunen. Ich danke auch dem Deutschen Bundestag, dass mithilfe des Bundes Kapazitäten, vor allen Dingen für Intensivbetten, erhöht werden konnten. Ich sehe die fabelhaften Leistungen, die im Öffentlichen Gesundheitsdienst, in den Gesundheitsämtern täglich erbracht werden, und ich bin auch froh, dass wir dort sowohl durch die Länder als auch durch den Bund Unterstützung geben konnten. Denn das sind genau die Orte, an denen die Infektionsketten nachvollzogen werden müssen, und davon hängt so vieles ab. Uns ist die internationale Zusammenarbeit gegen das Virus herausragend wichtig: im Kreis der Europäischen Union, unter den G-7-Staaten und den G-20-Ländern. Liebe Kolleginnen und Kollegen, wir können froh sein, dass wir diese letzten Wochen alle zusammen so gemeistert haben. Ich sehe darin aber auch eine Verpflichtung: die Verpflichtung, das gemeinsam Erreichte jetzt nicht zu gefährden. Wir haben doch nicht seit März alle vorher undenkbaren Einschränkungen in unserem Leben, in unserem Arbeiten und Wirtschaften, auch zeitweilige Einschränkungen unserer Rechte auf uns genommen, um jetzt, weil wir die Vorsicht ablegen, einen Rückfall zu riskieren. Es wäre doch deprimierend, wenn wir, weil wir zu schnell zu viel wollen, wieder zu Einschränkungen zurückkehren müssten, die wir alle hinter uns lassen wollen. Lassen Sie uns also mutig und wachsam sein! Lassen Sie uns schrittweise das öffentliche und wirtschaftliche Leben wieder öffnen und dabei die Entwicklung der Pandemie immer im Blick haben! Und vor allem: Lassen Sie uns an Arbeitsplätze, in Schulen, in Cafés und in die Sportvereine zurückkehren und dabei die neuen Grundregeln weiter beachten: Mindestabstand halten, Hände waschen, Respekt vor dem Schutzbedürfnis auch unserer Mitmenschen. Ich bin fest überzeugt: Wenn wir konsequent bleiben und so einen Rückfall verhindern, haben wir alle mehr davon. Denn dann ist auch in der nächsten Phase der Pandemie beides möglich: der Schutz der Gesundheit der Menschen und ein Vorgehen, das unsere Wirtschaft so schnell wie möglich wieder aufholen lässt und damit so viele Arbeitsplätze wie möglich sichert und ein gesellschaftliches Leben, an dem wir Freude haben können. Herzlichen Dank. 

\noindent\textbf{Comment:}
\begin{itemize}
    \setlength\itemsep{-3pt}
    \item (Beifall bei der CDU/CSU und der SPD)
\end{itemize}
\subsection{Chrupalla}
\noindent\textbf{Texts:} Sehr geehrte Frau Bundeskanzlerin, Sie sind jetzt auf die pandemischen Probleme eingegangen. Sie sind natürlich auf die wirtschaftlichen Folgen, was dieses Land betrifft, überhaupt nicht eingegangen. Deswegen meine Frage bzw. erst mal meine Einleitung dazu: Ihre Coronapolitik – auch die Coronapolitik der Bundesregierung – vernichtet aktuell in Deutschland über 2 Millionen Menschen ihre Existenzen. Wir haben über 10 Millionen Kurzarbeiter. Die Rücklagen der Sozialkassen schmelzen dahin; man geht von 14 Milliarden Euro bis Ende des Jahres aus. Wir haben über 300 000 mehr Arbeitslose als im Vormonat. Das Bruttoinlandsprodukt ist um 6,3 Prozent zusammengebrochen. Pro Woche kostet uns dieser Lockdown 42 Milliarden Euro. Die größte Rezession der Nachkriegsgeschichte, so titeln einige Zeitungen. Deswegen meine Frage: Wo ist eigentlich Ihr – Ihres und das der Bundesregierung – Finanzierungskonzept für nach dieser Krise, und können Sie hier und heute auch ausschließen, dass der Bürger durch erhöhte Abgaben und Steuern diese Kosten übernehmen und auch bezahlen muss?

\subsection{Merkel}
\noindent\textbf{Texts:} Erst einmal bin ich sehr froh, dass uns diese Pandemie in einer wirtschaftlich und auch, was die Haushaltspolitik und die Situation der Sozialversicherungen anbelangt, guten Situation ereilt hat. Das heißt, wir konnten sehr viele Sicherungsmaßnahmen durchsetzen. Ich bin auch sehr dankbar, dass die Mehrheit dieses Parlaments das ja auch getan hat durch die Abstimmungen. Dass Menschen in Kurzarbeit sind, zeigt, dass wir natürlich wirtschaftliche Folgen dieser Pandemie haben. Aber es zeigt, dass wir die Menschen nicht ins Nichts entlassen mussten,  sondern dass wir helfen können und dass wir Brücken bauen können. Und das wird auch weiter unser Ziel sein. Ich glaube, wenn wir die Dinge gut miteinander vereinbaren – wir können alle nicht voraussehen, wann wir einen Impfstoff haben, wann wir ein Medikament haben –, dann haben wir die Chance, es gut zu bewältigen. Aber ich sage nicht, dass niemand etwas merken wird. Wir werden alles daransetzen, dass möglichst viele Bürgerinnen und Bürger schnell an ihren Arbeitsplatz zurückkehren können, aber eben so, dass es verantwortbar ist.

\noindent\textbf{Comment:}
\begin{itemize}
    \setlength\itemsep{-3pt}
    \item (Beifall bei der CDU/CSU und der SPD sowie bei Abgeordneten der FDP)
\end{itemize}
\subsection{Chrupalla}
\noindent\textbf{Texts:} Ja. – Auf die konkrete Frage, ob Steuererhöhungen und Abgabenerhöhungen infolge der Kosten jetzt noch erfolgen, habe ich jetzt keine Antwort von Ihnen bekommen, deswegen die Nachfrage dazu, ob das auch geplant ist und in welcher Form. Natürlich sind Existenzen vernichtet worden. Was sagen Sie dem Gastronomen, dem Inhaber im Prinzip, der sein Geschäft aufgeben musste, der in Konkurs gehen musste? Der ist ja von staatlichen Hilfen weitestgehend ausgenommen worden.

\subsection{Merkel}
\noindent\textbf{Texts:} Nein, das ist er nicht; das wissen Sie ja auch. Es gibt Liquiditätshilfen, es gibt die Kurzarbeit für die Menschen,  und wenn jetzt die Restaurants wieder aufmachen, dann wird es auch eine Mehrwertsteuersenkung geben. Wir werden natürlich, angepasst, auch immer schauen: Wo müssen wir noch helfen? Wir haben nie gesagt, dass wir damit schon am absoluten Ende sind. Wir werden auch etwas zur Stimulierung der Wirtschaft tun. Ich kann Ihnen nur sagen: Stand heute sind keinerlei Erhöhungen von Abgaben und Steuern geplant.  – Es gehört ja zur Politik dazu, dass wir immer zum aktuellen Zeitpunkt antworten; sonst wären wir ja Zukunftsvorherseher, und das maße ich mir nicht an.

\noindent\textbf{Comment:}
\begin{itemize}
    \setlength\itemsep{-3pt}
    \item (Lachen bei Abgeordneten der AfD – Zuruf von der AfD: Stand heute!)
    \setlength\itemsep{-3pt}
    \item (Carsten Schneider [Erfurt] [SPD]: Direktzuschüsse! Das weiß er doch!)
\end{itemize}
\subsection{Spiering}
\noindent\textbf{Texts:} Sehr geehrte Frau Bundeskanzlerin, „Der Spiegel“ berichtet ausführlich über die Zustände bei den Saisonarbeitskräften. Mein eigener Landkreis ist stark betroffen. Es herrscht partiell eine unübersichtliche Datenlage. Sehr geehrte Frau Bundeskanzlerin, sowohl die als Saisonarbeitskräfte in der Landwirtschaft als auch die als Werksarbeiter in der Fleischindustrie Beschäftigten arbeiten seit Jahren unter unzumutbaren Zuständen. Enge Platzverhältnisse und mangelnde Hygiene haben inzwischen zu einer Vielzahl von Coronainfizierten geführt. Welche konkreten Maßnahmen plant die Bundesregierung zusammen mit den Bundesländern, damit die Missstände schnellstmöglich beseitigt werden? Stimmen Sie mit der SPD-Fraktion überein, dass wir eine grundlegende Reform dieser Arbeitsbereiche benötigen, um wettbewerbsfähige Arbeitsplätze zu erhalten, die Gesundheitsschutz, faire Löhne und soziale Mindeststandards gewährleisten?

\subsection{Merkel}
\noindent\textbf{Texts:} Die Bundesregierung hat ja bezüglich der Saisonarbeitskräfte ein Konzept verabschiedet, das auch mit dem Robert-Koch-Institut abgestimmt ist und das den Arbeitgebern in der Landwirtschaft sehr viel stärkere Auflagen auferlegt, als das in früheren Jahren der Fall war – angepasst natürlich sowohl an den Arbeitsschutz im Allgemeinen als auch an den in der Pandemie. Wir haben jetzt über die Situation in der Fleischindustrie noch mal erschreckende Nachrichten bekommen; Sie werden darüber ja auch gleich in der Aktuellen Stunde debattieren. Die Bundesregierung beabsichtigt, auch hierzu notwendige Änderungen zu beschließen. Nächste Woche Montag wird der Bundesarbeitsminister ein Konzept dazu vorlegen. Ich hoffe, dass wir uns dann auch einigen können. Gerade bei der Unterbringung gibt es erhebliche Mängel; das haben wir jetzt ja alle mitbekommen. Es muss dann geschaut werden, wer da in die Verantwortlichkeit genommen wird. Ich kann Ihnen jedenfalls sagen, dass auch ich nicht zufrieden bin mit dem, was wir da jetzt gesehen haben. Für die Kontrolle der Umsetzung dieser Dinge – auch für die Saisonarbeitskräfte – sind natürlich jeweils dann auch die vor Ort zuständigen Behörden zuständig; und die Arbeitgeber eben dafür, dass sie sich an die Bestimmungen halten. Wo das nicht der Fall ist – das hat die Bundeslandwirtschaftsministerin auch gesagt –, muss gehandelt werden.

\subsection{Spiering}
\noindent\textbf{Texts:} Ja, gerne. – Frau Bundeskanzlerin, ich würde gerne darauf eingehen. Die Arbeits- und Sozialministerkonferenz hat in einer Protokollerklärung genau auf diese Datenlage hingewiesen. Acht Bundesländer haben sie gegengezeichnet, indem sie die Bundesregierung auffordern, dafür Sorge zu tragen, dass hinsichtlich der Daten, die bis jetzt nicht an die Gesundheitsämter geliefert worden sind, sichergestellt wird, dass diese Datenlage geklärt wird. Zurzeit wird das über den Deutschen Bauernverband organisiert.

\subsection{Merkel}
\noindent\textbf{Texts:} Also, wenn die Bundesländer nicht selber zuständig sind, die Daten herbeizuführen. Manchmal warten auch wir als Bundesregierung auf Daten. Es ist jetzt nicht so, dass immer nur die Bundesregierung Daten liefern muss. Aber wo wir Daten beibringen können, werden wir uns auch darum bemühen, das zu tun. Was die Zahl der Saisonarbeiter anbelangt, haben wir ja sowieso ein festes Kontingent. Da ist eine feste Zahl klar. Da ist auch klar, wo die Leute hingegangen sind. Ich glaube, da müsste die Datenlage klar sein; in der Fleischindustrie ist das vielleicht nicht ganz so.

\subsection{Höferlin}
\noindent\textbf{Texts:} Vielen Dank, Herr Präsident. – Sehr geehrte Frau Bundeskanzlerin, wir haben von Ihnen gerade gehört, wie wichtig die Corona-Tracing-App für die Nachverfolgung von Kontakten, auch von Kontaktketten, von Infektionsketten sein kann. Jetzt ist es ja so, dass nicht erst seit gerade eben an einer solchen App gearbeitet wird, sondern daran wird nunmehr eigentlich seit fast sieben Wochen oder sogar schon über sieben Wochen gearbeitet, wenn man vom Beginn ausgeht. Diese Corona-Tracing-App ist, soweit ich das verstanden habe, federführend im Gesundheitsministerium angesiedelt. Jetzt hat man von Ihnen als Bundesregierung gehört: Das wird jetzt an ein Konsortium aus SAP und T-Systems vergeben. – Ich würde gerne von Ihnen wissen, warum die Entwicklung dieser Gesundheits-App faktisch ausgerechnet im Kanzleramt – jetzt vor allen Dingen auch unter starker Einbindung des Bundespresseamts – geführt wird. Gibt es bestimmte Gründe dafür, dass hier jetzt neue Maßnahmen getroffen werden, wie die App vorangetrieben wird? Wir haben heute gehört, dass sie im Juni fertig werden soll. Das ist ja kein wirkliches Vorantreiben.

\subsection{Merkel}
\noindent\textbf{Texts:} Der Kanzleramtsminister war ja gestern bei Ihnen. Ich weiß nicht, ob Sie gar nicht über die App gesprochen haben; aber wir können es hier auch noch mal tun. Das Kanzleramt hat im Allgemeinen eine koordinierende Funktion bei allen Projekten der Bundesregierung. Hier ist das Gesundheitsministerium zuständig, für Fragen des Datenschutzes ist das Bundesinnenministerium zuständig, und das Kanzleramt hat sich auch darum gekümmert. Sie wissen, dass wir verschiedene Ansätze verfolgen könnten. Wir haben uns jetzt für den dezentralen Ansatz entschieden, Frankreich zum Beispiel für einen zentralen Ansatz. Es ist erst mal eine allgemeine Architektur entwickelt worden. Wir haben uns für den dezentralen Ansatz entschieden, weil wir glauben, dass es dafür eine sehr viel höhere Akzeptanz gibt. Dass solche Entwicklungsarbeiten dauern können, ist jetzt nichts Neues. Diese App muss einen großen Vertrauensbeweis liefern. Ich bin sehr froh, dass Telekom und SAP mit in dieses Projekt eingestiegen sind. Wir brauchen die Schnittstellen von den Betriebssystembetreibern für die Mehrheit der Handys; die werden erst in diesen Tagen bereitgestellt. Ich glaube, dass die Arbeiten jetzt mit Hochdruck laufen. Aber auch hier gilt für mich natürlich: Gründlichkeit muss gewährleistet sein, und Datenschutz muss gewährleistet sein. – Es sind alle Datenschutzbehörden, sowohl das BSI als auch der Datenschutzbeauftragte, in die Arbeit mit eingebunden.

\subsection{Höferlin}
\noindent\textbf{Texts:} Gerne, ja. – Sie sind darauf eingegangen, dass das Kanzleramt koordiniert. Würden Sie mir oder uns vielleicht trotzdem erklären, warum über die Koordinierungsrolle hinaus plötzlich innerhalb des Kanzleramtes Dinge entschieden werden? Können Sie bestätigen, dass das so ist, dass zum Beispiel das Design der App, die Farbgebung und der Name der Corona-Warn-App jetzt plötzlich im Kanzleramt entschieden werden und nicht von denjenigen, die die App entwickeln, zum Beispiel von Telekom oder SAP? Hat das irgendeinen bestimmten Grund? Vielleicht können Sie auch etwas dazu sagen, wie lange wir noch warten müssen. Würden Sie sagen, dass die Corona-App noch vor einem Impfstoff kommen kann, oder kommt sie eher danach?

\subsection{Merkel}
\noindent\textbf{Texts:} Ich glaube schon, dass sie noch vor einem Impfstoff kommen kann. Also, Wunder können eigentlich in Sachen Impfstoff nicht passieren. Ich glaube, sie kommt vorher; das kann ich zusagen. Haben sich SAP und Telekom beschwert, dass wir auf ihre Designvorschläge nicht eingehen?  Das Bundespresseamt ist sozusagen für die Bewerbung von Produkten, hinter denen die Bundesregierung steht, zuständig. Dafür gibt es die entsprechenden Agenturen; die machen Vorschläge. Darüber entscheidet nicht das Kanzleramt, sondern darüber entscheiden wieder alle zuständigen Ressorts, und nach meinem Kenntnisstand haben wir das auch sehr friedlich gemacht. Also, wenn sich jemand bei Ihnen beschwert hat, sagen Sie es mir; vielleicht haben wir den vergessen. 

\noindent\textbf{Comment:}
\begin{itemize}
    \setlength\itemsep{-3pt}
    \item (Heiterkeit und Beifall bei Abgeordneten der CDU/CSU, der SPD und des BÜNDNISSES 90/DIE GRÜNEN)
    \setlength\itemsep{-3pt}
    \item (Heiterkeit und Beifall bei Abgeordneten der CDU/CSU, der SPD und der LINKEN)
\end{itemize}
\subsection{Henrichmann}
\noindent\textbf{Texts:} Sehr geehrte Frau Bundeskanzlerin! In meinem Heimatkreis Coesfeld sind in den letzten Tagen die Infektionszahlen in die Höhe geschossen. Es ärgert mich massiv, dass aufgrund von Verfehlungen in einem Unternehmen – Stichwort „Westfleisch“ – die Disziplin der Menschen vor Ort mit Füßen getreten wird. Es ärgert mich auch, dass unter anderem Gastronomen jetzt unter der Verlängerung des Shutdowns leiden, und es ärgert mich, dass jetzt Landwirte plötzlich in die Röhre gucken müssen.  Als Abgeordneter dieses Kreises habe ich die Erwartung, dass wir jetzt natürlich knallhart die Quarantäne durchsetzen – ich glaube, das teilen auch alle verantwortlichen Behörden vor Ort –, und ich halte es auch für wichtig, dass wir noch mal über die Unterbringung diskutieren; Sie haben das ja auch gerade angedeutet. Dies vorausgeschickt, hier meine Frage: Sie haben am 6. Mai 2020 in der Runde mit den Länderregierungschefs abgesprochen, dass es im Falle eines regional klar eingrenzbaren Infektionsgeschehens – da wird häufig beispielhaft als Einrichtung das Altenheim angeführt – Ausnahmetatbestände für Lockerungen geben kann, trotz eines Anstiegs. Ist in diesem Zusammenhang darüber nachgedacht worden, diese Regelung auch auf Unternehmen wie Westfleisch und solche Infektionsgeschehen zu beziehen, damit wir den Menschen vor Ort jetzt Hoffnung machen können, dass auch sie nicht mehr allzu lange unter den Restriktionen leiden müssen? – Danke.

\noindent\textbf{Comment:}
\begin{itemize}
    \setlength\itemsep{-3pt}
    \item (Timon Gremmels [SPD]: Ärgern Sie auch die Arbeitsbedingungen?)
\end{itemize}
\subsection{Henrichmann}
\noindent\textbf{Texts:} Die Entscheidungen darüber – das ist ein Beschluss, den wir gemeinsam mit den Ministerpräsidenten gefällt haben – treffen ja die Landesregierungen in Absprache mit den Landräten, also mit den örtlichen Behörden. Einrichtungen sind in der Tat Pflegeheime, Krankenhäuser oder eben ein Betrieb. Aber in diesem Fall geht es natürlich nicht um den Betrieb allein, sondern eben auch um die Unterbringung der Beschäftigten. Daher ist das Infektionsrisiko doch etwas breiter gestreut, als wenn Infektionen nur in diesem Betrieb vorkämen und niemand den Betrieb verlassen würde. Das heißt, man hat sich entschlossen, die Lockerungsmaßnahmen, die sonst in Nordrhein-Westfalen gelten, für eine bestimmte Zeit auszusetzen. Ich hoffe, dass mit diesem sehr entschiedenen Vorgehen der Landesregierung und der örtlichen Behörden – für das ich auch sehr dankbar bin – die Infektionen jetzt sehr schnell wieder eingegrenzt werden können. Nein.

\subsection{Weinberg}
\noindent\textbf{Texts:} Vielen herzlichen Dank, Herr Präsident. – Frau Bundeskanzlerin, ich bin krankenhauspolitischer Sprecher meiner Fraktion, und ich habe insofern sehr viel mit Pflegekräften zu tun und bin auch dauernd mit ihnen in der Diskussion. Es gibt einige Arbeitsschutzverstöße, Probleme in Bezug auf die Schutzausrüstungen und Ähnliches. Im Sinne von Learning Lessons gibt es eine Forderung der Pflegekräfte, die lautet: Es wäre ganz gut, wenn wir so etwas wie ein unabhängiges Krisenmonitoring hätten, an dem sich Verbände, Gewerkschaften, die Pflegeverbände und durchaus auch die Arbeitgeberseite, die Deutsche Krankenhausgesellschaft, beteiligen, um solche Fälle einfach auch mal aufzulisten und daraus eine Lagebeurteilung für eventuell zukünftig auftretende Ereignisse dieser Art zu erstellen. Was halten Sie von einer solchen Idee?

\subsection{Merkel}
\noindent\textbf{Texts:} Ich denke, wenn wir die Pandemie mal überwunden haben, dann wird man allen beteiligten Gruppen sehr dankbar dafür sein, dass sie uns ihre gelernten Dinge mitteilen. Je verbandsübergreifender – also von Arbeitnehmerverbänden bis Arbeitgeberverbänden – das ist, umso aussagekräftiger ist es und umso einfacher ist es auch für die Regierung, daraus Schlüsse zu ziehen. Wir werden das selber auch tun; denn wir haben ja auch Schwachstellen entdeckt, ob nun hinsichtlich der Produktion von Masken, strategischer Unabhängigkeiten, bestimmter Fragen, die das Miteinander von Bund, Ländern und Kommunen anbelangen, oder des großen Themas „Stärkung des Öffentlichen Gesundheitsdienstes“. Wir werden also alle gut daran tun, und ich werde jede Anregung sehr gerne mit aufnehmen.

\subsection{Weinberg}
\noindent\textbf{Texts:} Nur eine kurze Nachfrage: Eigentlich müsste man das ja nach Möglichkeit jetzt einrichten, damit dieses Learning Lessons auch greift. Wenn man das im Nachhinein macht, dann hat man zwar natürlich Erfahrungen, die man eventuell einbringen kann, aber jetzt gibt es ja die Ereignisse in den Krankenhäusern und Pflegeheimen vor Ort.

\subsection{Merkel}
\noindent\textbf{Texts:} Wenn Sie Kontakt zu dieser Gruppe haben und wenn Sie mit uns zusammenarbeiten und mir jede Woche einen Lagebericht schicken möchten, dann werde ich diesen nicht nur annehmen, sondern auch interessiert lesen und versuchen, die Erfahrungen mit einfließen zu lassen. Ich bin ja ein aufmerksamer Zeitmensch, um nicht „-genosse“ zu sagen.  Ich will mir jetzt hier nichts einbrocken. Schauen Sie, ich möchte, dass das alles möglichst gut funktioniert. Wenn wir jetzt insbesondere an die älteren Menschen in den Pflegeheimen denken: Sie sind doch zum großen Teil diejenigen, die wirklich am härtesten betroffen sind:  keine Besuche, ganz wenige Kontakte. Wir haben jetzt gesagt, dass es wenigstens eine feste Kontaktperson geben soll. Das gilt genauso für Behinderteneinrichtungen. Wenn wir da nicht aufmerksam wären und überlegen würden, was wir tun können, um das Leben dort etwas zu erleichtern und schöner zu machen – genauso wie das der dort Beschäftigten –, dann wäre das doch nicht in Ordnung. Ich bin da sehr aufnahmebereit.

\noindent\textbf{Comment:}
\begin{itemize}
    \setlength\itemsep{-3pt}
    \item (Heiterkeit und Beifall bei der CDU/CSU und der SPD sowie bei Abgeordneten der FDP, der LINKEN und des BÜNDNISSES 90/DIE GRÜNEN)
    \setlength\itemsep{-3pt}
    \item (Harald Weinberg [DIE LINKE]: Das ist richtig!)
\end{itemize}
\subsection{Rößner}
\noindent\textbf{Texts:} Vielen Dank, Herr Präsident. – Frau Bundeskanzlerin, ich möchte mich auf Medienberichte aus der vergangenen Woche beziehen, und zwar ging es darin um die Ermittlungserfolge in Sachen Bundestagshack. Es gab ja vor fünf Jahren – Sie erinnern sich – einen großen Hackerangriff auf den Bundestag. Man schätzt, dass damals 16 Gigabyte an Daten, Dokumenten, E-Mails abgegriffen worden sind. Unter anderem ist ja auch Ihr Abgeordnetenbüro davon betroffen gewesen. Ich nehme an, Sie teilen unsere Einschätzung aufgrund dieses Ermittlungserfolgs, dass sich die Zusammenarbeit mit den ausländischen Geheimdiensten bewährt hat und dass auch außerhalb des Instruments Hackback eine erfolgreiche Cyberabwehr möglich ist. Ich wollte Sie fragen, ob Sie Erkenntnisse darüber haben, welche Daten und vor allen Dingen zu welchem Zweck diese Daten aus Ihrem Büro abgegriffen worden sind.

\subsection{Merkel}
\noindent\textbf{Texts:} Nein. Ich habe den Eindruck: Da wurde relativ wahllos abgegriffen, was man kriegen konnte. Ich bin sehr froh, dass die Untersuchungen jetzt dazu geführt haben, dass der Generalbundesanwalt eine konkrete Person auf die Fahndungsliste gesetzt hat. Ich nehme diese Dinge sehr ernst, weil ich glaube, dass da sehr ordentlich recherchiert wurde, und ich darf sehr ehrlich sagen: Mich schmerzt es. Auf der einen Seite bemühe ich mich tagtäglich auch um ein besseres Verhältnis zu Russland, und wenn man auf der anderen Seite sieht, dass es harte Evidenzen dafür gibt, dass da auch russische Kräfte dabei sind, so vorzugehen, dann ist das schon ein Spannungsfeld, in dem wir da arbeiten – in dem Wunsch nach guten Beziehungen zu Russland –, das auch ich nicht ganz aus meinem Innern streichen kann. Das ist unangenehm, und wir werden natürlich alles tun, um den Wünschen des GBA zu entsprechen.

\subsection{Rößner}
\noindent\textbf{Texts:} Ja. – Vielleicht genau dazu die Nachfrage: Zu welcher neuen Einschätzung kommen Sie dann in der Bewertung der Zusammenarbeit mit dem russischen Geheimdienst – gerade im Hinblick darauf, dass es diese Verbindung dieses mutmaßlichen Täters gibt, der ja offenbar ganz eindeutig mit Fancy Bear, dem Militärgeheimdienst GRU usw. zusammengearbeitet hat oder ‑arbeitet?

\subsection{Merkel}
\noindent\textbf{Texts:} Leider ist die Einschätzung, zu der ich komme, nicht neu, weil es eine Facette in einer Vielzahl von Facetten gibt. Es gibt eine Strategie Russlands, die wir beachten müssen, und die können wir auch nicht einfach verdrängen: die Strategie der hybriden Kriegsführung, die auch Kriegsführung im Zusammenhang mit Cyberdesorientierung und Faktenverdrehung beinhaltet. Das ist nicht nur irgendwie ein Zufallsprodukt, sondern das ist durchaus eine Strategie, die dort angewandt wird. Trotzdem werde ich mich weiter um ein gutes Verhältnis zu Russland bemühen, weil ich glaube, dass es allen Grund gibt, diese diplomatischen Bemühungen immer fortzusetzen. Aber das macht es natürlich nicht einfacher.

\subsection{Höferlin}
\noindent\textbf{Texts:} Danke, Herr Präsident, dass Sie die Nachfrage zulassen. – Ich bin jetzt irritiert. Sie sagen, der Vorgang sei Ihnen unangenehm. Ich glaube, so nah war seit Guillaume kein fremder Geheimdienst mehr an einem Bundeskanzler der Bundesrepublik Deutschland dran. Ich finde den Vorgang schon eher ungeheuerlich. Ist Ihnen das nur unangenehm, oder haben Sie mit dem Außenminister und weiteren auch über Maßnahmen gesprochen, die irgendeine Folge für das Land zulassen bzw. möglich machen? Ich finde diesen Vorgang für dieses Land nämlich, ehrlich gesagt, mehr als unangenehm.

\subsection{Merkel}
\noindent\textbf{Texts:} Mehr als unangenehm, genau. Unangenehm ist aber eine Facette; ungeheuerlich finde ich ihn, nebenbei, auch. Ich bin auch nicht die einzige Betroffene des Deutschen Bundestages gewesen, sondern das waren ja viele Kollegen Abgeordneten. Für sie gilt das „ungeheuerlich“ genauso. Das stört natürlich eine vertrauensvolle Zusammenarbeit. Sie wissen, dass wir im Zusammenhang mit dem Mord im Tiergarten Sanktionen verhängt, also Ausweisungen veranlasst haben. Wir haben jetzt an dieser Stelle erst mal die Aufgabe, zu versuchen, die Person durch Fahndung zu finden. Aber natürlich behalten wir uns immer Maßnahmen vor – auch gegen Russland –, um das deutlich zu machen.

\subsection{Spangenberg}
\noindent\textbf{Texts:} Vielen Dank. – Frau Bundeskanzlerin, die Analyse des Auswertungsberichtes aus dem BMI vom 25. April 2020 in Kommentierung von neun bedeutenden Wissenschaftlern zeigt in schonungsloser Offenheit, dass die eingeleiteten Maßnahmen im Rahmen der Coronaproblematik völlig überzogen waren und sind. Insbesondere wird in diesem Bericht die Formulierung „Fehlalarm“ verwendet. Wie erklären Sie den Tausenden wirtschaftlich Geschädigten in Deutschland die offenbar unnötigen Einschränkungen der Grundrechte und diese Fehleinschätzung der Regierung, und welche personellen und politischen Konsequenzen werden Sie daraus ziehen? – Vielen Dank.

\subsection{Merkel}
\noindent\textbf{Texts:} Sie wissen ja sicherlich, dass die Regierung die Einschätzung, die in diesem Papier geäußert wurde, nicht teilt und dass wir zu anderen Bewertungen gekommen sind. Wir wissen, dass es Menschen gibt, die unter den Einschränkungen leiden müssen; wir haben bei der ersten Frage ja schon darüber gesprochen. Arbeitsplätze sind in Gefahr, die Wirtschaft geht durch eine Rezession; das ist klar. Wir müssen immer wieder gucken, wie es mit der Verhältnismäßigkeit unserer Maßnahmen aussieht. Wir sind zu der Überzeugung gekommen, dass diese Maßnahmen richtig waren, um einfach eine völlige Überforderung oder Überwältigung unseres Gesundheitssystems zu verhindern, genauso wie es richtig ist, jetzt schrittweise zu lockern. Das ist die gemeinsame Haltung der Bundesregierung und auch des Bundesinnenministeriums, und das hat das Bundesinnenministerium auch deutlich gemacht.

\subsection{Scheer}
\noindent\textbf{Texts:} Sehr geehrte Frau Bundeskanzlerin, in Ihrer Rede zum Petersberger Klimadialog haben Sie am 28. April 2020 ausdrücklich begrüßt, dass eine Emissionsreduktion auf 50 bis 55 Prozent im Vergleich zu 1990 erfolgen soll. Wie ist dies mit dem jüngsten Papier der CDU\/CSU-Fraktion vereinbar, in dem dieses Ziel nicht mitgetragen wird? Es wird darauf verwiesen, dass man dazu einen Lastenausgleich in der EU bräuchte, und alle wissen, dass das sehr schwer ist. Für mich und unsere Fraktion stellt sich daneben auch die Frage: Wie ist diese Äußerung auf dem Petersberger Klimadialog, die wir ja begrüßen, mit der anhaltenden Verweigerung der CDU\/CSU-Bundestagsfraktion vereinbar, wenn es darum geht, den Solardeckel abzuschaffen?

\subsection{Merkel}
\noindent\textbf{Texts:} Jetzt muss ich ein bisschen sortieren: Erst mal hat die CDU\/CSU-Bundestagsfraktion gestern ein Klimapapier verabschiedet, in dem sie sich zur Klimaneutralität im Jahre 2050 bekannt hat. Die Bundesregierung hat mit Unterstützung des Parlaments eine Vielzahl von Maßnahmen ergriffen. Wir werden noch weitere ergreifen müssen, um auch die Zwischenziele zu erreichen. Das Zweite ist: Es geht um die Frage, wie sich die Europäische Union in ihrer Zielsetzung weiterentwickeln wird. Die Kommissionspräsidentin hat gesagt, dass sie das Ziel 2030 von 40 Prozent auf 50 bis 55 Prozent anheben will. Das habe ich begrüßt. Die Unionsfraktion sagt, dass dazu aber auch neue Verhandlungen zur Lastenverteilung geführt werden müssen. Das ist für mich ganz selbstverständlich. Wir haben Mitgliedstaaten, die zum Beispiel bis 2030 Reduktionsraten haben, die sogar bei null liegen, wenn ich mich recht erinnere. Ich glaube, wenn wir alle miteinander in der Europäischen Union die Klimaneutralität 2050 erreichen wollen, dann wird man mit einem Reduktionsziel 2030 von null sicherlich nicht hinkommen. Das heißt, es wird ganz automatisch unter den Mitgliedstaaten – ich gebe Ihnen recht; es wird schwer; es war auch bis jetzt schon schwer; die Verhandlungen werden nicht leichter werden – Verschiebungen geben und auch unterschiedliche Lastenteilungen. Die Bundesrepublik wird sich daran beteiligen. Aber es kann nicht sein, dass der Schlüssel von 2020 oder 2017, als wir darüber verhandelt haben, einfach so bleibt. Der letzte Punkt war der PV-Deckel. Wie Sie wissen, gibt es in der Tat mühselige, aber hoffentlich irgendwann endende Verhandlungen über die Frage der Windenergie. Es gibt dann die politische Zusage, dass der PV-Deckel aufgehoben wird. Daran ändert sich auch nichts. Deren Mitglied die Bundeskanzlerin ist. 

\noindent\textbf{Comment:}
\begin{itemize}
    \setlength\itemsep{-3pt}
    \item (Heiterkeit und Beifall bei der CDU/CSU)
\end{itemize}
\subsection{Scheer}
\noindent\textbf{Texts:} Aber wie wollen Sie, liebe Frau Merkel, dann den Widerspruch auflösen, den wir hier akut haben – mit schon jetzt drohenden Arbeitsplatzverlusten –, wenn der Solardeckel und das nationale Beispiel in der Europäischen Union zu scheitern drohen, falls wir nicht unmittelbar handeln? Wir haben auch in dieser Woche das Erneuerbare-Energien-Gesetz auf der Tagesordnung. Insofern frage ich, ob Sie zur Kenntnis nehmen, dass inzwischen selbst auf internationaler Ebene dringend appelliert wird – auch im Kontext mit Konjunkturprogrammen –, den Schwerpunkt auf den Ausbau der erneuerbaren Energien zu setzen.

\subsection{Merkel}
\noindent\textbf{Texts:} Ja. Ich werde auch morgen die Gelegenheiten in der Vorbereitung des Bundesrates wieder nutzen, um noch mal mit den Ministerpräsidenten zu sprechen im Hinblick auf die Windenergie, damit wir diese Frage schnellstmöglich lösen. Ich stimme Ihnen zu: Wir sollten hier baldmöglichst eine Lösung finden. Ich werde mich auch dafür einsetzen. Dass es Ihnen zu lange dauert, verstehe ich auch. 

\noindent\textbf{Comment:}
\begin{itemize}
    \setlength\itemsep{-3pt}
    \item (Beifall bei Abgeordneten der SPD)
\end{itemize}
\subsection{Badum}
\noindent\textbf{Texts:} Sehr geehrte Frau Bundeskanzlerin, ich möchte ebenfalls gern zur Lastenverteilung in der EU fragen. Insbesondere vor dem Hintergrund, dass Sie vor Kurzem angemahnt haben, dass wir in der Coronakrise mehr europäische Zusammenarbeit, mehr Solidarität und Zusammenhalt brauchen, finde ich es sehr verwunderlich und sehr befremdlich, dass Sie nun – so auch in Ihrem Beitrag gerade – vorschlagen, dass Deutschland einen geringeren Klimabeitrag leistet, dass wir also bei der Klimazielerhöhung einen geringeren Beitrag einbringen. Das wäre ja das Gegenteil von mehr Solidarität und mehr Verantwortung. Daher die Frage an Sie: Welche Fakten haben sich dahin gehend geändert, dass Deutschland jetzt einen geringeren Beitrag einbringt? Meines Wissens nach sind wir noch immer der größte CO2-Emittent in der Europäischen Union. Mit unserem Pro-Kopf-Verbrauch von 10 Tonnen sind wir in der Spitzengruppe. Wir sind der größte Emittent. Wir müssen unserer Ansicht nach Verantwortung übernehmen. Was hat sich aus Ihrer Sicht geändert? Warum jetzt weniger Solidarität, weniger Verantwortung aus deutscher Sicht?

\subsection{Merkel}
\noindent\textbf{Texts:} Erstens. Wir sind in der Tat in der Spitzengruppe beim Pro-Kopf-Verbrauch. Deshalb haben wir uns sehr intensiv mit dem Kohleausstieg beschäftigt. Zweitens. Wir sind der größte Emittent, weil wir die meisten Einwohner haben. Aber beim Pro-Kopf-Verbrauch gibt es auch andere Spitzenländer. Wir sind uns doch sicherlich einig, wenn die Europäische Union der erste Kontinent sein will, der kein CO2 mehr emittiert, dass dann alle Länder auf null kommen müssen. Das heißt, auch andere Länder werden in ihren Ambitionen weitergehen müssen. Dass man, noch bevor der Kommissionsvorschlag vorliegt, bereits national zusagt: „Wir machen das, was wir immer gemacht haben“, während andere gar nicht gefragt werden, was sie machen, wäre taktisch doch wirklich unklug. Deshalb kann und muss über die Lastenverteilung neu verhandelt werden; das ist klar. Das wird auch im Europäischen Rat geschehen; so ist es jedes Mal gewesen. Es ist auch beim vorletzten Mal eine andere Lastenverteilung gewesen als beim letzten Mal. Auch beim nächsten Mal wird es eine andere sein. Dass Deutschland dazu beitragen muss, ist doch gar keine Frage. Wir haben im Übrigen das nationale Ziel von 55 Prozent Reduktion; das wissen Sie. Andere haben das nicht.

\subsection{Vogel}
\noindent\textbf{Texts:} Sehr geehrte Frau Bundeskanzlerin, Sie haben eben zu Recht darauf hingewiesen, dass in der wirtschaftlichen Dimension dieser Krise der Sozialstaat uns schützt. Ich denke, wir sind einer Meinung, dass wir auf eine solide Finanzierung dieses Sozialstaats achten müssen. Sie haben 2008, in der letzten großen Krise, die Rentengarantie eingeführt. Damals wurde mit vereinbart, dass das, wenn diese greift, in den folgenden Jahren verrechnet wird, sodass Renten und Löhne immer im Gleichklang steigen. Heute haben die Professoren Börsch-Supan und Rürup in der „Süddeutschen Zeitung“ darauf hingewiesen, dass Sie im Rentenpaket 2018 den Effekt – das ist der sogenannte Nachholfaktor – ausgesetzt haben. Das kann dazu führen, dass das dann, wenn im nächsten Jahr sehr wahrscheinlich die Rentengarantie wieder greift, in den Folgejahren nicht verrechnet wird. Das führt dazu, dass auf Dauer die Renten stärker steigen als die Löhne. Meine Frage ist: Halten Sie das für fair? Ist das nicht mit Blick auf die Generationengerechtigkeit eigentlich unverantwortbar? Müsste man nicht jetzt den Nachholfaktor wieder in Kraft setzen, also wieder einführen?

\subsection{Merkel}
\noindent\textbf{Texts:} Ich habe heute früh über diese Entdeckung der Rentenprofessoren gelesen. Wenn solchen renommierten Professoren das erst jetzt auffällt, dann verzeihen Sie vielleicht, dass ich in den vier Stunden, die mir seitdem verblieben sind, das Gesetz noch nicht gelesen habe.  Ich hatte mit Hubertus Heil heute auch so viel über die Fleischindustrie zu sprechen. Aber ich habe mir fest vorgenommen, das zu lesen. Wenn es so wäre, wie es heute in der Zeitung steht, nämlich dass der Nachholfaktor bis 2025 ausgesetzt ist, würden wir darüber noch mal reden. Das hängt dann aber mit den Haltelinien, die wir eingezogen haben, zusammen. Wie man das dann löst, muss man sehen. Ich muss aber erst mal gucken, ob das Faktum stimmt. Zweitens ist dann der Nachholfaktor auch nicht für immer ausgesetzt; das will ich noch mal sagen. Im Übrigen führt der Nachholfaktor dann dazu, dass die Renten nicht mehr so stark steigen wie die Löhne. Also, die Renten stiegen dann langsamer, weil man ja nachholen muss, was man bei den Löhnen nicht angeglichen hat, weil man die Rente hätte kurzzeitig senken müssen, was keiner möchte. Also: Ich stehe zu der Rentengarantie. Ich werde der Sache nachgehen. Falls es so ist, gilt es nur bis 2025 und nicht für immer. Glücklicherweise sind wir auch noch nicht in der Situation, dass die Renten sinken. Vielmehr steigen sie in diesem Jahr deutlich um 3 Prozent für die Menschen in den alten Bundesländern und um über 4 Prozent für die Menschen in den neuen Bundesländern. Noch mal: Darüber freuen wir uns. Ich halte nichts von Vorschlägen, die besagen, dass man das jetzt nicht machen sollte, in vorauseilender Antizipierung dessen, was vielleicht kommen könnte. Die Rentnerinnen und Rentner haben das verdient. 

\noindent\textbf{Comment:}
\begin{itemize}
    \setlength\itemsep{-3pt}
    \item (Beifall bei der CDU/CSU sowie bei Abgeordneten der SPD)
    \setlength\itemsep{-3pt}
    \item (Beifall bei Abgeordneten der SPD)
\end{itemize}
\subsection{Vogel}
\noindent\textbf{Texts:} Auch ich halte davon nichts. Ich bin sehr für die Rentensteigerung in diesem Jahr, aber auch dafür, dass das langfristig im Gleichklang steigt. Ich habe es mir schon angeschaut. Es ist in der Tat so, dass selbst dann, wenn die Haltelinie noch nicht erreicht ist, wenn also das Rentenniveau über 48 Prozent liegt, der Nachholfaktor für die folgenden Jahre ausgesetzt ist. Sollten Sie bei der Prüfung der Faktenlage zu demselben Urteil kommen wie ich bei meiner, wären Sie dann mit mir der Meinung, dass dann definitiv der Nachholfaktor, sofern es oberhalb der Haltelinie liegt, wieder in Kraft gesetzt werden müsste und alles andere mit Blick auf die Generationengerechtigkeit unverantwortbar wäre?

\subsection{Merkel}
\noindent\textbf{Texts:} Für diese Meinungsbildung brauche ich noch ein paar Stunden, Herr Vogel. Aber ich werde Sie dann informieren. 

\noindent\textbf{Comment:}
\begin{itemize}
    \setlength\itemsep{-3pt}
    \item (Heiterkeit und Beifall bei der CDU/CSU und der SPD)
\end{itemize}
\subsection{Krichbaum}
\noindent\textbf{Texts:} Vielen Dank, Herr Präsident. – Frau Bundeskanzlerin, die Coronakrise, die Coronapandemie hat uns an vielen Stellen in Europa kalt erwischt. An welchen Stellen sehen Sie die Notwendigkeit, zu stärkeren europäischen Ansätzen bei der Bekämpfung von Pandemien zu kommen? Denn für die Zukunft kann nicht ausgeschlossen werden, dass uns ähnliche Krankheitsverläufe, ähnliche Pandemien noch einmal heimsuchen.

\subsection{Merkel}
\noindent\textbf{Texts:} Ich glaube, dass wir mit der Tatsache, dass Gesundheitspolitik erst mal in der Zuständigkeit der Mitgliedstaaten liegt, weiter leben können. Das müssen wir aus meiner Sicht nicht vergemeinschaften. Aber wir brauchen schon ein paar bessere europäische Mechanismen; ich sage gleich etwas dazu. Wir sehen, wie schnell infolge zum Beispiel solcher Beschränkungen, die wir einführen mussten, um die Überforderung des Gesundheitssystems zu verhindern, dann auch andere Bereiche, die sehr europäisch sind, wie das Funktionieren des Binnenmarktes, betroffen sind. Da ist, glaube ich, erkennbar, dass da gegenseitig eine sehr viel bessere Information erfolgen muss, um Lkw-Schlangen, wie wir sie gesehen haben, zu verhindern, um unabgesprochene Grenzschließungen und vieles andere mehr zu verhindern. Wir brauchen mit Sicherheit auch im Blick auf die Gesundheit eine stärkere strategische Souveränität Europas, wenn es zum Beispiel um medizinische Versorgung geht, wenn es wie in diesem Falle um Masken geht. Daraus werden wir eine ganze Reihe von Lehren ziehen müssen; wir sind ja auch schon dabei, sie daraus zu ziehen. Wir selber haben schmerzlich erfahren, als wir durchaus mit guten Gründen gesagt haben: Wir erlassen ein limitiertes Exportverbot medizinischer Güter. – Damit haben wir uns zum Teil auch ins eigene Fleisch geschnitten, weil wir dann gemerkt haben, wie abhängig wir von Zulieferungen sind. Das alles noch mal zu reflektieren, das wird sicherlich die Aufgabe sein.

\subsection{Krichbaum}
\noindent\textbf{Texts:} Eine kurze Nachfrage. – Wir sind in Europa dafür durchaus institutionell gerüstet. Es gibt die ECDC – das ist eine Art Seuchenbekämpfungsagentur in der Nähe von Stockholm –, eine Agentur, die ein Schattendasein fristet und die eigentlich kaum bekannt ist. Aber hielten Sie es für denkbar, dass wir diese Agentur analog Frontex zu einer veritablen Behörde entwickeln? Frontex war früher ebenfalls eine – in Anführungszeichen – „simple Agentur“. Es fällt eben doch auf, dass wir hier in Europa institutionell einfach wenig gerüstet sind.

\subsection{Merkel}
\noindent\textbf{Texts:} Ja, wir haben gerade heute im Kabinett sehr lange über diese Agentur gesprochen, weil sie zum Beispiel europaweit das Infektionsgeschehen abbildet und Warnstufen ausgibt, auch weltweit. Diese Agentur sollte auf jeden Fall stärker befähigt werden, in dieser Pandemie eine noch wichtigere, koordinierende und von allen Mitgliedstaaten akzeptierte Rolle zu spielen, damit man auch zu gleichen Entscheidungskriterien kommen kann.

\subsection{Brantner}
\noindent\textbf{Texts:} Herzlichen Dank, Herr Präsident. – Sehr geehrte Frau Bundeskanzlerin, ich bin froh, dass wir vernehmen konnten, dass jetzt alle Grenzübergänge wieder geöffnet werden – Sie haben es gerade selber erwähnt –, dass Grenzkontrollen aber stichprobenartig oder in noch größerem Umfang durchgeführt werden sollen. Ich habe eine konkrete Frage an Sie. Bis jetzt ist grenzüberschreitendes Reisen als Einreise zum Ehepartner und zum eingetragenen Lebenspartner möglich. Was wollen Sie dafür tun, dass sich jene, die nach einem erweiterten Familienbegriff leben, also in einer Partnerschaft ohne Trauschein, ebenfalls in den nächsten Wochen sehen können, wenn diese Grenzkontrollen weiterlaufen?

\subsection{Merkel}
\noindent\textbf{Texts:} Der Bundesinnenminister hat heute dargelegt, dass genau das gewollt wird. Da gibt es dann immer die schöne Formulierung: „wenn es glaubhaft gemacht werden kann.“ Das ist in Zeiten von Corona manchmal vielleicht eine gewisse Schwierigkeit; wenn ich das so leicht scherzhaft sagen darf. Ansonsten sind wir der Meinung, dass das geschehen soll. Es wird in den nächsten Tagen ja auch nur noch stichprobenweise kontrolliert, also nicht mehr so flächendeckend, wie das jetzt der Fall war. Also, ich hoffe, dass diese in der Tat sehr beschwerlichen Situationen für Menschen, die auf beiden Seiten der Grenze leben, jetzt überwunden werden können. Das Ziel ist ja auch, wenn das Infektionsgeschehen das zulässt – das will ich noch mal ausdrücklich sagen –, dass dann, ab 15. Juni, die Grenzkontrollen im Schengenraum vollständig entfallen können, überall und auf jeden Fall.

\subsection{Schinnenburg}
\noindent\textbf{Texts:} Vielen Dank, Herr Präsident, dass Sie die Frage zulassen. – Frau Bundeskanzlerin, es schließt sich direkt an. Wir sprachen jetzt über direkte Grenzbeziehungen, Verwandtenbesuche und Ähnliches. Ich fasse die Frage mal weiter: Sind Sie nicht der Meinung, dass Grenzschließungen oder Grenzkontrollen außerordentlich wenig zur Eindämmung der Pandemie beitragen können, mindestens im Hinblick auf solche Länder, die ein ähnliches Infektionsgeschehen haben wie wir? Glauben Sie, dass Sie durch Grenzkontrollen die Pandemie nennenswert aufhalten können?

\subsection{Merkel}
\noindent\textbf{Texts:} Ich glaube, dass zwei Dinge eine Rolle spielen: zum einen die Vergleichbarkeit des Infektionsgeschehens und zum anderen der Aspekt der Vergleichbarkeit der Maßnahmen. Wenn – das war ja der letztlich der Auslöser für die deutsch-französischen Grenzkontrollen – zum Beispiel in Frankreich die Läden und die Restaurants schon geschlossen waren, dann gab es plötzlich eine große Bewegung Richtung Deutschland; das könnte aber umgekehrt genauso sein. Das heißt, wir müssen auch ein bisschen auf die Reziprozität der Maßnahmen achten. Im Zusammenhang mit Frankreich war während der Zeit, als der ganze Lockdown dort durchgeführt wurde, natürlich überhaupt nicht mehr die Situation gegeben wie an dem Tag, an dem die Grenzkontrollen eingeführt worden waren. Dazwischen lagen leider nur zwei Tage. Daraus hat sich dann ja auch die veränderte Betrachtungsweise ergeben. Also, wichtig ist die Reziprozität der Maßnahmen. Wir hatten zum Beispiel keine Grenzkontrollen an den Übergängen zwischen NRW, Niedersachsen und den Niederlanden, und das hat über Ostern nur geklappt, weil man miteinander gesprochen hat und gesagt hat: Wenn der eine die Freizeitparks zuhat, muss auch der andere die Freizeitparks zuhaben; sonst kommt es zu einer Mobilität, die wir dann nicht mehr genau überblicken. Das ist dort sehr gut gelungen, und das wird uns jetzt auch mit Frankreich gut gelingen. Ich habe mit dem Präsidenten auch selbst darüber gesprochen.

\subsection{De Masi}
\noindent\textbf{Texts:} Vielen Dank, Herr Präsident. – Frau Bundeskanzlerin, viele Menschen sind in der Coronakrise ja verständlicherweise verzweifelt, etwa Alleinerziehende, die nicht wissen, wie sie ihre Arbeit erledigen und gleichzeitig die Kinder betreuen sollen. Es soll in Deutschland sogar Kinder geben, die die Schule vermissen. Aber es ist ja so, dass es vor allem die Leute sind, die vom Leben nicht immer geküsst werden – die Kassiererinnen, die Pflegekräfte –, die jetzt den Laden am Laufen halten. Sehr bald ist die Hauptversammlung von BMW. Man schüttet Dividenden von über 700 Millionen Euro aus, unter anderem an die Quandts und Klattens. Die Quandts und Klattens tauchen ja auch immer wieder in Parteispendenberichten auf. Gleichzeitig werden Zehntausende von Beschäftigten in Kurzarbeit geschickt. Jetzt weiß ich, dass Kurzarbeit eine Versicherungsleistung ist; aber es ist ja durchaus möglich, dass wir noch Zuschüsse aus dem Bundeshaushalt für die Arbeitslosenversicherung brauchen. Vor diesem Hintergrund möchte ich Sie fragen, ob Sie die Auffassung Ihres Finanzministers teilen, dass diejenigen in diesem Land, die starke Schultern haben, mehr für den Wiederaufbau dieses Landes leisten müssen, zum Beispiel mit einer Vermögensabgabe, wie wir das in Deutschland nach dem Zweiten Weltkrieg schon mal mit einem Lastenausgleich hatten.

\subsection{Merkel}
\noindent\textbf{Texts:} Ich weiß nicht, ob der Finanzminister sich schon dezidiert für eine Vermögensabgabe ausgesprochen hat. Ich spreche mich nicht für eine Vermögensabgabe aus. Ich finde erst mal wichtig, dass wir noch ein paar Unternehmen haben, die Steuern an den Staat zahlen; denn sonst können wir auch denen, denen es nicht gut geht, leider nicht so viel Hilfe leisten, wie wir das gerne tun würden. Deshalb ist alles okay, solange es auf der Basis der Legalität stattfindet. Auch Parteispenden sind sozusagen rechtlich möglich und werden vom Steuerzahler sogar unterstützt, weil wir Parteien wichtig finden, wie Sie ja sicherlich auch nachvollziehen können. Also: Wir brauchen Unternehmen, die Gewinne machen, damit Steuern gezahlt werden. Wir brauchen eine stärkere Belastung der starken Schultern als der schwächeren Schultern – das ist vollkommen klar –; das ist die Grundlage unseres Sozialstaates. In diesem Rahmen werden wir die Diskussionen auch in der Zukunft führen.

\subsection{De Masi}
\noindent\textbf{Texts:} Die Frage, wer die Lasten in diesem Land trägt, wird uns ja noch eine Weile beschäftigen. Sie haben sich damals, in der letzten Finanzkrise, mit Herrn Steinbrück vor die Kameras gestellt und gesagt: Die Spareinlagen sind in Deutschland sicher. – Wären Sie denn bereit, sich auch mit Herrn Scholz vor die Kameras zu stellen und ein Versprechen für Ihre Amtszeit abzugeben, dass keine Renten, keine sozialen Leistungen gekürzt werden?

\subsection{Merkel}
\noindent\textbf{Texts:} Ich sage immer wieder und sage das auch heute hier gerne, dass die Steinbrück\/Merkel’sche Sparergarantie fortgilt. 

\noindent\textbf{Comment:}
\begin{itemize}
    \setlength\itemsep{-3pt}
    \item (Heiterkeit und Beifall bei Abgeordneten der CDU/CSU und der SPD)
\end{itemize}
\subsection{Schauws}
\noindent\textbf{Texts:} Sehr geehrter Herr Präsident! Liebe Frau Bundeskanzlerin, wie bewerten Sie die aktuellen Auswirkungen der Coronakrise auf die Gleichberechtigung? Erste wissenschaftliche Erkenntnisse deuten darauf hin, dass ein großer Teil der Frauen, insbesondere der Mütter, von einer andauernden Mehrfachbelastung aufgrund von Homeschooling und Kinderbetreuung, oft eben neben ihrer eigentlichen Berufstätigkeit, betroffen ist, und in den sozialen Medien schildern sehr viele Frauen ihre Betroffenheit. Nicht sehr wenige sind auch wütend, dass da aus ihrer Sicht zu wenig gemacht wird; denn sie wollen auch nicht ins Privatleben zurückgedrängt werden, sie wollen keine Retraditionalisierung. Deswegen frage ich Sie: Was wollen Sie als Chefin der Bundesregierung konkret gegen den Rückfall in eine traditionelle Rollenverteilung tun?

\subsection{Merkel}
\noindent\textbf{Texts:} Ich muss Ihnen sagen: Ich bin in diesen Tagen und Wochen wirklich noch mal sehr daran erinnert worden, dass wir eigentlich noch gar nicht so lange einen Rechtsanspruch auf einen Kitaplatz haben und dass das doch eine ganz, ganz wichtige Sache und glücklicherweise auch eine sehr, sehr gut angenommene Sache ist. Ich werde mich mit aller Kraft dafür einsetzen, dass wir nicht etwa eine Retraditionalisierung bekommen, sondern dass der Weg der gleichen Chancen für Männer und Frauen weiterführt. Es gibt im Übrigen auch viele Väter, die sich jetzt mit dem Homeschooling beschäftigt haben; es sind nicht nur Mütter. Aber ich stimme Ihnen darin zu, dass dann, wenn man die Summe der Stunden nimmt, wahrscheinlich die Mütter in sehr viel stärkerem Maße belastet sind. Mich spornt das an, noch mehr zu tun. Wir haben uns ja vorgenommen, als Koalition auch etwas für einen Rechtsanspruch auf Betreuung im Grundschulalter zu tun. Alle diese Dinge müssen fortgesetzt werden. Wir sind auch sehr froh, dass sich jetzt glücklicherweise die Notbetreuung durch die Öffnung der Kitas fortentwickeln kann und dass hoffentlich dann die Ausübung von Berufstätigkeit für Männer und Frauen wieder besser möglich wird. Wir werden das natürlich sehr genau beobachten.

\subsection{Schauws}
\noindent\textbf{Texts:} Vielen Dank. – Ich habe eine Nachfrage konkret zum politischen Raum. Sind Sie mit dem Anteil der Frauen in den politischen Entscheidungspositionen, die aktuell mit der Bekämpfung der Coronakrise befasst sind, zufrieden? Glauben Sie, dass so die Perspektive von Frauen ausreichend berücksichtigt wird? Insbesondere vor dem Hintergrund, dass Sie die Ministerin für Frauen und Familie nicht als ständiges Mitglied in Ihr Coronakabinett berufen haben, frage ich das jetzt auch ganz direkt Sie als Bundeskanzlerin.

\subsection{Merkel}
\noindent\textbf{Texts:} Streng genommen haben wir zwei Coronakabinette: ein Kernkabinett, das montags tagt, und ein erweitertes – da ist die Familienministerin dabei –, das in Sitzungswochen nicht mehr donnerstags tagen wird, weil wir festgestellt haben, dass wir sonst in Konflikt mit dem Bundestag kommen, und der hat natürlich Vorrang vor der Bundesregierung. Also, die Bundesfamilienministerin ist dort drin. Insofern sehe ich da überhaupt keinen Nachholbedarf, zumal Franziska Giffey bei uns und auch bei mir wirklich Gehör findet. Sie hat auch eine gute Stimme, die deutlich macht, was sie will. Also, da sehe ich keinen Nachholbedarf. „Bin ich zufrieden?“ Ich muss Ihnen sagen, dass ich mich freue, dass es ziemlich viele Professorinnen und Wissenschaftlerinnen im virologischen Bereich, im Bereich des Öffentlichen Gesundheitsdienstes und in den ethischen Bereichen gibt. Ich habe da jetzt viele kennengelernt, auch im Zusammenhang mit Beratung. Was die politischen Entscheidungsträger anbelangt, ändert sich natürlich jetzt nicht ad hoc etwas durch die Frage. Aber gerade im Gesundheitsbereich haben wir tendenziell mehr Frauen in Gremien als im Wirtschaftsbereich. Ob das nun wiederum schon für die Gleichberechtigung spricht, das weiß man auch nicht. Aber ich habe im Augenblick mit sehr vielen Frauen zu tun, die auch beraten, die zum Teil selber Beruf und Familie zusammenbringen müssen und deshalb auch aus eigenem Erleben sprechen.

\subsection{Kleinwächter}
\noindent\textbf{Texts:} Vielen Dank, Herr Präsident. – Frau Kanzlerin, am 5. Mai urteilte das Bundesverfassungsgericht, der oberste Wächter unseres Grundgesetzes, dass gewisse Beschlüsse der EZB zum Kauf von Staatsanleihen kompetenzwidrig seien. Damit wies es auch gewisse Urteile des EuGH als nicht nachvollziehbar und willkürlich zurück. EU-Kommissionspräsidentin von der Leyen reagierte prompt, sagte, dass sie Schritte gegen Deutschland prüft, bis hin zum Vertragsverletzungsverfahren. Ihren Äußerungen zufolge ist ja der Kern der europäischen Souveränität berührt, was auch immer das sei. Die Urteile des Europäischen Gerichtshofs für alle nationalen Gerichte seien bindend, und das EU-Recht habe für sie auch Vorrang vor nationalem Recht, offenbar auch vor dem Grundgesetz mit seiner Ewigkeitsgarantie. Meine Frage an Sie ist nun: Wie werden Sie als Bundeskanzlerin unser Bundesverfassungsgericht gegenüber offenbar übergriffigen EU-Institutionen vertreten? Was sagen Sie zu den Kritikern dieser Entscheidung, die dem Bundesverfassungsgericht eine Spaltung Europas vorwerfen? Spaltet das Bundesverfassungsgericht Europa? Spaltet unser Grundgesetz Europa?

\subsection{Merkel}
\noindent\textbf{Texts:} Erst mal haben wir die Entscheidung des Bundesverfassungsgerichts zu respektieren, und das tue ich selbstverständlich auch. Zweitens habe ich ein Interesse daran, dass die Möglichkeiten, die das Bundesverfassungsgericht eröffnet hat, auch genutzt werden, um den Konflikt kleiner zu machen und nicht größer zu machen. Drittens gibt es in weiten Teilen – ich sage ausdrücklich: in weiten Teilen – einen eindeutigen Vorrang des Europarechts vor dem nationalen Recht; das ist durch Rechtsprechung des Europäischen Gerichtshofs auch deutlich gemacht worden. Aber immer da, wo die Frage ist: „Welche Kompetenz genau hat denn nun ein Mitgliedstaat der europäischen Ebene gegeben?“, kann es natürlich an den Rändern dieser Auslegung der Kompetenz auch zu Fragestellungen kommen. Die manifestieren sich jetzt in diesem Urteil. Ich glaube, wir müssen jetzt mit einem klaren politischen Kompass an die Bearbeitung dieser Aufgaben gehen. Dieser Kompass heißt für mich: Ich möchte eine starke gemeinsame Währung, einen Euro. In diesem Sinne werden wir jetzt auch vorgehen. Dass die Kommissionspräsidentin in ihrer Verantwortlichkeit auch Fragen stellt, das ist normal. Das Vertragsverletzungsverfahren beinhaltet ja nicht nur, dass man einfach ein Vertragsverletzungsverfahren macht, sondern der erste Schritt ist ein Letter of Intent: Es werden Fragen an die Bundesregierung gestellt, und diese Fragen wird die Bundesregierung nach bestem Wissen und Gewissen und in voller Verantwortung für die Bundesrepublik Deutschland in europäischer Gesinnung beantworten.

\subsection{Kleinwächter}
\noindent\textbf{Texts:} Sehr gerne. – Das wünsche ich Ihnen sehr. Nun ist es ja so, dass das Urteil des Bundesverfassungsgerichts auch Versäumnisse der Bundesregierung festgestellt hat. Es geht konkret darum, dass die Verhältnismäßigkeit der entsprechenden Staatsanleihekäufe nicht ausreichend definiert war. Darin wurde im Endeffekt auch der Bundesregierung – und dem Bundestag, muss man dazusagen – vorgeworfen, hier nicht diese Verhältnismäßigkeitsprüfung eingefordert zu haben. Wo liegt da Ihre persönliche, auch politische Verantwortung als Bundeskanzlerin? Zum Zweiten: Wie werden Sie nun konkret – Sie haben gerade von einem „politischen Kompass“ gesprochen – dieses Urteil des Bundesverfassungsgerichtes umsetzen, auch im Hinblick auf die zukünftigen potenziellen Staatsanleihekäufe?

\subsection{Merkel}
\noindent\textbf{Texts:} Meine Aufgabe in diesem Bereich liegt darin, das Urteil zu respektieren. Das Bundesverfassungsgericht hat über Bundestag und Bundesregierung genau das gesagt, was Sie gesagt haben. Das müssen wir zur Kenntnis nehmen. Wie gesagt: Wir werden in klarer europäischer Ausrichtung dann unseren Beitrag dazu leisten. Darüber, wie der genau aussieht, werden wir Sie zu gegebener Zeit sicherlich informieren. Aber wir werden unseren Beitrag dazu leisten, dass ein starker Euro weiterbestehen kann. 

\noindent\textbf{Comment:}
\begin{itemize}
    \setlength\itemsep{-3pt}
    \item (Beifall bei Abgeordneten der CDU/CSU)
\end{itemize}
\subsection{Petry}
\noindent\textbf{Texts:} Vielen Dank, Herr Präsident. – Sehr geehrte Frau Bundeskanzlerin, es geht auch mir um das Urteil des Bundesverfassungsgerichts zum Staatsanleihekaufprogramm PSPP. Danach liegt keine unzulässige Staatsfinanzierung vor; das ist ein positives Ergebnis dieses Urteils. Gleichwohl ist der Nachweis der Verhältnismäßigkeit kritisiert worden. Deshalb sehen wir das Urteil im Hinblick auf den europäischen Rechtsraum als problematisch an. Zum einen ist die Zuständigkeit des EuGH hier, sage ich mal, hinterfragt. Zum anderen sind Bundestag und Bundesregierung aufgefordert, auf die EZB hinzuwirken, diese Verhältnismäßigkeit darzulegen. Europarechtlich ist aber auch auf Wunsch Deutschlands, wie Sie wissen, nicht vorgesehen, durch eine Einflussnahme die Unabhängigkeit der Zentralbanken, auch der Bundesbank, der nationalen Banken zu sichern. Deshalb die Frage: Wie kommen wir aus dieser Quadratur heraus? Ich würde Sie gerne fragen: Wie bewerten Sie dieses Urteil in diesem Lichte? Was werden Sie tun? Das ist zum Teil eben von Ihnen beantwortet worden. Eine weitere Frage: Würden Sie eine Vertiefung der Wirtschafts- und Finanzunion befürworten, das heißt auch eine institutionelle Weiterentwicklung, um nicht der EZB diese Aufgabenstellung dauerhaft zu übertragen, sondern gewisse Blockademöglichkeiten im Europäischen Rat zu überwinden?

\subsection{Merkel}
\noindent\textbf{Texts:} Wie gesagt, es geht ja nicht um meine Meinung zu diesem Urteil, sondern es geht darum, dass ich das Urteil zu respektieren habe. Das gebietet die Rolle des Bundesverfassungsgerichts und seiner Urteile. Und es geht jetzt darum, dass wir verantwortungsvoll handeln und so klug handeln, dass sozusagen der Euro weiterbestehen kann und soll und wird und dass auch die Bundesbank an den Aktivitäten der EZB teilnehmen kann; denn wir sind ja Mitglied der Europäischen Union und des Euro-Raums. Das wird in der Tat uns eher anspornen müssen, im Bereich der Wirtschaftspolitik mehr zu tun, um die Integration voranzubringen. Wir werden uns sicherlich sehr spezifisch mit dieser Fragestellung im Zusammenhang mit dem sogenannten Recovery Fund beschäftigen; denn da geht es um europäische Solidarität. Je stärker die europäische Antwort in dem Zusammenhang ist, umso sicherer kann auch die EZB ihre Arbeit machen; da gibt es durchaus eine Interkonnektion. Alles, was wir in dem Bereich der politischen Entscheidungen machen, wird dann ja auch gebilligt durch den Deutschen Bundestag. Dem Bundesverfassungsgericht ist immer sehr wichtig, dass es eine bestimmte Entscheidung des deutschen Parlaments, der gewählten Abgeordneten ist. Genau auf diesem Weg müssen wir auch weiterhin gehen. Wir dürfen nie vergessen, dass Jacques Delors vor Einführung des Euro gesagt hat: Es bedarf auch einer politischen Union; allein eine Währungsunion wird nicht reichen. – Wir sind da einige Schritte vorangekommen, aber wir sind nicht ausreichend vorangekommen. Das ist vollkommen evident. Das heißt also, es wird eher mehr Integration geben müssen als weniger, ohne dass ich heute hier schon spezifisch etwas sagen kann. Ich habe in meiner Regierungserklärung am 23. April hier erklärt: Es kann nicht sein, dass wir sagen: Vertragsveränderungen sind tabu. – Vielmehr sind Vertragsveränderungen immer auch bewusste Akte der Nationalstaaten; aber sie dauern. Wir müssen jetzt in der Pandemie natürlich auch Möglichkeiten finden, sehr schnell zu handeln.

\subsection{Petry}
\noindent\textbf{Texts:} Ja, eine Nachfrage. – Herzlichen Dank für diese – aus meiner Sicht – wirklich gute Antwort. Sie bietet eine europäische Perspektive und zeigt auch, dass wir als Bundestag gefordert sind, öfter Stellung zu beziehen. Vielleicht führt das dann auch zu etwas. Lob vom Koalitionspartner ist immer gut. Der Bundesfinanzminister hat einen bemerkenswerten Beschluss auf europäischer Ebene herbeigeführt, den ich fast als historisch bezeichnen würde. Dass man eine Einigung über die Weiterentwicklung des 500-Milliarden-Euro-Pakets auch hinsichtlich der ESM-Kreditlinien erzielt hat, ist eine große Leistung, finde ich.  Ich glaube, dass wir darauf aufbauend auch das Recovery-Fund-Programm betrachten können. Meine Nachfrage bezieht sich auf die Perspektive des mehrjährigen Finanzrahmens. Können Sie sich vorstellen, über eine institutionelle Weiterentwicklung auch in der Frage der Eigenmittel, nämlich des Anleiheankaufs, eine Gestaltungsmöglichkeit zu finden? Wer soll Anleihen aufkaufen? Ach so.  Irgendwie klappt das mit dem Mikrofon heute nicht so ganz; aber ich bin Ihnen, Herr Dr. Schäuble, dankbar für die Hilfestellung. Das meinte ich, genau. Nein, ich glaube, Sie meinten, dass die EU eine Anleihe begibt. Ja, Entschuldigung.

\noindent\textbf{Comment:}
\begin{itemize}
    \setlength\itemsep{-3pt}
    \item (Heiterkeit)
    \setlength\itemsep{-3pt}
    \item (Beifall bei Abgeordneten der SPD)
\end{itemize}
\subsection{Merkel}
\noindent\textbf{Texts:} Ich möchte mich jetzt zu den Spezifika nicht äußern. Ich habe sehr viel Sympathie dafür, dass wir im Rahmen des Recovery Fund eine Lösung finden, die nicht abgespalten vom normalen Budget der Europäischen Union ist. Das heißt, wir brauchen einfach mehr Finanzmittel in den ersten Jahren, um die Folgen der Pandemie zu bekämpfen. Wir sollten jedoch nicht sagen: „Das machen wir jetzt mal, aber wie es in den nächsten sieben Jahren ansonsten weitergeht, sagen wir nicht.“ Da besteht für mich ein enger Zusammenhang. Und was die Varianten betrifft, die wir da diskutieren, bin ich noch nicht so weit, dass ich jetzt öffentlich darüber spreche. Aber wir müssen handeln, und wir müssen der außergewöhnlichen Lage, in der wir sind, auch Rechnung tragen.

\subsection{Glaser}
\noindent\textbf{Texts:} Herzlichen Dank, Herr Präsident. – Frau Bundeskanzlerin, die EZB hat seit 2015 Staatsanleihen im Wert von rund 2,8 Billionen Euro von zum Teil hochverschuldeten Euro-Staaten aufgekauft.  Es ist völlig unvorhersehbar, wann und in welcher Höhe diese Anleihen in den Markt zurückgeführt werden. Und es ist ebenfalls unvorhersehbar, welche Ausfallrisiken auch mit Wirkung auf Deutschland mit diesem Anleihenerwerb verbunden sind. Diese Transaktionen sollen angeblich der Finanzstabilität dienen. Es könnte aber auch sein, dass es sich dabei um eine EU-rechtlich verbotene Staatsfinanzierung handelt.  Haben Sie daher Verständnis für die Sicht des Bundesverfassungsgerichts, dass die Verhältnismäßigkeit des Handelns der EZB infrage zu stellen ist?

\noindent\textbf{Comment:}
\begin{itemize}
    \setlength\itemsep{-3pt}
    \item (Carsten Schneider [Erfurt] [SPD]: Die nationalen Banken! Nicht die EZB!)
    \setlength\itemsep{-3pt}
    \item (Carsten Schneider [Erfurt] [SPD]: Falsch!)
\end{itemize}
\subsection{Merkel}
\noindent\textbf{Texts:} Also, erstens hat das Bundesverfassungsgericht sich ja gerade mit der Frage der Staatsfinanzierung auseinandergesetzt und seine Meinung dazu in der Form geäußert, dass es gesagt hat: Dieser Sachverhalt ist nicht gegeben. – Das Zweite ist: Ich habe Ihnen ja schon gesagt, dass ich das Urteil des Bundesverfassungsgerichts respektiere. Ihre Frage gibt mir aber noch mal Gelegenheit zu einer anderen Betrachtung der Frage des Euro. Der Euro ist eine Währung, die auch global Gewicht haben soll und eher mehr Gewicht bekommen soll, als sie es heute hat, denn weniger Gewicht. Damit ist der Euro auch eine Währung, die natürlich in Relation zum Handeln anderer steht. Die Europäische Zentralbank ist also eine Zentralbank, die im Vergleich zu anderen Zentralbanken der Welt – zur Fed, zur Bank of England, zu der japanischen und chinesischen Bank – steht und die sich natürlich in diesem Konzert auch bewähren muss. Sie hat also ihre vertraglichen Grundlagen in der Situation, dass die Europäische Union auf der einen Seite ein Konstrukt sui generis ist, wie man sagen würde, also kein Staat ist, aber auf der anderen Seite eine gemeinsame Währung hat, die sich bewähren soll. Ich habe ein Interesse – das meinte ich mit „starker Euro“ – an einem Euro, der mitspielen kann, der eine anleihefähige Währung ist, in der man seine Anlagen auch sicher wähnt. Deshalb ist das schon auch ein Spannungsfeld, ein globaler Akteur zu sein und gleichzeitig im europäischen Konzert zu arbeiten.

\subsection{Ebbing}
\noindent\textbf{Texts:} Vielen Dank, Herr Präsident. – Frau Bundeskanzlerin, ich freue mich, dass Sie in Ihrem letzten Podcast „Corona und Kultur“ betont haben, dass kulturelle Veranstaltungen für unser Leben von allergrößter Wichtigkeit sind. Digitale Alternativformate sind schnell und mit großer Kreativität entwickelt worden, können aber nicht das Liveerlebnis ersetzen – das wissen wir beide –; da ist eine andere Emotion drin. Sie haben vor Kurzem Bund und Länder aufgefordert, Konzepte zu entwickeln, wie Theater, Opernhäuser, Konzerthäuser und auch Kinos wieder aufmachen können. Meine Frage an Sie: Warum erfolgt das erst jetzt? Wann können wir mit Ergebnissen rechnen? Und werden die Ergebnisse mit den Ländern abgestimmt?

\subsection{Merkel}
\noindent\textbf{Texts:} Also, die Aufforderung geht ja vor allen Dingen an die Länder, auch an die Staatsministerin für Kultur, aber zuständig sind die Länder. Ich habe jetzt im Augenblick nicht den Überblick, ob es Länder gibt, die heute schon in Aussicht gestellt haben, Konzerte ab einem bestimmten Zeitpunkt wieder zu erlauben. Aber ich glaube, angesichts der Anforderungen an Abstand und Ähnliches wird man den Konzertsaal sicherlich nicht so voll besetzen können mit Zuschauern, wie man das bis zuletzt gemacht hat. Wir haben zum Beispiel das Europakonzert der Philharmoniker hier in Berlin gehabt, und auch Daniel Barenboim hat musiziert. Das heißt, auch für die Musiker gilt es dann, dafür zu sorgen, dass etwa die Bläser weit entfernt von den Geigen stehen. Aber da kann man ja kreativ sein. Ich glaube, jedes Liveerlebnis ist wirklich wünschenswert. Ich denke, dass das jetzt auch sehr schnell in Gang kommen wird.

\subsection{Ebbing}
\noindent\textbf{Texts:} Ja bitte, gerne. – Die Kinobetreiber benötigen nach meinen Informationen circa vier Wochen Vorlaufzeit, um ihre Kinos wieder aufmachen zu können, und einen einheitlichen Wiedereröffnungstermin; denn gerade nationale und internationale Filmdebuts brauchen sozusagen die bundesweite Aufmerksamkeit. Wird die Bundesregierung zusammen mit den Ländern einen gemeinsamen Termin vereinbaren, und wann ist mit einer Ankündigung zu rechnen?

\subsection{Merkel}
\noindent\textbf{Texts:} Also, an mir wird es nicht scheitern, dass da ein einheitlicher Termin gefunden wird. 

\noindent\textbf{Comment:}
\begin{itemize}
    \setlength\itemsep{-3pt}
    \item (Heiterkeit bei Abgeordneten der CDU/CSU, der SPD und des BÜNDNISSES 90/DIE GRÜNEN)
\end{itemize}
\subsection{Grundl}
\noindent\textbf{Texts:} Danke, Herr Präsident. – Frau Bundeskanzlerin, Ihre Videobotschaft vom Wochenende ist in der Kunst- und Kulturszene tatsächlich viel beachtet worden. Sie haben damit ein wichtiges Signal gegeben, nicht zuletzt, weil natürlich neben der gesellschaftlichen Relevanz auch die wirtschaftliche Relevanz von Kultur- und Kreativwirtschaft nicht zu vernachlässigen ist. Was haben Sie denn an konkreten Vorstellungen für Unterstützungsmaßnahmen, die Sie ja im Podcast auch angekündigt haben, Maßnahmen, die vielleicht speziell auf die Arbeitsrealität vieler Kreativer abgestimmt sind, die freiberuflich unterwegs sind?

\subsection{Merkel}
\noindent\textbf{Texts:} Also, wir haben ja schon einige Belange von Künstlerinnen und Künstlern mitgedacht bei den Programmen, die wir jetzt aufgelegt haben: Solo-Selbstständige, Grundsicherung. Auch die Staatsministerin für Kultur hat spezielle Programme aufgelegt. Aber wir haben vor, noch einmal einen Schritt zur Belebung der Wirtschaft insgesamt zu gehen. In dem Zusammenhang wird natürlich auch die Kultur eine wichtige Rolle spielen; denn – Sie sagten es schon – die Kreativwirtschaft ist von großer Bedeutung. Daran wird gearbeitet, und zum gegebenen Zeitpunkt können wir Ihnen das dann auch vorstellen. 

\noindent\textbf{Comment:}
\begin{itemize}
    \setlength\itemsep{-3pt}
    \item (Beifall bei der CDU/CSU sowie bei Abgeordneten der SPD und der FDP)
\end{itemize}
\end{document}