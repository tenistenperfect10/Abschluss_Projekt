% !TeX encoding = UTF-8
\documentclass{article}
\usepackage[T1]{fontenc}
\usepackage[utf8]{inputenc}
\DeclareUnicodeCharacter{202F}{\,}
\usepackage{graphicx}
\usepackage[ngerman]{babel}
\usepackage{hyperref}
\usepackage{enumitem}
\hypersetup{colorlinks=true, linkcolor=black}
\setlist[itemize]{topsep=-5pt}
\begin{document}
\title{Plenarprotokoll 19/81}
\date{}
\maketitle
\tableofcontents
\newpage
\section{Tagesordnungspunkt 18}
\subsection{Barley}
\noindent\textbf{Texts:} Werte Kolleginnen und Kollegen! Sehr geehrte Damen und Herren hier im Saal und an den Bildschirmen! Die Diskussion heute betrifft ein Thema, das viele Emotionen auslöst. Das hat damit zu tun, dass gerade für uns Frauen davon ein sehr persönlicher Lebensbereich berührt wird: unser Körper, unsere Sexualität, ganz grundlegende Lebensentscheidungen. Es ist gut und richtig, dass wir diese Debatte führen, sowohl politisch als auch gesellschaftlich. Es ist auch gut und richtig, dass sie sehr emotional geführt wird. Diese Debatte hat sich auf grundsätzliche Fragen ausgeweitet, die im Zusammenhang mit dem Schwangerschaftsabbruch stehen. Heute und hier im Deutschen Bundestag geht es allerdings um einen sehr konkreten Punkt, nämlich die Information über Schwangerschaftsabbrüche. Und da ist die Frage: Wie sieht die Lage derzeit aus für die Frauen, die ungewollt schwanger werden und Informationen darüber haben wollen, wo sie einen Schwangerschaftsabbruch vornehmen lassen können? Die Lage sieht nicht so gut aus; darüber waren sich hier die meisten, glaube ich, auch einig. Es gibt keine seriöse Stelle im Internet, wo sie eine Liste von Ärztinnen und Ärzten bekommen können, die Abbrüche vornehmen. Es gibt auch keine Hotline, wo sie diese Informationen bekommen. Und nach unseren Informationen ist es so, dass auch Schwangerschaftskonfliktberatungsstellen häufig nicht über solche Adressen verfügen oder sie zumindest nicht weitergeben. Wegen der Rechtsunsicherheit, die damit verbunden ist, verzichten auch Ärztinnen und Ärzte, die solche Schwangerschaftsabbrüche vornehmen, teilweise darauf, diese Informationen auf ihre eigene Homepage zu stellen. Deswegen besteht Handlungsbedarf, und die Position der SPD ist hinreichend zum Ausdruck gebracht worden. Wir haben jetzt in der Bundesregierung einen Gesetzentwurf erarbeitet, der sich darauf konzentriert, die Situation dieser Frauen in dieser sehr schwierigen Lage zu verbessern. Ich sage: Dieser Gesetzentwurf verbessert die Lage dieser Frauen wesentlich, und zwar gegenüber der jetzigen Lage  und auch gegenüber einer reinen Abschaffung des § 219a. Warum? Sie werden künftig, wenn dieses Gesetz kommen sollte, im Internet aus seriöser Quelle eine immer aktuelle Liste finden, bei welchen Ärztinnen und Ärzten sie einen Schwangerschaftsabbruch vornehmen lassen können und welche Methoden diese Ärztinnen und Ärzte anwenden. Sie werden seriöse Informationen zum Ablauf, zu der Methodik von Schwangerschaftsabbrüchen vorfinden. Sie werden diese Liste der Ärztinnen und Ärzte, die Abbrüche vornehmen, auch über das Hilfetelefon „Schwangere in Not“ bekommen können; dahin wenden sich viele, die sich in einer akuten Notlage befinden. Und diese Liste wird auch über die Beratungsstellen zur Verfügung gestellt werden können, die Anlaufstellen für Hilfesuchende sind.  Für die Ärztinnen und Ärzte bedeutet das, dass sie all diese Informationen auch über ihre Homepage den hilfesuchenden Frauen zur Verfügung stellen können. Das bedeutet auch ganz wesentlich Rechtssicherheit für die Ärztinnen und Ärzte.  Das heißt, die Frauen werden es künftig wesentlich leichter haben, in der Notlage, in der sie sich befinden, an die Informationen zu kommen, die sie brauchen. Ich bitte Sie deshalb heute, diesen Schritt zu tun, um die Situation dieser Frauen besser zu machen. Trotzdem müssen wir diese große gesellschaftliche Debatte weiterführen; da sind sich, glaube ich, fast alle in diesem Haus einig. Und wir müssen daraus auch in Zukunft weitere Schlüsse ziehen. Herzlichen Dank.  

\noindent\textbf{Comment:}
\begin{itemize}
    \setlength\itemsep{-3pt}
    \item (Beifall bei der SPD sowie bei Abgeordneten der CDU/CSU)
    \setlength\itemsep{-3pt}
    \item (Beifall bei der SPD)
    \setlength\itemsep{-3pt}
    \item (Beifall bei der AfD)
\end{itemize}
\subsection{Maier}
\noindent\textbf{Texts:} Sehr geehrter Herr Präsident! Meine Damen und Herren! Der Gesetzentwurf der Koalition zum Thema „Werbung für den Abbruch der Schwangerschaft“, § 219a StGB, trägt den Titel: Entwurf eines Gesetzes zur Verbesserung der Information über einen Schwangerschaftsabbruch Als ich das gelesen hatte, dachte ich mir: Das hätte ja noch viel schlimmer kommen können. Denn die Linken, die Grünen und zuletzt auch die FDP – das fand ich ganz besonders schlimm und peinlich – standen da wie eine Phalanx und forderten die Abschaffung des § 219a StGB.  Und dann die SPD, die diesen hier vorliegenden Gesetzentwurf mitträgt. Was haben Sie von der SPD im Vorfeld getönt! Zunächst hatte die SPD-Bundestagsfraktion im Dezember 2017 einen Gesetzentwurf beschlossen, der die Abschaffung des § 219a StGB vorsah.  Im SPD-Propagandablatt „Vorwärts“ erschien am 22. Februar 2018 ein Gastbeitrag von Frau Högl,  in dem es heißt: Wir als SPD-Bundestagsfraktion haben unseren Gesetzentwurf zunächst nicht in den Deutschen Bundestag eingebracht. Denn wir setzen weiter auf Gespräche mit CDU\/CSU, FDP, Grünen und Linken, um eine fraktionsübergreifende Lösung zu erarbeiten, die im Deutschen Bundestag eine Mehrheit findet. Die AfD hat man da gar nicht mit einbezogen.  Das war auch richtig so; denn in dieser Frage kann es mit uns keinen Kompromiss geben, meine Damen und Herren.  Man darf sich hier nichts vormachen: Das Ziel, den § 219a StGB ganz abzuschaffen, hat die SPD nie aufgegeben.  Ihre Jugendorganisation hat auf ihrem Bundeskongress Anfang Dezember 2018 sogar noch eins draufgesetzt.  Unter dem Titel „Für ein Recht auf reproduktive Selbstbestimmung: Legalisierung von Schwangerschaftsabbrüchen“ wurde beschlossen, sich für eine Streichung der aktuellen gesetzlichen Regelung der §§ 218 bis 219b StGB einzusetzen,  was einen Schwangerschaftsabbruch bis kurz vor der Geburt des Kindes rechtlich möglich machen würde.  Schämen Sie sich gar nicht?  Noch mal: Als ich diesen Entwurf gelesen habe, dachte ich: Das hätte noch viel schlimmer kommen können. Die SPD hat bei ihren Verhandlungen nämlich eins nicht erreicht, was aber das eigentliche Ziel der ganzen Gesetzesinitiative war: den Einstieg in den Ausstieg aus dem § 218 StGB.  Ich nehme an, dass Frau Barley auch deshalb hier so in dieser Trauerstimmung gesprochen hat.  Dieser Gesetzentwurf enthält bestenfalls eine Veränderung und Aktualisierung des Zugangs zu Informationen, ändert aber an der Grundstruktur des Schutzes des ungeborenen Lebens nichts, was in dem Referentenentwurf auch zutreffend an den Anfang gestellt wurde. Ärgerlich ist allerdings – hier hat sich die SPD wohl durchgesetzt –, dass für die Bürokratiekosten, die nun entstehen, zusammengerechnet 416 500 Euro pro Jahr ausgegeben werden müssen,  obwohl man nur die Daten aufbereiten müsste, die ohnehin nach § 18 des Schwangerschaftskonfliktgesetzes an das Statistische Bundesamt gemeldet werden müssen. Hier wird unnötig jedes Jahr der Gegenwert eines Einfamilienhauses in mittlerer Wohnlage zum Fenster rausgeschmissen.  Liebe SPD, wieder nichts Bedeutsames zustande gebracht, wieder hinter den Erwartungen der eigenen Klientel zurückgeblieben. Kein Wunder, dass sich immer mehr Ihrer linken Wähler von Ihnen abwenden.  Ihre sozialreformerischen Experimente, mit denen Sie jetzt Ihr Image aufbessern wollen, werden ebenso enden: Als Tiger gestartet, als Bettvorleger geendet.  Dieser Fall zeigt wieder einmal, welche Bedeutung Sie von der SPD in der Koalition und darüber hinaus in der politischen Landschaft haben. Entweder können Sie sich nicht wirklich durchsetzen, so wie hier, oder Sie setzen sich durch, und dann kommt dabei nur Schädliches für unser Land, für unser Volk heraus. Ziehen Sie endlich Konsequenzen. Machen Sie das, was Ihnen der junge, süße Kevin empfiehlt: Treten Sie aus der Regierung aus, kündigen Sie die Koalition, und machen Sie den Weg frei für Neuwahlen.  Wir von der AfD sehen weiterhin keinen Anlass, irgendetwas an dem § 219a StGB zu ändern. Bei 100 000 Abtreibungen pro Jahr ist indiziell nachgewiesen, dass es keinen Mangel an Informationsmöglichkeiten gibt. Das wurde nur als Vorwand herbeikonstruiert, um alte linke Forderungen nach der Legalisierung von Schwangerschaftsabbrüchen wieder auf die Tagesordnung zu setzen.  Wir lehnen auch die Verschwendung von Steuergeldern für ein neues, unnötiges Informationsmedium ab. Vielen Dank.  

\noindent\textbf{Comment:}
\begin{itemize}
    \setlength\itemsep{-3pt}
    \item (Timon Gremmels [SPD]: Die kriegen wir gerade noch so zusammen!)
    \setlength\itemsep{-3pt}
    \item (Beifall bei der AfD)
    \setlength\itemsep{-3pt}
    \item (Dr. Kirsten Tackmann [DIE LINKE]: Da kennen Sie sich als Mann ja aus!)
    \setlength\itemsep{-3pt}
    \item (Dr. Eva Högl [SPD]: Natürlich!)
    \setlength\itemsep{-3pt}
    \item (Dr. Kirsten Tackmann [DIE LINKE]: Leier! Leier!)
    \setlength\itemsep{-3pt}
    \item (Nicole Bauer [FDP]: Das fordern wir immer noch aus voller Überzeugung!)
    \setlength\itemsep{-3pt}
    \item (Zuruf von der SPD: Quatsch!)
    \setlength\itemsep{-3pt}
    \item (Christine Aschenberg-Dugnus [FDP]: Darum geht es hier gar nicht!)
    \setlength\itemsep{-3pt}
    \item (Zuruf von der AfD: Pfui!)
    \setlength\itemsep{-3pt}
    \item (Dr. Eva Högl [SPD]: Warum auch?)
    \setlength\itemsep{-3pt}
    \item (Beifall bei der AfD – Dr. Kirsten Tackmann [DIE LINKE]: Eben!)
    \setlength\itemsep{-3pt}
    \item (Gabriele Hiller-Ohm [SPD]: Widerlich!)
    \setlength\itemsep{-3pt}
    \item (Beifall bei der CDU/CSU)
    \setlength\itemsep{-3pt}
    \item (Beifall bei der AfD – Dr. Johannes Fechner [SPD]: Dann schauen Sie sich mal die Umfragen an! – Leni Breymaier [SPD]: Thema!)
    \setlength\itemsep{-3pt}
    \item (Beifall bei Abgeordneten der AfD)
    \setlength\itemsep{-3pt}
    \item (Beifall bei Abgeordneten der SPD)
    \setlength\itemsep{-3pt}
    \item (Dr. Eva Högl [SPD]: Ja!)
    \setlength\itemsep{-3pt}
    \item (Beifall bei Abgeordneten der LINKEN – Dr. Eva Högl [SPD]: Selbstverständlich! Das ist unsere Position!)
\end{itemize}
\subsection{Frei}
\noindent\textbf{Texts:} Herr Präsident! Liebe Kolleginnen und Kollegen! Ich glaube, immer dann, wenn es um die rechtlichen Rahmenbedingungen der ethischen Fragen von Beginn und Ende des menschlichen Lebens geht, geht es um die umstrittensten und auch herausforderndsten Themen, die wir hier im Deutschen Bundestag zu beraten haben. Wenn man sich an die Debatten der Jahre 1974 – da habe ich größere Schwierigkeiten – und 1992 erinnert, dann sieht man, dass das natürlich ganz umstrittene gesellschaftliche Debatten waren, bei denen es allerdings gelungen ist, am Ende zu einer Lösung, zu einem Kompromiss zu kommen, der nicht nur damals, sondern auch in den letzten mehr als 25 Jahren zu einer Befriedung geführt hat. Deswegen ist es natürlich auch nicht verwunderlich, dass man jetzt bei der Frage des Werbeverbots für Schwangerschaftsabbrüche ähnlich emotional mit diesem Thema umgeht. Das, was jetzt auf dem Tisch liegt, löst keinen großen Jubel aus. Das haben wir in den bisherigen Reden schon gehört, und das wird auch die weitere Debatte zeigen; denn es ist natürlich ein Kompromiss, der für viele von uns auch ein schmerzhafter Kompromiss ist – egal in welche Richtung die eigene Argumentation gegangen ist. Aber, ich glaube, wer sich ehrlich macht, muss sagen, dass bei dieser schwierigen Frage und bei den auseinanderklaffenden Vorstellungen hier im Haus und auch in der Gesellschaft nichts anderes als ein schmerzhafter Kompromiss möglich war. Deswegen ist die Lösung, die jetzt auf dem Tisch liegt, auch ein Erfolg an sich; das muss man, glaube ich, ganz klar formulieren.  Ich will offen bekennen: Aus meiner Sicht – und da spreche ich sicherlich auch für viele Kolleginnen und Kollegen hier im Haus – hätte es keiner Änderung des § 219a des Strafgesetzbuches bedurft.  Er ist für uns ein integraler Bestandteil dessen, was wir 1992 hier im Hause vereinbart haben.  Deswegen ist es für mich auch ganz besonders wichtig, dass es im Gesetzentwurf nicht nur um die schwierige Konfliktsituation der Frau – diese ist völlig zu Recht in den Fokus genommen worden –, sondern – das hat uns das Bundesverfassungsgericht in mehreren Entscheidungen aufgetragen – eben auch um das Lebensrecht des Ungeborenen geht. Das muss in einer solchen Güterabwägung letztlich auch betrachtet werden. Dem kommt dieser Gesetzentwurf nach. Deswegen kann man, glaube ich, schon sagen: Es ist ein guter Gesetzentwurf.  Wir müssen an der einen oder anderen Stelle das Ergebnis durchaus vom Ende denken. Es ist die Kritik geäußert worden, dass Frauen in schwierigen Lebens- und Konfliktsituationen nicht über genügend Informationen verfügen können. Wenn man die 100 000 Schwangerschaftsabbrüche – in manchen Jahren sind es sogar 130 000 Schwangerschaftsabbrüche – ins Verhältnis zu den jährlich etwa 700 000 Geburten in Deutschland setzt, darf man Zweifel an diesem Argument haben. Jeder von uns, der mal ins Internet geht und eine Suchmaschine betätigt, wird sehr schnell an alle möglichen Informationen kommen. Deswegen, glaube ich, ist das in der Tat kein durchschlagendes Argument. Nichtsdestotrotz: Wenn Frauen in schwierigen Lebenssituationen der Auffassung sind, dass diese Informationen auch nach einer Beratung für sie nicht leicht zugänglich sind, dann sollten wir alles dafür tun, dass sich das ändert. Dieser Gesetzentwurf macht das möglich. Die Bundeszentrale für gesundheitliche Aufklärung wird beispielsweise die entsprechende Ärzteliste, monatlich aktualisiert, im Netz zur Verfügung stellen, sodass diese Informationen für Frauen leicht zugänglich werden.  Umgekehrt wird auch Rechtssicherheit für Ärzte geboten. Deswegen möchte ich an der Stelle sagen: Wer zu diesem Gesetzentwurf die Auffassung vertritt, dass es keine Rechtssicherheit für Ärzte gibt, muss sich wirklich fragen, ob es ihm am Ende um die Rechtssicherheit geht oder ob es nicht viel eher darum geht, dass man die Grenzen des Rechts auszuloten versucht oder vielleicht auch ganz bewusst übertritt, weil dieses Recht nicht mit den persönlichen Vorstellungen in Einklang zu bringen ist.  Diese Frage muss man, glaube ich, schon klar beantworten. Uns als Union geht es darum, dass Werbung für Schwangerschaftsabbrüche auch zukünftig unter Strafe steht. Das ist mit dem Gesetzentwurf tatsächlich gelungen. Ein zweiter Punkt ist uns wichtig. Wir wollen, dass die Beratung im Mittelpunkt steht. Wir sollten im parlamentarischen Verfahren im Blick haben, ob wir für die Pluralität der Beratung noch etwas erreichen können. Das Entscheidende aber ist: Beratung muss an erster Stelle stehen. Beratung ist das Wichtigste – als Hilfestellung, als Unterstützung und keinesfalls als Bevormundung. Herzlichen Dank.  

\noindent\textbf{Comment:}
\begin{itemize}
    \setlength\itemsep{-3pt}
    \item (Beifall bei Abgeordneten der CDU/CSU und der SPD)
    \setlength\itemsep{-3pt}
    \item (Beifall bei Abgeordneten der CDU/CSU – Dr. Gesine Lötzsch [DIE LINKE]: Das glaube ich unbesehen!)
    \setlength\itemsep{-3pt}
    \item (Beifall bei der CDU/CSU)
    \setlength\itemsep{-3pt}
    \item (Ulle Schauws [BÜNDNIS 90/DIE GRÜNEN]: Und das ist falsch!)
    \setlength\itemsep{-3pt}
    \item (Beifall bei der FDP – Dr. Kirsten Tackmann [DIE LINKE]: Es reden bisher nur Männer!)
    \setlength\itemsep{-3pt}
    \item (Dr. Kirsten Tackmann [DIE LINKE]: Und die Männer? Schwangerschaften entstehen nicht ohne Befruchtung!)
    \setlength\itemsep{-3pt}
    \item (Beifall bei Abgeordneten der CDU/CSU – Dr. Gesine Lötzsch [DIE LINKE]: Weil es falsches Recht ist! – Dr. Kirsten Tackmann [DIE LINKE]: Sie sind doch sonst immer für Rechtssicherheit!)
\end{itemize}
\subsection{Thomae}
\noindent\textbf{Texts:} Sehr geehrter Herr Präsident! Verehrte Kolleginnen! Verehrte Kollegen! Die Regierung legt uns heute einen Gesetzentwurf zur Verbesserung der Information über einen Schwangerschaftsabbruch vor. Das ist ein reichlich prosaischer Name – wenn man sich mal überlegt, wie die Koalition ihre Gesetze bislang nennt: Starke-Familien-Gesetz, Gute-Kita-Gesetz, jetzt kommt ein Geordnete-Rückkehr-Gesetz aus dem Innenministerium. Hier haben Sie sich nichts einfallen lassen. Ich schlage vor: Nennen Sie Ihr Gesetz doch Koalitionsfriedensrettungsgesetz; denn nur darum geht es Ihnen! Das ist der Zweck dieses Gesetzes.  Es geht Ihnen nicht wirklich um eine Verbesserung der Information über Schwangerschaftsabbrüche.  Wenn es Ihnen um eine Informationsverbesserung ginge, dann wäre doch der Maßstab, auf den es uns immer angekommen ist, dass die sachliche Information über Schwangerschaftsabbrüche, die ein Arzt vornimmt, nicht mehr strafbar sein soll. Als Beispiel nenne ich die Gießener Ärztin, Frau Hänel. Das war für uns immer die Messlatte. Auch für Sie, liebe Kolleginnen und Kollegen der SPD, war das immer der Maßstab bzw. die Messlatte.  Ihr Gesetzentwurf bleibt deutlich unter dieser Messlatte. Der Arzt darf weiterhin nicht frei informieren, darf weiterhin nicht aus seiner beruflichen Erfahrung schöpfen.  Er darf nicht sagen: Das sind die Fragen der Frauen und Mädchen, die zu mir kommen. Ich beantworte diese aus meiner beruflichen Erfahrung und stelle diese Informationen zur Verfügung. – Der Arzt ist doch derjenige, der das weiß.  Die Frauen und Mädchen können sich weiterhin nicht bei den Ärzten, zu denen sie gehen wollen, frei informieren. Das ist keine echte Verbesserung der Information für die Frauen.  Woher kommt das riesige Misstrauen, das Sie den Ärzten entgegenbringen? Kammerversammlungen müssen erst darüber befinden, wie informiert werden soll, welche Texte sich anbieten. Eine Behörde, die Bundeszentrale für gesundheitliche Aufklärung, muss sich Gedanken darüber machen, wie sie aufklärt. Das zeigt die ganze Absurdität des Vorhabens.  Sie wenden das schärfste Schwert des Rechtsstaates an – die Ultima Ratio, das letzte Mittel –, um Rechtsgüter zu schützen, nämlich unser Strafrecht, um strafbares Unrecht zu ahnden. Schon die Vorstellung, dass eine sachliche Information über erlaubte Tätigkeiten strafbares Unrecht sein könnte, ist doch absurd. Es ist absurd, zu glauben, dass das strafbar sein muss.  Sie treiben es sogar noch auf die Spitze. Sie sagen: Was bei den einen, nämlich bei der Bundesärztekammer und bei der Bundeszentrale für gesundheitliche Aufklärung, gesetzlicher Auftrag ist, ist, wenn es ein Arzt macht, so schlimm, dass er mit dem schärfsten Schwert des Staates rechnen muss, dem Strafrecht, der Ultima Ratio unseres Staates.  Ihre Logik ist, dass dieselbe sachliche Information bei dem einen gesetzlicher Auftrag und bei dem anderen strafbares Unrecht ist. Das ist absurd.  Sie benutzen das Strafrecht als Allzweckwaffe der Politik statt als Ultima Ratio, als letztes Mittel des Rechtsstaates. Das kann hier nicht sinnvoll sein.  Welches gesetzlich geschützte Rechtsgut verletzt der Arzt eigentlich, wenn er sachliche Informationen über eine erlaubte Tätigkeit veröffentlicht, die man sowieso woanders, bei der Bundesärztekammer oder bei der Bundeszentrale für gesundheitliche Aufklärung, finden kann? Welches Rechtsgut wird hier verletzt, dass das Strafrecht bemüht werden muss?  Wenn Sie diese Frage – damit komme ich zum Schluss, Herr Präsident – nicht überzeugend beantworten können, dann ist Ihr Gesetzentwurf nicht nur absurd – das sind wir ja schon gewohnt –, er ist vor allem verfassungswidrig.  Deswegen kündige ich Ihnen an – das ist mein letzter Satz –, dass wir in der parlamentarischen Beratung einen Änderungsantrag einbringen werden, der zeigen wird, wie man es nach unserer Auffassung verfassungskonform regeln kann. Wir kündigen aber auch an, dass wir, wenn Sie Ihren Gesetzentwurf nicht verändern, eine Normenkontrollklage anstrengen werden. Wir laden Sie, die Linken, die SPD und die Grünen, ein, sich uns anzuschließen. Vielen Dank.  

\noindent\textbf{Comment:}
\begin{itemize}
    \setlength\itemsep{-3pt}
    \item (Beifall bei der FDP sowie bei Abgeordneten der LINKEN und des BÜNDNISSES 90/DIE GRÜNEN)
    \setlength\itemsep{-3pt}
    \item (Beifall bei der FDP und dem BÜNDNIS 90/DIE GRÜNEN sowie bei Abgeordneten der LINKEN)
    \setlength\itemsep{-3pt}
    \item (Ulle Schauws [BÜNDNIS 90/DIE GRÜNEN]: Gute Frage!)
    \setlength\itemsep{-3pt}
    \item (Beifall bei Abgeordneten der FDP und der LINKEN – Christine Aschenberg-Dugnus [FDP]: Unglaublich!)
    \setlength\itemsep{-3pt}
    \item (Beifall bei der FDP, der LINKEN und dem BÜNDNIS 90/DIE GRÜNEN)
    \setlength\itemsep{-3pt}
    \item (Christine Aschenberg-Dugnus [FDP]: Genau!)
    \setlength\itemsep{-3pt}
    \item (Beifall bei der FDP sowie bei Abgeordneten des BÜNDNISSES 90/DIE GRÜNEN)
    \setlength\itemsep{-3pt}
    \item (Beifall bei der FDP und dem BÜNDNIS 90/DIE GRÜNEN sowie bei Abgeordneten der LINKEN – Dr. Kirsten Tackmann [DIE LINKE]: Komplett absurd!)
    \setlength\itemsep{-3pt}
    \item (Beifall bei der LINKEN)
    \setlength\itemsep{-3pt}
    \item (Michael Grosse-Brömer [CDU/CSU]: Ihr wolltet nicht regieren! Deswegen müsst ihr keine Kompromisse machen!)
\end{itemize}
\subsection{Möhring}
\noindent\textbf{Texts:} Danke. – Herr Präsident! Kolleginnen und Kollegen! Ich kann mich den Worten des Kollegen Thomae schon mal vollumfänglich anschließen. Ich möchte aber die Gelegenheit nutzen und auch noch mal in Erinnerung rufen, warum es diese Debatte um den § 219a überhaupt gibt; denn ich habe schon den Eindruck, dass die für den Entwurf verantwortlichen SPD-Ministerinnen zumindest das Ausmaß des Problems und auch die Beschlusslage der eigenen Partei und Fraktion verdrängt haben.  Der § 219 ist ein Relikt aus der Nazizeit.  Durch § 219a werden Ärztinnen und Ärzte kriminalisiert und in ihrer freien Berufsausübung eingeschränkt. Sie dürfen nicht darüber informieren, ob sie die Leistung anbieten, und sie dürfen auch nicht darüber informieren, wie sie Abbrüche durchführen. In der medizinischen Ausbildung kommt diese Leistung übrigens gar nicht mehr vor. Der § 219a birgt als Frauenbild die verantwortungslose Schwangere, die keine Informationen verarbeiten kann, nicht alleine entscheidungsfähig ist und auf Werbung hereinfällt. In der Folge führen immer weniger Praxen noch Abbrüche durch. Mit Ihrem Kompromiss, der jetzt auch noch in einem Affenzahn durch den Bundestag gepeitscht werden soll, werden diese Folgen nicht umgekehrt, sondern vielleicht sogar verschärft. Frau Barley, Sie sagen, es wird Rechtssicherheit geschaffen. Tja, Ärztinnen und Ärzte dürfen dann schreiben, dass sie Schwangerschaftsabbrüche anbieten. Aber es wird auch festgeschrieben, dass sie nicht über die Methoden und Abläufe informieren und nur zu Behördenseiten verlinken dürfen. Das kann man natürlich auch „Rechtssicherheit“ nennen, nicht? Die Fachleute haben dann die Sicherheit, dass sie sich mit fachlicher Information nicht weiterhin strafbar machen. Das, finde ich, ist ganz toll. Ehrlich.  Sie setzen im Ergebnis Informationen weiterhin mit Werbung gleich und meinen wohl, dass Frauen durch fachliche Infos zu sehr verwirrt werden. Bei aller Wertschätzung, Ministerin Barley: Ich kann wirklich nicht verstehen, dass Sie das als Erfolg verkaufen.  Die jetzt vor Gericht stehenden Ärztinnen und Ärzte würden sich mit ihren Webseiten immer noch strafbar machen. Kollege Thomae hat darauf hingewiesen: Das war eigentlich mal der Maßstab, den Sie uns hier versprochen haben, liebe Kolleginnen und Kollegen. Was macht ihr da bloß? Jetzt habt ihr nicht mal das Durchhaltevermögen, die entsprechende Empörung auszuhalten, macht lieber Verrenkungen und versucht, die Debatte möglichst vor den Europawahlen zu beenden. Glaubwürdigkeit, liebe Kolleginnen und Kollegen von der SPD, geht anders.  Ihr macht das vor allem in einer Zeit, in der Abtreibungsgegner militanter werden, sich weltweit mit rechten Kräften vernetzen und gegen jede Selbstbestimmung zu Felde ziehen. Das ist verantwortungslos. Dass schwangere Frauen durch Gehsteigbelästigungen eingeschüchtert werden sollen und die Union ständig das Thema Lebensschutz mit dem Thema Schwangerschaftsabbruch in Verbindung bringt, zeigt doch, was für ein Klima geschaffen wird.  Da müssen fortschrittliche Kräfte doch gemeinsam gegenhalten. Aber die Änderung des § 219a stärkt solchen Leuten den Rücken. Mal ganz abgesehen davon, liebe Kolleginnen und Kollegen der SPD: Es gibt politische Fragen, bei denen gibt es ein Richtig und ein Falsch. Da hilft überhaupt kein technokratisches Gerede. Denn innerhalb des Strafgesetzbuches kann es keine Lösung geben. Solange das Strafrecht als schärfstes juristisches Schwert Schwangerschaftsabbrüche regelt, gibt es kein Recht auf körperliche Selbstbestimmung, sondern allerhöchstens eine Erlaubnis. Frauen brauchen aber keine Erlaubnis. Sie brauchen das klare und das eindeutige Recht, dass nur sie alleine entscheiden können, ob sie eine Schwangerschaft fortführen oder ob sie sie beenden wollen.  Deswegen sagt meine Fraktion ganz klar: Der § 219a muss gestrichen werden! Keine miesen Kompromisse! Vielen Dank.  

\noindent\textbf{Comment:}
\begin{itemize}
    \setlength\itemsep{-3pt}
    \item (Thorsten Frei [CDU/CSU]: Ganz falsch!)
    \setlength\itemsep{-3pt}
    \item (Beifall beim BÜNDNIS 90/DIE GRÜNEN)
    \setlength\itemsep{-3pt}
    \item (Beifall bei der LINKEN sowie bei Abgeordneten des BÜNDNISSES 90/DIE GRÜNEN und des Abg. Stephan Thomae [FDP])
    \setlength\itemsep{-3pt}
    \item (Beifall bei der LINKEN sowie bei Abgeordneten des BÜNDNISSES 90/DIE GRÜNEN)
    \setlength\itemsep{-3pt}
    \item (Dr. Gesine Lötzsch [DIE LINKE]: Richtig!)
    \setlength\itemsep{-3pt}
    \item (Beifall bei der LINKEN und dem BÜNDNIS 90/DIE GRÜNEN sowie bei Abgeordneten der FDP)
    \setlength\itemsep{-3pt}
    \item (Beifall bei der LINKEN und der FDP sowie bei Abgeordneten des BÜNDNISSES 90/DIE GRÜNEN)
    \setlength\itemsep{-3pt}
    \item (Beifall bei der LINKEN – Michael Grosse-­Brömer [CDU/CSU]: Wie erstaunlich! Lebensschutz ist Ihnen nicht wichtig, das hört man bei der Rede! Das ist unfassbar, was Sie da sagen!)
    \setlength\itemsep{-3pt}
    \item (Beifall bei der LINKEN sowie bei Abgeordneten des BÜNDNISSES 90/DIE GRÜNEN – Alexander Hoffmann [CDU/CSU]: Unterirdisch, Frau Kollegin!)
\end{itemize}
\subsection{Schauws}
\noindent\textbf{Texts:} Herr Präsident! Verehrte Kolleginnen und Kollegen! Wenn ich ein Mathematikstudent in den Endzwanzigern wäre, eine Abneigung gegen Sexualaufklärung und nicht den Hauch einer Ahnung von Frauen hätte, wenn mein Hobby wäre, das Internet nach Ärztinnen und Ärzten zu durchforsten, die Schwangerschaftsabbrüche durchführen, um sie reihenweise anzuzeigen, dann, meine Damen und Herren, würde ich mich über dieses restriktive Gesetz der Bundesregierung freuen. Warum? Weil § 219a im Strafgesetzbuch stehen bleibt, trotz der Gerichtsverfahren gegen Ärztinnen und Ärzte, trotz einer deutlichen Mehrheit in allen Umfragen für die Streichung. Schon die kleinste Abweichung vom erlaubten Begriff auf der Homepage der Ärztinnen und Ärzte würde es möglich machen, sie auch weiterhin anzuzeigen. Darum stelle ich Ihnen von der Bundesregierung eine zentrale Frage: Wem dient dieses komplizierte Gesetz?  Dient es den Ärztinnen und Ärzten? Gibt es ihnen Rechtssicherheit? Stärkt es ihre Berufsfreiheit? Können sie die Staatsaufgabe, Schwangere in jeder Lebenslage bestmöglich zu versorgen, ohne Einschränkung erfüllen? Oder dient dieses Gesetz den Frauen in diesem Land? Nein! Dieser Kompromiss dient weder den Frauen noch den Ärztinnen und Ärzten, meine Damen und Herren.  Dieses Gesetz signalisiert den Abtreibungsgegnern: Weitermachen, Stimmung machen gegen Frauen, gegen Ärztinnen und Ärzte, gegen Selbstbestimmung. Und das ist fatal.  Zu Recht sind die Entrüstung und der Widerstand gegen dieses Gesetz groß, und zwar auch, weil die Union die Auseinandersetzung um § 219a zu einer Debatte um § 218 macht. Und da sage ich noch einmal: Das ist unredlich. Das steht hier nicht auf der Tagesordnung.  Sie gehen sogar so weit und übernehmen eins zu eins Positionen der Abtreibungsgegner. Diese Studie für sage und schreibe 5 Millionen Euro ist ein unmissverständlicher Beleg dafür. Ich frage Sie ganz konkret: Wissen Sie, wo Sie da gerade hinsteuern?  Stattdessen machen Sie Verrenkungen, um diesen Paragrafen – er ist aus dem Jahre 1933 – weiter zu legitimieren und ihn ja nicht streichen zu müssen. Akrobatik ist das, was Sie hier machen. Jetzt wäre die einmalige Chance – das sage ich klar in Richtung Union –, die Lage für ungewollt schwangere Frauen umfassend zu verbessern und dieses unerträgliche Schüren von Misstrauen gegen sie zu beenden,  die Leistungen der Ärztinnen und Ärzte ohne Vorbehalt anzuerkennen und eine gute, einvernehmliche Lösung ohne den § 219a zu finden. Aber Sie in der Union haben sich entschieden, genau diesen Kurs nicht zu fahren, weiter zu entmündigen und eine, wie ich finde, gefährliche Stimmung zu erzeugen  gegen die klare Haltung etlicher Organisationen, auch aus den Kirchen, und gegen die Mehrheit der Bevölkerung. Und diese Verantwortung tragen Sie von der Sozialdemokratie mit. Ich sage Ihnen ganz klar: Wir lassen Ihnen das nicht durchgehen, auch wenn Sie das hier nächste Woche abschließen wollen.  Meine Damen und Herren, es ist ja nicht das erste Mal, dass hier herablassend über Frauen gesprochen wird, wenn es um ihre reproduktiven Rechte geht.  – Hören Sie gerne mal zu. – Ich sage nur: Smarties. Herr Spahn scheint also erneut besorgt, dass die Frauen nicht eigenverantwortlich entscheiden können. Er will für eine horrende Summe genau hingucken, wie schlecht es ihnen wirklich geht. Verstehen Sie mich nicht falsch: Studien sind wichtig, wenn es Erkenntnisdefizite gibt. Aber wenn es diese nicht gibt und wenn es dieses von Abtreibungsgegnern behauptete Post-Abortion-Syndrom erst recht nicht gibt, dann muss man sehr, sehr kritisch fragen: Wofür wollen Sie diese 5 Millionen Euro wirklich bewilligen?  Wenn Sie sich um Frauen sorgen, dann nehmen Sie dieses Geld in die Hand für eine bessere Versorgung durch Hebammen und Geburtsstationen. Das würde Frauen helfen.  Ich komme zum Schluss. – Aber hören Sie auf, zu suggerieren, Frauen wüssten nicht, was sie tun. Sie von Union und SPD verantworten mit, dass laufende Klagen nicht eingestellt werden und betende Abtreibungsgegner weiter Beratungsstellen und Arztpraxen belagern. Das ist ein unhaltbarer Zustand für alle.  Ich sage: Unsere Oppositionsgesetzentwürfe zur Streichung liegen vor. Es liegt auch ein Gesetzentwurf in der Schublade der SPD. Sie haben es jetzt in der Hand. Lassen Sie uns gemeinsam § 291a streichen! Vielen Dank.  

\noindent\textbf{Comment:}
\begin{itemize}
    \setlength\itemsep{-3pt}
    \item (Stephan Thomae [FDP]: Dem Koalitionsfrieden!)
    \setlength\itemsep{-3pt}
    \item (Beifall bei der SPD sowie bei Abgeordneten der CDU/CSU)
    \setlength\itemsep{-3pt}
    \item (Beifall beim BÜNDNIS 90/DIE GRÜNEN)
    \setlength\itemsep{-3pt}
    \item (Beifall beim BÜNDNIS 90/DIE GRÜNEN und bei der LINKEN – Elisabeth Winkelmeier-­Becker [CDU/CSU]: Wir haben das Thema nicht auf die Tagesordnung gesetzt!)
    \setlength\itemsep{-3pt}
    \item (Dr. Petra Sitte [DIE LINKE]: Die wissen das! – Dr. Kirsten Tackmann [DIE LINKE]: Das ist das Schlimme, dass die das wissen!)
    \setlength\itemsep{-3pt}
    \item (Ingmar Jung [CDU/CSU]: Sagen Sie einmal, was die Grünen in Hessen in dieser Frage machen!)
    \setlength\itemsep{-3pt}
    \item (Beifall beim BÜNDNIS 90/DIE GRÜNEN, bei der FDP und der LINKEN)
    \setlength\itemsep{-3pt}
    \item (Beifall beim BÜNDNIS 90/DIE GRÜNEN und bei der LINKEN sowie bei Abgeordneten der FDP)
    \setlength\itemsep{-3pt}
    \item (Beifall beim BÜNDNIS 90/DIE GRÜNEN, bei der FDP und der LINKEN – Michael Grosse-Brömer [CDU/CSU]: Das ist falsch!)
\end{itemize}
\subsection{Giffey}
\noindent\textbf{Texts:} Sehr geehrter Herr Präsident! Meine Damen und Herren! Bei diesem Thema – man merkt es an der hitzigen Diskussion – könnten die Positionen nicht unterschiedlicher sein. Das ist auch innerhalb der Regierungskoalition so, und das ist kein Geheimnis. Unsere schwierige Aufgabe war es, zwischen der Position, den § 219a nicht anzurühren, und der SPD-Position, ihn abzuschaffen, einen Kompromiss zu finden, einen Weg, den alle drei Regierungspartner mitgehen können. Unser Ziel in den Verhandlungen war – diese Position haben wir mit Katarina Barley immer vertreten –, einerseits das Informationsrecht der Frauen zu stärken und anderseits Rechtssicherheit für die Ärztinnen und Ärzte zu schaffen.  Das waren die beiden wesentlichen Punkte, mit denen wir auch in die Verhandlungen gegangen sind. Es ist ganz klar: Jede Frau, die vor der Frage steht, ein Kind trotz schwieriger Umstände zur Welt zu bringen oder nicht, die Schwangerschaft abzubrechen oder nicht, befindet sich in einer Notsituation. Sie ist in einer außergewöhnlichen Lebenssituation, und es ist ganz klar, dass weder eine Frau in dieser Lage noch eine Ärztin, ein Arzt sich diese Entscheidung leicht macht.  Wir haben dafür gestritten, dass die Informationen im Vergleich dazu, wie sie jetzt sind, verbessert werden: dass der Zugang zu Informationen verbessert wird, dass es endlich – das ist schon lange gefordert worden – eine verlässliche Liste über Ärztinnen und Ärzte geben wird, die das machen, und dass auch öffentlich darüber informiert wird, welche Methoden sie anwenden. Wir haben erreicht – auch das war eine wesentliche Forderung –, dass in Zukunft jeder Arzt, jede Ärztin in Deutschland darüber informieren darf, dass er bzw. sie Schwangerschaftsabbrüche vornimmt. Das war bisher nicht so.  Es ist ein Fortschritt, dass wir das geschafft haben und dass das nun möglich ist. Wir haben vereinbart, dass von der Bundeszentrale für gesundheitliche Aufklärung, der Bundesärztekammer und dem Bundesamt für Familie und zivilgesellschaftliche Aufgaben diese Informationen in den 1 600 Beratungsstellen bundesweit auch verfügbar gemacht werden. Ich würde gerne weiter ausführen. – Wir haben mit den Apotheken in Deutschland eine Zusammenarbeit darüber, dass sie über unser Hilfetelefon „Schwangere in Not“ informieren. Das Angebot dieses Hilfetelefons besteht schon jetzt, und wir wollen dies noch weiter ausbauen. Rund um die Uhr in 18 Sprachen werden alle Frauen informiert, die in dieser Notsituation sind. Ich sage hier einmal die Nummer: 0800 40 40 020. Dieses Hilfetelefon ist rund um die Uhr an jedem Tag des Jahres verfügbar. Es informiert Frauen über Ärzte, die Schwangerschaftsabbrüche vornehmen, über die Methoden, die sie anwenden, über die Hilfsangebote, über die Beratungsstellen. Es war uns ganz wichtig, dass wir zu einem solchen Angebot kommen. Das ist gelungen.  Es ist aber ganz klar: Wir wollen, dass es darüber hinausgeht. Wir sehen, dass wir Versorgungslücken in Deutschland haben. Wir sehen, dass wir klare Verbesserungen bei der Qualifizierung der Ärztinnen und Ärzte brauchen. Deshalb haben wir – und das ist mit dem Bundesgesundheitsminister vereinbart – auch ausgemacht –, dass er ein Konzept dafür vorlegen muss, wie er diese Qualifizierung der Ärzte fortentwickeln und verbessern will. Das wird er bis zum Ende des Jahres auch tun.  Wir wollen, dass Ärztinnen und Ärzte genau wissen, welche Informationen sie auf ihre Internetseite stellen können und zu welchen Informationen sie verlinken können. Das haben wir vereinbart. Das ist für einige von Ihnen vielleicht nicht ausreichend. Aber es ist das, was möglich war. Das ist eine deutliche Verbesserung der Situation.  Das kommt den Zielen zugute, die wir uns gesetzt haben, nämlich die Informationen für die Frauen zu verbessern und auch für die Rechtssicherheit der Ärztinnen und Ärzte. Vielen herzlichen Dank. 

\noindent\textbf{Comment:}
\begin{itemize}
    \setlength\itemsep{-3pt}
    \item (Beifall bei der SPD sowie bei Abgeordneten der CDU/CSU)
    \setlength\itemsep{-3pt}
    \item (Beifall bei der SPD)
    \setlength\itemsep{-3pt}
    \item (Katja Keul [BÜNDNIS 90/DIE GRÜNEN]: Beides nicht geschafft!)
    \setlength\itemsep{-3pt}
    \item (Beifall bei Abgeordneten der SPD)
    \setlength\itemsep{-3pt}
    \item (Ulle Schauws [BÜNDNIS 90/DIE GRÜNEN]: Reden Sie es doch nicht schön! Das ist das Schlimmste daran!)
\end{itemize}
\subsection{Keul}
\noindent\textbf{Texts:} Vielen Dank, Herr Präsident. – Frau Ministerin, ich habe eine Frage, weil Sie gesagt haben, es gebe jetzt Rechtssicherheit für die Ärztinnen und Ärzte. Das kann man so oder so sehen. Sicher ist jedenfalls, dass sowohl Frau Hänel als auch alle anderen Ärztinnen, die bislang angezeigt worden sind, nach Ihrem Gesetz weiterhin strafrechtlich verurteilt werden müssen. Ist Ihnen das eigentlich klar? Was ist das also für eine Rechtssicherheit, die sie haben? Sie haben keine Rechtssicherheit, zu informieren, sondern Rechtssicherheit, weiterhin strafrechtlich verurteilt zu werden. 

\noindent\textbf{Comment:}
\begin{itemize}
    \setlength\itemsep{-3pt}
    \item (Beifall bei Abgeordneten des BÜNDNISSES 90/DIE GRÜNEN)
\end{itemize}
\subsection{Giffey}
\noindent\textbf{Texts:} Wir reden hier über ein Gesetz für die Zukunft und nicht für die Vergangenheit, für Verfahren, die schon laufen, und zwar nach den bisherigen Regeln. Wir reden davon, dass Ärztinnen und Ärzte in Zukunft wissen, dass sie über die Tatsache informieren dürfen, dass sie Schwangerschaftsabbrüche vornehmen.  Das ist ganz klar eine Sicherheit, die alle Ärztinnen und Ärzte in Deutschland haben werden und wir reden auch darüber, dass Ärztinnen und Ärzte in Zukunft wissen, welche Informationen sie auf ihre Internetseite stellen dürfen. Im Beratungsgespräch dürfen sie natürlich auch darüber hinaus informieren. Jeder Arzt, jede Ärztin hat die Freiheit und die Sicherheit, im Beratungsgespräch all das in eigenen Worten zu sagen. Es geht hier einzig und allein um die Internetinformation. Diese wird zukünftig ganz klar über die Bundeszentrale für gesundheitliche Aufklärung, die Bundesärztekammer und über das BAFzA zur Verfügung gestellt werden. Das heißt, ein Link auf der Homepage des Arztes wird in die Lage versetzen, diese Information zu geben.   

\noindent\textbf{Comment:}
\begin{itemize}
    \setlength\itemsep{-3pt}
    \item (Beifall bei der CDU/CSU)
    \setlength\itemsep{-3pt}
    \item (Beifall bei Abgeordneten der SPD)
    \setlength\itemsep{-3pt}
    \item (Katja Keul [BÜNDNIS 90/DIE GRÜNEN]: Habe ich ja!)
    \setlength\itemsep{-3pt}
    \item (Beifall bei der SPD sowie bei Abgeordneten der CDU/CSU – Ulle Schauws [BÜNDNIS 90/DIE GRÜNEN]: Das ist doch keine Informationsfreiheit!)
\end{itemize}
\subsection{Winkelmeier-Becker}
\noindent\textbf{Texts:} Herr Präsident! Meine lieben Kolleginnen und Kollegen! Meine Damen und Herren! Seit gut einem Jahr führen wir in diesem Haus eine sehr kontroverse Debatte über die Abschaffung des Werbeverbots für Abtreibungen. Heute liegt erstmals ein Kompromiss als Grundlage vor, den die Minister der Koalition erarbeitet haben. Erst einmal herzlichen Dank dafür, dass Sie das auf den Weg gebracht haben. Die Kurzbotschaft dieses Kompromisses lautet: Das Werbeverbot nach § 219a StGB bleibt in der Substanz erhalten. Das war für uns essenziell. Aber wir greifen auch zwei Anliegen auf, die diejenigen, die für eine Abschaffung sind, in den Mittelpunkt gestellt haben, und zwar einmal aus dem Blickwinkel der Frauen und zum anderen aus dem Blickwinkel der Ärzte: Wir verstehen, dass Frauen bei einer ungewollten Schwangerschaft in einer schwierigen Situation sind und Informationen suchen. Heutzutage schauen sie typischerweise ins Netz und suchen dort nach Informationen, wo sie gegebenenfalls eine Abtreibung durchführen lassen können, nach Informationen über medizinische Fragen usw. Deshalb haben wir uns vorgenommen, valide sicherzustellen, dass die Frauen an diese Informationen kommen. Wir haben die Bundeszentrale für gesundheitliche Aufklärung im Blick. Sie soll dafür sorgen, dass medizinisch verlässliche Informationen zur Verfügung gestellt werden. Ich bin mir sicher, dass dort nicht der Fehler gemacht wird, einen Fötus mit Schwangerschaftsgewebe zu vergleichen. Wir haben die Bundesärztekammer gebeten, sich darum zu kümmern, dass es immer eine aktuelle Liste über entsprechende Ärztinnen und Ärzte gibt. Da gab es bisher Defizite; das muss man einräumen. Solche Informationslücken werden in Zukunft verlässlich geschlossen. Für diese beiden Punkte wäre eigentlich keine Gesetzesänderung nötig gewesen. Das hätte man auf dem einfachen Wege über eine Vereinbarung oder eine Anweisung machen können; aber es schadet auch nicht. Meine These ist übrigens, dass die Frage, wo und wie eine Abtreibung erfolgen kann, in der Phase des Ob, also wenn es darum geht, ob sich die Frau für das Kind entscheidet oder für eine Abtreibung, nicht die entscheidende Rolle spielt. Das kommt zu einem späteren Zeitpunkt, wenn die Entscheidung gefallen ist. Dann möchten die Frauen natürlich alle Details wissen. Da hat es nie eine Beschränkung gegeben, weder nach der heutigen Rechtslage noch zu einem anderen Zeitpunkt.  Der Arzt ist verpflichtet, dann, wenn es zu einer Abtreibung kommen soll, jede Information zu geben. Ja. Danke, Frau Kollegin, dass Sie meine Zwischenfrage zulassen. – Sie bringen Ihren Gesetzentwurf heute das erste Mal in den Bundestag ein, und schon am Montag werden wir im Rechtsausschuss eine Expertenanhörung dazu haben. Das ist ein irrsinnig schnelles Tempo. Sie haben heute viele Aspekte gehört, mit denen verfassungsrechtliche Bedenken zu diesem Gesetzentwurf vorgebracht wurden. Diese werden wir wahrscheinlich auch in der Anhörung mitgeteilt bekommen. Mich würde interessieren: Halten Sie das Verfahren in dieser Form und in diesem Tempo für seriös? Wie sollen sich die Experten in dieser kurzen Zeit zu Ihrem Gesetzentwurf äußern? Können Sie zusagen, dass Sie, je nachdem, was die Experten am Montag sagen, genug Zeit einräumen, damit diese möglicherweise angebrachten verfassungsrechtlichen Bedenken in Ihren Gesetzentwurf einfließen können? Oder sind diese Anhörung am Montag und das ganze Verfahren eine Farce, und wollen Sie das Ganze hier im Parlament schnell durchpeitschen?  Danke für Ihre Frage. – Ich habe eingangs gesagt, dass wir diese Debatte seit Anfang dieser Legislaturperiode führen. Das ganze Pro und Kontra ist nun wirklich in allen Varianten zigmal abgewägt worden, und alle Argumente sind ausgetauscht worden.  Die Eckpunkte zu unserem Gesetzentwurf – bleiben Sie bitte stehen – liegen seit Ende des letzten Jahres vor und sind jetzt nur noch einmal konkretisiert worden. Von daher sehe ich überhaupt keine Einschränkung in der umfassenden Diskussion. Alles liegt auf dem Tisch, und alles ist bekannt. Ich glaube, die Positionen sind halbwegs klar und sicher.  Ob ein Verfahren ordnungsgemäß abläuft – ja oder nein? –, das richtet sich nach der Verfassung und der Geschäftsordnung. Da sehe ich überhaupt kein Problem.  Ich möchte auf die Situation zurückkommen, in der die Frau noch über das Ob eines Schwangerschaftsabbruchs entscheidet. Da ist aus unserer Sicht weiterhin entscheidend, dass die Beratung der richtige Ort ist, um die anderen Fragen zu entscheiden: Was sagt der Partner dazu? Welche Hilfen habe ich, wenn ich mich für ein Leben mit dem Kind entscheide? Auch die Grundrechte des Kindes, etwa sein Lebensrecht, werden in der Beratung erörtert. Deshalb ist dort der wichtigste Platz. Für alle, die es noch nicht mitbekommen haben: Am Ende der Beratung ist die Frau frei in ihrer Entscheidung. Sie braucht niemandem etwas zu erklären. Sie braucht nichts zu rechtfertigen. – Deshalb: Alle, die hier weiter die Freiheit der Frau reklamieren, haben bitte zur Kenntnis zu nehmen, dass die Frau diese Entscheidung frei treffen kann. Es ist der Ausdruck des höchsten Respektes gegenüber der Frau, dass sie diese Entscheidung alleine trifft und auch alleine verantwortet.  Ich fände es im Übrigen auch gut, wenn wir zusätzlich eine Liste der Beratungsstellen ins Netz stellen würden.  Mit dem Kompromiss legen wir auch den Streit darüber bei, ob Ärzte auf ihrer Homepage über die Tatsache, dass sie Abtreibungen durchführen, informieren dürfen. Gerade war von den unausgelasteten Mathematikstudenten die Rede. Für sie habe auch ich keine Sympathie. Sie kommen mir ein bisschen vor wie die Deutsche Umwelthilfe der Abtreibungsärzte.  Die DUH ist auch nicht unbedingt mit viel Sympathie verbunden. Was uns hier vereint, ist der Wille, den gesellschaftlichen und politischen Kompromiss, den wir in diesem Land haben, weiterzutragen, ihm ein Update zu geben. Ich glaube, in diesem Sinne werden wir die Beratungen führen. Herzlichen Dank.  

\noindent\textbf{Comment:}
\begin{itemize}
    \setlength\itemsep{-3pt}
    \item (Ulle Schauws [BÜNDNIS 90/DIE GRÜNEN]: Das stimmt nicht, Frau Kollegin!)
    \setlength\itemsep{-3pt}
    \item (Beifall bei Abgeordneten der CDU/CSU)
    \setlength\itemsep{-3pt}
    \item (Beifall bei der CDU/CSU – Michael Grosse-­Brömer [CDU/CSU]: So ist das! Sehr gut!)
    \setlength\itemsep{-3pt}
    \item (Beifall bei der CDU/CSU)
    \setlength\itemsep{-3pt}
    \item (Beifall bei der LINKEN – Stephan Thomae [FDP]: Letzteres! – Michael Grosse-Brömer [CDU/CSU]: Einmal zu schnell, einmal zu langsam!)
    \setlength\itemsep{-3pt}
    \item (Beifall bei der CDU/CSU – Michael Grosse-­Brömer [CDU/CSU]: Das will er nicht wissen!)
    \setlength\itemsep{-3pt}
    \item (Beifall bei Abgeordneten der CDU/CSU – Ulle Schauws [BÜNDNIS 90/DIE GRÜNEN]: Das war jetzt wirklich unterirdisch!)
\end{itemize}
\subsection{Hoffmann}
\noindent\textbf{Texts:} Sehr geehrter Herr Präsident! Geschätzte Kolleginnen und Kollegen! Meine sehr verehrten Damen und Herren! Die §§ 218 bis 219b StGB stellen ein fein austariertes Konstrukt dar. Dieses Konstrukt beinhaltet Schutz in zwei Richtungen: zum einen den Schutz des ungeborenen Lebens, zum anderen die Idee, Frauen in einer ganz schwierigen Lebenssituation Hilfestellung zu geben. In dem Moment, wo Sie aus diesem Konstrukt auch nur einen Teil rauslösen, ist dieser Schutz nicht mehr lückenlos gewährleistet.  Gerade das Werbeverbot in § 219a stellt einen ganz wichtigen Baustein dar. Die Zielrichtung, die dort verfolgt wird, können, glaube ich, losgelöst von der inhaltlichen Diskussion heute und den Emotionen, alle Fraktionen hier verfolgen. Die Idee dahinter ist nämlich: Eine Frau in dieser schwierigen Situation soll maximale, umfassende Hilfestellung bekommen, um diesen Konflikt aufzulösen. Der Gesetzgeber geht davon aus, dass es diese umfassende Hilfe eben dann nicht gibt, wenn die Frau sich bei ihrer Entscheidung, auch wenn sie sich diese Entscheidung nicht einfach macht, unter Umständen von relativierenden, oberflächlichen Momenten leiten lässt. Das sind zum Beispiel Sätze wie: „Die Abtreibung geht bei mir anonym“, „Sie können bar bezahlen“, „Ihr Bauch gehört doch Ihnen“.  Oder es ist ein Satz wie der von Frau Dr. Hänel, die sagt – Herr Präsident, ich zitiere –: Jedes faschistische System zielt auf die reproduktiven Rechte von Frauen.  Der Gesetzgeber will vielmehr, dass die Frau Gelegenheit hat, in diesem Konfliktprozess  den Konflikt aufzuarbeiten, damit sie Zeit hat, für sich die Frage zu beantworten: Warum will ich dieses Kind?  Liebe Kolleginnen und Kollegen, wir wissen doch alle – ich verstehe das Amüsement gerade nicht –,  dass die Probleme, die dahinterstehen, weitreichender und tiefgreifender sind. Da geht es nämlich darum, dass die Frau das Kind nicht will, weil es Rahmenbedingungen gibt, die von Gewalt geprägt sind. Die Frau will das Kind nicht, weil der Mann Alkoholiker ist. Die Frau will das Kind nicht, weil der Mann unter Umständen so viel Druck auf sie ausübt. – Wir wissen wir doch alle, wie wir hier sitzen, dass das eigentliche Problem dann mit der Abtreibung doch nicht gelöst ist, sondern dass es weiter besteht.  – Wir reden von Gleichberechtigung, und jetzt sollen meine Argumente weniger wert sein, weil ich ein Mann bin?  Insofern, liebe Kolleginnen und Kollegen, ist das Werbeverbot wichtig. Das Ziel muss sein, dass die Entscheidung „Ich treibe ab“ im Konfliktberatungsgespräch oder danach getroffen wird und eben nicht schon vorher anhand bestimmter relativierender Parameter – weil es günstig ist, weil es unkompliziert ist, weil es vielleicht kein großes Ding ist – und dass man ins Konfliktberatungsgespräch nicht nur geht, um sich den Schein abzuholen.  Wir von der Unionsfraktion sind sehr froh, dass das Werbeverbot bei diesem Kompromiss bestehen bleibt.  Es ist feiner geschliffen und mit Sicherheit etwas praxistauglicher gemacht. Aber wenn wir das beherzigen, dann glaube ich, dass die Debatte, vor allem vor dem Hintergrund unseres gemeinsamen Ziels, Frauen in dieser schwierigen Situation Hilfestellung zu geben, in den weiteren Beratungen weitaus weniger emotional – – sein dürfte als heute. Vielen Dank für die Aufmerksamkeit. 

\noindent\textbf{Comment:}
\begin{itemize}
    \setlength\itemsep{-3pt}
    \item (Beifall bei Abgeordneten der CDU/CSU und der FDP)
    \setlength\itemsep{-3pt}
    \item (Dr. Kirsten Tackmann [DIE LINKE]: Mann, Mann, Mann! So kann nur ein Mann reden!)
    \setlength\itemsep{-3pt}
    \item (Beifall bei der CDU/CSU sowie des Abg. Jürgen Braun [AfD])
    \setlength\itemsep{-3pt}
    \item (Beifall bei Abgeordneten der CDU/CSU)
    \setlength\itemsep{-3pt}
    \item (Katharina Dröge [BÜNDNIS 90/DIE GRÜNEN]: Ist eher Entsetzen, ehrlich gesagt!)
    \setlength\itemsep{-3pt}
    \item (Widerspruch bei Abgeordneten der LINKEN und des BÜNDNISSES 90/DIE GRÜNEN)
    \setlength\itemsep{-3pt}
    \item (Dr. Kirsten Tackmann [DIE LINKE]: Das ist auch ein Konflikt der Männer! Es gibt bei Menschen keine Selbstbefruchtung! Hände aus den Taschen!)
    \setlength\itemsep{-3pt}
    \item (Cornelia Möhring [DIE LINKE]: Verbietet den Frauen doch am besten das Gebären!)
    \setlength\itemsep{-3pt}
    \item (Katja Dörner [BÜNDNIS 90/DIE GRÜNEN]: Was ist das für ein Frauenbild?)
    \setlength\itemsep{-3pt}
    \item (Stephan Thomae [FDP]: Das bestreite ich aber! – Ulle Schauws [BÜNDNIS 90/DIE GRÜNEN]: Die Geschichte stimmt einfach nicht!)
    \setlength\itemsep{-3pt}
    \item (Dr. Kirsten Tackmann [DIE LINKE]: Ich weiß nicht, was Sie für Frauen kennen!)
\end{itemize}
\section{Tagesordnungspunkt 16}
\subsection{Karliczek}
\noindent\textbf{Texts:} Sehr geehrter Herr Bundestagspräsident! Liebe Kolleginnen und Kollegen! Meine sehr verehrten Damen und Herren! Künstliche Intelligenz ist der Treiber des Wandels, unseres Zusammenlebens und unserer Arbeit. Sie kann unser Leben spürbar leichter machen. Sie kann zur Lösung großer Fragestellungen entscheidend beitragen. Allein im Gesundheitsbereich wird eine bessere Datenanalyse schnellere, zielgenauere Therapien für schwere Krankheiten wie beispielsweise Krebs ermöglichen. Aber auch unser Verkehr kann sicherer werden. Oder sie kann auch Antworten liefern für eine kluge Energiewende. Das zeigt: Künstliche Intelligenz hat riesiges Potenzial! Für unsere Wirtschaft, aber auch für jeden Einzelnen von uns. Sie hat schon längst angefangen, unser Leben zu verändern. Lassen Sie mich kurz die vier Säulen unserer Strategie erklären. Erstens. Die künstliche Intelligenz ist für unsere gesamte Wirtschaft wichtig. Deshalb werden wir unsere Wirtschaft und natürlich insbesondere unsere mittelständische Wirtschaft unterstützen, künstliche Intelligenz für sich nutzbar zu machen. Hier setzen wir auf unsere Forschung, auf eine enge Zusammenarbeit von Forschung, Entwicklung und Praxis. Nur dann werden wir im Wettbewerb zwischen den USA und China mithalten können. Unser Ziel ist es, Deutschland zum weltweit führenden KI-Standort zu machen.  Wir investieren bis 2025  3 Milliarden Euro. Damit leisten wir einen entscheidenden Beitrag für unsere internationale Wettbewerbsfähigkeit. Zweitens. Der Wettbewerb, vor dem wir stehen, ist in erster Linie ein Wettbewerb um Talente. Auch in sie investieren wir. Wir schaffen die Rahmenbedingungen dafür, dass die Besten der Welt zu uns kommen. Wir richten 100 neue KI-Professuren an den Hochschulen ein. Wir legen ein ambitioniertes Nachwuchsprogramm auf; denn auch die nächste Generation von KI-Spezialisten soll in Deutschland zu Hause sein.  Drittens. Wir vernetzen uns. Deutschland ist in einer guten Ausgangslage. Wir verfügen über eine international wettbewerbsfähige, gut aufgestellte Forschungslandschaft. Wir fördern die KI-Forschung seit vielen, vielen Jahren. Wir haben exzellentes Know-how im Deutschen Forschungszentrum für Künstliche Intelligenz. Wir haben vier Zentren gegründet: in Berlin, München, Dortmund und Tübingen. All dies werden wir zu einem Netzwerk mit weiteren Standorten – die Big-Data-Zentren und das DFKI – ausbauen. Wir wollen eine Exzellenz, die in ganz Deutschland sichtbar ist. Forschung und Lehre können wir nur noch vernetzt denken: national, europäisch, aber auch international. Unsere Stärke liegt in der Arbeitsteiligkeit. So haben wir zum Beispiel mit Japan verabredet, noch enger als bisher zum autonomen Fahren gemeinsam zu forschen. Vor allem aber werden wir mit Frankreich zusammenarbeiten. Die vor uns liegende technologische Entwicklung ist so bedeutend, dass wir mindestens auf europäischer Ebene, wenn nicht sogar in der gesamten demokratischen Welt gemeinsam unsere Werte als Leitplanken des technologischen Fortschritts verankern müssen. Denn individuelle Freiheit, Toleranz und demokratisches Miteinander sind die Grundpfeiler unseres Zusammenlebens. Diese Prinzipien werden wir auch im technologischen Wandel hochhalten. Mit ihnen werden wir „KI made in Germany“ zum Leuchten bringen.  Das ist die vierte Säule, auf der unsere Strategie basiert. Wir sind nicht in China. Totale Kontrolle durch den Staat werden wir niemals akzeptieren. Wir sind aber auch nicht in den USA. Wir gehen einen anderen, einen eigenen Weg. Wir lassen uns von unserem christlichen Menschenbild leiten. Jeder technologische Fortschritt hat sich dahinter einzureihen. Wir sind überzeugt: Künstliche Intelligenz muss dem Menschen dienen. Menschenwürde, Persönlichkeitsrechte und individuelle Freiheit sind die Grundlagen unseres Zusammenlebens. Unser Grundgesetz wird in diesem Jahr 70 Jahre alt, und ich glaube, darauf können wir sehr stolz sein. Die neue Technologie kommt, und dazu führen wir einen intensiven Dialog. Was sind die Chancen? Was sind die Risiken? Wir gehen in die Gesellschaft hinein. Deshalb starten wir im März unser Wissenschaftsjahr 2019 zum Thema „künstliche Intelligenz“. KI-Systeme sind lernende Systeme, und unsere KI-Strategie ist es auch. Herzlichen Dank.  

\noindent\textbf{Comment:}
\begin{itemize}
    \setlength\itemsep{-3pt}
    \item (Beifall bei der AfD)
    \setlength\itemsep{-3pt}
    \item (Beifall bei der CDU/CSU)
    \setlength\itemsep{-3pt}
    \item (Beifall bei der CDU/CSU sowie bei Abgeordneten der SPD)
\end{itemize}
\subsection{Cotar}
\noindent\textbf{Texts:} Sehr geehrter Herr Präsident! Liebe Kollegen! Die Regierung möchte, dass Deutschland in Zukunft beim Thema „künstliche Intelligenz“ an der Weltspitze mitspielt. Dafür hat sie letztes Jahr ihr sogenanntes Strategiepapier vorgelegt. Nein, ich korrigiere mich: Nicht die Bundesregierung hat dieses Papier vorgelegt, sondern die Unternehmensberatungsgesellschaft Roland Berger hat das getan. Wie so oft fehlt es der Regierung anscheinend an Fachkräften. Den Verstand muss man sich von außen einkaufen. Vielleicht sollten Sie bei der nächsten Bundestagswahl von Anfang an Unternehmensberater als Kandidaten aufstellen. Das wäre ehrlicher und würde dem Steuerzahler viel Geld sparen, meine Damen und Herren.  Aber kommen wir zu Ihrem Papier. Von einer Strategie kann und möchte ich hier nicht wirklich sprechen. Es ist eher Stückwerk. Künstliche Intelligenz ist eine der wichtigsten Schlüsseltechnologien der Zukunft. Für manche ist sie sogar die Stunde null der Wirtschaft. Hier sind massive Investitionen nötig, um unser Land nach vorne zu bringen, um sich den Herausforderungen stellen zu können. China und die USA machen es vor. Die Regierung will nun, wie wir gehört haben, bis 2025  500 Millionen Euro pro Jahr zur Verfügung stellen – ein Tropfen auf den heißen Stein. Das ist bei weitem nicht genug, um, wie angekündigt, ganz vorne mitspielen zu können. Zum Vergleich: In China sollen bis 2030 umgerechnet 150 Milliarden Dollar investiert werden. Rechnen Sie sich aus, welches Land in Zukunft führender Standort für KI sein wird.  Die zweite Baustelle: die Bildung. Die Bundesregierung möchte 100 KI-Professuren schaffen, wie wir gehört haben. Klingt gut. Aber woher kommen diese Topprofessoren, und welche Gehälter wollen Sie ihnen zahlen? Glauben Sie, dass KI-Spezialisten für typisch deutsche Unigehälter nach Deutschland kommen werden, während sich andere Länder um sie reißen? Von der Regierung kein einziges Wort zur konkreten Umsetzung! Sie wollen – Zitat – „international attraktive und konkurrenzfähige Arbeitsbedingungen“ schaffen. Prima! Aber wie sollen die konkret aussehen? Auch darüber kein Wort in Ihrem Papier! Nächster Punkt: der Datenschutz. Ohne Daten kein Lernen, ohne Lernen keine künstliche Intelligenz. Kaum ein Wort der Bundesregierung zu einer Datenpolitik, die nicht nur den Bürger schützt, sondern auch der Entwicklung der künstlichen Intelligenz entgegenkommt. Stattdessen schaffen wir uns mit Verordnungen wie der DSGVO immer mehr Hemmnisse, die dafür sorgen, dass wir den Ansprüchen der KI gar nicht gerecht werden können.  Auch hier sind andere Länder weiter, ist der Mindset der Gesellschaft ein anderer. Sie glauben, Deutschland und die EU setzen Standards in Sachen Datenschutz, und andere werden ihn kopieren. Mit Verlaub, das ist eine genauso irrige Annahme wie bei der deutschen Energiewende.  Der Regierung fehlt schlicht der Ehrgeiz beim Thema „künstliche Intelligenz“. Mit diesem Papier schafft sie keine Wende. Mit diesem Papier wird sie die Menschen draußen nicht begeistern, ihnen die Ängste nicht nehmen. Genau das wäre doch eigentlich nötig. Die Vorschläge der AfD: Schaffen Sie ein bis zwei themenoffene Leuchtturmprojekte mit ordentlichem Budget zwecks internationaler Sichtbarkeit. Richten Sie die nationale Förderung der KI-Forschung themenoffener aus. Denken Sie an Gesundheit, Pflege, Sicherheit. Schaffen Sie endlich die Voraussetzungen für die Verständnisförderung in Schulen. Nehmen Sie mehr Geld in die Hand. Versprechen Sie Finanzierungsinstrumente wie den Tech Growth Fund nicht nur, planen Sie die Gelder auch im Haushalt ein. Weiten Sie die KI-spezifische Unterstützung von kleinen und mittleren Unternehmen deutlich aus.  Richten Sie zur Förderung von Technologietransfer und Unternehmensgründungen nicht nur internationale und überregionale Kompetenzzentren ein, sondern auch regionale Innovationscluster; bestes Beispiel das Cyber Valley in Baden-Württemberg. Unterstützen Sie bestehende Initiativen von Open Data und Open Access. Fördern Sie Normung und Standardisierung. Fördern Sie den Unternehmergeist in Deutschland. Verstärken Sie die gesellschaftliche Debatte über KI. Machen Sie den Menschen Mut! Zeigen Sie Ehrgeiz, und wagen Sie einmal wirklich einen großen Wurf! Nur so können wir tatsächlich vorne mitspielen. Vielen Dank.  

\noindent\textbf{Comment:}
\begin{itemize}
    \setlength\itemsep{-3pt}
    \item (Beifall bei der SPD)
    \setlength\itemsep{-3pt}
    \item (Beifall bei der AfD)
\end{itemize}
\subsection{Schüle}
\noindent\textbf{Texts:} Sehr geehrter Herr Präsident! Liebe Kolleginnen und Kollegen! Liebe Zuhörer und Zuschauer! Im Normalfall schätze ich meinen Koalitionspartner wirklich, auch wenn wir oft unterschiedlicher Meinung sind. Das ist hier im Hause auch keinem entgangen. Als ich gestern aber die Pressemitteilung der Kollegen Schön und Schipanski gelesen habe, wurde meine Einstellung zur heutigen Debatte ein bisschen strapaziert. Aber der Reihe nach. Wir diskutieren heute die KI-Strategie der Bundesregierung. Diese Debatte wird für meinen Eindruck manchmal etwas zu spöttelnd, zu vereinfachend oder aber sogar panikverursachend geführt. Da ist der Vorwurf immer: zu wenig, zu langsam, zu unambitioniert. Ich wiederhole gerne noch einmal, was darin steht: Zwölf KI-Zentren werden gegründet. 100 zusätzliche Professuren werden eingerichtet. Jedes ostdeutsche Bundesland erhält ein Zukunftszentrum. Über die nächsten fünf Jahre unterstützen wir ein europäisches KI-Innovationscluster.  Das alles wird 3 Milliarden Euro kosten. Also, so etwas wie „zu unambitioniert, zu langsam und zu wenig“ kann ich da nicht erkennen. Das ist auch kein Hausaufgabenheft, das ist eine Umsetzungsstrategie. Natürlichen müssen wir darüber diskutieren, wie wir mit der datengetriebenen Ökonomie umgehen. Es ist ja nun schon fast eine Binsenweisheit, dass wir mit unseren klassischen industriellen Erfolgsmodellen in Europa und in Deutschland unter Wettbewerbsdruck geraten. Aber weder lassen wir uns auf die Einschätzung ein, der nächste Technologieschritt werde uns automatisch ins dystopische Unheil führen – frei nach dem Motto: Terminator statt C-3PO –, noch sind wir, die Sozialdemokraten und Andrea Nahles, das rote Gespenst, das staatlich verordnet, Daten offenzulegen. Der erste Satz ist nur eine pessimistische Einstellung, der zweite Satz ist eine Unterstellung aus Ihrer Pressemitteilung, und das finde ich unlauter.  Richtig ist, dass auch der Zugang zu Daten über Entwicklung und Innovationskraft eines Landes entscheidet. Über das Wie können wir noch streiten, liebe Kolleginnen und Kollegen, aber lassen Sie uns die Debatte seriös führen. Hören Sie auf damit, zu behaupten, wir würden die kleinen und mittelständischen Unternehmen zwingen, ihre Daten offenzulegen, sie mit asiatischen, chinesischen oder amerikanischen Plattformen zu teilen. Hören Sie auf damit! Im Übrigen waren gestern in unserer Anhörung die kleinen und mittelständischen Unternehmen sogar dafür, dass es mehr Datenoffenheit gibt.  Was wir brauchen, sind Anreizsysteme für Datenkooperationen. Das ermöglicht nämlich Innovation und Wettbewerb. Dazu wird auch Falko Mohrs gleich noch etwas sagen. Und nein, ich finde nicht, dass unser wertvoller Rohstoff, unsere Daten, ausschließlich ein paar wenigen Digitalunternehmen zugänglich sein soll.  Warum sollen Monopole im Übrigen nur in klassischen Wirtschaftsbereichen aufgebrochen werden dürfen? Ich bin fest davon überzeugt, dass Europa und speziell Deutschland die einmalige Chance besitzen, bei der Entwicklung – da stimme ich Ihnen zu, Frau Karliczek – selbstdenkender Maschinen an erster Stelle zu stehen, und zwar durch den sorgsamen Umgang mit Daten und durch eine europäische Forschungs-, Investitions- und Innovationslandschaft; da können wir nämlich in der Tat noch eine Schippe drauflegen. Dabei werden wir weder das Verhalten der Menschen in unserem Land überwachen wie in China, noch werden wir es ausschließlich ökonomisch verwerten wie in den Vereinigten Staaten von Amerika. Ja, bei uns steht der Mensch im Mittelpunkt, aber nicht als Objekt, sondern als mündiger Bürger, liebe Kolleginnen und Kollegen.  Die Pessimisten behaupten, niemals würden KI-Experten aus Harvard, Berkeley oder dem MIT nach Potsdam, Greifswald oder Dresden kommen. Die Grünen haben sich in der letzten Sitzung des Bildungsausschusses sogar zu der Behauptung verstiegen, deutsche KI-Forschende würden keinerlei Rolle auf internationalem Parkett spielen. Da frage ich Sie: Wofür halten Sie eigentlich das DFKI in Saarbrücken? Für eine Kita? Dort wird seit 30 Jahren geforscht, und zwar im Bereich der künstlichen Intelligenz: 1 000 Wissenschaftler aus 60 Nationen und mittlerweile 240 Projekte.  Beispiele gefällig, Frau Christmann? Gerne! Die Wissenschaftler Schmeier, Kiefer und Bernardi forschen beispielsweise an Pepper. Pepper ist ein kleiner künstlicher Kumpel für ein krankes Kind, der ihm helfen soll, wieder gesund zu werden.  Es wird daran geforscht, der Mutter und dem Kind einen schnellen Weg zum nächsten Zug zu zeigen, wenn im Hauptbahnhof wieder der Aufzug ausgefallen ist. Ja, es wird auch daran geforscht, KI-Start-ups mit der Landwirtschaft zusammenzubringen, um den besten Zeitpunkt der Düngung von Gerste zu ermitteln; für die Biertrinker unter uns vielleicht eine relevante Information.  Auf dieser deutschen Forschungstradition möchten wir aufbauen. Ja, wir werden KI zur Schlüsseltechnologie für unser Land machen – für unser Land, aber vor allem für die Menschen. Herzlichen Dank.  

\noindent\textbf{Comment:}
\begin{itemize}
    \setlength\itemsep{-3pt}
    \item (Beifall bei der SPD)
    \setlength\itemsep{-3pt}
    \item (Beifall bei der FDP)
    \setlength\itemsep{-3pt}
    \item (Beifall bei Abgeordneten der SPD)
\end{itemize}
\subsection{Brandenburg}
\noindent\textbf{Texts:} Sehr geehrter Herr Präsident! Frau Ministerin! Liebe Kolleginnen und Kollegen! Wir debattieren ein weiteres Mal das Best-of der ministerialen KI-Thesen; denn das Wort „Strategie“ dürfen wir ja auf Ihren Wunsch hin nicht mehr benutzen. Auf unsere Kleine Anfrage hat uns Herr Meister nämlich erklärt, dass sogenannte Performance-Indikatoren – KPIs – zur Messung der Erreichung von Zielen in einer ganzheitlichen Strategie wie der unseren nicht angebracht seien.  Das war eine sehr interessante Information. Ich habe dann überlegt, wer von Ihnen die Franzosen anruft, die sehr wohl Indikatoren zur Erfolgsmessung in ihrer Strategie verankert haben.  Wenn wir diesen Gedanken weiterspinnen, dann stellt sich doch die Frage: Wann ist diese Strategie denn dann eigentlich erfolgreich? Wenn in zehn Jahren alles noch so ist wie jetzt, weil nichts schiefgegangen ist? Wenn wir Technologieführerschaft haben? Wenn wir Vollbeschäftigung haben? Oder aber erst dann, wenn Milch und Honig fließen und die Chinesen anrufen und sagen: „Wir hören auf; ihr habt gewonnen“? Liebe Kolleginnen und Kollegen, es ist vollkommen unklar, wann diese Strategie erfolgreich ist. Es wird außerdem zunehmend unklar, wer eigentlich wofür zuständig ist.  Wenn man sich auf internationaler Bühne hinter vorgehaltener Hand mit unseren Kollegen unterhält, dann erzählen sie, dass sie mit Zuständigen in Deutschland gesprochen hätten – viermal, fünfmal –, dass teilweise aber kein Rückruf gekommen ist und dass sie teilweise überhaupt nicht wissen, wer wirklich den Hut aufhat. Das, liebe Kolleginnen und Kollegen, ist schon bedenklich; denn während andere Länder mit klaren Zielsetzungen, klaren Zuständigkeiten und gebündelter Kompetenz an dieses Thema herangehen, kommt Deutschland mit einem Debattierbus mit drei federführenden Ministerien, 16 Bundesländern, einer Staatsministerin für Digitales und einem Bundesminister für besondere Aufgaben angefahren. Dies ist ein Problem.  Das alles könnte man relativ humorvoll nehmen, wenn man damit nicht wirklich die Hoffnungen vieler Gründer und kluger Menschen in diesem Land riskieren würden. Wir haben den ersten Teil der digitalen Transformation faktisch nicht unbedingt gewonnen. Jetzt muss es um Geschwindigkeit gehen. Daher meine Bitte: Definieren Sie klare, messbare Zielkriterien, die die Bürger, Sie und wir verfolgen können; denn nur dann wissen wir, wann wir gegensteuern und unterstützen müssen, Frau Ministerin.  Gründen Sie ein Digitalministerium, welches den Querschnitt bündelt und im Prinzip – um es betriebswirtschaftlich zu sagen, Herr Röspel – ein Programm-Management in den verschiedenen Bereichen durchführen kann. Frau Ministerin, rufen Sie bitte Ihre Freundin, die Kanzlerin, an, und lassen Sie sie eine Aufbruchrede an die Nation halten, genauso wie das viele andere Staatsoberhäupter getan haben,  und vermitteln Sie ein positives Bild, wohin es gehen soll.  – Man kann es versuchen. Letzter Gedanke, dann sind Sie mich auch wieder los. Die Entwicklung der Algorithmen von morgen ist ein internationaler Teamsport. Die Einführung von offenen und transparenten Messkriterien ist nicht Ausdruck der Gängelung durch die Opposition. Das ist vielmehr ein gängiges Mittel für die Messung des Erfolgs von Strategien. Deswegen tun Sie uns bitte den Gefallen, und erstellen Sie klare Messpunkte! Danke schön.  

\noindent\textbf{Comment:}
\begin{itemize}
    \setlength\itemsep{-3pt}
    \item (Beifall bei der FDP)
    \setlength\itemsep{-3pt}
    \item (Beifall bei der FDP – Zuruf der Abg. Joana Cotar [AfD])
    \setlength\itemsep{-3pt}
    \item (René Röspel [SPD]: Und die betriebswirtschaftlich orientiert sind!)
    \setlength\itemsep{-3pt}
    \item (Beifall bei der LINKEN)
\end{itemize}
\subsection{Sitte}
\noindent\textbf{Texts:} Herr Präsident! Meine Damen und Herren! Künstliche Intelligenz berührt die Menschen auf jeden Fall erst einmal emotional. Die einen sind verunsichert und reagieren eher ablehnend, andere freuen sich darauf. Ein Grund mehr, die KI als das zu bezeichnen, was sie wirklich ist, nämlich als maschinelles Lernen oder eben gar als automatisierte Statistik. Unter diesem Blickwinkel lassen sich bestimmte Vorstellungen, die es in der Gesellschaft gibt, zurückdrängen. Das mag insgesamt ein bisschen banal oder langweilig klingen. Verglichen mit futuristischen Vorstellungen von sogenannter starker KI und der Ersetzung des Menschen ist es das vielleicht tatsächlich. Aber wir sprechen für die vorhersagbare Zukunft von selbstlernenden Systemen, die quasi Erfahrungen in der Mustererkennung sammeln und dann entscheiden. In einzelnen Bereichen – das wissen wir längst – wie Bilderkennung, bei Go- oder Schachspielen sind diese Systeme mindestens genauso gut wie Menschen oder teilweise schon deutlich besser. Aber wie bei jeder Technologie liegt die Verantwortung für den Einsatz und die ethische Konzeption für diesen Einsatz bei uns, bei den Menschen.  Ich habe in den unzähligen Debatten zur KI mal einen sehr schönen Satz gehört, den ich Ihnen hier nicht ersparen möchte: Nicht vor künstlicher Intelligenz müssen wir uns fürchten, sondern vor menschlicher Dummheit.  Die KI-Strategie der Bundesregierung beginnt mit der Aussage, dass die Technologie dem Menschen dienen soll. Gegen diese Aussage habe ich nichts einzuwenden. Schaut man sich die KI-Strategie aber konkret an, dann vermisst man genau diese Ebene. Die Logik dieses Papiers besteht darin, rein wettbewerbsorientiert, rein standortpolitisch zu denken. Wer von KI made in Germany spricht, dem muss ich sagen, dass er das gesamte Wesen dieses Prozesses nicht verstanden hat.  Derzeit werden in vielen Ländern KI-Strategien entwickelt. Wenn man sich die im Einzelnen anschaut, dann stellt man fest: Sie sind alle ziemlich ähnlich. Man bekommt auch immer die gleichen Beispiele vorgeführt. Da fragt man sich doch ernsthaft: Wieso entwickelt man das nicht arbeitsteilig? Wieso werden solche Dinge wie KI-Wettrüsten an die Spitze gestellt? Wir brauchen eine gesellschaftliche Diskussion über die Ziele und die Ethik von KI. Das ist unsere gemeinsame Aufgabe, die wir hier leisten müssen. Da sind eben Fragen zu stellen: Was ist gesellschaftlich sinnvoll? Was ist gewollt? Wie kann die Diskriminierung von Menschen und Gruppen verhindert werden? Wie sichern wir gute Arbeit? Welche Sicherheitsstandards brauchen wir? Und natürlich auch: Welche ordnungspolitischen Entscheidungen müssen gefällt werden, um die Konzentration von Monopolmacht zurückzudrängen? Da brauchen wir Transparenz und Kontrolle. Wir brauchen klare Verantwortlichkeiten und schließlich eine klare Entscheidung zum Thema „Einsatz KI-basierter Waffensysteme“. Dazu können wir heute schon klar Nein sagen.  Die Bundesregierung hat keine Strategie vorgelegt. Sie hat lediglich einen Maßnahmenkatalog der Ministerien zusammengefasst. Das halten wir für gesellschaftspolitisch höchst fahrlässig. Mithin geht es in einer rastlosen, an Daten nimmersatten Informationsgesellschaft um nichts Geringeres als um die Ausgestaltung menschlicher Freiheit im 21. Jahrhundert. Ich danke Ihnen, liebe Kollegen.  

\noindent\textbf{Comment:}
\begin{itemize}
    \setlength\itemsep{-3pt}
    \item (Beifall beim BÜNDNIS 90/DIE GRÜNEN)
    \setlength\itemsep{-3pt}
    \item (Beifall bei der LINKEN, der SPD und dem BÜNDNIS 90/DIE GRÜNEN)
    \setlength\itemsep{-3pt}
    \item (Beifall bei der LINKEN sowie bei Abgeordneten des BÜNDNISSES 90/DIE GRÜNEN)
    \setlength\itemsep{-3pt}
    \item (Beifall bei der LINKEN)
\end{itemize}
\subsection{Janecek}
\noindent\textbf{Texts:} Sehr geehrter Herr Präsident! Sehr geehrte Frau Karliczek! Sehr geehrter Herr Minister in Abwesenheit Altmaier! Eine kleine Vorbemerkung sei mir gestattet: Ich hoffe wirklich sehr, dass die Umsetzungsstrategie der Bundesregierung nicht auf den gleichen Fakten basiert wie Herrn Scheuers Mobilitätsstrategie.  Seit gestern wissen wir, dass die Forscher, die er anführt, sich um den Faktor 1  000 verrechnet haben. Ich hoffe, dass wir hier nicht in diese Richtung kommen.  Herr Altmaier bemüht ja das Bild eines KI-Airbus. Nun frage ich mich: Was ist das für ein Bild? Kriegen wir künftig ein großes KI-Unternehmen oder wie bei Sie mens Tausende von KI-Mitarbeitern in einem KI-Industrie giganten? Beglücken wir die Welt mit den Ideen einer zentralistischen Instanz? Ich glaube, das ist ein völlig schiefes Bild. Ich glaube, die Frage ist doch: Wie schaffen wir es, dass die Tausenden von verschiedenen Anwendungen – in der Medizin, in der Umweltschutztechnik, in der öffentlichen Verwaltung – in der Gesellschaft gemeinwohlorientiert wirken und übrigens auch unseren Ressourcenverbrauch in der Ökonomie nicht steigern, sondern ganz klar senken?  Das Letzte, was wir brauchen, sind marktbeherrschende monopolistische oder oligopolistische Markt- und Machtstrukturen durch KI. Auch darauf muss die Bundesregierung eine ganz klare Antwort geben: Wie schaffen wir es, in unserem Land einen fairen Wettbewerb zu gewährleisten? Im Süden Deutschlands, wo ich herkomme, haben wir zum Beispiel 1 500 Nischenweltmarktführer. Um die geht es. Auch für die kleinen Unternehmen müssen wir einen Wettbewerb garantieren, bei dem alle dabei sind.  Wir – Anna Christmann und ich – haben vor kurzem quasi den Guru der französischen KI-Strategie Cédric Villani – Sie nicken bei der SPD – bei uns zu Gast gehabt, eine sehr beeindruckende Persönlichkeit, Träger der Fields-Medaille. Sie müssen sich mal anschauen, was die Franzosen schon vorgelegt haben. Wir müssen das Rad nicht neu erfinden. Wir finden aber, dass es bei Ihnen ein paar Leerstellen gibt. Eine Leerstelle ist zum Beispiel das Thema „ökologische Wirtschaft“. Es ist in der französischen Strategie ganz klare Maßgabe, KI dafür einzusetzen, den Ressourcenverbrauch zu senken, intelligente Steuerungssysteme im Bereich der Mobilität, der Energienetze einzusetzen. Da erwarten wir von Ihnen, dass Sie auch etwas vorlegen; denn in Zeiten der Klimakrise können wir nicht über KI reden, ohne diese Themen ganz klar nach vorne zu stellen.  Ich rede gerne über Positives; aber man muss auch negative Szenarien sehen. Der Stromverbrauch durch die Digitalisierung, durch die KI ist nach der französischen Strategie – das ist ein Szenario – im Jahre 2030 bis zu zehnmal höher als heute. Wir brauchen also definitiv auch eine Green-IT-Strategie, um diese Problematik aufzufangen, sonst gibt es ein klimapolitisches Desaster.  Der entscheidende Punkt für uns ist – auch Frau Sitte hat das angesprochen –: Wir werden diese die Technologie betreffenden Fragen in der Gesellschaft nicht beantworten können, wenn wir die Menschen nicht mitnehmen. Deswegen brauchen wir einen klaren sozialen Anker. Wir schlagen vor, nach dem Vorbild von Schweden und Großbritannien eine soziale Innovationsstiftung zu gründen, damit KI-Anwendungen eben nicht nur von der Industrie, von den Großen genutzt werden, sondern auch in der Fläche, bei den Menschen vor Ort. Übrigens gilt das auch für Menschen mit Behinderung, für die Inklusion, für all diese Themen. Technologie muss für alle da sein. Das ist unser grüner Ansatz. Ich danke Ihnen.  

\noindent\textbf{Comment:}
\begin{itemize}
    \setlength\itemsep{-3pt}
    \item (Beifall beim BÜNDNIS 90/DIE GRÜNEN sowie der Abg. Dr. Petra Sitte [DIE LINKE])
    \setlength\itemsep{-3pt}
    \item (Beifall beim BÜNDNIS 90/DIE GRÜNEN)
    \setlength\itemsep{-3pt}
    \item (Beifall bei Abgeordneten des BÜNDNISSES 90/DIE GRÜNEN)
    \setlength\itemsep{-3pt}
    \item (Beifall beim BÜNDNIS 90/DIE GRÜNEN sowie bei Abgeordneten der LINKEN)
    \setlength\itemsep{-3pt}
    \item (Beifall bei der CDU/CSU sowie bei Abgeordneten der SPD)
\end{itemize}
\subsection{Steier}
\noindent\textbf{Texts:} Herr Präsident! Sehr geehrte Damen und Herren! Liebe Kolleginnen und Kollegen! Schauen wir in der Geschichte ein bisschen zurück: Wir haben in den 90er-Jahren Standards im Bereich der künstlichen Intelligenz gesetzt, gerade im Bereich der Industrie 4.0. Damals war es die Regierung von Helmut Kohl, die die künstliche Intelligenz als Thema besetzt hat. Anfang der 2000er-Jahre haben wir erlebt, dass durch neue Entwicklungen gerade im Hardware- und Softwarebereich in einem dynamischen Consumermarkt Entwicklungen in den USA stattfanden und dort Standards gesetzt wurden. Wenn wir jetzt auf die weitere Entwicklung schauen, dann müssen wir sehen, dass es darum geht, die digital getriebenen Anwendungen mit der Physik, der Mechanik zu vereinbaren. Da wird neue Wertschöpfung geschaffen. Da kommt die traditionelle Stärke unserer Industrie, unserer Wirtschaft zur Geltung. Die Bundesregierung gibt in der KI-Strategie eine entscheidende Antwort. Sie bündelt die Kräfte, die wir in Forschung und Industrie haben, um neue Standards zu setzen und Entwicklungen voranzutreiben. Lassen Sie mich eine entscheidende Frage stellen: Wie gehen wir dabei mit den Daten um? Es wurde eben viel über Daten diskutiert. Wir müssen uns die Frage stellen: Wie schaffen wir einerseits einen ethischen und moralischen Rahmen, und wie schaffen wir andererseits genug Freiraum, um auch Entwicklungen zuzulassen? Eine weitere entscheidende Frage ist: Wie stellen wir zukünftig sicher, dass wir Daten hier bei uns speichern können? Die Bundesregierung hat auch darauf entscheidende Antworten geliefert. Wir müssen nach der Vorlage der Strategie jetzt an die konkrete Ausarbeitung gehen. Die Zeit dafür ist gekommen. Wir müssen sicherstellen, dass diese Daten bei uns weiterhin gespeichert, genutzt und verwertet werden können.  Die Ausarbeitung der KI-Strategie liegt jetzt vor uns. Wir müssen nun die entsprechenden Schritte gehen. 50 Millionen Euro stehen bereits in diesem Jahr, also 2019, im Bundeshaushalt zur Verfügung. Jetzt gilt es, diese zu konkretisieren und mit gemäßigtem und schnellem Schritt voranzugehen. Weiterhin müssen wir dafür sorgen, dass wir auch bei uns eine Infrastruktur vorhalten, die hier Antworten gibt. Wir müssen dafür sorgen, dass wir die Chipentwicklung bei uns vor Ort vorantreiben, dass wir zum einen die Wertschöpfungstiefe auch bei uns sicherstellen, dass wir zum anderen aber durch eine Chipproduktion, also eine Entwicklung bei uns, für Datensicherheit sorgen und ein Einfallstor hier schließen können. Ich höre aus der Industrie bereits positive Stimmen, die weiterhin am deutschen Standort interessiert sind. Das gilt es jetzt hier konkret umzusetzen.  Wir müssen bei der Softwareentwicklung dafür sorgen, dass wir Daten speichern können und diese weiter nutzen können. Jetzt gilt es, diese in der Ausarbeitung der KI-Strategie voranzutreiben. Dann können wir auch hier weiter dafür sorgen, dass die Wertschöpfung in Deutschland stattfinden kann. Aber wir dürfen auch in der Datenpolitik nicht übertreiben. Es wurde eben angesprochen: Die DSGVO hat sehr hohe Standards gesetzt, Standards insbesondere im Bereich der Persönlichkeitsdaten. Wir müssen aber auch weiterdenken und die Frage beantworten: Wie gehen wir mit anonymisierten Daten um? Gerade im Gesundheitsbereich lässt sich hier großes Wachstumspotenzial verwirklichen. Wir müssen uns die Frage stellen, wie wir mit weiteren synthetischen Daten umgehen, wie wir dort zusätzliche funktionale Sicherheit herstellen können. Nur mit Qualität und funktionaler Sicherheit haben wir in der Vergangenheit immer wieder auf dem Weltmarkt überzeugen können. Das gilt es auch hier bei synthetischen Daten sicherzustellen.  Ich nenne ein Beispiel. Vielleicht hat der eine oder andere im Frühjahr letzten Jahres das sogenannte autonom fahrende Fahrzeug gesehen, das in Amerika leider einen Fahrradfahrer überfahren hat. Wenn man sich das genauer anschaut, dann sieht man, dass dieses Fahrzeug nur mithilfe von visuellen Daten versucht hat, sich durch den Verkehr zu quälen. Das wurde nur mit realen Daten gespeist. Wer schon einmal in den USA gewesen ist – ich war über Weihnachten noch mal dort –, der weiß: Man sieht kaum einen Fahrradfahrer auf der Straße. Auf der Autobahn sieht man noch viel weniger Fahrradfahrer. Das heißt, das trainierte Fahrzeug, das mit realen Daten über die Straßen gefahren ist, hat den Fahrradfahrer nur als Rauschen erkannt. Da gilt es, neue Standards zu setzen. Es kommen synthetische Daten rein, es kommt eine Modellierung rein, damit wir mit zusätzlicher Datenmodellierung für Sicherheit sorgen können. Hier können wir unsere Erfahrung weiter ausbauen, auch Vertrauen gewinnen, wenn wir durch Qualität und Sicherheit überzeugen. Das gilt es, jetzt umzusetzen. Ich freue mich schon auf die konkrete Ausarbeitung. Ich helfe der Bundesregierung auch gerne weiter. Vielen Dank.  

\noindent\textbf{Comment:}
\begin{itemize}
    \setlength\itemsep{-3pt}
    \item (Beifall bei der AfD)
    \setlength\itemsep{-3pt}
    \item (Beifall bei der CDU/CSU)
    \setlength\itemsep{-3pt}
    \item (Beifall bei der CDU/CSU sowie bei Abgeordneten der SPD)
    \setlength\itemsep{-3pt}
    \item (Beifall bei Abgeordneten der CDU/CSU sowie des Abg. René Röspel [SPD])
    \setlength\itemsep{-3pt}
    \item (Beifall der Abg. Antje Lezius [CDU/CSU])
\end{itemize}
\subsection{Jongen}
\noindent\textbf{Texts:} Herr Präsident! Werte Abgeordnete! So langsam spricht es sich auch in Deutschland herum: Die künstliche Intelligenz – KI oder englisch AI – ist dabei, die Industrie, die Wirtschaft, die Arbeitswelt, in der Folge auch Kultur und Gesellschaft in atemberaubendem Tempo zu revolutionieren. Auf immer mehr Feldern, wo früher der Mensch benötigt wurde, um zu analysieren, um Entscheidungen zu treffen, um geistige und körperliche Arbeit zu verrichten, übernehmen selbstlernende Algorithmen im Verein mit Robotern das Ruder. Nicht einmal als Endverbraucher ist der Mensch mehr vor der Übernahme durch Algorithmen sicher, siehe das selbstfahrende Auto. Die politisch entscheidenden Fragen lauten dabei: Wie kann es gelingen, die vielen Berufsfelder und Arbeitsplätze, die durch KI schlicht überflüssig werden, im selben Umfang und in der nötigen Geschwindigkeit durch neue zu ersetzen? Oder werden wir bald von einer Antiquiertheit des Menschen auch in Bezug auf die Arbeit sprechen müssen? So viel ist sicher, werte Kollegen: Die vielen Milliarden, die Sie in die Ertüchtigung von Migranten für den Arbeitsmarkt stecken,  sind schon allein wegen der KI in den Sand gesetzt, weil diese geringqualifizierten Arbeitsplätze die ersten sind, die wegfallen, meine Damen und Herren.  Sodann: Werden wir die oberste Kontrollinstanz über die rapide wachsende Macht der Algorithmen bleiben, oder wird uns dieser Prozess im wahrsten Sinne des Wortes über den Kopf wachsen? Und mit „wir“ meine ich nicht nur philosophisch das menschliche Subjekt, sondern vor allem politisch den Bürger als Souverän in der Demokratie. Das Strategiepapier Künstliche Intelligenz der Bundesregierung deutet nicht darauf hin, dass die Koalition überzeugende Antworten auf diese Fragen hätte. Für unsere Kanzlerin war bis vor kurzem selbst das Internet noch „Neuland“. Insofern muss man es wohl honorieren, dass der politische Handlungsbedarf in Sachen KI jetzt erkannt worden ist. Eine wirkliche Strategie, das heißt ein Gesamtkonzept, eine leitende Idee, sucht man in Ihrem Papier vergebens, werte Kollegen. Es ist eine Aneinanderreihung von Absichtserklärungen, bei denen oft eher der fromme Wunsch Vater des Gedankens ist, als eine realistische Lageeinschätzung.  So schreiben Sie: „Artificial Intelligence ... made in Germany“ soll zum weltweit anerkannten Gütesiegel werden. Es ist ja schön, dass Sie auf „Made in Germany“ überhaupt noch Wert legen, nachdem Sie dabei sind, die deutsche Automobilindustrie durch völlig sinnfreie Stickoxidgrenzwerte und andere Schikanen systematisch zu ruinieren. Aber darauf, wie Sie den gigantischen Rückstand aufholen wollen, in dem sich Deutschland und Europa gegenüber China und den USA bereits befinden, bleiben Sie eine Antwort weitgehend schuldig. 100 hochkarätige neue KI-Professuren wollen Sie an deutschen Universitäten besetzen, dabei sind die deutschen Besoldungsvorschriften völlig ungeeignet, auch nur 10 internationale Spitzenkräfte ins Land zu holen. Selbst die europäische Konkurrenz, Beispiel Holland, hängt uns regelmäßig hier ab. Sie wollen Standards setzen, um den Markt zu bestimmen, unter anderem auch ethische Standards. Aber glauben Sie wirklich, werte Kollegen, dass China sich an die Standards von Diversität und Geschlechtergerechtigkeit anpassen wird, die Sie der künstlichen Intelligenz einimpfen wollen? Vor allem die Grünen sind in ihrem Antrag wieder einmal wahre Weltmeister der politischen Korrektheit und sehen ihre Chance, sich eine virtuelle Welt nach ihren ideologischen Vorstellungen zu modellieren, die dann auf die reale Welt zwingend zurückwirken soll – eine große Gefahr, die die AfD in der Enquete-­Kommission KI sehr wachsam im Auge behalten wird, meine Damen und Herren.  Wir meinen, dass sehr viel mehr unternommen werden muss, auch sehr viel mehr Geld in die Hand genommen werden muss, wenn Deutschland als KI-Standort nicht völlig abgehängt werden soll und tatsächlich führend werden soll.  Die AfD wird hierzu bald konkrete Anträge stellen. Wir freuen uns auf die weitere Debatte. Vielen Dank.  

\noindent\textbf{Comment:}
\begin{itemize}
    \setlength\itemsep{-3pt}
    \item (Beifall bei der SPD)
    \setlength\itemsep{-3pt}
    \item (Beifall bei der AfD)
    \setlength\itemsep{-3pt}
    \item (Dr. Karamba Diaby [SPD]: Ich bin sehr gespannt!)
    \setlength\itemsep{-3pt}
    \item (Zurufe von der SPD: Och!)
\end{itemize}
\subsection{Mohrs}
\noindent\textbf{Texts:} Sehr geehrter Herr Präsident! Meine Damen und Herren! Am Dienstag gab es eine interessante Überschrift, die da lautete: Maschine gegen Mensch: IBMs KI verliert knapp gegen menschlichen Redner in politischer Debatte. – Die KI hat dabei vor allem Archive aus Zeitungsartikeln für den Aufbau der eigenen Argumente genutzt. Der menschliche Redner hat an der Stelle vielmehr das breite gesellschaftliche Bild gezeigt. Egal ob gewonnen oder nicht: Das zeigt am Ende die rasante Entwicklung, die wir mit der KI in den letzten Jahren und Jahrzehnten weltweit erlebt haben. 1997 gewinnt die KI gegen den menschlichen Schachweltmeister, 2011 gegen den Menschen im TV-Quiz, 2016 gab es den chinesischen Go-Moment. Denke ich an die Debatte zurück, dann kann ich mir ehrlicherweise einen Verweis auf meinen Vorredner nicht ganz ersparen. Ich könnte mir vorstellen, dass es bei manchen hier im Bundestag nicht so lange dauert, bis die KI auch die menschliche Debatte und Qualität der Argumente überflügeln wird.  Im Grunde ist es aber genau das nicht, was wir mit der künstlichen Intelligenz verfolgen. Es geht nicht darum, dass KI den Menschen ersetzt, sondern in unserer Strategie geht es darum, dass KI den Menschen ergänzt, die Fähigkeiten erweitert, dass die KI assistiert. Dass die künstliche Intelligenz ein Riesenpotenzial zur Veränderung im Arbeitsalltag und auch in der Wirtschaft beiträgt, sehen wir an allen Ecken und Enden. Wir haben Studien, dass das Bruttoinlandsprodukt bis 2030 allein durch den Einsatz von künstlicher Intelligenz um über 11 Prozent ansteigen wird, 430 Milliarden Euro. Das allein ist ein wichtiger Grund, warum wir die künstliche Intelligenz in der Strategie, in der Arbeit der Bundesregierung und unserer Fraktion in den Mittelpunkt stellen. Und wichtig ist dabei: Technologischer Fortschritt ist nie Selbstzweck, sondern muss der Gesellschaft immer einen Fortschritt und eine positive Veränderung bringen.  Es ist klar, dass die Strategie der Bundesregierung ein wichtiger Auftakt, ein Startpunkt für unsere Arbeit mit der künstlichen Intelligenz ist. Und ganz ehrlich, lieber Mario Brandenburg, da hilft es auch nichts, wenn du die ewige Schallplatte der FDP auflegst, wie ihr es in jeder digitalpolitischen Debatte tut, nämlich dass ein Digitalministerium das einzig Wahre und alles andere überflüssig ist. Meine Güte! Nachdem ihr eure Strategie im Bundestagswahlkampf auf das Thema Digitalisierung beschränkt habt, lasst euch doch mal nach anderthalb Jahren was Neues einfallen!  Diese Schallplatte kann keiner mehr hören. Frau Cotar, eine Sache zu Ihnen. Sie beschweren sich darüber, dass hier externer Sachverstand genutzt wurde. Ich würde Ihnen ehrlicherweise vielleicht mal zu ein wenig externem Sachverstand raten.  Dann wären Ihre Argumente vielleicht etwas besser. Ihre Fraktionsvorsitzende, auf deren Platz Sie sitzen, brüstet sich ja immer damit, dass sie doch die Unternehmerin, die Start-up-Gründerin, die Unternehmensberaterin sei. Ich weiß gar nicht, ob sie in ihrem Berufsleben lang genug in einer Firma gewesen ist, um das wirklich von sich behaupten zu können.  Herr Jongen, Sie haben gesagt, dass Sie in der KI-Enquete wachsam sein werden. Ich würde mir wünschen, Sie wären nicht nur wachsam, sondern würden sich auch beteiligen. Wachsam zu sein, reicht an der Stelle nämlich überhaupt nicht.  Meine Damen und Herren, für uns ist klar: Wir verfolgen mit der Strategie Künstliche Intelligenz der Bundesregierung eindeutige, klare Ziele. Wir wollen, dass Deutschland, dass Europa Weltmarktführer in dem Bereich wird. Dabei geht es nicht nur um den Bereich der Forschung und Innovation, in dem wir gut aufgestellt sind. Dabei geht es uns auch darum, dass wir den Transfer sicherstellen. Das werden wir schaffen. Herzlichen Dank.  

\noindent\textbf{Comment:}
\begin{itemize}
    \setlength\itemsep{-3pt}
    \item (Beifall bei der SPD sowie bei Abgeordneten der CDU/CSU)
    \setlength\itemsep{-3pt}
    \item (Beifall bei der FDP)
    \setlength\itemsep{-3pt}
    \item (Beifall bei Abgeordneten der SPD und des BÜNDNISSES 90/DIE GRÜNEN – Joana Cotar [AfD]: Das ist billig, Herr Mohrs!)
    \setlength\itemsep{-3pt}
    \item (Beifall bei Abgeordneten der SPD sowie des Abg. Dr. Anton Hofreiter [BÜNDNIS 90/DIE GRÜNEN])
    \setlength\itemsep{-3pt}
    \item (Beifall bei der SPD und dem BÜNDNIS 90/DIE GRÜNEN sowie bei Abgeordneten der CDU/CSU – Mario Brandenburg [Südpfalz] [FDP]: Ihr könnt es ja auch einfach mal umsetzen!)
    \setlength\itemsep{-3pt}
    \item (Beifall bei der SPD sowie des Abg. Andreas Steier [CDU/CSU])
    \setlength\itemsep{-3pt}
    \item (Beifall bei Abgeordneten der SPD und des BÜNDNISSES 90/DIE GRÜNEN)
    \setlength\itemsep{-3pt}
    \item (Beifall bei der SPD sowie bei Abgeordneten des BÜNDNISSES 90/DIE GRÜNEN)
\end{itemize}
\subsection{Kluckert}
\noindent\textbf{Texts:} Verehrter Herr Präsident! Meine Damen und Herren! Innovationen und neue Entwicklungen brauchten schon immer Mut. Genauso braucht KI Mut. Wenn wir auf die Finanzen schauen, dann sehen wir da vor allem 3  Milliarden Euro zusätzlich. Doch eine echte Strategie für eine Innovation braucht eben auch echte Haushaltsmittel.  Auch 3 Milliarden Euro in den nächsten sechs Jahren können Peanuts werden, wenn sie einfach nur auf verschiedenste Einzelprojekte verteilt werden. Hier mal ein Vergleich: China investiert das 30‑Fache, und in China gibt es auch keine Restriktionen bei der Bezahlung von Forschern. Wenn wir hier in Deutschland nicht aufpassen, dann wandern unsere klügsten Köpfe ab.  Aber eines ist auch klar: Geld ist nicht alles, auch nicht bei KI. KI braucht auch Mut bei den Rahmenbedingungen. Wir reden hier seit Jahren und auch heute schon wieder über Probleme, Fragen und Risiken. Aber was muss denn genau im Kartellrecht, im Verbraucherschutzrecht, im Bereich des Datenschutzes oder bei der IT-Sicherheit getan werden? Wo genau verläuft denn die Grenze zwischen öffentlichen und nichtöffentlichen Daten, und was schlussfolgern wir denn daraus? Die Fragen sind ja seit Jahren im Raum, aber völlig ungeklärt. Das liegt nicht zuletzt daran, dass in der Bundesregierung einfach niemand dafür wirklich zuständig ist.  Sie, Frau Ministerin, und Ihre Kollegen von SPD und Union sprechen hier vor allem über die ethischen Fragen; aber sie müssen eben auch einmal beantwortet werden. Denn, Herr Mohrs, technischer Fortschritt – das ist ganz richtig – muss den Menschen dienen, aber er muss auch gelebt werden. Wir können ihn sowieso nicht aufhalten.  Ich will Ihnen da gerne mal ein Beispiel nennen: Seit Jahrzehnten setzen wir Roboter zur Entschärfung und zur Aufspürung von Bomben ein. Warum akzeptieren wir eigentlich gleichzeitig 3 300 Verkehrstote jedes Jahr, obwohl die KI die Mobilität mit dem autonomen Fahren doch so viel sicherer machen könnte,  und das auch noch mit unserer Technologie? Ein zweites Beispiel. Wir haben junge deutsche Firmen, die in der Krebsfrüherkennung extrem weit sind und mithilfe von KI sehr viele Leben retten wollen. Dies wird ihnen aber durch den Gemeinsamen Bundesausschuss unmöglich gemacht; denn er braucht geschlagene zwei Jahre, bis eine Bewertung von Hilfsmitteln vorgenommen wird. In zwei Jahren, meine verehrten Damen und Herren, sind diese Firmen aber entweder insolvent oder eben abgewandert.  Was wir endlich brauchen, ist erstens eine schonungslose Analyse. Wir müssen jede einzelne Norm kennen, die in Deutschland die Entwicklung von KI hemmt. Zweitens brauchen wir echte Entscheidungen, damit hier Unternehmen überhaupt investieren und Wissenschaftler hierherkommen wollen. Drittens brauchen wir – das ist sehr dringend – Reallabore, damit wir mal ausprobieren können, was denn funktionieren könnte.  Dann könnten wir das, was funktioniert, für ganz Deutschland übernehmen. KI braucht Mut, liebe Bundesregierung. Ich hoffe, Sie haben ein bisschen mehr Mut als in den letzten Jahren und auch mehr Mut, als diese Strategie erahnen lässt, und setzen auch etwas um.  

\noindent\textbf{Comment:}
\begin{itemize}
    \setlength\itemsep{-3pt}
    \item (Beifall bei der FDP)
    \setlength\itemsep{-3pt}
    \item (Beifall bei der FDP – Falko Mohrs [SPD]: Will ja auch keiner! – Zuruf des Abg. Gustav Herzog [SPD])
    \setlength\itemsep{-3pt}
    \item (Beifall bei Abgeordneten der FDP – René Röspel [SPD]: Es geht um Sicherheit! Das muss man schon abwägen!)
    \setlength\itemsep{-3pt}
    \item (Beifall bei der LINKEN)
    \setlength\itemsep{-3pt}
    \item (Andreas Steier [CDU/CSU]: Haben wir doch!)
    \setlength\itemsep{-3pt}
    \item (Beifall bei der FDP sowie der Abg. Anke Domscheit-Berg [DIE LINKE])
\end{itemize}
\subsection{Tatti}
\noindent\textbf{Texts:} Sehr geehrter Herr Präsident! Liebe Kolleginnen und Kollegen! Eine Strategie ist die Beschreibung einer langfristigen Planung zur Erreichung genau beschriebener Ziele. Zu einer Strategie gehören die erforderlichen Ressourcen und überprüfbare Zwischenziele, damit anhand von Kennzahlen gemessen werden kann, ob die Maßnahmen auch erfolgreich sind. Wenn ich mir das Kapitel zur Arbeitswelt im Strategiepapier zur künstlichen Intelligenz anschaue, lese ich, dass der Einsatz von KI zu einer – ich zitiere – „neuen Stufe der Veränderung“ führe, „mit deutlichen Unterschieden zur bisherigen Automatisierung und Digitalisierung.“ Ziel der Bundesregierung ist – und das freut mich besonders –, dass KI „verantwortungsvoll und gemeinwohlorientiert“ entwickelt und genutzt wird und zur „Steigerung des Wohlergehens der Erwerbstätigen“ führt. Das finde ich wunderbar.  Dann kommen die Maßnahmen, die zur Erreichung dieser hehren Ziele führen sollen: eine Nationale Weiterbildungsstrategie. Der Fokus auf Weiterbildung ist auch völlig richtig. Allerdings sind weder die Nationale Weiterbildungsstrategie noch der Ausbau und die Förderung der beruflichen Weiterbildung neue Ideen.  Das alles steht bereits in der Umsetzungsstrategie „Digitalisierung gestalten“ und im verabschiedeten Qualifizierungschancengesetz aus dem vergangenen Jahr. Ich gehe jetzt nicht wieder auf die Mängel des Qualifizierungschancengesetzes ein, aber ich frage: Wo ist denn die neue Stufe?  Die Bundesregierung schreibt, sie wolle die betriebliche Mitbestimmung und den Beschäftigtendatenschutz weiterentwickeln. Ich begrüße es natürlich, dass sich die Bundesregierung, wenn auch im Schneckentempo, den Forderungen der Gewerkschaften und meiner Fraktion anzunähern scheint. Aber auch diese Willensbekundung ist nicht neu, sondern ergab sich bereits aus der Debatte zur Digitalisierung der Arbeitswelt. Was da drinsteht, ist überhaupt nichts Neues.  KI bringt also eine neue Stufe, die Bundesregierung offensichtlich nicht. Letztes Beispiel: die Errichtung eines deutschen KI-Observatoriums, das die Auswirkungen von KI beobachten soll. Aufgrund dieser scharfen Beobachtungen soll dann später irgendwann vielleicht die unverbindliche Auditierung von KI wenigstens geprüft werden. Die Bundesregierung will offensichtlich unsere Zukunft durch bloßes Beobachten und Prüfaufträge gestalten. Also, das ist wirklich mal ein spannender Ansatz.  Plattformen wie Deliveroo stehlen sich aus ihrer Verantwortung als Arbeitgeber, was die Bundesregierung aufmerksam beobachtet. Unternehmen wie Amazon überwachen ihre Beschäftigten auf Schritt und Tritt mit digitalen Technologien, und Sie beobachten in aller Ruhe, was der Markt so treibt. Beobachten ist aber – sei es noch so ambitioniert – keine Ausrede für die Untätigkeit dieser Bundesregierung.  Das Papier hat, was die Arbeitswelt und die Beschäftigten angeht, den Namen „Strategie“ nicht verdient. Werfen Sie es einfach weg! Legen Sie uns was Vernünftiges vor!  

\noindent\textbf{Comment:}
\begin{itemize}
    \setlength\itemsep{-3pt}
    \item (Beifall beim BÜNDNIS 90/DIE GRÜNEN)
    \setlength\itemsep{-3pt}
    \item (Beifall bei der LINKEN – René Röspel [SPD]: Das ist Technikfolgenabschätzung! Das ist ein völlig richtiger Ansatz! Jetzt widersprechen Sie sich aber selber!)
    \setlength\itemsep{-3pt}
    \item (Beifall bei der LINKEN – Falko Mohrs [SPD]: Wiederholen macht es aber nicht richtiger, Frau Kollegin!)
    \setlength\itemsep{-3pt}
    \item (Beifall bei der LINKEN sowie des Abg. Dr. Anton Hofreiter [BÜNDNIS 90/DIE GRÜNEN] – Matthias W. Birkwald [DIE LINKE]: Ach! Das ist ja ein Ding!)
    \setlength\itemsep{-3pt}
    \item (Zuruf von der LINKEN: Das stimmt!)
    \setlength\itemsep{-3pt}
    \item (Beifall bei der LINKEN sowie bei Abgeordneten der SPD)
    \setlength\itemsep{-3pt}
    \item (Beifall bei der LINKEN)
\end{itemize}
\subsection{Christmann}
\noindent\textbf{Texts:} Sehr geehrter Herr Präsident! Liebe Kolleginnen und Kollegen! Wir hatten diese Woche im Forschungsausschuss den EU-Forschungskommissar Carlos Moedas zu Besuch, und er hat an etwas sehr Wichtiges erinnert, nämlich daran, dass das weltweite Internet durchaus in Europa, beim CERN, erfunden worden ist. Nur leider sind dann von den Anwendungen, die sich daraus so vielfältig ergeben haben, nicht mehr so viele aus Europa gekommen. Jetzt haben wir mit der KI aber wieder eine Technologie, die durch Grundlagenforschung durchaus noch mal ganz entscheidend weiterentwickelt werden kann. Das heißt, wir haben wieder die Stunde von Europa. Europa ist stark in Grundlagenforschung, und wenn wir jetzt unsere Kräfte bündeln, dann können wir weltweit aus Europa heraus die Standards für KI setzen.  Europa könnte richtig zeigen, was es kann. Aber was macht die Bundesregierung? Sie verlieren sich in kleinsten Aktivitäten wie 20 KI-Trainer hier und 12 Zentren dort, statt glasklar auf ein europäisches Forschungsnetzwerk zu setzen.  Sie setzen – das haben Sie eben auch mehrfach gesagt – auf „AI made in Germany“ statt auf „AI made in Europe“, und das ist ein Fehler.  Es gibt da längst auch die europäischen Initiativen ­ELLIS und CLAIRE, Forschungsnetzwerke, die gerne Unterstützung hätten. Sie lassen sie aber leider links liegen. Dabei muss es unbedingt unser Ziel sein, dass so eine weitreichende Technologie wie maschinelles Lernen von Menschen entwickelt wird, die unsere Werte teilen. Dann hat sie nämlich das Potenzial, bei ökologischen und sozialen Herausforderungen zu helfen und uns den Alltag zu erleichtern. Überlassen wir es hingegen autokratischen Systemen wie China, kommt Social Scoring dabei heraus. Das müssen wir verhindern.  Es wäre daher dringend an der Zeit, dass Sie eine klare Initiative für eine europäische Entwicklung für KI und für globale Standardsetzung für ihre Anwendung erarbeiten. Aber mit der Umsetzung hat es die Bundesregierung ja leider nicht so sehr. Denn von den angekündigten Milliarden sind bisher nur 50 Millionen Euro dieses Jahr überhaupt in den Haushalt eingestellt, und – das kommt noch dazu – sie sind ja noch nicht mal verteilt. Auf unsere Nachfrage hin hat die Bundesregierung uns mitgeteilt, man arbeite gerade noch am Gesamtkonzept zur Verteilung dieser 50 Millionen Euro. Immerhin haben Sie erkannt, dass die Strategie offenbar noch nicht dafür taugt, überhaupt zu wissen, wofür man das Geld ausgibt. Aber wir stellen fest: Sie haben 3 Milliarden Euro ins Schaufenster gestellt, und jetzt wissen Sie noch nicht mal, was Sie mit den 50 Millionen Euro dieses Jahr machen sollen. Das ist zu wenig.  Dabei gäbe es wirklich genug Ideen, was man sofort tun könnte. Wir haben dazu Vorschläge gemacht – schauen Sie gerne in unseren Antrag, der seit Herbst vorliegt –: unter anderem 100 Millionen Euro für ein europäisches Forschungsnetzwerk, aber zum Beispiel auch für soziale Innovationen und für mehr Frauen in der KI-Entwicklung. Denn es kommt auch darauf an, wer KI entwickelt, also auf Diversität. Da haben Sie auf Nachfrage vor allem den Girls’ Day erwähnt. Ich finde den Girls’ Day super,  aber ich glaube, das kann nicht der ganze Weg sein, um die KI-Expertinnen zu gewinnen, die wir brauchen. Danke schön.  

\noindent\textbf{Comment:}
\begin{itemize}
    \setlength\itemsep{-3pt}
    \item (Beifall beim BÜNDNIS 90/DIE GRÜNEN – Andreas Steier [CDU/CSU]: Das steht doch in der Strategie drin!)
    \setlength\itemsep{-3pt}
    \item (Beifall beim BÜNDNIS 90/DIE GRÜNEN)
    \setlength\itemsep{-3pt}
    \item (Beifall bei der CDU/CSU)
    \setlength\itemsep{-3pt}
    \item (Beifall der Abg. Dr. Daniela De Ridder [SPD])
    \setlength\itemsep{-3pt}
    \item (Beifall beim BÜNDNIS 90/DIE GRÜNEN und bei der LINKEN sowie bei Abgeordneten der FDP)
\end{itemize}
\subsection{Schön}
\noindent\textbf{Texts:} Sehr geehrter Herr Präsident! Liebe Kolleginnen und Kollegen! Wie kann man die Ernte um 30 Prozent steigern? Indem man mit KI Pflanzenschädlinge bestimmt und zielgenau bekämpft. Wie kann man Stromkosten senken? Indem man mit KI ein kluges Management von Wetterdaten, Strompreisen und Verbrauchsbedarfen macht, das Ganze geschickt mixt und entsprechend steuert. Wie kann man Krebsdiagnosen und -therapien verbessern und dadurch zur Bekämpfung von Krebs beitragen? Indem man mit KI die Kompetenzen der Ärzte erweitert und Therapieempfehlungen generiert. All das wird heute schon gemacht, und zwar von deutschen Unternehmen: von deutschen Start-ups, die deutsches Wissen aus der Forschung generiert und in kluge Geschäftsmodelle überführt haben und damit die Probleme der Welt in den nächsten Jahren lösen. Das sind erfolgreiche deutsche Unternehmen, und darauf können wir sehr stolz sein.  Natürlich muss die Opposition zur KI-Strategie der Bundesregierung sagen, dass es zu wenig ist, dass es noch ein bisschen mehr sein sollte und schneller gehen müsste und was Sie sonst noch so gesagt haben. Aber was mich optimistisch stimmt, ist, dass die Debatte heute eigentlich schon von Optimismus geprägt war. Ich finde, das brauchen wir auch. Denn Zukunft wird aus Mut gemacht. Wir müssen optimistisch in die Zukunft schauen.  – Ich freue mich, dass die Grünen der gleichen Meinung sind. – Wir müssen anpacken, statt nur zu meckern. Wir brauchen mehr Optimismus und mehr Start-up-Mentalität, und das, liebe FDP, kann man nicht messen. Das ist entweder da, oder es ist nicht da. KI stellt einiges auf den Kopf, und deshalb müssen auch wir einiges auf den Kopf stellen, und zwar deutlich schneller als bisher. Es ist gut, dass wir über 100 neue KI-Professuren schaffen. Hier entstehen die Ideen von morgen. Liebe Frau Christmann, Sie kritisieren, dass das ganze Thema angeblich nicht europäisch eingebettet ist. Aber es gibt diese europäischen Verbünde. Es gibt CLAIRE, wo gerade das Deutsche Forschungszentrum für Künstliche Intelligenz die Federführung übernimmt und wo aus Deutschland heraus der europaweite Forschungsverbund gesteuert und geprägt wird. Wir sind da mit dabei, und deshalb ist Ihre Kritik an der Stelle völlig fehl am Platz.  Dass sich in den Ländern im Bereich der digitalen Bildung einiges bewegt, ist gut. Ich finde allerdings, es reicht nicht, nur nach der Infrastruktur zu rufen, die der Bund bezahlt, sondern auch die Inhalte müssen sich ändern. Computing bzw. digitale Grundkompetenzen braucht zukünftig jeder. Jeder einzelne Schüler muss nicht nur wissen, wie er mit dem PC arbeitet, sondern er muss auch wissen, wie er mit Daten arbeitet. Das muss sich schneller als bisher ändern. Deshalb ist es richtig, dass die Ministerin nicht nur eine KI-Strategie vorlegt, sondern auch eine nationale MINT-Strategie, dass wir Wert darauf legen, dass jeder Schüler mit dem Thema MINT und mit IT zu tun hat, und dass wir hier gerade auch einen Schwerpunkt auf die Frauen und Mädchen legen.  Verwaltung muss sich ändern, Regulierung und Gesetzgebung ebenfalls. Das Thema Reallabore wurde angesprochen. Natürlich brauchen wir neue Methoden. Aber das ist Teil der KI-Strategie. Speziell Reallabore und Testfelder werden in der Strategie erwähnt und sogar schon umgesetzt. Im BMWi gibt es längst ein Programm für Reallabore. Das gilt es zu nutzen. Wir müssen Innovationen testen und ausprobieren können, damit wir auch als Gesetzgeber wissen, wo man die Regulierung entsprechend anpassen muss. Wir brauchen neue Freiräume, und das ist die große Chance, die auch in der KI-Strategie der Bundesregierung enthalten ist.  Unternehmen müssen agiler werden und Daten teilen. Denn Fakt ist: Wir brauchen gerade zum Training von künstlicher Intelligenz unglaublich viele Daten, und zwar in guter Qualität. Deshalb ist es gut, dass die Bundesregierung dafür gekämpft hat, dass wir in Brüssel jetzt bei der Urheberrechtsreform eine Regulierung schaffen, die KI-freundlich ist und es ermöglicht, mit Text- und Data-Mining auch wirklich diese Datenschätze zu heben, die wir vor Ort haben. Bei der ganzen Diskussion um Artikel 13 der Urheberrechtsreform, die wir in diesen Tagen gerade in den sozialen Netzwerken erleben, gerät das leider aus dem Blick. Wir brauchen hier eine KI-freundliche Regulierung, und gerade dafür hat die Bundesregierung in den letzten Tagen gesorgt. Es gilt, den Datenreichtum der Unternehmen besser zu nutzen. An der Stelle finde ich interessant, was die SPD in den letzten Tagen vorgelegt hat. In vielen Punkten sind wir einer Meinung. Auch wir sind der Meinung, dass man Open Data von staatlicher Seite besser nutzen muss. Das war übrigens eine Initiative, die wir in der letzten Legislaturperiode gemeinsam vorangetrieben haben und die nun die Bundesregierung in die Tat umsetzen muss. Aber, um noch mal auf diesen Punkt einzugehen, wer soll dazu gezwungen werden, seine Daten offenzulegen? Sie sprechen in Ihrem Konzept von Unternehmen mit Marktmacht. Wir haben aber auch viele Mittelständler, die Marktmacht haben und die auf einem großen Datenreichtum sitzen.  Ich will nicht, dass wir diesen Datenreichtum öffnen, damit sich andere daraus bedienen.  Hier brauchen wir Schnittstellen und Interoperabilität. Wir müssen dafür sorgen, dass die Unternehmen diese Datenschätze gemeinsam nutzen können, aber am besten hier, damit nicht andere Großkonzerne darauf zugreifen. Deshalb ist Ihr Konzept an der Stelle zu hinterfragen. Ich freue mich aber, dass Experten etwa in der Datenethikkommission und auch in der Enquete-Kommission gerade parallel an Antworten auf diese Fragestellungen sitzen. Ich denke, dass sich in den nächsten Wochen und Monaten sehr viele kluge Vorschläge ergeben. Die müssen wir umsetzen, und dann kommen wir in unserem Land auch gut voran.  

\noindent\textbf{Comment:}
\begin{itemize}
    \setlength\itemsep{-3pt}
    \item (Beifall bei der SPD)
    \setlength\itemsep{-3pt}
    \item (Beifall bei Abgeordneten der CDU/CSU)
    \setlength\itemsep{-3pt}
    \item (Andrea Nahles [SPD]: Sie haben es noch nicht gelesen!)
    \setlength\itemsep{-3pt}
    \item (Beifall bei der CDU/CSU)
    \setlength\itemsep{-3pt}
    \item (Dr. Jens Zimmermann [SPD]: Da ist der Mut ganz schön schnell verloren gegangen!)
    \setlength\itemsep{-3pt}
    \item (Beifall bei der CDU/CSU sowie bei Abgeordneten der SPD)
    \setlength\itemsep{-3pt}
    \item (Beifall bei Abgeordneten der CDU/CSU – Dr. Anna Christmann [BÜNDNIS 90/DIE GRÜNEN]: Hören Sie überhaupt zu?)
    \setlength\itemsep{-3pt}
    \item (Lachen und Beifall beim BÜNDNIS 90/DIE GRÜNEN)
\end{itemize}
\subsection{Röspel}
\noindent\textbf{Texts:} Herr Präsident! Meine sehr verehrten Damen und Herren! Am Ende einer Debatte mit wenig Redezeit kann man es sich vielleicht leisten, auf ein paar grundsätzliche Fragen oder Betrachtungsweisen zurückzukommen. Wir haben gerade viel darüber gehört, dass es wichtig ist, KI, künstliche Intelligenz, in Deutschland weiter zu erforschen, zu fördern und zu unterstützen. Das ist alles richtig, und das ist auch gut so, und das soll auch weiterhin gemacht werden. Wir haben von Frau Schüle, Falko Mohrs, aber auch von Nadine Schön ganz viele Beispiele dafür gehört, was in Deutschland tatsächlich schon gemacht wird. Das ist eine große Fülle. Das ist auch gut, und wir sind international tatsächlich relativ gut aufgestellt. Aber trotzdem muss man an der einen oder anderen Stelle sagen: Diese Debatten sind häufig eine Instrumentendebatte, also wir diskutieren hier politisch: Welche Details muss man vielleicht ändern? Welche Instrumente sind besser? Was ist falsch in der Strategie? Was ist zu detailliert in der Strategie? Wo ist das, was die Bundesregierung vorgeschlagen hat, zu offen, zu groß? Dann ist die Frage: Ist es eigentlich sinnvoll, dass wir hier In­strumentendebatten führen? Ich bin für politischen Streit gerne zu haben. Aber ist es im Bereich der künstlichen Intelligenz und der Frage, wie man sie voranbringt, sinnvoll, politisch über Instrumente zu streiten?  Sollten nicht vielmehr die Bundesregierung und die Regierungskoalition für gute Vorschläge aus der Opposition offen sein? Oder sollte man nicht auch umgekehrt sagen: „Da läuft die Strategie der Bundesregierung gut“? Müssten wir nicht eigentlich gemeinsam parlamentarisch eine Idee entwickeln, wie künstliche Intelligenz bzw. vor allen Dingen die Rahmenbedingungen für Wissenschaftlerinnen und Wissenschaftler so gestaltet werden können, dass sie freier, besser und verlässlicher forschen können?  Denn am Ende entscheiden gar nicht wir politisch über das, was dabei herauskommt, sondern das tun die Wissenschaftlerinnen und Wissenschaftler. Deswegen sollten wir uns politisch konzentrieren und auch über die großen Fragen diskutieren. Häufig ist in der Diskussion eben auch der Wettbewerbsgedanke zu spüren. Da gucken wir sehr panisch darauf: Was machen die USA? Wie viel Geld geben sie dafür aus? Was geben wir aus? Sind 3 Milliarden Euro viel oder wenig? Übrigens zu Ihrer Zahl von 50 Millionen Euro, Anna Christmann: Wenn man sich die unterschiedlichen Etats der Bundesregierung anguckt, dann stellt man fest: Wir geben jetzt schon über 140 Millionen Euro für künstliche Intelligenz aus. Wenn man den gestrigen Parlamentarischen Abend des Deutschen Zentrums für Luft- und Raumfahrt besucht hat, konnte man erfahren, wie viel Geld über die institutionelle Finanzierung für künstliche Intelligenz ausgegeben wird. Also, wir geben schon viel Geld aus. Aber ist das wirklich die wahre Betrachtungsweise? Die USA sind doch viermal größer als Deutschland. 1 400 Millionen Chinesen stehen 80 Millionen Deutschen gegenüber. Das Land ist 17‑mal größer als unseres. Glauben wir allen Ernstes, dass wir das 17‑Fache an Geld ausgeben und mit den Chinesen vielleicht gleichziehen können? Umgekehrt: Macht man eine Pro-Kopf-Betrachtung, stellt man fest: All unsere gesellschaftlichen Daten in Deutschland sind viel besser. Also, ich will darauf hinaus: Müssen wir uns eigentlich das Ziel setzen, Weltmarktführer in diesem Bereich zu werden, in dem die Konkurrenz groß ist? Oder müssen wir nicht vielmehr die Frage stellen: Welches Ziel haben wir eigentlich gesellschaftlich? Wo wollen wir hin? Darüber können wir auch eine politische Debatte führen.  Wie können wir künstliche Intelligenz nutzen, damit die Menschen in unserem Land besser leben? Vielleicht werden wir nicht Technologie- und Weltmarktführer, wenn wir diese Ziele verfolgen und uns fragen: Wie werden Menschen besser leben? Wie können wir Arbeit so verändern, dass Menschen nicht Angst haben, ihren Arbeitsplatz zu verlieren, ohne dass etwas anderes kommt?  Dazu kann man sich auch einmal die Verdi-Vorschläge angucken. Wie können wir eigentlich die Effizienzgewinne nutzen, um damit Sozialsysteme, Pflegesysteme, Gesundheitssysteme oder das Ehrenamt zu stärken? Alle diese Fragen der gesellschaftlichen Weiterentwicklung, finde ich, müssen wir diskutieren. Und dann muss man überlegen: Wie ist KI eigentlich zu entwickeln und zu nutzen? Da haben jedenfalls wir als Sozialdemokratie eine Gesellschaft im Kopf, in der die Menschen frei, selbstbestimmt und ohne Druck von außen leben und in der KI tatsächlich zum Nutzen der Gesellschaft angewandt wird.  Es gab dazu die kluge Entscheidung dieses Parlaments, eine Enquete-Kommission einzusetzen. Wenn die tatsächlich gut arbeitet – das ist ja festzustellen –, werden wir interfraktionell genau diese Ziele diskutieren – mit Ausnahme einer Fraktion, die nicht auf Zusammenführung, sondern nur auf Spaltung ausgerichtet ist, mit dem Ergebnis, dass der AfD-Sprecher, der tatsächlich Ahnung von der Sache hat, aus der AfD ausgetreten ist; das muss auch einmal gesagt werden.  Sie verlieren Ihren Sachverstand und brauchen mehr externen Sachverstand. Der Wunsch an die Enquete-Kommission ist, dass tatsächlich die gesellschaftlichen Fragen diskutiert werden. Dann können wir politisch gemeinsam Vorstellungen entwickeln, wofür wir KI eigentlich brauchen. Vielen Dank.  

\noindent\textbf{Comment:}
\begin{itemize}
    \setlength\itemsep{-3pt}
    \item (Beifall bei der SPD sowie bei Abgeordneten der CDU/CSU)
    \setlength\itemsep{-3pt}
    \item (Beifall bei der SPD sowie bei Abgeordneten der CDU/CSU, der FDP und des BÜNDNISSES 90/DIE GRÜNEN)
    \setlength\itemsep{-3pt}
    \item (Beifall bei Abgeordneten der SPD sowie des Abg. Dieter Janecek [BÜNDNIS 90/DIE GRÜNEN])
    \setlength\itemsep{-3pt}
    \item (Beifall bei der CDU/CSU)
    \setlength\itemsep{-3pt}
    \item (Beifall bei Abgeordneten der SPD)
\end{itemize}
\subsection{Kemmer}
\noindent\textbf{Texts:} Herr Präsident! Liebe Kolleginnen und Kollegen! Jetzt, zum Ende der Debatte, glaube ich, kann man zusammenfassen, dass schon viele wichtige Punkte angesprochen worden sind. Deswegen will ich mich auf einen Bereich fokussieren, in dem ich besonderen Handlungsbedarf sehe: auf den Einsatz von KI im Mittelstand. Vor diesem Hintergrund ist es gut, dass in der KI-Strategie die KMU besonders genannt werden und vor allem auch deren Innovationskraft und Wettbewerbsfähigkeit gestärkt werden sollen. Der Mittelstand hat unser Land stark gemacht.  Wir haben eine große Vielfalt von Betrieben. Sie sind wichtige Zentren vor Ort und in der Fläche für Innovationen. Sie sind letztendlich das wirtschaftliche Rückgrat in den Gemeinden, aber auch unseres gesamten Landes.  Aber wie sieht das künftig aus? Ich wünsche mir, dass wir auch in 20 Jahren noch Weltmarktführer und Hidden Champions in ganz Deutschland verteilt haben. Ja, wir brauchen KI-Ökosysteme in den Ballungszentren. Wir brauchen Leuchttürme. Aber wir brauchen KI eben auch in der Fläche. Ich denke, das ist realistisch; aber das ist kein Selbstläufer. Vieles wird davon abhängen, wie gerade im Mittelstand die Chancen von KI genutzt werden. Da haben wir einen gewissen Aufholbedarf. Wenn man sieht, wie viele Mittelständler heute in KI investieren, dann ist das doch noch ziemlich zurückhaltend. Im Vergleich zur Industrie ist es eben so, dass Entwicklungsabteilungen, dass IT-Abteilungen mit dem Tagesgeschäft meistens schon ziemlich gut ausgelastet sind und dass auch in den Führungsetagen wenig Zeit ist, sich mit dem Thema KI auseinanderzusetzen. Hier gilt es, aufzupassen, dass wir Potenziale nicht verschenken, dass wir auch ganz klar sehen: Viele Schlüsseltechnologien können durch KI auf neue Innovationsstufen gehoben werden. – Aber es geht eben nicht nur um Produktivitätssteigerungen. Es geht um komplett neue Geschäftsmodelle. Es geht um neue Produkte. Es geht auch darum, Organisationskulturen und Prozesse ganz neu zu denken.  Hier sind wir auch politisch gefragt, die Weichen richtig zu stellen. Viele wichtige Punkte in der Strategie mit Blick auf KMU wurden schon genannt. Wir haben die Kompetenzzentren, in denen künftig KI-Trainer KMU beraten können, wie KI-Anwendungen relevant einzubringen sind. Wir haben die Reallabore, mit denen auf der einen Seite gute Experimentierräume für die Unternehmen geschaffen werden können; auf der anderen Seite stellt sich für uns als Gesetzgeber aber dabei die Frage: Wo besteht denn konkreter Handlungsbedarf? Ja, gerne.  Frau Kemmer, Sie hatten gerade sehr nett gesagt, dass wir Organisationsstrukturen in den Firmen neu definieren müssten. Ich weiß jetzt nicht, ob Sie schon einmal in einer Firma tätig gewesen sind. Ich zum Beispiel war als Diplom-Informatiker bei einem großen Versicherungskonzern für die Einführung einer KI-Anwendung begleitend zuständig,  und zwar einer zur Interpretation von Kundenanschreiben. Wir wissen genau, was in einem solchen Fall passiert: Es gibt die nächste Rationalisierungswelle. Was machen Sie als Regierung, wenn Sie sagen, Sie wollen Organisationsstrukturen überdenken und neu definieren?  Wie unterstützen Sie denn in dem Fall die Arbeitnehmer? Wie unterstützen Sie die Bundesagentur für Arbeit dabei, das vorauszudenken? Wie werden in dem Fall die weiteren Ausbildungen durchgeführt, gerade im Hinblick darauf, dass auch durchaus qualifizierte Arbeitsplätze jetzt sukzessive wegfallen werden? Könnten Sie das mal etwas konkretisieren? Ja, das mache ich sehr gerne. Ich will zunächst mal anmerken, dass hier schon wieder der Unterschied zwischen der AfD und den anderen Fraktionen im Haus deutlich wird. Es geht nicht darum, Ängste zu schüren. Es wäre schön, wenn Sie sich mal mit Fakten und Studien auseinandersetzen würden. Wir haben die Studie des Fachkräftemonitors, die besagt, dass wir durch die digitale Transformation in den nächsten Jahren ein Plus von über 1 Million Jobs in unserem Land haben werden. Ja, natürlich, wir werden auch Arbeitsplätze verlieren. Deswegen ist aber insgesamt das Thema „Weiterbildung und lebenslanges Lernen“ ganz wichtig.  Entscheidend ist: Wir haben ein Plus an Jobs und kein Minus. Politisch ist es die Aufgabe, dass wir die Rahmenbedingungen richtig gestalten und nicht nur Ängste schüren, weil wir so in diesem Land nicht vorankommen.  Zurück zu den wichtigen Punkten in der Strategie. Mein dritter Punkt – Stichwort: steuerliche Forschungsförderung – wurde auch schon angesprochen. Klar ist auch: Wir können und müssen von staatlicher Seite viel Geld in die Forschung geben. Aber das kann eben nicht nur staatliches Geld sein. Wir brauchen private Investitionen. Und vor allem müssen wir darauf achten, dass sich die Innovationsausgaben nicht auf immer weniger Unternehmen konzentrieren, sondern in der Fläche erhalten bleiben. Vierter Punkt. Wir müssen Plattformen schaffen, um den Mittelstand und Start-ups, insbesondere KI-Start-ups, besser zusammenzubringen, um für mehr Austausch und Transfer zu sorgen. Bei aller Liebe, liebe Linke, zur Kritik, die Bundesregierung sei untätig: Sie haben in der Debatte als einzige Fraktion keinen eigenen Antrag eingebracht. Vielleicht fangen Sie an dieser Stelle erst mal bei sich selber an.  Insgesamt wird im Bereich KI bis 2022 ein Potenzial für den europäischen Raum in Höhe von 10 Milliarden Euro vorausgesagt. Deswegen will ich zum Schluss einfach nur sagen: Liebe Grüne, KI made in Germany und KI made in Europe – das gehört zusammen. Das ist kein Gegensatz, sondern das ist ineinander verzahnt.  Wir setzen dafür die Rahmenbedingungen. Sie haben noch vor einigen Monaten kritisiert, es gebe keine Strategie, jetzt kritisieren Sie es gebe eine Strategie, wo wir natürlich für unser Land eine Vision und die entsprechenden Maßnahmen benennen. Von daher: Die Strategien Deutschlands und Europas gehören natürlich zusammen. Wir arbeiten daran. Herzlichen Dank. 

\noindent\textbf{Comment:}
\begin{itemize}
    \setlength\itemsep{-3pt}
    \item (Beifall bei Abgeordneten der CDU/CSU und der SPD)
    \setlength\itemsep{-3pt}
    \item (Albert Rupprecht [CDU/CSU]: Genau so ist es!)
    \setlength\itemsep{-3pt}
    \item (Beifall bei Abgeordneten der CDU/CSU – Zuruf der Abg. Dr. Petra Sitte [DIE LINKE])
    \setlength\itemsep{-3pt}
    \item (Michael Grosse-Brömer [CDU/CSU]: Sie ist doch gar keine Regierung! Sie ist Abgeordnete!)
    \setlength\itemsep{-3pt}
    \item (Beifall bei Abgeordneten der CDU/CSU)
    \setlength\itemsep{-3pt}
    \item (Falko Mohrs [SPD]: Warum sind Sie denn da nicht geblieben?)
    \setlength\itemsep{-3pt}
    \item (Beifall des Abg. René Röspel [SPD])
    \setlength\itemsep{-3pt}
    \item (Beifall bei der CDU/CSU sowie bei Abgeordneten der SPD)
    \setlength\itemsep{-3pt}
    \item (Beifall bei der CDU/CSU sowie bei Abgeordneten der SPD, der FDP und des BÜNDNISSES 90/DIE GRÜNEN – René Röspel [SPD]: Das ist das Problem, wenn der AfD die Fachkräfte verloren gehen!)
    \setlength\itemsep{-3pt}
    \item (Beifall bei Abgeordneten der CDU/CSU sowie des Abg. René Röspel [SPD])
    \setlength\itemsep{-3pt}
    \item (Dieter Janecek [BÜNDNIS 90/DIE GRÜNEN]: Kommt jetzt irgendwas mit Flüchtlingen?)
\end{itemize}
\section{Tagesordnungspunkt 9}
\subsection{Spaniel}
\noindent\textbf{Texts:} Sehr geehrter Herr Präsident! Sehr geehrte Damen und Herren! Halb Deutschland ist betroffen durch Fahrverbote. Diese Fahrverbote sind durch Klageverfahren in Luftreinhalteplänen verankert worden. Und diese Klageverfahren wurden vor allen Dingen von der Deutschen Umwelthilfe durchgeführt.  Politiker von CDU\/CSU und FDP haben in Interviews und Talkshows großspurig angekündigt, gegen die Deutsche Umwelthilfe vorzugehen. Das Verbandsklagerecht bzw. die Gemeinnützigkeit solle aberkannt werden. Allerdings ist es wie so oft: Den großspurigen Ankündigungen folgte: nichts.  Die erste Vorlage, die sich dieses Themas annimmt, kommt hier und heute von der AfD.  Etliche Bürger haben mich angeschrieben, und wir haben uns die europäische Gesetzgebung zu diesem Thema mal genauer angeschaut. Wieder einmal haben wir festgestellt: Das Problem liegt nicht in der EU-Richtlinie begründet, sondern in der ungeschickt oder gezielt böswilligen Umsetzung dieser Richtlinie in deutsches Recht.  Die AfD will mit dem vorliegenden Gesetzentwurf das Umwelt- und Verbandsklagerecht keinesfalls abschaffen. Im Gegenteil: Das Verbandsklagerecht soll auf Landkreise und Gemeindeebene ausgedehnt werden, damit zukünftig auch Bürger eines kleinen Gebiets die Möglichkeit haben, sich umweltrechtlich Gehör zu verschaffen, beispielweise gegen die Verschandlung unserer Natur durch Windräder.  Wir wollen nicht die Ausschaltung von Umweltorganisationen, wir wollen aber endlich Fehlentwicklungen im Verbandsklagerecht beheben. Wir wollen nicht, dass Organisationen in Deutschland das Verbandsklagerecht missbrauchen, um sich selbst zu bereichern. Diesen Organisationen geht es weder um saubere Luft noch um die Natur. Es ist schlicht ein Geschäftsmodell. Das Verbandsklagerecht im Zuge der Umsetzung einer europäischen Richtlinie gibt es seit zwölf Jahren. Eine Bestandsaufnahme zeigt, dass dringender Handlungsbedarf für eine Gesetzesänderung besteht. Das Problem will ich Ihnen kurz erklären: Vereinigungen mit wenigen Hundert Mitgliedern können volkswirtschaftlich wichtige Projekte aufhalten oder gar verhindern. Viele dieser Vereinigungen verfolgen kommerzielle Ziele oder sind Abmahnvereine. Durch die Intransparenz der Finanzierungsstrukturen ist es möglich, dass Wettbewerber sowie ausländische Organisationen oder Staaten heimischen Unternehmen direkt schaden können. Eine Gesetzesänderung ist deshalb dringend erforderlich.  Wir wollen, dass sich anerkannte Umweltvereinigungen zukünftig ausschließlich statt wie bislang vorwiegend dem Ziel des Umweltschutzes zu widmen bzw. zu unterwerfen haben. Wir wollen, dass Finanzierungsstrukturen offengelegt werden. Wo kommen die Spender her? Sind es ausländische Unternehmen, sind es fremde Staaten? Momentan ist das alles verschleiert. Wir wollen, dass Spenden von außerhalb der EU nicht mehr angenommen werden dürfen, um zu unterbinden, dass ausländische Organisationen unsere Wirtschaft lahmlegen können.  Zukünftig sollen Vereinigungen nur noch anerkannt werden, wenn sie mindestens 0,1 Prozent der Wahlberechtigten in ihrem Tätigkeitsgebiet als Mitglieder aufweisen. Das entspricht übrigens dem Europawahlrecht. Die großen Umweltorganisationen NABU und BUND haben damit kein Problem. Es ist aber ein Unding, dass eine Organisation mit wenigen Hundert Mitgliedern halb Deutschland lahmlegen kann. Das wollen wir ändern.  Zum Abschluss noch ein paar Worte an die Kollegen von CDU und CSU. Es reicht nicht, wenn Sie sich medial echauffieren, sich betroffen fühlen und über die Aberkennung von Gemeinnützigkeit diskutieren. Bei diesem Gesetzentwurf müssen Sie Farbe bekennen. Sie können beweisen, dass Sie sich für die Interessen der Bürger in unserem Land einsetzen oder ob Sie nur Sonntagsreden halten.  Nutzen Sie diese Chance! Machen Sie endlich wieder Politik im Interesse unseres Landes und seiner Bürger! Wenn Sie es nicht machen, dann geben wir den Takt vor. Vielen Dank.  

\noindent\textbf{Comment:}
\begin{itemize}
    \setlength\itemsep{-3pt}
    \item (Beifall bei der AfD – Ralph Lenkert [DIE LINKE]: Sie schützen wieder einmal die Betrüger in der Autoindustrie! – Weitere Zurufe von der LINKEN)
    \setlength\itemsep{-3pt}
    \item (Beifall bei der AfD)
    \setlength\itemsep{-3pt}
    \item (Zurufe vom BÜNDNIS 90/DIE GRÜNEN)
    \setlength\itemsep{-3pt}
    \item (Beifall bei der CDU/CSU)
    \setlength\itemsep{-3pt}
    \item (Beifall bei Abgeordneten der AfD)
    \setlength\itemsep{-3pt}
    \item (Beifall bei der AfD – Zurufe vom BÜNDNIS 90/DIE GRÜNEN)
\end{itemize}
\subsection{Schweiger}
\noindent\textbf{Texts:} Sehr geehrter Herr Präsident! Liebe Kolleginnen und Kollegen! Ja, wir werden Farbe bekennen, aber sicherlich nicht die, die Sie sich an dieser Stelle wünschen. Deswegen will ich zunächst einmal sachlich auf die Dinge eingehen. Im Umwelt- und Naturschutzrecht gibt es in der Tat die Besonderheit, dass Umweltverbände Klagen vor Verwaltungsgerichten erheben können, ohne in den eigenen Rechten betroffen zu sein. Diese sogenannte Umweltverbandsklage hat ihre gesetzlichen Grundlagen im Umwelt-Rechtsbehelfsgesetz und im Bundesnaturschutzgesetz sowie natürlich in den landesrechtlichen Bestimmungen, die da ja oftmals sehr verschieden sind. Zentrale Forderungen des vorliegenden Gesetzentwurfs zielen daher zunächst richtigerweise auf die Vermeidung von missbräuchlichen Klagemöglichkeiten der Verbände ab. Als unbestimmte Rechtsauslegung, was missbräuchlich ist und was nicht, werden Einschränkungen vorgesehen, die aus Sicht unserer Fraktion verwaltungstechnisch nicht praktikabel sind bzw. eher zu weiteren strittigen Auseinandersetzungen führen werden. So soll zum Beispiel das Verbandsklagerecht nur noch Verbänden zugestanden werden, deren Mitgliederzahl in ihrem Tätigkeitsbereich – das wurde gesagt – einem Tausendstel der dort Wahlberechtigten entspricht. Ganz abgesehen von der Frage, wie das abschließend ermittelt und auf dem aktuellen Stand gehalten werden soll, stehen bei Klagebefugnis weiterhin alle Rechtsinstanzen mit erheblichen Verfahrensdauern offen. Verbände mit Verbandsklagerecht im Umweltschutz sollen ausschließlich dem Umweltschutz verpflichtet sein. Eine regelmäßige Prüfung der Einstufung als gemeinnützige Organisation ist ein durchaus richtiger Ansatz im vorliegenden Entwurf. Allerdings eröffnen die Kriterien im Gesetzentwurf wiederum Auslegungsmöglichkeiten. Damit wird dieser Ansatz in der Praxis nicht halten, was er auf dem Papier verspricht. Wenn – wie in jüngster Vergangenheit – zum Beispiel von der schon genannten Deutschen Umwelthilfe Klagen geführt werden, die sich offenbar nicht mehr nur an konkreten Einzelfällen orientieren, sondern eher schon als flächendeckend und von grundsätzlicher Natur zu bezeichnen sind, ist die von mir eingangs genannte Überprüfungsmöglichkeit der Gemeinnützigkeit durchaus ein legitimes Mittel. Das ist auch vor dem Hintergrund, dass Prozesse und Abmahnungen nicht zum Geschäftszweck werden dürfen, sicherlich richtig. Um nicht missverstanden zu werden: Mehr Transparenz in Fragen der Finanzierung solcher Verbände wäre nicht nur bei der Deutschen Umwelthilfe wünschenswert, sondern auch bei einer Vielzahl anderer NGOs. Dies würde auch die redlichen Verbände, die es ja auch gibt, vor einem Generalverdacht bewahren.  Aber diese Prüfung – und da unterscheiden sich unsere Vorstellungen von denen der Vorlage – muss natürlich ergebnisoffen erfolgen und darf nicht mit eingeschränkten Zielrichtungen versehen werden. Die Änderungen der Umweltgesetzgebung der letzten Jahre führten unzweifelhaft zur Zunahme von Verbandsklagen; insbesondere zieht natürlich die Möglichkeit, dass von den klagenden Verbänden im laufenden und in weiteren Verfahren noch Sachverhalte neu eingebracht werden können,  die Verfahren in die Länge. Die Einführung der Präklusion, mit der das immer wieder neue Einbringen nachträglicher Argumente ausgeschlossen werden soll, ist der richtige Ansatz.  Das hat auch die Justizministerkonferenz Ende letzten Jahres so gesehen und einen Antrag aus Bremen angenommen. Aber auch Breite und Tiefe der Klagen nahmen zu. Finale Rechtsauslegungen durch Gerichte haben zu Entwicklungen geführt, die diskutabel sind. So richten sich mittlerweile Klagemöglichkeiten zum Schutz der Natur gegen den Trassenausbau, der ja gerade notwendig ist, um den Ausbau der erneuerbaren Energien zu forcieren. Solche Entwicklungen sind aber nicht durch den vorgelegten Gesetzentwurf behebbar, zumal der Entwurf als Ziel benennt, dass die Behinderung von Infrastrukturmaßnahmen ausgeschlossen werden soll. Das fängt in der Planungsphase an, worauf ja das im letzten Jahr verabschiedete Gesetz zur Planungsbeschleunigung zielt. Auch für die Durchführung wäre ein Gesetz nach dem Beispiel des Infrastrukturbeschleunigungsgesetzes – wie nach der Wiedervereinigung – zielführender, da es grundsätzliche rechtsstaatliche Möglichkeiten nicht aushebelt, aber mit Instanzeinschränkungen maßgeblich zu Verfahrensverkürzungen beigetragen hat.  Hier sind weitere Initiativen der Koalition in dieser Legislatur geplant. Ich würde mich freuen, wenn sie breit unterstützt würden. Insgesamt werden wir dem vorliegenden Entwurf in dieser Form nicht zustimmen; denn als Resümee ist zu sagen: Gut gedacht ist nicht immer gut gemacht!  In diesem Sinne freue ich mich auf die weiteren Erörterungen. Vielen Dank.  

\noindent\textbf{Comment:}
\begin{itemize}
    \setlength\itemsep{-3pt}
    \item (Beifall bei der FDP)
    \setlength\itemsep{-3pt}
    \item (Steffi Lemke [BÜNDNIS 90/DIE GRÜNEN]: Welche sind unredlich? – Oliver Krischer [BÜNDNIS 90/DIE GRÜNEN]: Ist doch unglaublich!)
    \setlength\itemsep{-3pt}
    \item (Lachen bei der AfD)
    \setlength\itemsep{-3pt}
    \item (Beifall bei Abgeordneten der CDU/CSU)
    \setlength\itemsep{-3pt}
    \item (Steffi Lemke [BÜNDNIS 90/DIE GRÜNEN]: Die haben ja auch recht gehabt!)
    \setlength\itemsep{-3pt}
    \item (Beifall bei der CDU/CSU)
    \setlength\itemsep{-3pt}
    \item (Beifall der Abg. Marie-Luise Dött [CDU/CSU])
\end{itemize}
\subsection{Müller-Böhm}
\noindent\textbf{Texts:} Herr Präsident! Meine verehrten Kolleginnen und Kollegen! Meine sehr geehrten Damen und Herren! Der von der AfD vorgelegte Gesetzentwurf beschäftigt sich mit Rechtsbehelfen für Verbände in Umweltangelegenheiten. Aber was heißt das eigentlich konkret? Konkret geht es hierbei um die Deutsche Umwelthilfe. Und da muss man auch einmal sagen: Das ist natürlich ein proble­matischer Fall. Wir reden über einen Verein, der gerade einmal 361 Mitglieder hat, aber dafür immerhin 100 Mitarbeiter, und mit diesen Mitgliedern und Mitarbeitern ein ganzes Land tyrannisiert und ihm irgendwie seinen Willen aufzwingen will,  einen Verein, der sinnvolle Instrumente des Rechtsstaats pervertiert, um eigene Politik zu betreiben. Und genau so darf es nicht weitergehen, meine Damen und Herren!  Die Deutsche Umwelthilfe ist im Grunde ein plumper Abmahnverein. Sie benutzt sehr kritische Grenzwerte.  Übrigens, liebe AfD, haben wir im Deutschen Bundestag auch schon einen eigenen Antrag zu dem Thema eingebracht. Ihre Ausführungen gerade eben treffen also einfach nicht zu. Wir fordern ganz klar, die Grenzwerte temporär auszusetzen, damit wir hier Fahrverbote vorübergehend auf jeden Fall vermeiden können.  Sie aber scheinen das zu vergessen. Die Deutsche Umwelthilfe lässt sich dann auch noch von bestimmten Autokonzernen mit ganz gezielten Geschäftsinteressen teilweise finanzieren und fördern,  um somit ihren Kulturkampf gegen das Auto voranzutreiben. Ich sage es ganz deutlich: Diese Umweltinquisition namens Deutsche Umwelthilfe ist genauso kurzsichtig wie ihr historischer Beiname. Es bringt rein gar nichts, absolut nichts, das Auto einfach von bestimmten Straßen weg- und in andere Straßen hineinfahren zu lassen. Dieser überzogene Hass auf den Individualverkehr mit dem Automobil verschlimmert gerade in Metropolen wie in meiner Heimat, dem Ruhrgebiet, die bereits vorhandenen Mobilitätsprobleme enorm. Genau das hat die Deutsche Umwelthilfe zu verantworten, und das ist einfach unverantwortlich.  Aber schauen wir uns ruhig einmal die Geschäftspraktiken der Deutschen Umwelthilfe näher an: Wer über 2 Millionen Euro mit Abmahnschikanen gegen Einzelhändler erwirtschaftet, sich dabei auch noch selbstlos gibt und dann Mitarbeitern Gehälter zahlt, von denen manche Bürgerinnen und Bürger nur träumen können, der missbraucht Rechtsbehelfe eigentlich nur zur persönlichen Bereicherung und gängelt damit die Gesellschaft. Meine sehr verehrten Damen und Herren, der Rechtsstaat ist eben kein Versorgungswerk, und Rechtsbehelfe sind kein Renditeprodukt.  Wir wissen doch, der Rechtsstaat lebt von der Akzeptanz, davon, dass unsere Gesetze angemessen angewandt werden. Die Deutsche Umwelthilfe hingegen nutzt den Rechtsstaat einfach nur aus und beschmutzt nebenbei auch noch die seriöse Arbeit mancher sehr guter Umweltverbände.  Was dabei nicht so stehen gelassen werden darf, ist, dass sie all das auch noch mit einer moralischen Überlegenheit tut, die so unglaublich arrogant daherkommt, dass sie gerade die Fleißigen dieser Gesellschaft mit einem „ökologische Marktüberwachung“ genannten Titel auch noch permanent abmahnen will. Das ist absolut unanständig und unfassbar.  Ich muss jetzt aber auch noch auf die deutsche Bundesregierung zu sprechen kommen. Ich bin jetzt einmal so naiv und nehme an, sie tut etwas in der Richtung, wird da jetzt aktiv. Aber nein! Sie fördert die Deutsche Umwelthilfe bei deren Klimakreuzzug auch noch finanziell über Förderprogramme und Gemeinnützigkeit. Ich frage Sie jetzt ganz ehrlich: Wie sollen wir eigentlich irgendeinem Handwerker, kleinen Mittelständler oder einem Autopendler, der tagtäglich fleißig zur Arbeit fahren muss, erklären, dass er brav Steuern zahlt, Sie aber über Gemeinnützigkeit und Fördermittel hintenrum dafür sorgen, dass diese Leute noch schlechter zur Arbeit kommen? Erklären Sie das mal draußen jemandem!  Es stellt sich einfach die Frage, liebe Union – tut mir leid –, warum Ihr wirtschaftspolitischer Sprecher sagt, die Deutsche Umwelthilfe sei semikriminell, aber ihr dann dagegen selbst kaum etwas tut. Das ist doch unglaubwürdig. Also, was brauchen wir? Wir brauchen eine vernünftige Lösung, um den Missbrauch von Rechtsbehelfen abzustellen. Dabei allerdings Umweltverbände, die anständige Arbeit leisten, zu sabotieren, kann auch nicht das Ziel sein. Das wäre ein Kollateralschaden, den wir nicht in Kauf nehmen können. Die Lösung muss so sein, dass der Rechtsstaat hier eine angemessene und ausgewogene Antwort findet. Vielen Dank.  

\noindent\textbf{Comment:}
\begin{itemize}
    \setlength\itemsep{-3pt}
    \item (Beifall bei der FDP)
    \setlength\itemsep{-3pt}
    \item (Beifall bei der FDP sowie bei Abgeordneten der CDU/CSU und der AfD – Zurufe vom BÜNDNIS 90/DIE GRÜNEN)
    \setlength\itemsep{-3pt}
    \item (Beifall bei der FDP sowie bei Abgeordneten der AfD und des Abg. Klaus-Peter Willsch [CDU/CSU] – Widerspruch beim BÜNDNIS 90/DIE GRÜNEN)
    \setlength\itemsep{-3pt}
    \item (Beifall bei der FDP sowie bei Abgeordneten der AfD – Steffi Lemke [BÜNDNIS 90/DIE GRÜNEN]: Sie wollen doch eine Rechtsstaats­partei sein! Das ist eine unsägliche Rede!)
    \setlength\itemsep{-3pt}
    \item (Beifall bei der FDP sowie bei Abgeordneten der AfD – Steffi Lemke [BÜNDNIS 90/DIE GRÜNEN]: Eine reine AfD-Rede!)
    \setlength\itemsep{-3pt}
    \item (Beifall bei Abgeordneten der FDP und der AfD)
    \setlength\itemsep{-3pt}
    \item (Beifall bei der FDP und der AfD sowie der Abg. Marie-Luise Dött [CDU/CSU])
    \setlength\itemsep{-3pt}
    \item (Beifall bei der FDP sowie bei Abgeordneten der CDU/CSU und der AfD – Widerspruch bei der LINKEN und dem BÜNDNIS 90/DIE GRÜNEN)
    \setlength\itemsep{-3pt}
    \item (Beifall bei Abgeordneten der SPD)
    \setlength\itemsep{-3pt}
    \item (Steffi Lemke [BÜNDNIS 90/DIE GRÜNEN]: Rechenfehler der Lungenärzte!)
    \setlength\itemsep{-3pt}
    \item (Beifall bei der FDP sowie bei Abgeordneten der CDU/CSU und der AfD – Sabine Leidig [DIE LINKE]: Das ist ja eins zu eins AfD!)
    \setlength\itemsep{-3pt}
    \item (Beifall bei der FDP sowie bei Abgeordneten der CDU/CSU und der AfD)
\end{itemize}
\subsection{Mindrup}
\noindent\textbf{Texts:} Sehr geehrter Herr Präsident! Liebe Kolleginnen und Kollegen! Die AfD-Fraktion legt hier einen Gesetzentwurf zur Änderung des Umwelt-Rechtsbehelfsgesetzes vor, der die Voraussetzungen an die Anerkennung von Umweltverbänden verschärfen soll. Ziel des Gesetzentwurfs ist es, eine vermeintliche Gefahr des Missbrauchs des Verbandsklagerechts in Umweltangelegenheiten auszuschließen. Ich könnte mich jetzt kurz fassen und sagen: „Das ist alles Unsinn“; aber damit würde ich es der AfD zu einfach machen. Deswegen habe ich ein bisschen genauer hingeschaut,  was hinter dem Antrag steht. Der Antrag basiert erst einmal auf dem falschen Gerücht, wir würden in Deutschland schärfere Umweltvorschriften als in anderen Ländern haben und wir würden Umweltschutzorganisationen in Deutschland stärkere Rechte als in anderen Ländern einräumen. Das stimmt nicht. Deswegen muss ich noch etwas zu dem Rechtsrahmen sagen, der hier dahintersteckt. Rechtsrahmen und Voraussetzung für dieses deutsche Recht ist die Århus-Konvention, die 1998 in Dänemark unterschrieben wurde. Deutschland hat sie allerdings erst nach der rot-grünen Regierungsübernahme 1998 ratifiziert, als letzter EU-Vertragsstaat. Da sieht man schon, dass dahinter ein gewisser Sprengstoff steht.  Die Århus-Konvention ist die einzige internationale Vereinbarung im Umweltschutz, die auch Bürgerinnen und Bürgern und Nichtregierungsorganisationen Rechte verleiht. Es gibt keinen anderen internationalen Vertrag, der das tut. Deswegen ist das eine ganz wichtige Konvention.  Für diese Konvention haben sich vor allen Dingen die Bürgerbewegten aus den osteuropäischen Ländern und der ehemaligen DDR eingesetzt; sie haben das aus ihrer Erfahrung der Unterdrückung heraus gemacht. Wer jetzt die Rechte der Bürgerinnen und Bürger einschränken will, die uns dieses Erbe überlassen haben, der tritt das Erbe der friedlichen Revolution mit Füßen.  Weiterhin liegt dem Vorschlag der Fraktion der AfD die Annahme zugrunde, dass die Anerkennungsregeln für Umweltverbände in Deutschland zu weit gefasst seien. Das Gegenteil ist der Fall. Die Anerkennungsregeln für Umweltverbände sind in Deutschland mit die stärksten in der EU. Es gibt andere Länder, die überhaupt keine staatliche Zulassung für Klagerechte verlangen. Deswegen steht der Vorschlag der AfD weder mit völkerrechtlichen noch mit europarechtlichen Rahmenbedingungen im Einklang. Es ist sogar anders: Die Frage, ob die geltenden strengen Anerkennungsvoraussetzungen in Deutschland überhaupt mit der Århus-Konvention übereinstimmen, ist strittig. Es gibt gerade ein Verfahren gegen die Bundesrepublik Deutschland, weil unsere Anerkennungsverfahren zu streng sind, weil wir zum Beispiel Stiftungen nicht zulassen. Der Beschwerdeausschuss der internationalen Konvention hat schon getagt, und es sieht so aus, als würden wir das Verfahren verlieren. Auch die Europäische Kommission sagt, unser Recht erfülle dort nicht die internationalen Rahmenbedingungen. Es ist also genau andersherum, als Sie es hier erzählen. Die 7. Vertragsstaatenkonferenz wird 2021 tagen. Ich nehme an, dass wir hier als Gesetzgeber wieder etwas zu tun haben werden: Wir werden dieses Recht anpassen müssen, wenn wir nicht anschließend wieder ein Strafverfahren von der EU bekommen möchten. Das ist die Wahrheit: Wir sind eben nicht die Vorreiter in dem Fall, sondern wir müssten mehr tun – ganz im Gegensatz zum dem, was Sie hier sagen.  Ich komme also zu einem ersten Zwischenfazit: Der AfD geht es um nichts anderes als um die Einschränkung von Bürgerrechten.  Wir können hier auch ganz offen reden – Sie haben es ja in Ihrer Rede getan; aber Sie haben es in dem Gesetzentwurf nicht getan –: Ihnen geht es um die Deutsche Umwelthilfe. Nein. Die Antwort bleibt Nein.  Also, es geht um die Deutsche Umwelthilfe; das ist ja hier eben auch gesagt worden. Ich möchte mir gar nicht vorstellen, wie es wäre, wenn die Deutsche Umwelthilfe von Migrantinnen und Migranten gegründet worden wäre; das als kleine Nebenbemerkung hier.  Es geht Ihnen gar nicht darum, Lösungen für ein reales Problem zu finden. Mir ist nicht bekannt, dass die Deutsche Umwelthilfe Software von Autos manipuliert hätte. Mir ist auch nicht bekannt, dass die Deutsche Umwelthilfe Autos verkauft hätte, die dreckiger sind als versprochen.  Mir gefällt an der Deutschen Umwelthilfe auch nicht alles; das kann ich Ihnen sagen. Aber die Deutsche Umwelthilfe hat ihre Rechte wahrgenommen, die ihr unser Gesetz gibt, und sie hat vor Gericht fast überall gewonnen; das muss man doch zur Kenntnis nehmen. Was wollen Sie mit Ihrem Gesetzentwurf? Es geht Ihnen gar nicht darum, hier Lösungen zu finden, sondern Sie wollen ablenken von den realen Problemen. Wir müssen die Luft in unseren Städten sauberer machen; das ist doch vollkommen klar. Darauf haben vor allen Dingen die einen Anspruch, die Probleme haben: Das sind die Asthmatiker und die kleinen Kinder. Wir müssen doch eine Politik machen, die die schützt, die zu schützen sind.  Und wir müssen natürlich klar sagen, was wir machen müssen: Wir brauchen eine bessere Verkehrspolitik; das ist völlig unstrittig. Es gibt ja Beschlüsse der Bundesregierung, die da in die richtige Richtung gehen: Stärkung des Umweltverbundes, Stärkung des öffentlichen Nahverkehrs.  Und wir brauchen saubere Autos, um unsere Gesundheit zu schützen – ganz egoistisch –, aber wir brauchen auch saubere Autos, weil in der Welt zukünftig nur noch saubere Autos verkauft werden können.  Das wissen die Kolleginnen und Kollegen in den Werken auch; sie wissen, dass unsere ganze Exportindustrie daran hängt, dass wir das schaffen. Aber wir brauchen auch Nachrüstung; vollkommen klar, das haben wir als SPD immer gesagt. Da sind die Automobilindustrie und der Bundesverkehrsminister in der Pflicht. Ich kann Ihnen sagen: Mit dieser Strategie werden wir die Grenzwerte wieder einhalten und beides schaffen: die Verbraucherrechte schützen – die von Ihnen immer zitierten Menschen, die wenig Geld haben, schützen – und die Gesundheit schützen. Das geht nämlich zusammen.  Jetzt möchte ich noch einen weiteren Aspekt beleuchten: was ich hier wiederfinde an Prinzipien bei der AfD. Sie bedienen wieder das Sündenbockprinzip. Es wird an der Stelle ganz verschlagen, wo Sie sagen, es gibt internationale Mächte – ich weiß gar nicht, wer das sein soll –,  die sich jetzt quasi der Umweltschutzorganisationen bedienen, um unser Land zu schädigen. – Dafür müssten Sie einen Beleg bringen. Das tun Sie aber nicht. Es ist ganz klar gesetzlich ausgeschlossen, dass die Umweltschutzorganisationen durch Klagen eigene Vorteile haben; dann müssten sie auch den Weg vor die Gerichte antreten. Das tun sie aber nicht. Die AfD will das Problem also gar nicht lösen. Das Missbrauchspotenzial, das Sie ansprechen, ist gar nicht gegeben. Diese Verbreitung von Verschwörungstheorien ist gefährlich, weil sie die Stimmung in unserem Land vergiftet.  Als ich Ihren Gesetzentwurf gelesen habe, habe ich mir die Frage gestellt: Wer ist denn als Nächstes dran? Die unabhängigen Gerichte,  die die Deutsche Umwelthilfe bestätigen? Die Anwälte, die das machen? Die Gewerkschaften? Unabhängige Forschungsinstitute? Wenn man Ihre Reden hört, braucht man nicht viel Fantasie, um diese Liste zu verlängern.  Die Menschen in unserem Land erkennen, was hier gerade passiert: Sie treten die freiheitlich-demokratische Grundordnung mit Füßen.  Wenn Sie davon reden, dass hier eine Umweltschutzorganisation Spenden offenlegen soll, dann fangen Sie doch erst mal bei sich selber an!  Ich kann Ihnen eines sagen: Wir werden Sie damit nicht durchkommen lassen. Die Menschen in diesem Land erkennen, welch Geistes Kind Sie sind. Danke schön. 

\noindent\textbf{Comment:}
\begin{itemize}
    \setlength\itemsep{-3pt}
    \item (Beifall bei Abgeordneten der SPD sowie der Abg. Sylvia Kotting-Uhl [BÜNDNIS 90/DIE GRÜNEN])
    \setlength\itemsep{-3pt}
    \item (Beifall bei SPD sowie bei Abgeordneten der LINKEN und des BÜNDNISSES 90/DIE GRÜNEN – Dr. Alexander Gauland [AfD]: Schämen Sie sich! So ein Stuss!)
    \setlength\itemsep{-3pt}
    \item (Beifall bei der SPD sowie bei Abgeordneten der LINKEN und des BÜNDNISSES 90/DIE GRÜNEN)
    \setlength\itemsep{-3pt}
    \item (Beifall bei Abgeordneten der SPD und des BÜNDNISSES 90/DIE GRÜNEN – Dr. Bernd Baumann [AfD]: So ein Blödsinn!)
    \setlength\itemsep{-3pt}
    \item (Beifall bei der SPD sowie bei Abgeordneten der LINKEN und des BÜNDNISSES 90/DIE GRÜNEN – Karsten Hilse [AfD]: So ein Schwachsinn! Das kann doch wohl nicht wahr sein!)
    \setlength\itemsep{-3pt}
    \item (Beifall bei SPD sowie bei Abgeordneten der LINKEN und des BÜNDNISSES 90/DIE GRÜNEN – Jürgen Braun [AfD]: Geben Sie mal Ihre Zeitungen ab!)
    \setlength\itemsep{-3pt}
    \item (Beifall bei Abgeordneten der SPD, der LINKEN und des BÜNDNISSES 90/DIE GRÜNEN)
    \setlength\itemsep{-3pt}
    \item (Karsten Hilse [AfD]: Wir wissen es!)
    \setlength\itemsep{-3pt}
    \item (Steffi Lemke [BÜNDNIS 90/DIE GRÜNEN]: So ist das!)
    \setlength\itemsep{-3pt}
    \item (Beifall der Abg. Cansel Kiziltepe [SPD])
    \setlength\itemsep{-3pt}
    \item (Beifall bei Abgeordneten der SPD)
    \setlength\itemsep{-3pt}
    \item (Zuruf von der AfD: Vielen Dank!)
    \setlength\itemsep{-3pt}
    \item (Beifall bei Abgeordneten der SPD und des BÜNDNISSES 90/DIE GRÜNEN)
    \setlength\itemsep{-3pt}
    \item (Karsten Hilse [AfD]: So ein Blödsinn, echt!)
    \setlength\itemsep{-3pt}
    \item (Beifall bei Abgeordneten der SPD, der LINKEN und des BÜNDNISSES 90/DIE GRÜNEN – Enrico Komning [AfD]: Ihre Stimmung!)
\end{itemize}
\subsection{Spaniel}
\noindent\textbf{Texts:} Vielen Dank, Herr Präsident. – Also, erst mal stelle ich fest, dass die SPD-Fraktion offensichtlich kein Problem damit hat, dass die Deutsche Umwelthilfe viele Klageverfahren gewonnen hat  und das halbe Land terrorisiert.  Das scheint offensichtlich von Ihnen gebilligt zu werden und in Ihrem Interesse zu sein.  Zweitens. Die Århus-Konvention sagt ganz klar, dass innerstaatliches Recht regelt, welche Organisation anerkannt wird und welche nicht. Das innerstaatliche Recht der Bundesrepublik Deutschland bestimmen wir hier. Somit ist eine Änderung dieser innerstaatlichen Anerkennung sehr wohl konform mit dieser Konvention. Wir haben gesagt, wir wollen Transparenz. Es kann doch nicht in Ihrem Interesse sein, dass es keine Transparenz darüber gibt,  wer hinter Umweltorganisationen steht – das sagt sehr viel über Ihr Verständnis von Demokratie.  Selbstverständlich wollen wir auch – wir wollen auch – eine demokratische Verankerung von solchen Organisationen. Es kann nicht sein, dass 300 oder 400 Menschen in diesem Land darüber bestimmen, wie in unserem Land Fahrverbote oder sonstige Einschränkungen geregelt werden. Aus diesem Grunde sagen wir ganz klar: Das, was in der Gesetzgebung momentan verankert ist, soll mit unserem Gesetzentwurf korrigiert werden. Das stärkt die Interessen der Bürger in diesem Land und verbessert die Demokratie und die Akzeptanz. – Wenn Sie das nicht wollen, dann spricht das Bände. 

\noindent\textbf{Comment:}
\begin{itemize}
    \setlength\itemsep{-3pt}
    \item (Beifall bei der AfD)
    \setlength\itemsep{-3pt}
    \item (Ulli Nissen [SPD]: Was reden Sie da für einen Schwachsinn!)
    \setlength\itemsep{-3pt}
    \item (Armin-Paulus Hampel [AfD]: Doch! Natürlich!)
    \setlength\itemsep{-3pt}
    \item (Zuruf des Abg. Ralph Lenkert [DIE LINKE])
    \setlength\itemsep{-3pt}
    \item (Beifall bei Abgeordneten der AfD)
\end{itemize}
\subsection{Mindrup}
\noindent\textbf{Texts:} Liebe Kolleginnen und Kollegen, Ihre Ausführungen zur Århus-Konvention zeigen, dass Sie die Prinzipien der Århus-Konvention nicht verstanden haben.  Sie wollen sie nur der Form nach einhalten. Aber das Grundprinzip der Århus-Konvention ist Beteiligung, Öffentlichkeit und Klagerechte, und genau das wollen Sie einschränken. Sie haben gesagt, es gebe keine Transparenz. Natürlich gibt es die; denn die Organisation, über die wir die ganze Zeit reden, ist gemeinnützig. Finanzämter prüfen so etwas. An anderer Stelle gab es eine Anhörung, weil Sie diese Gemeinnützigkeit angegriffen haben. Diese Organisation legt ihre Informationen offen; sonst wüssten Sie das alles ja gar nicht. Das ist also wieder ein fadenscheiniges Argument. Jetzt kommen wir zu dem entscheidenden Punkt: Sie tun immer wieder so, als sei die Deutsche Umwelthilfe das Problem.  Das Problem sind aber die Überschreitungen der Grenzwerte, die Betrügereien und die falschen Regulierungen. Das haben unabhängige Gerichte festgestellt.  Ich habe Respekt vor der Gewaltenteilung. Ich achte unabhängige Gerichte. Das tun Sie offenbar nicht. Wer dies nicht tut, achtet nicht unseren Rechtsstaat.  

\noindent\textbf{Comment:}
\begin{itemize}
    \setlength\itemsep{-3pt}
    \item (Beifall bei der SPD sowie bei Abgeordneten der LINKEN und des BÜNDNISSES 90/DIE GRÜNEN)
    \setlength\itemsep{-3pt}
    \item (Beifall bei der SPD und dem BÜNDNIS 90/DIE GRÜNEN sowie bei Abgeordneten der LINKEN)
    \setlength\itemsep{-3pt}
    \item (Beifall bei Abgeordneten der SPD, der LINKEN und des BÜNDNISSES 90/DIE GRÜNEN)
    \setlength\itemsep{-3pt}
    \item (Enrico Komning [AfD]: Richtig!)
    \setlength\itemsep{-3pt}
    \item (Beifall bei der LINKEN)
\end{itemize}
\subsection{Leidig}
\noindent\textbf{Texts:} Herr Präsident! Verehrte Kolleginnen und Kollegen! Liebe Gäste! Ich kann nahtlos an das anschließen, was der Kollege Mindrup hier prima vorgelegt hat. Seit Monaten macht die politische Rechte hierzulande mobil gegen den gemeinnützigen Verein Deutsche Umwelthilfe.  Warum? Weil die Umwelthilfe dafür kämpft, dass die Vorschriften zum Schutz von Umwelt und Menschen eingehalten werden. Tatsächlich hat sie nämlich schon vor vielen Jahren belegt, dass Autohersteller die gesetzlichen Grenzwerte nicht einhalten. Sie hat Briefe geschrieben. Sie hat Messungen vorgelegt. Sie hat Studien gemacht. Sie hat in Ministerien vorgesprochen. Und sie hat gefordert, dass echte Messungen am Auspuff vorgenommen werden. Der Verband der Automobilindustrie war dagegen und hat sich durchgesetzt. So konnten Hunderttausende Fahrzeuge mit Betrugssoftware verkauft werden. Die Deutsche Umwelthilfe hat gegen diese Gesetzesverstöße geklagt, und das ist gut so.  Vor einem Jahr hat das Bundesverwaltungsgericht ein Grundsatzurteil gesprochen und festgestellt, dass beschränkte Fahrverbote für bestimmte Dieselfahrzeuge möglich sind, wenn nur so die Luftqualität in belasteten Städten erhalten werden kann. Seither überschlagen sich die Parteien hier rechts im Haus mit Vorstößen, wie man die Umwelthilfe außer Gefecht setzen kann. Die AfD will ihre Befugnisse einschränken. Die FDP will die Projektmittel aus dem Bundeshaushalt streichen. Die CDU will beides und stellt dazu noch die Gemeinnützigkeit infrage.  Sie wollen eine unbequeme Stimme zum Schweigen bringen. Sie wollen den Leuten weismachen, dass die Umwelthilfe an den Fahrverboten schuld ist. Das ist wirklich ein plattes Ablenkungsmanöver.  Ich kann nur empfehlen, vor allen Dingen Ihnen von der FDP, die Ausführungen des ehemaligen Innenministers Gerhart Baum zu lesen. Ich zitiere: Es ist das Versagen der Politik, das zu dieser Situation geführt hat. Die Politik hat über Jahrzehnte gemeinsame Sache mit der Automobilindustrie gemacht und so notwendige Innovationen verhindert. Dort liegt die Verantwortung für die aktuelle Bedrohung der Umwelt und der individuellen Mobilität und auch für die Bedrohung der Arbeitsplätze.  Ich sage Ihnen: Was die Parteien hier rechts und Die Linke ganz klar unterscheidet, ist: Sie wollen die Automobilgesellschaft weiterfahren lassen,  koste es, was es wolle. Wir hingegen wollen soziale und ökologische Alternativen stark machen.  Die Deutsche Umwelthilfe nimmt rund 8 Millionen Euro im Jahr aus Spenden, Bußgeldern und öffentlichen Projektmitteln ein. Sie gibt für Umwelt- und Verbraucherschutz auch wieder 8 Millionen Euro aus. Sie sagen, das sei ein fragwürdiges Geschäftsmodell. Wir sagen: Fragwürdig ist das Geschäftsmodell der Autokonzerne,  die immer größere Autos bauen, weil sie daran am meisten verdienen, die Grenzwerte unterlaufen, sich vor Entschädigungszahlungen drücken und Milliardengewinne an die Aktionäre ausschütten. Damit muss endlich Schluss sein.  Die Deutsche Umwelthilfe bekommt vom Bund Geld für gemeinnützige Projekte, bis zu 5 Millionen Euro jährlich. Die Automobilindustrie bekommt Hunderte Millionen Euro aus dem Bundeshaushalt für eigennützige Zwecke. Sie sagen: Kein Steuergeld mehr für die Umwelthilfe. – Wir sagen: Kein Steuergeld für die Gewinne von Daimler, BMW, Porsche und VW.  Sie verlangen vollständige Transparenz von der Umwelthilfe. Das ist völlig in Ordnung. Aber wir verlangen vollständige Transparenz für alle politischen Akteure, auch für den Verband der Automobilindustrie, der in den Ministerien und im Kanzleramt ein und aus geht.  Sie sagen: Wer Umweltrecht einklagt, schadet der deutschen Wirtschaft. – Wir sagen: Es sind die Spitzenmanager und ihre Seilschaften in der Politik, die den größten Schaden anrichten.  Sie wollen den unbequemen Teil der Zivilgesellschaft schwächen. Wir wollen das Gegenteil. Wir lassen nicht zu, dass mächtige Unternehmen wie Daimler, Amazon und Co die Demokratie in Gefahr bringen. Auch deshalb brauchen wir Vereine wie die Deutsche Umwelthilfe, den BUND, Greenpeace, Attac, Robin Wood und all die anderen, die sich nicht einschüchtern lassen. Vielen Dank für eure Arbeit!  

\noindent\textbf{Comment:}
\begin{itemize}
    \setlength\itemsep{-3pt}
    \item (Beifall bei der LINKEN – Dr. Alexander Gauland [AfD]: Die Linke hat ja Erfahrungen im Ruinieren von Ländern!)
    \setlength\itemsep{-3pt}
    \item (Beifall beim BÜNDNIS 90/DIE GRÜNEN)
    \setlength\itemsep{-3pt}
    \item (Armin-Paulus Hampel [AfD]: Genau!)
    \setlength\itemsep{-3pt}
    \item (Dr. Alexander Gauland [AfD]: Ihr wollt das Land ruinieren, wie ihr schon einmal eines ruiniert habt!)
    \setlength\itemsep{-3pt}
    \item (Beifall bei der LINKEN sowie bei Abgeordneten der SPD und des BÜNDNISSES 90/DIE GRÜNEN)
    \setlength\itemsep{-3pt}
    \item (Beifall bei der LINKEN und dem BÜNDNIS 90/DIE GRÜNEN sowie bei Abgeordneten der SPD)
    \setlength\itemsep{-3pt}
    \item (Beifall bei der LINKEN sowie bei Abgeordneten des BÜNDNISSES 90/DIE GRÜNEN – Dr. Alexander Gauland [AfD]: Ruiniert weiter!)
    \setlength\itemsep{-3pt}
    \item (Armin-Paulus Hampel [AfD]: Sie ist gemein, aber nicht nützig!)
    \setlength\itemsep{-3pt}
    \item (Beifall bei der LINKEN sowie bei Abgeordneten der SPD)
    \setlength\itemsep{-3pt}
    \item (Beifall bei der LINKEN)
    \setlength\itemsep{-3pt}
    \item (Michael Grosse-Brömer [CDU/CSU]: Zu Recht!)
\end{itemize}
\subsection{Kotting-Uhl}
\noindent\textbf{Texts:} Sehr geehrter Herr Präsident! Meine Kolleginnen und Kollegen! Kollegen von der AfD, egal was Sie vorlegen: Man fragt sich, ob Sie eigentlich die letzten 60 Jahre im Winterschlaf verbracht haben.  Haben Sie denn gar nichts mitbekommen von notwendig gewordenen Veränderungen, von gesellschaftspolitischen Entwicklungen? Egal ob die Gleichstellung, die Globalisierung, die großen Atomunfälle oder die Klimaveränderung: Alles scheinen Sie verschlafen zu haben. Immer öfter greifen Sie die Grundlagen des Rechtsstaats an, weil Sie sie nicht verstehen.  Das Verbandsklagerecht ist eine großartige zivilgesellschaftliche Errungenschaft, weil es ermöglicht, Rechte der Allgemeinheit einzuklagen. Mit dem 2006 in Kraft getretenen Umwelt-Rechtsbehelfsgesetz wurde dem Erhalt unserer natürlichen Lebensgrundlagen vor Gericht eine Stimme gegen rein wirtschaftliche Interessen gegeben. Was für ein großartiger und dringlicher Fortschritt! Aber Sie verstehen das nicht.  Ihr Gesetzentwurf will klagebefugte Umweltvereinigungen und die darin engagierten Bürgerinnen und Bürger so gängeln, dass eine effektive Arbeit nahezu unmöglich wird. Die Abmahntätigkeit von Verbänden empört Ihr Anstandsgefühl. Dabei lassen Sie aber völlig außer Acht, dass eine Abmahnung ja immer auf rechtswidriges Verhalten folgt. Sie zielen auf die Schwächung des Verbraucherschutzes und solidarisieren sich mit denjenigen, die sich rechtswidrig verhalten.  Sowohl die von Ihnen geforderte Erhöhung der erforderlichen Mitgliederzahl als auch die Aushöhlung der Finanzierung der Umweltverbände haben nur eines im Sinn: die Schwächung der demokratischen Teilhabe.  Wer 60 Jahre verschlafen hat, legt einen Gesetzentwurf wie diesen vor: europarechts- und völkerrechtswidrig; denn sowohl das Europarecht wie auch das Völkerrecht sehen vor, dass das Umweltrecht auch mit Mitteln der gerichtlichen Verfahren effektiv durchzusetzen sein muss.  Verbände wie die DUH setzen sich für die Durchsetzung des Rechts ein – mit Erfolg. Die DUH hat bisher noch keine Klage verloren. Aber für die AfD sagt das nichts über die Berechtigung des Anliegens aus, sondern über die besondere Bösartigkeit des Klägers. Recht und Gesetz spielen für Sie so wenig eine Rolle wie Wissenschaftlichkeit.  Zumindest Letzteres gilt leider auch für andere in diesem Haus. Vielleicht wären ja selbst manchem in der AfD die eigenen Vorlagen peinlich, wenn nicht sogar ein Minister der amtierenden Bundesregierung Ihnen mit seinen verdrehten politischen Maßstäben den Weg bereiten würde.  Wie gefährlich entfernt sich ein Minister von sachlichen Grundlagen, wenn er frohlockt, nun kämen endlich Fakten in die Stickoxiddebatte, allein aufgrund der Einzelmeinung eines wild gewordenen Facharztes, der nie wissenschaftlich publiziert hat,  aber mal eben den gesamten Forschungsstand zu Luftschadstoffen infrage stellt,  dessen vermeintliche Expertise sich nun als von Rechenfehlern und falschen Ausgangswerten gespickt herausstellt.  Wer die Arbeitskraft eines ganzen Ministeriums zur Verfügung hat, der sollte, bevor er die gesamten Grundlagen der Grenzwerte infrage stellt, sein Ministerium erst einmal die Rechnung eines Dr. Köhler überprüfen lassen.  Die gesamte Debatte um die Luftschadstoffe hat völlig den Boden verloren.  Dazu haben Sie von der Union und der FDP kräftig beigetragen. Sie haben ein Klima geschaffen, in dem eine betrügerische Automobilindustrie plötzlich Welpenschutz braucht, in dem man einem Bundesverkehrsminister die Verdrehung von Tatsachen durchgehen lässt und in dem ein Umweltverband, der sich für die Durchsetzung des Rechts einsetzt, an den Pranger gestellt wird.  Gemeinsam mit der AfD sägen Sie an den Grundwerten des Rechtsstaats, wenn Sie einem Umweltverband, der Ihnen zu erfolgreich wird, die Gemeinnützigkeit absprechen wollen, ihn, wie Herr Schweiger es tat, unredlich nennen und ihn, wie Herr Müller-Böhm es tat, gar zum Inquisitor erklären. Bei der AfD habe ich keine Hoffnung,  dass sie aus ihrem demokratischen Tiefschlaf erwacht. Aber Sie von der Union und Sie von der FDP sollten sich darauf besinnen, was die Grundlagen unseres zivilgesellschaftlichen Lebens in Recht und Freiheit sind, und sich nicht mit dem ganz rechten Teil in diesem Haus gemein machen.  

\noindent\textbf{Comment:}
\begin{itemize}
    \setlength\itemsep{-3pt}
    \item (Beifall beim BÜNDNIS 90/DIE GRÜNEN sowie bei Abgeordneten der SPD und der LINKEN – Udo Theodor Hemmelgarn [AfD]: Ich sage nur: Toyota!)
    \setlength\itemsep{-3pt}
    \item (Beifall beim BÜNDNIS 90/DIE GRÜNEN)
    \setlength\itemsep{-3pt}
    \item (Beifall beim BÜNDNIS 90/DIE GRÜNEN sowie bei Abgeordneten der LINKEN – Zuruf des Abg. Dr. Dirk Spaniel [AfD])
    \setlength\itemsep{-3pt}
    \item (Heiterkeit und Beifall beim BÜNDNIS 90/DIE GRÜNEN sowie bei Abgeordneten der SPD)
    \setlength\itemsep{-3pt}
    \item (Dr. Alexander Gauland [AfD]: Gott sei Dank haben Sie keine Hoffnung! Das lässt mich sehr hoffen! – Armin-Paulus Hampel [AfD]: Wir sind die Hoffnung!)
    \setlength\itemsep{-3pt}
    \item (Beifall beim BÜNDNIS 90/DIE GRÜNEN sowie bei Abgeordneten der LINKEN)
    \setlength\itemsep{-3pt}
    \item (Beifall bei der CDU/CSU)
    \setlength\itemsep{-3pt}
    \item (Widerspruch von der AfD – Michael Grosse-­Brömer [CDU/CSU]: Und 100 Kollegen!)
    \setlength\itemsep{-3pt}
    \item (Beifall beim BÜNDNIS 90/DIE GRÜNEN sowie bei Abgeordneten der SPD und der LINKEN – Dr. Alexander Gauland [AfD]: Was Ihnen nicht passt, gehört nicht hinein!)
    \setlength\itemsep{-3pt}
    \item (Beifall beim BÜNDNIS 90/DIE GRÜNEN sowie bei Abgeordneten der SPD)
    \setlength\itemsep{-3pt}
    \item (Beifall beim BÜNDNIS 90/DIE GRÜNEN sowie bei Abgeordneten der LINKEN – Sven-Christian Kindler [BÜNDNIS 90/DIE GRÜNEN]: So sieht es aus!)
    \setlength\itemsep{-3pt}
    \item (Beifall des Abg. Klaus Mindrup [SPD])
    \setlength\itemsep{-3pt}
    \item (Beifall beim BÜNDNIS 90/DIE GRÜNEN – Dr. Alexander Gauland [AfD]: Und Sie schaffen ein Klima, das dieses Land ruiniert! Die Grünen sind eine Gefahr für Deutschland!)
    \setlength\itemsep{-3pt}
    \item (Beifall beim BÜNDNIS 90/DIE GRÜNEN sowie bei Abgeordneten der SPD und der LINKEN)
    \setlength\itemsep{-3pt}
    \item (Dr. Alexander Gauland [AfD]: Hat den Boden endlich erreicht!)
    \setlength\itemsep{-3pt}
    \item (Beifall beim BÜNDNIS 90/DIE GRÜNEN sowie bei Abgeordneten der SPD und der LINKEN – Dr. Alexander Gauland [AfD]: So ein grüner Stuss!)
\end{itemize}
\subsection{Kuffer}
\noindent\textbf{Texts:} Herr Präsident! Liebe Kolleginnen und Kollegen! Es geht darum, wie man dem mehr als zweifelhaften Geschäftsmodell der Deutschen Umwelthilfe die Grundlage entzieht.  Es ist einfach eine rechtstechnische Frage. Es ist schon abstoßend – das muss ich Ihnen wirklich sagen –, mit welchem Vokabular hier in der Debatte zur Lösung einer rechtstechnischen Frage aufgewartet wird.  Es wird – ganz egal, ob von links oder von rechts in diesem Haus – überhaupt kein Beitrag zur Lösung geleistet. Ich erinnere mich an die erste Wortmeldung des Kollegen Spaniel, der sagt: Es reicht nicht, wenn Sie sich medial echauffieren. – Das sagt die Partei, die uns heute wieder einen Gesetzentwurf vorlegt, von dem sie entweder weiß oder zumindest wissen müsste, dass er nicht im Mindesten mit der Rechtsprechung des Europäischen Gerichtshofs in Einklang zu bringen ist. Nicht im Mindesten! Ganz ehrlich: Den Tag, an dem Sie einmal hier vorne etwas vorlegen, was Sie vorher gelesen und studiert haben, streiche ich mir im Kalender an. Ich weiß nicht, ob das in dieser Legislaturperiode noch sein wird. Aber ich bin echt gespannt, ob es irgendwann passieren wird. Der Deutscher Bundestag hat im Sommer 2017, liebe Kolleginnen und Kollegen, die Novellierung des Umwelt-Rechtsbehelfsgesetzes beschlossen. Damit sind wir einer völkerrechtlichen Verpflichtung aus der Århus-Konvention und den daraus abgeleiteten EU-Richtlinien in Deutschland nachgekommen. Wir sind damals der unmissverständlichen Aufforderung des EuGH nachgekommen, die Vorgaben der einzelnen Richtlinien gemäß der Århus-Konvention eins zu eins umzusetzen. Sie können versichert sein, dass die Unionsfraktion damals alle rechtlichen Möglichkeiten zur Planungsbeschleunigung ausgeschöpft hat, die mit dem Europarecht vereinbar waren. Aber da sind die Grenzen halt eng. Man sieht das beim Thema „materielle Präklusion“. Sie wissen, dass uns der Europäische Gerichtshof in seinem Urteil vom 15. Oktober 2015 hier einen Riegel vorgeschoben hat. Das sitzt uns bis heute im Nacken, weshalb wir uns bei der Durchsetzung unserer politischen Wünsche immer wieder limitiert sehen.  Mit Ihrem Gesetzentwurf kommen Sie, liebe Kolleginnen und Kollegen von der AfD, einfach zu spät. Sie wollen uns wieder etwas vorschlagen, was mit höherrangigem Recht schlicht unvereinbar ist. Jetzt einmal zum Inhalt des Gesetzes. Nach Artikel 10a der Öffentlichkeitsbeteiligungsrichtlinie sollen die Mitgliedstaaten der EU das Ziel wahren, der betroffenen Öffentlichkeit einen weiten Zugang zu Gerichten zu gewähren. In Deutschland dürfen Umweltverbände und -organisationen daher klagen, wenn sie a) nach ihrer Satzung nicht nur vorübergehend die Ziele des Umweltschutzes fördern und b) gemeinnützige Zwecke im Sinne von § 52 der Abgabenordnung verfolgen. Letzterer Punkt ist mir besonders wichtig. Die Deutsche Umwelthilfe steht diesbezüglich in der Kritik. Ich sage auch: Sie steht zu Recht in der Kritik. Die CSU und auch die CDU haben auf ihren Parteitagen deshalb beschlossen, eine Prüfung der Gemeinnützigkeit der Deutschen Umwelthilfe anzustoßen.  Gleiches hat die Bundesregierung angekündigt.  Förderungsbewilligung für den Erhalt öffentlicher Mittel bei zweifelhaften Methoden kritisch zu beleuchten, ist legitim. Aber deshalb Europarecht zu brechen, ist kein probates Mittel, und es ist für uns kein probates Mittel, mit dem wir diesem Ziel nachgehen werden. Das kann ich Ihnen ganz deutlich sagen.  Das führt mich zu Ihrem Gesetzentwurf, aus dem ich mal zwei Punkte zitieren will. Erstens. Sie wollen die Zulassung von Verbänden vor Gericht verbieten, wenn diese „Spenden von außerhalb des Geltungsbereiches dieses Gesetzes“ erhalten.  Der Humor ist wirklich speziell. Kehren Sie doch erst einmal vor Ihrer eigenen Haustüre!  Spendengelder aus der Schweiz mit unklarer Identität des Spenders gehören wohl kaum zu einer hinnehmbaren Praxis im Parteienrecht.  Dass sich jemand wie Sie dann über derartige Praktiken echauffiert, ist einfach falsch. Sorry.  Nein, danke.  Zweitens. Den Zugang zu Gerichten wollen Sie Umweltverbänden künftig nur noch gewähren, wenn diese „mindestens eins vom Tausend der wahlberechtigten Einwohner in ihrem Tätigkeitsbereich“ als Mitgliederzahl aufweisen können. In Deutschland könnten demnach Umweltverbände nur noch dann klagen, wenn sie rund 62 000 Mitglieder oder mehr haben.  – Ja, genau. Das zeigt wieder: Sie haben es nicht gelesen. – Gucken Sie sich das EuGH-Urteil vom 15. Oktober 2009 an. Darin werden Sie finden, dass bereits die Vorgabe einer Mindestanzahl von 2 000 Mitgliedern unionsrechtlich problematisch bewertet worden ist. Lesen Sie es halt mal, Himmel noch mal, bevor Sie uns hier dauernd mit solchen Schaufensterdebatten aufhalten.  Wenn Sie diese pauschale Ausgrenzung – es geht nämlich genau darum, die solide arbeitenden Verbände nicht zu treffen – zum Bestandteil Ihrer Gesetzgebung machen wollen, dann führen Sie damit die Öffentlichkeitsbeteiligung als elementares Gut unseres Rechtsstaats ad absurdum. Deshalb sage ich Ihnen zum Schluss: Es bleibt dabei: Wir lehnen Ihren Entwurf ab. Wir halten uns an Europa und an Völkerrecht.  Wir machen sozialverträgliche, ökologisch und ökonomisch sinnvolle Klimaschutzpolitik, bei der sich die Bürger weiterhin beteiligen sollen.  Vielen Dank. 

\noindent\textbf{Comment:}
\begin{itemize}
    \setlength\itemsep{-3pt}
    \item (Lachen der Abg. Sylvia Kotting-Uhl [BÜNDNIS 90/DIE GRÜNEN] – Oliver Krischer [BÜNDNIS 90/DIE GRÜNEN]: Die AfD verbieten! – Zuruf des Abg. Armin-Paulus Hampel [AfD])
    \setlength\itemsep{-3pt}
    \item (Beifall der Abg. Marie-Luise Dött [CDU/CSU] – Sylvia Kotting-Uhl [BÜNDNIS 90/DIE GRÜNEN]: Das ist der erste Schritt! Und der zweite kommt!)
    \setlength\itemsep{-3pt}
    \item (Beifall bei Abgeordneten der CDU/CSU und der SPD)
    \setlength\itemsep{-3pt}
    \item (Armin-Paulus Hampel [AfD]: CDU und Spenden! – Dr. Bernd Baumann [AfD]: CDU-Spendenaffäre! – Weitere Zurufe von der AfD)
    \setlength\itemsep{-3pt}
    \item (Karsten Hilse [AfD]: Sie halten lieber Sonntagsreden!)
    \setlength\itemsep{-3pt}
    \item (Dr. Bernd Baumann [AfD]: Angekündigt? Die hat den Antrag schon vorliegen!)
    \setlength\itemsep{-3pt}
    \item (Beifall bei Abgeordneten der CDU/CSU – Dr. Alexander Gauland [AfD]: Das ist der Unterschied zwischen uns und euch! Das ist ja das Dilemma!)
    \setlength\itemsep{-3pt}
    \item (Beifall bei Abgeordneten der CDU/CSU und der AfD – Oliver Krischer [BÜNDNIS 90/DIE GRÜNEN]: Genau das ist es! Dass Sie jetzt darüber reden!)
    \setlength\itemsep{-3pt}
    \item (Beifall bei Abgeordneten der AfD – Steffi Lemke [BÜNDNIS 90/DIE GRÜNEN]: Wo ist Herr Scheuer eigentlich?)
    \setlength\itemsep{-3pt}
    \item (Beifall bei der CDU/CSU sowie bei Abgeordneten der SPD)
    \setlength\itemsep{-3pt}
    \item (Dr. Alexander Gauland [AfD]: Ich würde auch keine zulassen bei Spenden!)
    \setlength\itemsep{-3pt}
    \item (Beifall bei Abgeordneten der CDU/CSU und der SPD – Klaus Mindrup [SPD]: Und verstehen!)
    \setlength\itemsep{-3pt}
    \item (Dr. Dirk Spaniel [AfD]: Ja!)
    \setlength\itemsep{-3pt}
    \item (Karsten Hilse [AfD]: Weiter verarscht fühlen! Ganz genau!)
    \setlength\itemsep{-3pt}
    \item (Beifall bei der CDU/CSU – Dr. Alexander Gauland [AfD]: Das heißt, es ändert sich nichts! Wie immer bei der CDU!)
    \setlength\itemsep{-3pt}
    \item (Beifall der Abg. Marie-Luise Dött [CDU/CSU])
\end{itemize}
\section{Tagesordnungspunkt 19}
\subsection{Zimmermann}
\noindent\textbf{Texts:} Verehrter Herr Präsident! Liebe Kolleginnen und Kollegen! Die Dramatik des Pflegenotstands kann man daran erkennen, dass selbst Bundesminister Spahn von einer notwendigen Grundsatzdebatte über die Finanzierung der Pflege spricht und Herr Lauterbach wieder mal die Bürgerversicherung für die Pflege fordert, obwohl wir doch gar nicht im Wahlkampf sind.  Wir haben Konzepte vorgelegt. Wir wollen, dass schnell gehandelt und nicht nur geredet wird.  Deshalb schlagen wir einen ersten Schritt vor, um die Pflege besser zu finanzieren, einen ersten Schritt, um mehr Geld für gute Pflege bereitzustellen, ohne dass die Versicherten weiter belastet werden,  einen ersten Schritt zu wirklicher Generationengerechtigkeit, damit die jungen, gesünderen Menschen in der privaten Pflegeversicherung nicht nur für sich vorsorgen, einen ersten Schritt zu wirklicher Solidarität, damit alle Besserverdienenden gute Pflege für alle mitfinanzieren.  Es ist doch absurd: Die Rücklagen der gesetzlichen Pflegeversicherung reichen nicht mal drei Monate. Die Rücklagen der privaten Pflegeversicherung von 34 Milliarden Euro reichen fast 30 Jahre. Das ist doch wirklich absurd, meine Damen und Herren.  Und warum ist das so? Weil Junge, Gesunde, meist Besserverdienende, die noch keine Pflege brauchen, in den privaten Versicherungen sind. Das ist ungerecht, und das wollen wir ändern.  Um diese Systemfrage geht es. Das zeigt auch der Antrag der FDP. Sie bleiben sich treu, verehrte Kolleginnen und Kollegen: mehr private Vorsorge zum Wohle der Versicherungen, mehr Wettbewerb sogar zwischen den privaten Pflegeversicherungen  und noch mehr Pflegevorsorgefonds, die Versicherungsbeiträge am Finanzmarkt parken und so der Versorgung entziehen – als hätte der private Markt mit dem Pflegenotstand nichts zu tun, als würde noch mehr Wettbewerb die Pflege besser machen. Das ist so absurd. Das will ich an dieser Stelle gar nicht weiter kommentieren.  Meine Damen und Herren, Die Linke fordert konsequent einen Paradigmenwechsel. Wir wollen gute Pflege für alle solidarisch finanzieren. Das heißt, jeder und jede zahlt den gleichen Anteil vom gesamten Einkommen, im Übrigen auch von Kapitaleinnahmen, in die Pflegeversicherung ein, und zwar ohne Beitragsbemessungsgrenze. Die Pflegeversicherung finanziert alle Leistungen, die für eine gute und bedarfsgerechte Pflege notwendig sind. Als Einstieg in diese Pflegevollversicherung werden die Eigenanteile sofort gedeckelt.  Gute, bedarfsgerechte Pflegeangebote müssen für alle Menschen verfügbar und bezahlbar sein. Niemand wird bei der Versorgung bevorzugt oder benachteiligt. Deshalb wird die private in die soziale Pflegeversicherung überführt. Liebe Kolleginnen und Kollegen der SPD, ich würde vorschlagen, Sie folgen einfach mal Ihrem Gewissen und nicht dem Koalitionszwang. Sie haben bereits mehr Solidarität angekündigt; jetzt können Sie Ihr Versprechen einlösen. Wir als Linke haben tatsächlich nicht den Ehrgeiz, die einzige Partei zu bleiben, die sich wirklich für soziale Gerechtigkeit einsetzt.  Auch viele Pflegekräfte wollen sofortige Veränderungen. Nicht nur in Niedersachsen kämpfen sie gemeinsam und auch mit uns Linken für gute Arbeitsbedingungen und gute Löhne. Die Pflegekräfte haben die Nase voll von Entscheidungen über ihren Kopf hinweg. Sie wollen endlich mitbestimmen und nicht nur mit verwalten.  Das werden sie auch tun. Der Protest auf der Straße lässt sich nicht mehr aufhalten. Ich freue mich darüber sehr. Der Protest auf der Straße, wie er jetzt von Niedersachsen ausgeht, ist das Salz und ist der Pfeffer in der Suppe für unsere Arbeit hier im Deutschen Bundestag.  Meine Damen und Herren, glauben Sie als Regierung wirklich, dass Ihre halbherzigen arbeitgeberfreundlichen Reformen den jetzigen Zustand ändern werden? Haben Sie Konzepte tatsächlich für alle Pflegebereiche, zum Beispiel für die ambulante Pflege? Dort wird die tarifliche Bezahlung der Pflegekräfte oft nicht durch die Kassen refinanziert. Es reicht nicht, tarifliche Bezahlung ins Gesetz zu schreiben. Es muss auch gesichert sein, dass tatsächlich alle 109 Krankenkassen den Tariflohn gleichermaßen akzeptieren und bezahlen.  Sie verabschieden ein Gesetz für 13 000 zusätzliche Pflegekräfte in Pflegeheimen. Sie könnten auch 30 000 oder 300 000 zusätzliche Stellen versprechen. Das bleiben leere Worte, wenn Sie keine Finanzgrundlage schaffen, um Pflegekräfte sofort besser zu bezahlen, damit niemand mehr aus dem Beruf flieht. Meine Damen und Herren, wenn Sie nicht den Ausverkauf der Pflegeeinrichtungen an Renditejäger unterbinden, bleiben die Menschen mit Pflegebedarf und die Pflegekräfte Spielball im internationalen Finanzmarkt. Pflegekräfte sind keine billigen Arbeitskräfte, die ihren Beruf aus Dankbarkeit ausüben und mit einer Schachtel Pralinen abgespeist werden können. Menschen mit Pflegebedarf haben ein Recht darauf, professionell und mit Würde versorgt zu werden, egal wo sie wohnen und egal wie dick ihr Geldbeutel ist. Angehörige und Familien müssen sich darauf verlassen können, dass sie überall und jederzeit die Unterstützung bekommen, die sie benötigen.  Die Überführung der privaten Pflegeversicherung in die soziale Pflegeversicherung wäre ein Anfang. Damit entsteht eine nachhaltige Finanzierungsgrundlage. Weitere Schritte müssen natürlich folgen. Vielen Dank für Ihre Aufmerksamkeit.  

\noindent\textbf{Comment:}
\begin{itemize}
    \setlength\itemsep{-3pt}
    \item (Beifall bei der LINKEN sowie der Abg. Kordula Schulz-Asche [BÜNDNIS 90/DIE GRÜNEN])
    \setlength\itemsep{-3pt}
    \item (Nicole Westig [FDP]: Weil es nachhaltig ist!)
    \setlength\itemsep{-3pt}
    \item (Tino Sorge [CDU/CSU]: Ist doch nichts Neues!)
    \setlength\itemsep{-3pt}
    \item (Beifall bei der CDU/CSU)
    \setlength\itemsep{-3pt}
    \item (Erich Irlstorfer [CDU/CSU]: Reden Sie doch nicht einen solchen Propaganda-Unsinn!)
    \setlength\itemsep{-3pt}
    \item (Beifall bei der LINKEN)
\end{itemize}
\subsection{Rüddel}
\noindent\textbf{Texts:} Sehr geehrter Herr Präsident! Sehr geehrte Damen und Herren! Die Linke bleibt ihrer Linie treu,  obwohl das in der Pflegepolitik zu keinem Ergebnis führt. In der letzten Legislaturperiode wollte man die private Pflegezusatzversicherung abschaffen, dann die private Pflegeversicherung. Jetzt will man wieder die private Pflegeversicherung abschaffen und gleichzeitig die Rücklagen einkassieren. Das ist verfassungswidrig. Bevor wir uns Enteignungsfantasien hingeben, sollten wir konkret über Aufgaben reden, die sich für die Pflege stellen und die wir dann im Sinne der Betroffenen lösen sollten.  Wir müssen Strukturen angehen. Hier sehe ich vier Schwerpunkte: Personal, Digitalisierung, Eigenanteile und Sektorengrenzen. Zum Personal. Mit Beginn des Jahres haben wir ein Paket in Kraft gesetzt für mehr Stellen, attraktive Arbeitsbedingungen und Hilfen bei der Betreuung zu Hause. Ermöglicht werden damit 13 000 zusätzliche Stellen in der stationären Altenpflege. Sie werden über Behandlungspflege und damit über die Krankenversicherung kostenneutral für die Pflegebedürftigen finanziert. Jetzt sind die Träger und – das spreche ich explizit an – die Kassen gefordert, das konsequent und unbürokratisch umzusetzen.  Denn Dreh- und Angelpunkt für eine menschenwürdige Pflege sind und bleiben möglichst viele qualifizierte Fachkräfte. Die Ausbildungszahlen entwickeln sich Gott sei Dank sehr positiv. Wir bauen zudem auf Wiedereinstiegsprogramme und viele Rückkehrer aus Teilzeit in Vollzeit. Wir müssen faire Löhne zahlen und alles daransetzen, den Pflegeberuf durch den Abbau überflüssiger Bürokratie und mehr gesellschaftliche Anerkennung attraktiver zu machen.  Zur Digitalisierung. Bei der Digitalisierung setzen wir auf altersgerechte Assistenzsysteme und E-Health-Lösungen. Sowohl in der häuslichen Umgebung wie auch in den Pflegeheimen können digitale Innovationen die Pflegekräfte wirklich entlasten, damit sie mehr Zeit für Zuwendung für ihre Patienten haben. Wir müssen ferner Synergieeffekte im System besser nutzen; denn auch die medizinische Versorgung in der Fläche wird morgen anders aussehen als heute. Konsequente Digitalisierung und Vernetzung, Prozesssteuerung und technische Assistenz werden hohe Priorität haben. Die soziale Pflegeversicherung ist keine Vollkaskoversicherung. Deshalb ist – unabhängig von familiärer Unterstützung – Eigenverantwortung unverändert gefordert.  Wir brauchen Neuanreize zur privaten Vorsorge, und wir sollten auch Projekte zur betrieblichen Pflegevorsorge fördern. Da gehe ich mit dem Antrag der FDP einher.  Zu den Eigenanteilen. Wir müssen die systembedingten Steigerungen der Eigenanteile für stationär versorgte Pflegebedürftige eingrenzen. In Kliniken wird die Pflege voll über die Krankenversicherung finanziert; im Heim gibt es einen Festbetrag. Die allseits erwünschte bessere Bezahlung von deutlich mehr Pflegekräften in der Altenpflege erhöht deshalb die individuelle Belastung der Pflegebedürftigen. Hier besteht Handlungsbedarf. Wir müssen uns Gedanken machen, wie wir die dynamische Steigerung der Eigenanteile für die Pflegebedürftigen einbremsen. Der Bundesgesundheitsminister hat in diesem Kontext kürzlich die Notwendigkeit neuer Finanzierungsmodelle betont. Dem stimme ich ausdrücklich zu.  Schließlich die Sektorengrenzen. Nach meiner Überzeugung sollte künftig ausschließlich die Qualität der Versorgung ausschlaggebend sein und nicht der Ort der Pflege. Deshalb sollten wir Sektorengrenzen abbauen und Unterschiede in der Versorgung einebnen. Innovative Versorgungsformen könnten auch die Angehörigenpflege in alle Versorgungskonzepte einbinden und es damit den Familien ermöglichen, sich bei hauswirtschaftlichen Tätigkeiten nicht nur im ambulanten Bereich, sondern auch im stationären Bereich einzubringen, um die Kosten zu begrenzen. Was wir brauchen, sind viele kreative Ideen für gute Pflege und keine Ideologie.  

\noindent\textbf{Comment:}
\begin{itemize}
    \setlength\itemsep{-3pt}
    \item (Beifall bei der AfD)
    \setlength\itemsep{-3pt}
    \item (Beifall bei der CDU/CSU und der FDP)
    \setlength\itemsep{-3pt}
    \item (Beifall bei der FDP sowie bei Abgeordneten der CDU/CSU)
    \setlength\itemsep{-3pt}
    \item (Beifall bei der CDU/CSU)
    \setlength\itemsep{-3pt}
    \item (Beifall bei Abgeordneten der CDU/CSU sowie der Abg. Christine Aschenberg-Dugnus [FDP])
    \setlength\itemsep{-3pt}
    \item (Beifall bei der CDU/CSU sowie des Abg. Dr. Andrew Ullmann [FDP])
    \setlength\itemsep{-3pt}
    \item (Beifall der Abg. Karin Maag [CDU/CSU])
    \setlength\itemsep{-3pt}
    \item (Beifall bei der LINKEN)
\end{itemize}
\subsection{Schneider}
\noindent\textbf{Texts:} Herr Präsident! Meine sehr geehrten Damen und Herren! Liebe Zuschauer! Mein Kollege Paul Podolay wird sich gleich mit dem Antrag der Linken auseinandersetzen. Ich werde mich mit dem Antrag der FDP befassen. Ein Kritikpunkt ist: Wenn Sie Transparenz in der Pflege fordern, dann sollten Sie vielleicht auch eine etwas kritischere Bestandsaufnahme dessen machen, was die Regierung bisher getan hat. Wenn man das tut, kommt man nämlich ziemlich schnell zu dem Ergebnis, dass es bei der Pflege im Zentrum eigentlich nicht um die finanziellen Fragen geht, sondern dass es andere Fragen gibt, die zu beantworten sind, vor allen Dingen zum Thema Personal. Ich mache jetzt mal die Bestandsaufnahme für Sie: Im Koalitionsvertrag war vorgesehen, dass die Bezahlung der Pflegekräfte verbessert wird. Jetzt haben Sie festgestellt, dass das im deutschen Tarifsystem ein bisschen schwierig ist. Tatsächliche Verbesserungen – bislang Fehlanzeige! Sie haben dort 13 000 zusätzliche Stellen schaffen wollen. Wenn man es umrechnet, ist das für eine Pflegeeinrichtung mit 40 Beschäftigten eine halbe Stelle.  Hinzu kommt, dass die meisten neuen Stellen bisher wohl eher auf dem Papier bestehen dürften. Das heißt auch hier: Tatsächliche Verbesserungen – Fehlanzeige! Sie wollten die Attraktivität des Pflegeberufs verbessern und haben dazu Ausbildungsgänge zusammengefasst. Aus mehreren Ausbildungsgängen wurde jetzt ein Pflegeberuf. Dadurch wird die Ausbildung schwieriger, vielleicht für den einen oder anderen Bewerber etwas zu schwierig. Ob man damit tatsächlich den Pflegeberuf attraktiver gestaltet, versehe ich zumindest mit einem großen Fragezeichen. Sie sehen schon: Das Thema Pflege hat sicherlich viel mit Finanzen zu tun; aber es gibt andere wichtige Aspekte. Vor allen Dingen gibt es halt Personalknappheit. Da könnte natürlich eine Lösung sein, dass man mehr Angehörige motiviert, zu Hause die Pflege ihrer Angehörigen wahrzunehmen.  Das wäre in der Tat dann auch finanziell zu flankieren, und dazu finde ich im FDP-Antrag überhaupt nichts, obwohl das sehr viel mit Finanzen zu tun hat. Im Koalitionsvertrag finde ich dazu einiges. Ich mache also mal weiter mit der Bestandsaufnahme: Sie wollten die Situation der pflegenden Angehörigen verbessern. Nun, es gab da leichte Verbesserungen im Bereich der Rehamaßnahmen, es gab eine Verbesserung im Bereich der Brückenteilzeit. Da sind Sie sicherlich auf dem richtigen Weg. Andererseits wollten Sie die pflegenden Angehörigen auch bei der Rente besserstellen. Das soll jetzt ein Stück weit über die Grundrente erfolgen. Aber ich habe doch erhebliche Bedenken, ob das ausreichend ist. In Summe stelle ich fest, dass der Antrag der FDP durch seine Fixierung auf die Finanzierung an den tatsächlichen Problemen weitgehend vorbeigeht, vor allen Dingen, wenn ich mir anschaue, was Sie da machen wollen: Sie wollen umwidmen. Sie wollen Geld, das eigentlich für die Altersversorgung gedacht ist, für die Pflege umwidmen. Das bedeutet aber doch letztendlich, irgendwo Löcher zu stopfen, indem man an anderer Stelle schon vorhandene Löcher noch größer macht oder neue erzeugt. Das ist meiner Meinung nach keine zielführende Idee, die Sie da angeboten haben. Wir werden Ihren Vorschlag im Ausschuss diskutieren. Ich meine, wir sollten auf jeden Fall die finanziellen Aspekte im Zusammenhang mit pflegenden Angehörigen hinzufügen. Insofern stimmen wir der Überweisung in den Ausschuss zu und freuen uns auf die Diskussionen dort. Ich danke Ihnen.  

\noindent\textbf{Comment:}
\begin{itemize}
    \setlength\itemsep{-3pt}
    \item (Beifall bei der SPD sowie bei Abgeordneten der CDU/CSU)
    \setlength\itemsep{-3pt}
    \item (Beifall bei der AfD)
    \setlength\itemsep{-3pt}
    \item (Harald Weinberg [DIE LINKE]: Oh, der große Pflegedienst!)
    \setlength\itemsep{-3pt}
    \item (Zuruf von der SPD: Besser als nichts!)
\end{itemize}
\subsection{Baehrens}
\noindent\textbf{Texts:} Herr Präsident! Sehr geehrte Damen und Herren! Liebe Kolleginnen und Kollegen! Gegensätzlicher könnten die Anträge ja nicht sein: Während Die Linke fordert, die Milliardenrücklagen der privaten Pflegeversicherung zu nutzen, um die soziale Pflegeversicherung zu stärken, fordert die FDP unter der Überschrift „Mehr Ehrlichkeit in der Pflegedebatte“ quasi ein Konjunkturprogramm für die private Versicherungswirtschaft.  Spannende Debatten zeichnen sich ab. Meine Sympathie liegt dort, wo wir ganz konkret die Pflege heute verbessern können. Ehrlichkeit in der Pflegedebatte, liebe Kolleginnen und Kollegen von der FDP, brauchen wir tatsächlich, aber in einem anderen Sinne, als Sie es uns mit Ihrem Antrag nahelegen.  Ich erinnere an die letzte Pflegedebatte und an Ihre, Frau Westig, falsche Behauptung, mit dem Pflegepersonal-Stärkungsgesetz sei nichts für die ambulante Pflege getan worden. Das war unehrlich; denn das Personal in der ambulanten Pflege haben wir gestärkt, indem wir dafür gesorgt haben, dass endlich Tarifgehälter auch in der häuslichen Krankenpflege anerkannt werden.  Nun fordern gerade Sie Ehrlichkeit in der Debatte. Sie schreiben in Ihrem Antrag – ich zitiere –: Die private Pflegepflichtversicherung zeigt … aufgrund ihres kapitalgedeckten Finanzierungssystems mit Bildung von Altersrückstellungen, dass Generationengerechtigkeit auch in der Pflege funktionieren kann. Was für ein Unsinn, noch dazu in Zeiten von Niedrigzinsen!  Mit ein paar wenigen Kennzahlen möchte ich Sie gern zu mehr Ehrlichkeit in der Pflegedebatte ermuntern: Gute 9 Millionen Versicherte in der privaten Pflegeversicherung haben über 34 Milliarden Euro Rücklagen angesammelt, Geld also, das nicht für die Verbesserung der Pflege heute eingesetzt wird. Und woran liegt das? Das beitragspflichtige Einkommen der Privatversicherten liegt um 60 Prozent über dem der gesetzlich Versicherten. Da entzieht sich eine ganz große Gruppe der Solidarität in unserer Gesellschaft.  Und während in der normalen Pflegeversicherung rund 350 Euro je Versicherten im Jahr ausgegeben werden, belaufen sich die Leistungsausgaben der privaten Pflegeversicherung lediglich auf 90 Euro pro Jahr. Selbst wenn man die Beihilfeleistungen für Beamte mitrechnet, erhalten die gesetzlich Versicherten immer noch etwa dreimal mehr Leistungen als die Privatversicherten. – Die private Pflegeversicherung hat Versicherte mit wesentlich höherem Einkommen und einem wesentlich geringeren Krankheits- und Pflegerisiko. Während also 10 Prozent der Versicherten Geld auf die hohe Kante legen, sollen sich 90 Prozent der Versicherten um ihre Pflege Sorgen machen? Das kann nicht gerecht sein.  Wer es mit der Generationengerechtigkeit in der Pflege ehrlich meint und wer eine ehrliche Pflegedebatte will, der muss sich klar dazu bekennen: Alle Menschen haben ein Recht auf würdevolle Pflege, egal welchen sozialen, beruflichen, finanziellen Hintergrund sie haben. – Darum muss die soziale Pflegeversicherung auf eine solide Grundlage gestellt werden.  Die FDP schreibt es in ihrem Antrag selbst: Gesetzliche und private Pflegeversicherung haben gleiche Leistungen, gleiche Zugangsvoraussetzungen. Darum sollten Sie den nächsten Denkschritt auch noch gehen und sich dafür aussprechen, das Doppelsystem in der Pflegeversicherung abzuschaffen.  Wir brauchen eine gesetzliche Pflegeversicherung, in die alle Bürgerinnen und Bürger einzahlen und die so leistungsfähig ist, dass niemand mehr Angst haben muss, mit den Kosten für die Pflege überfordert zu werden. Wie war das noch mit dem letzten Versuchsballon der FDP zur privaten Pflegevorsorge? Und wohin wechselte der Kurzzeit-Gesundheitsminister der FDP? Natürlich zu einem der größten Anbieter der Pflege-Bahr-Policen. Nein, auf solche Klientelpolitik für Versicherer werden wir als SPD uns nicht einlassen. Ja, wir brauchen eine ehrliche Pflegedebatte. Ehrlich ist es, zu sagen: Gute Pflege kostet. Eine gute Pflege für alle Menschen in unserer Gesellschaft können wir uns nur gemeinsam und solidarisch leisten. Vielen Dank.  

\noindent\textbf{Comment:}
\begin{itemize}
    \setlength\itemsep{-3pt}
    \item (Beifall bei der FDP)
    \setlength\itemsep{-3pt}
    \item (Beifall bei der SPD und dem BÜNDNIS 90/DIE GRÜNEN)
    \setlength\itemsep{-3pt}
    \item (Beifall bei Abgeordneten der SPD sowie der Abg. Kordula Schulz-Asche [BÜNDNIS 90/DIE GRÜNEN])
    \setlength\itemsep{-3pt}
    \item (Beifall bei Abgeordneten der SPD und der CDU/CSU)
    \setlength\itemsep{-3pt}
    \item (Beifall bei der SPD sowie bei Abgeordneten des BÜNDNISSES 90/DIE GRÜNEN und der Abg. Pia Zimmermann [DIE LINKE])
    \setlength\itemsep{-3pt}
    \item (Beifall bei Abgeordneten der SPD und des BÜNDNISSES 90/DIE GRÜNEN)
    \setlength\itemsep{-3pt}
    \item (Beifall der Abg. Emmi Zeulner [CDU/CSU] und Kordula Schulz-Asche [BÜNDNIS 90/DIE GRÜNEN])
    \setlength\itemsep{-3pt}
    \item (Beifall bei der LINKEN sowie bei Abgeordneten der SPD)
    \setlength\itemsep{-3pt}
    \item (Beifall bei der SPD sowie bei Abgeordneten des BÜNDNISSES 90/DIE GRÜNEN)
\end{itemize}
\subsection{Westig}
\noindent\textbf{Texts:} Herr Präsident! Liebe Kolleginnen und Kollegen! Liebe Frau Zimmermann, wenn Sie unseren Antrag absurd finden, dann haben wir wohl etwas richtig gemacht.  Herr Schneider, Sie haben vielleicht nicht zugehört, aber wir haben in den vergangenen Monaten viele Debatten über die Pflege geführt, und ich habe eigentlich immer eine kritische Bestandsaufnahme gemacht. Hier war die Finanzierung das Thema. Deswegen haben wir das einmal etwas aufgefächert. Allerdings bin ich sehr gespannt auf die Vorschläge der AfD, darauf, wie Sie zu besserer Pflege beitragen wollen.  Ich habe zum Beispiel einen Vorschlag zur häuslichen Pflege, über den Sie einmal nachdenken können: Die häusliche Pflege wird ja in der Regel, zu einem ganz großen Anteil, von Frauen erledigt. Sie könnten ja ein Programm vorschlagen, wie man auch Männer in die häusliche Pflege bringen könnte.  Darauf wäre ich sehr gespannt.  Frau Baehrens, zum Pflegepersonal-Stärkungsgesetz. Es gibt Ängste in der ambulanten Pflege. Ich habe das neulich noch bei mir vor Ort auf der Kommunalen Konferenz Alter und Pflege erfahren können. Sie finden keine Leute, und durch das Pflegepersonal-Stärkungsgesetz und die Vorschriften im Zusammenhang mit den Krankenhäusern wird diese Sorge noch verstärkt.  Von daher ist dieses Gesetz für die ambulante Pflege wirklich ein Problem.  Ich rede auch gerne über Möglichkeiten, die Pflege heute zu verbessern, aber ich stehe hier nicht, um Politik nur für diese Legislaturperiode zu machen. Ich möchte auch Politik für nachfolgende Generationen machen,  für die Generationen, die jetzt nicht auf die Straße gehen können, weil sie noch gar nicht geboren sind. Ich habe den Anspruch, auch Politik für morgen zu machen, nicht nur für heute. Deswegen finde ich es sehr wichtig, dass wir diese Diskussion führen und die schrumpfende junge Generation nicht länger gegen die wachsende alte Generation ausspielen.  Die Pflegeversicherung ist bewusst als Teilkasko angelegt. Die aktuelle Pflegepolitik wiegt die Menschen aber in der Illusion, sie sei eine Vollkaskoversicherung. Dadurch, dass die Pflegebeiträge ständig erhöht werden, glauben die Menschen: Ich zahle mehr Beitrag, also bekomme ich eine gute Pflege. – Nein, ohne Eigenvorsorge ist keine gute Pflege leistbar.  Wenn Sie die soziale und die private Pflegeversicherung zusammenführen, wird es vielleicht für einen kleinen Moment besser, aber doch nicht langfristig.  Der Minister hat vor einiger Zeit gesagt, er wolle eine Grundsatzdebatte darüber führen, wie wir die Pflege künftig generationengerecht finanzieren. Heute ist er leider mal wieder nicht anwesend.  Wir haben vor einiger Zeit bereits den ersten Aufschlag gemacht und einen Entschließungsantrag vorgelegt. Jetzt machen wir den nächsten Schritt. Wir haben an das Ministerium eine Kleine Anfrage gestellt. Dort wollte man allerdings nichts von Generationengerechtigkeit wissen. Es ging beispielsweise darum, was man im Zusammenhang mit dem Pflegevorsorgefonds machen kann, damit dort keine Negativzinsen mehr anfallen. Hier ist nichts in der Planung. Frau Weiss, vielleicht können Sie das, was der Minister in Interviews sagt, nämlich dass er darüber sprechen will, und das, was das Ministerium macht, ein wenig zusammenführen. Wir sind jedenfalls bereit, diese Debatte zu führen; wir haben einen Aufschlag gemacht. Ich freue mich auf die Diskussion. Denken wir auch an morgen!   

\noindent\textbf{Comment:}
\begin{itemize}
    \setlength\itemsep{-3pt}
    \item (Pia Zimmermann [DIE LINKE]: Indem Frauen genauso viel verdienen wie Männer zum Beispiel!)
    \setlength\itemsep{-3pt}
    \item (Beifall bei der FDP)
    \setlength\itemsep{-3pt}
    \item (Beifall bei der FDP sowie bei Abgeordneten der CDU/CSU – Beatrix von Storch [AfD]: Für Diverse, auch in der Pflege! – Gegenruf des Abg. Reinhard Houben [FDP]: Das ist Ihr Niveau bei den Zwischenrufen! – Weiterer Zuruf von der AfD: Dafür haben wir Gleichstellungsbeauftragte!)
    \setlength\itemsep{-3pt}
    \item (Beifall beim BÜNDNIS 90/DIE GRÜNEN)
    \setlength\itemsep{-3pt}
    \item (Niema Movassat [DIE LINKE]: Die haben keine Vorschläge!)
    \setlength\itemsep{-3pt}
    \item (Harald Weinberg [DIE LINKE]: Manche!)
    \setlength\itemsep{-3pt}
    \item (Claudia Moll [SPD]: Stimmt nicht!)
    \setlength\itemsep{-3pt}
    \item (Erich Irlstorfer [CDU/CSU]: „Mal wieder“ ist nicht angebracht!)
    \setlength\itemsep{-3pt}
    \item (Beifall bei der FDP – Marianne Schieder [SPD]: Was ist denn die Alternative?)
    \setlength\itemsep{-3pt}
    \item (Beifall bei der FDP – Harald Weinberg [DIE LINKE]: Schwachsinn!)
\end{itemize}
\subsection{Schulz-Asche}
\noindent\textbf{Texts:} Herr Präsident! Meine Damen und Herren! Nein, Herr Präsident, Sie sind nicht doof. Das kann man wirklich nicht behaupten. Sie sind ein guter Präsident.  Meine sehr geehrten Damen und Herren, wenn wir verhindern wollen, dass sich der Pflegenotstand in eine Pflegekatastrophe verwandelt, dann müssen wir die Probleme in unserer Gesellschaft endlich ehrlich benennen –  konsequent, ohne Ideologie und vor allem im Interesse der Menschen, die älter werden, ihrer Familien und ihrer Versorgungssicherheit.  Der demografische Wandel führt auf der einen Seite zu einer steigenden Zahl von pflegebedürftigen Menschen und auf der anderen Seite dazu, dass wir für diese auch zusätzliche Pflegefachkräfte brauchen. Gleichzeitig sehen wir, dass die ausgebildeten Fachkräfte der geburtenstarken Jahrgänge in Rente gehen oder bereits aus dem Beruf ausgestiegen sind. Die Folge ist, dass immer mehr Familien mit ihren pflegebedürftigen Angehörigen alleingelassen oder finanziell überfordert werden. Pflegedienste kündigen bereits laufende Verträge, Pflegeheime legen Betten still; an allen Ecken und Enden fehlt es an Personal. Das gilt übrigens nicht nur für die Altenpflege, sondern zum Beispiel auch für die Versorgung schwerstkranker Kinder. Warum haben wir einen Pflegenotstand? Weil man in Deutschland sehr lange der Meinung war, dass Pflegefachkräfte am unteren Ende der Gehaltsskala gehalten werden können, nach dem Motto: Pflegen kann eigentlich jeder. – Dabei ist professionelle Altenpflege genau das Gegenteil. Langzeitpflege ist ein Beruf der Zukunft mit hohen Ansprüchen an Qualifizierung und Qualität.  Es ist sicher einer der größten Fehler gewesen, dass diese Koalition das bei der Ausbildungsreform im letzten Jahr wieder bewusst nicht umgesetzt hat.  Aber wenn wir es nicht schaffen, die Altenpflege zu einem attraktiven, selbstverantwortlichen und selbstbewussten Beruf zu machen – das gilt übrigens auch für Niedersachsen –, dann werden wir nicht genug junge Menschen finden, die bereit sind, in der Pflege zu arbeiten. Und das ist eine Katastrophe.  Wir haben heute auch deswegen einen Pflegenotstand, weil die Bereitschaft, die Pflegeversicherung nach ihrer Einführung den wachsenden Herausforderungen anzupassen, bei den nachfolgenden Regierungen nie vorhanden war. Am deutlichsten wird das daran, dass der Kostenanteil, den die Pflegebedürftigen heute selber zur Pflege beitragen, in den letzten Jahren immer weiter angestiegen ist, was dazu führt, dass Familien tatsächlich finanzielle Probleme bekommen. Die Anbeter der privaten Vorsorge hatten mit der FDP in der vorletzten Legislaturperiode ja den sogenannten Pflege-Bahr eingeführt. Wenn Sie sich die Zahlen angucken, sehen sie: Das war ein absoluter Flop.  Deshalb stellt sich natürlich die Frage, warum die FDP heute wieder mit der alten Leier von noch mehr kapitalgestützter Finanzierung in der Pflegeversicherung um die Ecke kommt – und dies auch noch in Zeiten von Niedrig- und Negativzinsen. Warum lernen Sie nicht endlich aus Ihren ideologischen Fehlern der Vergangenheit?  Auch die von Ihnen gewünschte Stärkung der privaten Pflegeversicherung ist genau das Gegenteil von Generationengerechtigkeit. Privatversicherte erhalten heute im Rahmen des Sozialgesetzbuchs XI die gleichen Leistungen wie diejenigen, die in der sozialen Pflegeversicherung versichert sind. Aber die niedrigen Prämien der privaten Pflegeversicherung locken junge, gesunde und einkommensstarke Menschen an. Gerade deshalb befindet sich die private Pflegeversicherung derzeit vor einer dramatischen demografischen Herausforderung. Die Zahl der Pflegebedürftigen in der Privatversicherung wird bis zum Jahr 2060 um über 280 Prozent steigen. Und jetzt sagen Sie mir bitte, was Sie glauben, wie dann trotz aller Rücklagen die Versicherungsprämien noch gezahlt werden können und wer diese Prämien überhaupt noch zahlen kann. Das ist keine Generationengerechtigkeit, sondern das genaue Gegenteil.  Ganz kurz zum Antrag der Linken, der ja im Prinzip in die richtige Richtung geht: Ich empfehle Ihnen einfach einmal, den Sachverständigen zuzuhören, die Sie selber zu Stellungnahmen zu Ihren Anträgen eingeladen haben. Dann wissen Sie, dass die Abschaffung der Beitragsbemessungsgrenze in Deutschland nicht möglich ist, weil die Pflegeversicherung dadurch steuerähnlich würde, was verfassungswidrig wäre. Halten Sie sich also einfach mal an Ihre eigenen Experten!  Meine Damen und Herren, es ist offensichtlich, dass die Finanzierung der Pflege gerade in Zeiten des demografischen Wandels keine Frage von Markt und Kapital, sondern von Menschlichkeit und Solidarität ist.  Wenn wir eine gute Pflege wollen, wenn wir Pflegekräfte in ausreichender Zahl und mit bester Qualifikation wollen, wenn wir den Menschen Versorgungssicherheit geben wollen und trotzdem die Beitragssätze der Pflegeversicherung in den nächsten 40 Jahren stabil halten wollen, dann brauchen wir eine Pflege-Bürgerversicherung, die die finanziellen Lasten auf allen Schultern gerecht verteilt. Das ist heute Generationengerechtigkeit.  Lassen Sie mich am Ende ganz kurz an die älteren Menschen in diesem Land wenden, in Ost und West, in Stadt und Land. Wir brauchen für gute Pflege in der Zukunft den Zusammenhalt der Gesellschaft. Deswegen bitte ich Sie: Halten Sie immer dagegen, wenn Hass und Hetze gegen einzelne Gruppen, egal welcher Hautfarbe und welcher Religion, in unserem Land voranschreiten. Gehen Sie dagegen an! Wir alle brauchen die Solidarität und den Zusammenhalt unserer Gesellschaft. Ich bitte Sie alle, daran teilzuhaben und mitzuwirken. Danke schön.  

\noindent\textbf{Comment:}
\begin{itemize}
    \setlength\itemsep{-3pt}
    \item (Beifall beim BÜNDNIS 90/DIE GRÜNEN und bei der FDP sowie des Abg. Erich Irlstorfer [CDU/CSU])
    \setlength\itemsep{-3pt}
    \item (Beifall beim BÜNDNIS 90/DIE GRÜNEN)
    \setlength\itemsep{-3pt}
    \item (Beifall bei der CDU/CSU)
    \setlength\itemsep{-3pt}
    \item (Beifall beim BÜNDNIS 90/DIE GRÜNEN sowie der Abg. Emmi Zeulner [CDU/CSU])
    \setlength\itemsep{-3pt}
    \item (Erich Irlstorfer [CDU/CSU]: So ist es!)
    \setlength\itemsep{-3pt}
    \item (Beifall der Abg. Dr. Kirsten Kappert-Gonther [BÜNDNIS 90/DIE GRÜNEN])
    \setlength\itemsep{-3pt}
    \item (Beifall beim BÜNDNIS 90/DIE GRÜNEN sowie bei Abgeordneten der FDP)
    \setlength\itemsep{-3pt}
    \item (Beifall beim BÜNDNIS 90/DIE GRÜNEN sowie bei Abgeordneten der SPD)
    \setlength\itemsep{-3pt}
    \item (Beifall bei der FDP sowie der Abg. Marianne Schieder [SPD] – Tino Sorge [CDU/CSU]: Der Valentinstag ist doch vorbei!)
    \setlength\itemsep{-3pt}
    \item (Beifall beim BÜNDNIS 90/DIE GRÜNEN sowie des Abg. Erich Irlstorfer [CDU/CSU])
\end{itemize}
\subsection{Irlstorfer}
\noindent\textbf{Texts:} Herr Präsident! Liebe Kolleginnen! Liebe Kollegen! Ich denke, die Debatte zeigt sehr deutlich, wie notwendig es ist, bei einem so großen Sachverhalt wie der Pflege immer wieder in die Diskussion zu gehen, sich immer wieder auszutauschen und die Dinge immer wieder zu hinterfragen. Es ist notwendig – unser Minister hat das angekündigt und ist auch bereits dabei –, dass wir eine Diskussion über die gesetzliche und die private Pflegeversicherung führen. Das ist ein Thema, das gesellschaftspolitisch relevant ist. Ich glaube aber auch, dass wir gewisse Eckpfeiler brauchen, die die Systematik festzurren. Hier über Vollkasko\/Teilkasko und die Generationengerechtigkeit zu philosophieren und dabei teilweise Jung gegen Alt in Stellung zu bringen, finde ich fatal. Es ist ein gemeinsames Ziel dieser Großen Koalition, aber auch der Oppositionsparteien, die Generationengerechtigkeit im Kontext einer Situation in Deutschland, in der wir den Anfang des demografischen Wandels erleben, neu auszurichten.  Natürlich stelle ich mir schon die Frage, welche Punkte im Endeffekt eine Lösung bedeuten könnten. Die Finanzierung muss immer wieder überprüft werden, vor allem ordnungspolitisch. Wenn wir der Meinung sind, dass wir Beitragserhöhungen in der Pflegeversicherung benötigen, dann ist es nur richtig, das so zu formulieren und es den Menschen auch zu erklären. Ich komme zum Punkt Personal. Wir haben in der letzten Legislaturperiode viele Dinge geregelt, gerade in der Ausbildung. Ich bin mir hundertprozentig sicher, dass wir mehr Akquise benötigen, dass wir auch mehr Motivation benötigen, wenn es um die Aus- und Weiterbildung geht. Ich bin mir auch hundertprozentig sicher, dass wir die notwendigen finanziellen Mittel und auch den notwendigen politischen Willen haben, diese Dinge zu regeln. Aber wir müssen die Menschen motivieren. Wir dürfen hier keine Horrorszenarien an die Wand malen, sondern müssen die Menschen motivieren, diesen Beruf zu wählen.  Ich glaube aber auch, meine sehr geehrten Damen und Herren, dass die Stabilisierung des Arbeitsmarkts im Bereich Pflegekräfte nur in einem engen Miteinander mit den Unternehmerinnen und Unternehmern gelingen kann. Ich halte nichts davon, wenn man laufend private Einrichtungen verteufelt. Das ganze System der Pflegeversicherung wäre ohne private Investitionen und ohne private Betreiber nicht möglich.  Dass wir für eine Tarifgebundenheit sind, ist natürlich völlig klar.  Dass wir für ordentliche Arbeitsbedingungen sorgen wollen, dass wir eine ordentliche Bezahlung wollen, ist die Basis, ist die Grundlage. Aber hören Sie doch bitte auf, die Arbeitnehmer und die Arbeitgeber gegeneinander aufzuhetzen und Szenarien zu entwerfen, die es gar nicht gibt.  Ich möchte es noch einmal klar formulieren, weil es hier die Tendenz gibt, zu sagen: Wer in der Pflege Geld verdienen möchte, handelt unanständig. – Das ist Unsinn, meine sehr geehrten Damen und Herren. Wenn man in solche Einrichtungen investiert, ist das eine Zukunftsinvestition. Und es ist auch normal, dass man, wenn man ordentlich unternehmerisch tätig ist, auch Geld verdient.  Ich möchte unterstreichen, dass wir hier vor allem einen Mix an Unternehmen benötigen: aus dem privaten Bereich, aus den Kirchen, aus den Kommunen, aus freien Trägern – mit allem, was dazugehört. Aber in der Diskussion brauchen wir den Ansatz des Miteinanders. Nur miteinander können wir die Dinge verbessern. Die Spielwiese in der Politik ist nicht die Straße, sondern der Verhandlungstisch. Ich rufe Sie daher auf: Bringen Sie sich ein, um die Situation zu verbessern!  Ich möchte aber darum bitten: Lassen Sie uns das Ganze in einer angemessenen Geschwindigkeit beginnen. Die Große Koalition ist bereit, der Minister auch und wir als Parlament und Abgeordnete sowieso. Herzlichen Dank.  

\noindent\textbf{Comment:}
\begin{itemize}
    \setlength\itemsep{-3pt}
    \item (Beifall bei der AfD)
    \setlength\itemsep{-3pt}
    \item (Beifall bei Abgeordneten der CDU/CSU und der FDP)
    \setlength\itemsep{-3pt}
    \item (Beifall bei der CDU/CSU und der FDP)
    \setlength\itemsep{-3pt}
    \item (Beifall bei Abgeordneten der CDU/CSU)
    \setlength\itemsep{-3pt}
    \item (Beifall bei der CDU/CSU)
    \setlength\itemsep{-3pt}
    \item (Harald Weinberg [DIE LINKE]: Sagen Sie mal was zur Reform!)
    \setlength\itemsep{-3pt}
    \item (Beifall bei Abgeordneten der CDU/CSU und der FDP – Harald Weinberg [DIE LINKE]: Nicht zu glauben!)
\end{itemize}
\subsection{Podolay}
\noindent\textbf{Texts:} Verehrter Herr Präsident! Sehr geehrte Damen und Herren! Die Herausforderungen, vor denen das deutsche Pflegesystem steht, sind enorm; sie werden uns noch über Jahrzehnte beschäftigen. Die Altersstruktur unserer Gesellschaft verändert sich dramatisch. Lebten im Jahr 2013 noch 81 Millionen Menschen in Deutschland, so wird die Einwohnerzahl nach Berechnungen des Statistischen Bundesamtes im Jahr 2060 bei 68 Millionen liegen. Schon im Jahr 2030 werden die Menschen mit einem Alter ab 65 Jahren etwa 29 Prozent der Bevölkerung ausmachen.  Das ist fast jeder Dritte. An dieser demografischen Entwicklung wird deutlich, dass die umlagefinanzierte Pflegeversicherung den Bürgern auf Dauer keine verlässliche Absicherung gegen die Pflegerisiken im Alter bieten wird. Der Charakter der gesetzlichen Pflegeversicherung führt automatisch dazu, dass den Versicherten enorme Eigenanteile verbleiben, die durch sie zu tragen sind. Die hiermit verbundenen Ungerechtigkeiten werden durch die vollständige Aufgabe der kapitalgedeckten privaten Pflegeversicherung aber nicht gelöst. Außerdem wird der demografische Wandel vor den Türen der privaten Pflegeversicherung ebenfalls nicht haltmachen. Mit der Integration der privaten Pflegeversicherung in die gesetzliche Pflegeversicherung müssten dann auch diejenigen Familienmitglieder kostenfrei mitversichert werden, die bislang in der privaten Pflegeversicherung Beiträge zahlen. Die Einführung einer Pflege-Bürgerversicherung lenkt von dem systemeigenen Problem ab, nämlich die ausufernden Kosten im Pflegebereich in den Griff zu bekommen. Wie bereiten wir uns auf die künftige demografische Entwicklung und den Pflegenotstand vor? Sicher nicht dadurch, dass ein funktionierendes Pflegeversicherungssystem in einem anderen System aufgelöst wird, welches nicht funktioniert. Wir setzen daher auf mehr Prävention im Gesundheitswesen, damit die Zahl die Pflegefälle nicht so exorbitant steigt, wie es in der letzten Zeit geschieht. Die Menschen müssen gesund alt werden. Gesunde Ernährung, Bewegung und geistige Aktivierung – das sind die wichtigen Gesichtspunkte, die vor Pflegebedürftigkeit im Alter bewahren können.  Natürlich muss eine nachhaltige Familienförderung das gesunde Altwerden in der Familie finanziell ermöglichen. Das ist nicht nur für unsere Senioren humaner, sondern auch für die Gesellschaft kostengünstiger. Der Antrag der Linken ist unausgereift; das ist ideologischen Zielen geschuldet. Ich weiß, wovon ich rede. Ich habe schließlich 36 Jahre in einem kommunistisch regierten Land gelebt.  Sämtliche Probleme, die mit einer Auflösung der privaten Pflegeversicherung verbunden sind, werden aber komplett ignoriert. Deshalb lehnt die AfD den Antrag der Linken ab. Vielen Dank für Ihre Aufmerksamkeit.  

\noindent\textbf{Comment:}
\begin{itemize}
    \setlength\itemsep{-3pt}
    \item (Beifall bei der SPD)
    \setlength\itemsep{-3pt}
    \item (Beifall bei der AfD)
    \setlength\itemsep{-3pt}
    \item (Reinhard Houben [FDP]: Deswegen brauchen wir Zuwanderung!)
    \setlength\itemsep{-3pt}
    \item (Zuruf von der LINKEN: Na ja!)
\end{itemize}
\subsection{Moll}
\noindent\textbf{Texts:} Sehr geehrter Herr Präsident! Liebe Kolleginnen und Kollegen! In der letzten Legislaturperiode haben wir viel auf den Weg gebracht: mehr Leistungen für Pflegebedürftige, Entlastung für Angehörige und eine bessere Pflegeberatung. Die Reform des Pflegebedürftigkeitsbegriffs ist die größte Reform der sozialen Pflegeversicherung seit ihrer Einführung. Deutlich mehr Menschen können nun erstmals Leistungen der Pflegeversicherung beziehen, und der Bedarf an Pflege kann sehr viel besser verfolgt werden, als es mit dem alten System möglich war.  In dieser Legislaturperiode haben wir weitergemacht. Ja, ich weiß, das ist nur ein Anfang. Eines können Sie mir glauben: Mir geht das auch zu langsam. Ich hätte es auch am liebsten gestern. Aber jeder weiß, dass das nicht so einfach ist. In weiten Teilen der Bevölkerung ist die Erhöhung des Pflegesatzes zur Finanzierung der Versicherungen akzeptiert. Wir sind damit in diesem Bereich auf einem guten Weg. Die Linke fordert, das Zwei-Klassen-System in der Pflegeversicherung zu beenden. Ich sage: Das wollen wir auch.  Doch weder wollen wir das in einer Hauruckaktion von heute auf morgen,  noch wollen wir die Menschen, die heute in der privaten Pflegeversicherung sind, zum Beispiel die Freiberufler, die Selbstständigen und die Beamten, in die gesetzliche Pflegeversicherung zwingen. Vielmehr wollen wir die gesetzliche Pflegeversicherung für diese Berufsgruppen öffnen, um sie langfristig zu einer gemeinsamen Pflegeversicherung für alle weiterzuentwickeln. Wir haben das einmal Pflege-Bürgerversicherung genannt. Eines ist ganz klar – da muss ich Ihnen zustimmen –: Das aktuelle System ist nicht zukunftsfähig, und es ist auch nicht gerecht. Aktuell erfolgt eine Risikoselektion zugunsten der privat Pflegeversicherten. Die Folge ist, dass einige wenige Menschen in der privaten Pflegeversicherung von dem System profitieren, indem sie zum Teil deutlich geringere Beiträge zahlen müssen. Das geht zulasten der Mehrheit der Versicherten in der gesetzlichen Pflegeversicherung durch höhere Beiträge. Langfristig möchten wir weg von der Teilkosten- hin zu einer Vollkostenversicherung, am besten im Rahmen einer Bürgerversicherung. Indem wir das Nebeneinander von privater und gesetzlicher Pflegeversicherung auflösen, können wir eine bessere Pflege für alle erreichen,  die für die Betroffenen und ihre Familien finanzierbar ist. Das heißt, mit mehr Solidarität in der Pflegeversicherung erreichen wir eine Verbesserung für sehr viele Menschen in unserem Land. Die soziale Pflegeversicherung und die private Pflegeversicherung haben gleiche Leistungsansprüche: die Beitragsfreiheit für Kinder und die Orientierung der Beiträge der privaten Pflegeversicherung an der sozialen Pflegeversicherung. – Das war jetzt ein komischer Satz.  Ja.  Daher ist die Bürgerversicherung in der Pflegeversicherung relativ hürdenfrei einzuführen.  – Nicht? In der Pflegeversicherung müssen Beiträge aus allen Einkommensarten gezahlt werden. Beamte und Beamtinnen müssen die Wahlfreiheit haben, und neue Beamtinnen und Beamte müssen in die Bürgerversicherung aufgenommen werden können. Für uns ist klar, dass wir angesichts einer immer weiter steigenden Anzahl von Pflegebedürftigen – bis 2045 sind es bereits 5 Millionen Menschen – eine Umgestaltung der Pflegeversicherung in Angriff nehmen müssen. Wie Sie sehen, sind wir da schon deutlich weiter als die Linksfraktion, die gar kein Konzept vorlegt,  sondern nur fordert: Macht mal!  

\noindent\textbf{Comment:}
\begin{itemize}
    \setlength\itemsep{-3pt}
    \item (Beifall bei der SPD sowie bei Abgeordneten der CDU/CSU)
    \setlength\itemsep{-3pt}
    \item (Beifall bei der SPD)
    \setlength\itemsep{-3pt}
    \item (Beifall bei der FDP)
    \setlength\itemsep{-3pt}
    \item (Heiterkeit)
    \setlength\itemsep{-3pt}
    \item (Lachen bei der LINKEN)
    \setlength\itemsep{-3pt}
    \item (Beifall bei der SPD – Harald Weinberg [DIE LINKE]: Genau so ist es!)
    \setlength\itemsep{-3pt}
    \item (Heiterkeit und Beifall bei der SPD und der CDU/CSU)
    \setlength\itemsep{-3pt}
    \item (Beifall bei der SPD sowie bei Abgeordneten der LINKEN)
    \setlength\itemsep{-3pt}
    \item (Beifall bei Abgeordneten der SPD – Pia Zimmermann [DIE LINKE]: Na, dann!)
    \setlength\itemsep{-3pt}
    \item (Susanne Ferschl [DIE LINKE]: Sie haben ja noch zehn Jahre Zeit!)
\end{itemize}
\subsection{Ullmann}
\noindent\textbf{Texts:} Sehr geehrter Herr Präsident! Liebe Kolleginnen und Kollegen! Eines vorab: Durch das impertinente Wiederholen Ihres Modells einer einheitlichen Zwangsversicherung in der Pflege wird Ihr Ansatz nicht richtiger. Wettbewerb, Qualität und Generationengerechtigkeit spielen bei Ihnen keine Rolle.  Ein solcher Populismus macht mich traurig. Mir fehlt auch einfach die Fantasie, mir vorzustellen, wie ein solches Zwangssystem gerechter sein soll, vor allem, wie es finanziert werden soll.  Sehr geehrte Damen und Herren, meiner Generation geht es relativ gut. Aber haben Sie an die jungen Menschen gedacht, junge Menschen wie meine Kinder, die heute 18 und 21 Jahre alt sind, denen wir allein schon mit der Finanzierung der Rente unserer Generation eine schwere Last zu tragen geben?  Sie wollen mit Ihrem vorliegenden Antrag, liebe Kolleginnen und Kollegen von Die Linke, den jungen Menschen eine weitere Bürde auferlegen,  eine Bürde, die von meiner Generation kaum einer bereit wäre zu tragen; denn Ihre Idee der einheitlichen Zwangsversicherung muss finanziert werden. Gerne. Es ist sehr nett, dass Sie die Zwischenfrage zulassen. Die Diskussion über Generationengerechtigkeit haben wir schon öfter geführt. Auch Sie wiederholen dieses Thema ständig. Die Wiederholung ist eine wichtige rhetorische Figur, habe ich mal gelernt. Ich möchte Sie fragen, ob Sie Herrn Mackenroth und das Mackenroth-Theorem kennen? Mackenroth war ein Sozialpolitiker. Er hatte eine etwas schlechte Vergangenheit: Er war ehemals Nazi, dann war er bei der Union. Herr Mackenroth hat eine richtige These aufgestellt: Jeder Sozialaufwand einer Periode ist logischerweise immer aus der Wirtschaftsleistung der gleichen Periode zu erbringen. – Das gilt für private Vorsorge im Übrigen genauso wie für die soziale Vorsorge. Es gibt keine andere Quelle als die jeweilige Wirtschaftsleistung. Wenn unser Bruttoinlandsprodukt wächst, ist die Finanzierung der Sozialleistungen, auch der Pflegeversicherung, eine Frage der Verteilung.  Wir brauchen hier keine Debatte über Wirtschaftstheoretiker aufzusetzen.  – Herr Weinberg, einen Moment! – Vielmehr geht es darum, zu zeigen, dass wir Generationengerechtigkeit erreichen wollen. Es gibt verschiedene Blickwinkel. Aber was nützt uns das, wenn wir die relative Belastung immer höher und höher schrauben und die nächsten Generationen, die Kindeskinder, überhaupt kein Geld mehr übrig haben, sondern alles in die Umlage geht? Wir müssen erkennen, dass es andere Wege gibt als die Umlagefinanzierung.  Es hat sich gezeigt, dass die private Pflegevorsorge funktioniert. Man sollte zu einer Mischung kommen. Das ist unser Modell.  – Wir haben einen Pflegevorsorgefonds.  – Ja. Herzlichen Dank. – Reden wir über Generationengerechtigkeit. Unsere Gesellschaft wird immer älter, und einer immer größeren Zahl von Pflegebedürftigen stehen immer weniger junge Menschen gegenüber. Ich kenne keinen, der seinen Kindern zur Last fallen möchte. Ich verstehe niemanden, der fremden Kindern zur Last fallen will. Deswegen fordern wir als Freie Demokraten eine nachhaltige und generationengerechte Politik.  Hierzu gehören neben einem stabilen Rentenkonzept auch Konzepte zur Weiterentwicklung der Kranken- und Pflegeversicherung. Die GroKo ist bisher fantasielos. Sie denkt nur von Beitragssteigerung zu Beitragssteigerung. Meine Kollegin Nicole Westig hat aufgezeigt, wohin der Weg gehen soll. Das muss die Basis sein, auf der wir eine Politik der Vernunft und Weitsicht entwickeln. Der Vorschlag, den Sie, liebe Fraktion Die Linke, mit Ihrem Antrag vorgelegt haben, ist ein Hohn für alle arbeitenden Menschen. Sie entlasten weder die Arbeitnehmer mit kleineren noch die Arbeitnehmer mit mittleren Einkommen. Stattdessen wollen Sie die nachkommenden Generationen belasten und ihnen von Anfang an Mühlsteine um den Hals hängen.  Die Babyboomer wollen sozial verantwortlich, selbstbestimmt und generationengerecht älter werden. Das von Ihnen vorgelegte Konzept ist reine Ideologie. Sie, liebe Linke, lehnen private Vorsorge aus ideologischen Gründen ab. Sie lehnen eine kapitalgedeckte Säule zur Finanzierung der Pflege aus ideologischen Gründen ab. Ihnen geht es um Gleichmacherei und um Umverteilung.  Uns hingegen geht es um Generationengerechtigkeit und nachhaltige Finanzierbarkeit. Nur so kann es funktionieren, und nicht anders. Danke schön.  

\noindent\textbf{Comment:}
\begin{itemize}
    \setlength\itemsep{-3pt}
    \item (Beifall bei der FDP – Pia Zimmermann [DIE LINKE]: Umverteilung, das ist richtig!)
    \setlength\itemsep{-3pt}
    \item (Beifall bei der FDP)
    \setlength\itemsep{-3pt}
    \item (Pia Zimmermann [DIE LINKE]: Ja, genau! Sag ich doch!)
    \setlength\itemsep{-3pt}
    \item (Beifall bei der FDP – Pia Zimmermann [DIE LINKE]: Schlechte Rede war das!)
    \setlength\itemsep{-3pt}
    \item (Zuruf des Abg. Harald Weinberg [DIE LINKE])
    \setlength\itemsep{-3pt}
    \item (Beifall bei der FDP – Harald Weinberg [DIE LINKE]: Bei Ihnen wird es auch nicht besser!)
    \setlength\itemsep{-3pt}
    \item (Beifall bei der FDP sowie bei Abgeordneten der CDU/CSU)
    \setlength\itemsep{-3pt}
    \item (Beifall bei der CDU/CSU)
    \setlength\itemsep{-3pt}
    \item (Beifall bei der FDP – Pia Zimmermann [DIE LINKE]: So ein Quatsch!)
    \setlength\itemsep{-3pt}
    \item (Beifall bei der FDP – Pia Zimmermann [DIE LINKE]: Am besten einen Pflegevorsorgefonds einrichten!)
    \setlength\itemsep{-3pt}
    \item (Beifall bei der LINKEN – Zuruf von der CDU/CSU: Wie war jetzt die Frage? – Gegenruf des Abg. Niema Movassat [DIE LINKE]: Ob er es kennt!)
\end{itemize}
\subsection{Kippels}
\noindent\textbf{Texts:} Sehr geehrter Herr Präsident! Meine sehr verehrten Kolleginnen und Kollegen! Die heutige Debatte zeigt, dass man ein gewisses Phänomen bei diesem Thema offensichtlich nicht vermeiden kann. In der Regel lautet der Generalvorwurf der Opposition, wenn Problemlösungen von der Regierung vorgelegt werden: Das ist viel zu spät, das ist nur reaktiv, man hätte das Problem doch viel früher erkennen können.  Nun haben wir die besondere Situation, dass der Minister zu einem frühen Zeitpunkt auf eine Problemstellung, die sich in Zukunft entwickeln könnte, hinweist und anregt, eine Debatte über das Thema zu führen. Ich glaube allerdings, seine Fantasie hat nicht gereicht, sich vorzustellen, dass die erste Reaktion der Opposition – bedauerlicherweise auch unseres Koalitionspartners; Frau Baehrens, verzeihen Sie mir den Vorwurf – in dem Reflex besteht, auf die Bürgerversicherung auszuweichen. Das sind alte Kamellen, um es in der Sprache der fünften Jahreszeit zu sagen. Es ist vor allen Dingen, wie gern angenommen wird, keine Wunderdroge, um Finanzierungsprobleme in medizinischen Versorgungssystemen zu lösen.  In der Tat ist anhand der letzten beiden Beitragserhöhungen in der Pflegeversicherung die Entwicklung erkennbar, dass aufgrund der Struktur der Pflegebedürftigen, der Veränderung der gesellschaftspolitischen Situation, der Kostensituation im Pflegesystem eine Beitragserhöhung – nach normalen rechnerischen Kalkulationen – unvermeidbar sein wird. Aber eine Analyse setzt meines Erachtens zunächst die Kontrolle im Bereich der Kostenentstehung voraus. Da ist es ganz wichtig, sich in Erinnerung zu rufen, dass die Pflegeversicherung im Gegensatz zur Krankenversicherung damals, bei ihrer Einführung vor über 20 Jahren, eine ganz andere Grundausrichtung hatte. Es sollte eine Versorgungsergänzung sein – kein komplett in sich geschlossenes Versorgungssystem –, indem der Patient von fachkundiger Seite, Arzt und Krankenhaus, geführt und therapiert wird. In der Pflegeversicherung haben wir hingegen einen ganz entscheidenden und nicht zu unterschätzenden eigenen Einflussbereich des zu Pflegenden, des Patienten. Er trifft wahnsinnig wichtige Entscheidungen für die Wahrung seiner Würde, die Erhaltung seiner Selbstbestimmung. Dies kann natürlich auch Kosten auslösen, die dann neben den Pflegesätzen, welche anhand der Pflegegrade festgelegt sind, in die Kalkulation einfließen.  Wenn wir uns entscheidend mit der Problemlösung befassen wollen, müssen wir nach den zahlreichen Verbesserungen, die wir in den letzten Jahren in der Pflege mit ihren Instrumenten herbeigeführt haben – die natürlich auch mit Kostensteigerungen einhergingen –, eine gewisse Steuerung vornehmen, und zwar dergestalt, dass der Pflegende angesichts der Vielzahl der Leistungsangebote auch eine Beratung erhält, was für ihn in der konkreten Pflegesituation nützlich und hilfreich ist. Dies könnte insbesondere die Familie, die sich aus persönlichen, familiären oder auch gesundheitlichen Gründen nicht in der Lage sieht, den Familienangehörigen selbst zu pflegen, dabei unterstützen, dem Betroffenen vor allen Dingen Hinweise zu geben. Und – das hat Kollege Rüddel in seiner Rede zu diesem Tagesordnungspunkt heute ausführlich dargestellt – die Pflege muss sich auch neuen und modernen Hilfsmitteln widmen, die Selbstständigkeit und Eigenverantwortlichkeit im eigenen Wohnumfeld ermöglichen. Denn häufig ist der Weg in die stationäre Pflege im Grunde genommen eine Flucht in die behütete Situation, weil es durch andere Hilfsmaßnahmen leider nicht möglich ist, zu Hause, in der Eigenverantwortlichkeit, zu verbleiben, obwohl vielleicht nur relativ wenige Defizite ausgeglichen werden müssen.  Die Initiative unseres Ministers ging dahin, eine Grundsatzdebatte anzustoßen. Sie sollte mit einer Grundsatzanalyse der Problemstellung beginnen. Wir sollten wertfrei und unideologisch zunächst einmal die Kostenauslöser und die Kostendimensionen in den jeweiligen Bereichen betrachten und in einem zweiten Ansatz nach Lösungen suchen. Aber, sehr verehrte Kolleginnen und Kollegen von den Linken – dies auch als kleinen Hinweis an unseren Koalitionspartner –: Bitte nicht mit alten Kamellen, sondern mit neuen Ideen! Herzlichen Dank.  

\noindent\textbf{Comment:}
\begin{itemize}
    \setlength\itemsep{-3pt}
    \item (Beifall bei der SPD)
    \setlength\itemsep{-3pt}
    \item (Beifall bei der CDU/CSU und der FDP)
    \setlength\itemsep{-3pt}
    \item (Beifall bei Abgeordneten der CDU/CSU)
    \setlength\itemsep{-3pt}
    \item (Beifall bei der CDU/CSU)
    \setlength\itemsep{-3pt}
    \item (Kordula Schulz-Asche [BÜNDNIS 90/DIE GRÜNEN]: Hätte man ja auch! Die Enquete des Bundestages war in den 90er-Jahren!)
\end{itemize}
\subsection{Heidenblut}
\noindent\textbf{Texts:} Sehr geehrter Herr Präsident! Sehr geehrte Kolleginnen und Kollegen! Meine sehr geehrten Damen und Herren! Das, was wir heute erleben, ist so etwas wie ein Antragsspagat; denn gegensätzlicher könnten die Anträge kaum sein, die uns heute vorliegen. Das zeigt auch die Diskussion; das macht die Sache ja auch ganz interessant. Und ja, Herr Kollege Schneider, die Anträge richten sich beide auf die Finanzierung der Pflege. Ich habe zwar auch viel an dem Antrag der FDP auszusetzen, aber ihr vorzuwerfen, dass sie sich um die Finanzierung der Pflege kümmere und dies der größte Fehler des Antrags wäre, finde ich wirklich absurd.  Wenn sie etwas gut gemacht hat, dann, dass sie sich mit der Finanzierung beschäftigt. Zu dem Rest komme ich nicht mehr; das haben die Kollegen schon gesagt.  Sie haben richtig gesagt: Das, worum wir uns in der Pflege kümmern müssen, ist in allererster Linie die Frage: Wie kommen wir an mehr Personal? Wie gelingt es, dieses Personal vernünftig zu bezahlen?  Und wie sorgen wir dafür, dass die Arbeitsbedingungen und viele andere Dinge dort auch stimmen? Um das zu schaffen, brauchen wir die entsprechende Finanzierung, brauchen wir die Mittel; denn das geht nicht einfach so. Die Wege, die hierfür gefunden werden, sind unterschiedlich – sehr unterschiedlich zwischen der FDP und den Linken –, aber es sind zumindest Vorschläge da, wie wir das Ganze finanzieren können. Herr Schneider: Nein, wir wissen durchaus, wie wir unterstützen können – es geht um die Akzeptanz der Tarife –, dass die Mitarbeiterinnen und Mitarbeiter mehr Geld bekommen. Sie müssen sich unsere Gesetze nur einmal ansehen! Wir haben dafür gesorgt, dass die Tarife im Bereich der Krankenpflege übernommen werden und dort entsprechend bezahlt wird.  Besser kann man das doch nicht sicherstellen. Aber wissen Sie, mich wundert auch nicht, dass Sie bei Anhörungen im Gesundheitsausschuss im Zweifel sprachlos sind und keine Fragen mehr haben. Man muss halt zuhören und überdenken, was gesagt wird. Dann kann man auch Fragen dazu stellen.  Genauso interessant finde ich allerdings den Vorwurf der FDP an die Kolleginnen und Kollegen von den Linken, sie kämen – ich will das erste Wort nicht wiederholen – immer wieder mit denselben Sachen, nämlich mit der Bürgerversicherung. Herr Kollege Kippels, jetzt müssen Sie ganz stark bleiben: Ja, wir sind nach wie vor für die Bürgerversicherung. – Ich finde, das kann man gar nicht oft genug wiederholen, damit auch Sie irgendwann dafür sind; denn das überzeugt letzten Endes.  – Ja, die Argumente bringen wir auch – das ist kein Problem – und haben sie auch schon gebracht. Kollegin Moll hat eine ganze Reihe von Punkten dazu genannt. Ein wesentliches Argument ist: Wir müssen die Finanzierung der vielen guten Leistungen, die wir in der Koalition durchaus gemeinsam durchsetzen können, auf die Beine stellen, und dazu brauchen wir andere Systeme. Sicher ist auch in Ihrem Antrag der eine oder andere Aspekt, den man gebrauchen kann. Aber – um das ehrlich zu sagen – da Sie anderen Wiederholungen vorwerfen –: So ganz neu sind die Dinge, die Sie da gebracht haben, auch nicht unbedingt. Vielleicht ärgern Sie sich nach wie vor darüber, dass Sie sie in der letzten Legislaturperiode nicht wiederholen konnten. Da sind die Linken Ihnen halt ein bisschen voraus; die haben das öfter tun können.  Insofern: Wir müssen uns mit der Finanzierung beschäftigen. Wir müssen noch viel Geld in die Hand nehmen und in die Pflege investieren. Einen Punkt hat Kollege Rüddel ganz am Anfang aufgegriffen, nämlich die Zuzahlung, die Eigenanteile gerade in der stationären Pflege. Da müssen wir ran, die müssen wir dringend begrenzen. Die AWO hat dazu gerade eine große Petition gestartet. Da sollten wir als Koalition eine Lösung finden. So verstehe ich auch die Vorschläge von Herrn Spahn, über die Finanzierung nachzudenken. Lassen Sie mich zum Schluss in Richtung FDP noch eines sagen. Ich habe festgestellt, dass Sie sich immer wieder auch mit den schönen Titeln, die wir für unsere Gesetze haben, beschäftigen und sagen, wir müssten da vielleicht noch weitere schöne Titel finden. Ich hätte einen schönen Titel für Ihren Antrag, der die Sache auch transparent und griffig machen würde: „Hilf dir selbst, sonst hilft dir keiner!“ – Antrag der FDP. Vielleicht nutzen Sie das! Das machen Sie in Nordrhein-Westfalen ja auch. Danke.  

\noindent\textbf{Comment:}
\begin{itemize}
    \setlength\itemsep{-3pt}
    \item (Beifall bei der SPD sowie bei Abgeordneten der CDU/CSU)
    \setlength\itemsep{-3pt}
    \item (Beifall bei der SPD)
    \setlength\itemsep{-3pt}
    \item (Heiterkeit und Beifall bei Abgeordneten der SPD)
    \setlength\itemsep{-3pt}
    \item (Beifall bei der SPD und der LINKEN – Karin Maag [CDU/CSU]: Wo sind die Argumente? Es sollten auch Argumente kommen!)
    \setlength\itemsep{-3pt}
    \item (Beifall bei der SPD und der FDP sowie bei Abgeordneten des BÜNDNISSES 90/DIE GRÜNEN)
    \setlength\itemsep{-3pt}
    \item (Heiterkeit und Beifall bei der SPD sowie bei Abgeordneten der FDP)
    \setlength\itemsep{-3pt}
    \item (Heiterkeit bei Abgeordneten der CDU/CSU und der FDP)
    \setlength\itemsep{-3pt}
    \item (Beifall bei der CDU/CSU)
    \setlength\itemsep{-3pt}
    \item (Beifall bei der SPD und der LINKEN – Michael Theurer [FDP]: Eigenverantwortung!)
\end{itemize}
\subsection{Zeulner}
\noindent\textbf{Texts:} Sehr geehrter Herr Präsident! Meine lieben Kolleginnen und Kollegen! Liebe Frau Westig, wie gesagt, Ihnen persönlich kann ich keinen Vorwurf machen, Sie sind erst seit dieser Legislatur im Gesundheitsausschuss. Aber es liegt in der Natur der Sache: Im Bereich der Pflege ist die FDP vorher einfach sehr unterentwickelt gewesen. Deswegen freue ich mich auch, dass Sie hier die Fahne für die Pflege hochhalten.  Was soll ich zum Antrag der FDP sagen? Dass ich als gelernte Krankenschwester einmal einer Wirtschaftspartei wirtschaftliche Zusammenhänge erklären muss, hätte ich nicht gedacht. Geschätzte Kollegen von der FDP, Ihr Antrag ist leider Unfug. Sie fordern in Ihrem Antrag vor allem den Ausbau der privaten Vorsorge in der sozialen Pflegeversicherung, und das, obwohl Sie wissen, dass die privaten Zuzahlungen der Patientinnen und Patienten in den Pflegeheimen immer weiter steigen  und mittlerweile jeder dritte Heimbewohner aufgrund Pflegebedürftigkeit zusätzlich Sozialhilfe in Anspruch nehmen muss. Sie sagen den Menschen also: Ihr müsst nur besser privat vorsorgen, dann ist alles kein Problem! – Damit verkennen Sie das eigentliche Versprechen, mit dem der Gesetzgeber die Einführung der sozialen Pflegeversicherung begründet hat, und dieses Versprechen ist für mich auch 25 Jahre nach ihrer Einführung bindend. Die Politik hat den Menschen damals versprochen, dass diejenigen, die ihr Leben lang gearbeitet und eine durchschnittliche Rente erworben haben, im Pflegefall eben nicht zum Sozialfall werden. Wir brauchen also für die fünfte Säule der gesetzlichen Sozialversicherung eine neue Antwort. Eine alte Antwort gibt einmal mehr Die Linke. Lassen Sie doch die Privatversicherten endlich in Ruhe! Die gesetzliche und die private Versicherung, das sind Systeme, die sind unterschiedlich konzipiert, da sind unterschiedliche Finanzierungssysteme unterlegt. Wenn jemand privat vorsorgen oder eine private Versicherung abschließen möchte, dann soll er das auch zukünftig tun. Um ehrlich zu sein: Ich bin von Ihrem Antrag einfach enttäuscht. Die Pflege ist das Megathema der Zukunft – für die Älteren, aber eben auch für die Jüngeren in unserem Land. Und das ist Ihnen gerade einmal eine halbe Seite wert? Die Namen Ihrer Fraktionsmitglieder auf dem Antrag sind länger als das, was Sie inhaltlich beitragen.  Deswegen lehnen wir beide Anträge ab, wenn sie dann in den Ausschuss überwiesen sind. Ich bin der Meinung: Wir brauchen eine echte Lösung; denn Pflege muss verlässlich sein – für alle, nicht nur für diejenigen, die es sich leisten können.  Deswegen finde ich es spannend, darüber zu diskutieren, das System umzukehren. Das heißt für mich, die Eigenanteile der Pflegebedürftigen müssten gedeckelt werden, und der flexible Anteil muss von der Versichertengemeinschaft – unter Umständen auch mithilfe von Steuermitteln, die wir aber in der Sozialhilfe wieder einsparen könnten – getragen werden. Denn Pflegebedürftigkeit kann jeden treffen; aber keiner weiß, wie lange sie dauert und in welchem Schweregrad sie stattfindet. Mit einem festen Eigenanteil kämen wir auch der Idee einer verlässlichen zusätzlichen privaten Vorsorge wieder näher, die Sie ja auch andenken. Wenn wir das angehen, müssen wir die Bedürfnisse der Menschen auch wirklich ernst nehmen. Das heißt für mich, dass der ambulante Bereich natürlich zwingend mitgedacht werden muss; denn die Menschen wollen, solange es geht, in ihrer vertrauten Umgebung bleiben. Für den ambulanten Bereich muss aber ein System gedacht werden, das den Anreiz setzt, keine ausufernden Leistungen, welche für die Versorgung nicht gebraucht werden, in Anspruch zu nehmen. Hier bieten die wissenschaftlichen Vorstellungen von Professor Dr. Klie und Professor Dr. Rothgang ganz spannende Ansätze.  Ich danke unserem Bundesgesundheitsminister Jens Spahn ausdrücklich für seinen Vorstoß,  diese Debatte zu führen. Er hat wie kein anderer Minister vorher für die Pflege seinen Teil geleistet,  und er setzt zügig das um, was im Koalitionsvertrag auch vereinbart wurde. Er geht dabei mutig voran.  Es ist unser erklärtes Ziel, die Abwärtsspirale bei der Pflege endlich zu durchbrechen. Das geht vor allem durch bessere Arbeitsbedingungen, und dafür haben wir auch Maßnahmen eingebracht. Meiner Ansicht nach muss man im Bereich der Pflege vieles weiter- bzw. mitdenken, zum Beispiel das Care-und-Case-Management. Wir haben auch eine Reform der Kontrollsysteme im Koalitionsvertrag vereinbart, also eine bessere Überwachung der Situation in den Pflegeheimen, weil da teilweise wirklich Verhältnisse sind, die wir so nicht wollen. Es gibt gute Pflegeheime; aber wir müssen die schlechten ganz massiv aussortieren – wenn sie denn da sind – und die guten stützen; denn das haben die Menschen verdient. Ich bin der festen Überzeugung, dass wir gerade im Bereich der Pflege die gleiche Sorgfalt anwenden müssen, die wir für unsere Kinder und für unsere jungen Leute ganz selbstverständlich anwenden. Das muss für das Alter eben auch gelten, sodass wir wirklich ein Leben in Würde auch am Ende der Tage gewährleisten können. In diesem Sinne freue ich mich wirklich auf eine gute und spannende zukünftige Debatte über dieses sehr wichtige Thema. Vielen Dank. 

\noindent\textbf{Comment:}
\begin{itemize}
    \setlength\itemsep{-3pt}
    \item (Beifall bei Abgeordneten der CDU/CSU und der FDP sowie der Abg. Verena Hartmann [AfD])
    \setlength\itemsep{-3pt}
    \item (Beifall bei Abgeordneten des BÜNDNISSES 90/DIE GRÜNEN)
    \setlength\itemsep{-3pt}
    \item (Michael Theurer [FDP]: Da sind wir auf Ihre Vorschläge gespannt!)
    \setlength\itemsep{-3pt}
    \item (Beifall bei Abgeordneten der CDU/CSU)
    \setlength\itemsep{-3pt}
    \item (Beifall bei der CDU/CSU)
    \setlength\itemsep{-3pt}
    \item (Beifall bei der CDU/CSU und dem BÜNDNIS 90/DIE GRÜNEN)
    \setlength\itemsep{-3pt}
    \item (Marianne Schieder [SPD]: Oje, oje!)
    \setlength\itemsep{-3pt}
    \item (Beifall bei der CDU/CSU – Niema Movassat [DIE LINKE]: Er ist nicht entschuldigt!)
    \setlength\itemsep{-3pt}
    \item (Niema Movassat [DIE LINKE]: Wo ist denn der Herr Spahn?)
\end{itemize}
\section{Tagesordnungspunkt 21}
\subsection{Rottmann}
\noindent\textbf{Texts:} Sehr geehrter Herr Präsident! Sehr geehrte Damen und Herren! Ein wesentliches Handicap der EU ist ja Folgendes: Hinter dem Pfusch der nationalen Umsetzung werden ihre Leistungen für die Bürgerinnen und Bürger manchmal überhaupt nicht mehr erkennbar. Heute haben wir dafür wieder ein Beispiel: Die EU erlässt eine Richtlinie zum Geheimnisschutz, an der es eigentlich gar nicht viel auszusetzen gibt. Die EU will europäische Unternehmen effektiver gegen den Verlust ihrer Betriebsgeheimnisse schützen. Das ist gut. Sie sieht aber auch die Risiken eines zu weiten Geheimnisbegriffs, und das ist auch gut. Das Interesse von Unternehmen am Schutz von Geheimnissen wird in der Richtlinie gut austariert mit dem Schutz von Hinweisgebern und mit der Medien-, Meinungs- und Informationsfreiheit. Die Richtlinie sieht ausdrücklich vor, dass die Mobilität von Arbeitnehmerinnen und Arbeitnehmern durch einen überbordenden Geheimnisschutz nicht beeinträchtigt werden darf. Sie regelt ausdrücklich, dass die Erfüllung von Informationsaufgaben von Arbeitnehmervertretungen gegenüber der Belegschaft weiterhin sichergestellt bleiben muss. Sie fordert außerdem Sanktionen für eine missbräuchliche Berufung auf Betriebsgeheimnisse, damit daraus nicht ein Instrument der Einschüchterung wird. Sie verweist auf die Notwendigkeit besonderer Schutzmechanismen vor Gericht, damit Geheimnisschutz überhaupt effektiv durchgesetzt werden kann. Brüssel hat das gut gemacht.  Versemmelt worden ist es dann allerdings hier in Berlin. Aus einer guten Richtlinie wird ein Risikogesetz für Journalistinnen und Journalisten, für Hinweisgeber, für Arbeitnehmer, für Betriebsrätinnen oder für die Entwickler freier Software. Die Sachverständigenanhörung zum Umsetzungsgesetz von Katarina Barley ist ein Desaster gewesen.  Da können Sie uns und der Linken gerne mal einen Kaffee dafür ausgeben, dass wir mit unseren Anträgen die Arbeit machen, für die Sie ein ganzes Justizministerium haben.  Jetzt brauchen Sie die beiden Anträge nur noch neben Ihren Gesetzentwurf zu legen und diesen entsprechend umzuarbeiten, und der schlimmste Schaden ist abgewendet. Was ist aber davon zu halten, wenn die Kommission, das EU-Parlament und 27 EU-Staaten bessere Regelungen zustande bringen als eine Koalition aus zwei Fraktionen und drei Parteien? Was zum Teufel ist eigentlich mit der SPD los, wenn der DGB dort mit seinen fundierten Einwänden überhaupt kein Gehör mehr findet, sondern sich an uns wenden muss?  Und was ist von einer Justizministerin zu halten, die europäisches Recht national so sehr verkorkst, vor allem durch ihre europapolitischen Ambitionen?  Die größten Sorgen mache ich mir aber um die laufenden Trilogverhandlungen zur Whistleblower-Schutzrichtlinie in Brüssel. Fast wöchentlich sitzen bei mir im Büro Hinweisgeber, die aufgrund der unklaren Rechtslage in Deutschland gravierende persönliche Folgen für hochanständiges, integres Verhalten tragen. Es ist an der Zeit, dass dieser Staat etwas für diese Menschen tut.  Wir wissen als Abgeordnete nicht, ob Sie von der Bundesregierung sich in diesen Verhandlungen für die Meinungsfreiheit und für die Hinweisgeber einsetzen oder für die Compliance-Interessen der Unternehmen. Es gibt keinerlei Anlass, dieser Bundesregierung in dieser Frage Vertrauen zu schenken, und das ist das Bitterste an diesem Gesetzentwurf.  

\noindent\textbf{Comment:}
\begin{itemize}
    \setlength\itemsep{-3pt}
    \item (Erhard Grundl [BÜNDNIS 90/DIE GRÜNEN]: Das stimmt!)
    \setlength\itemsep{-3pt}
    \item (Beifall beim BÜNDNIS 90/DIE GRÜNEN)
    \setlength\itemsep{-3pt}
    \item (Beifall beim BÜNDNIS 90/DIE GRÜNEN sowie bei Abgeordneten der LINKEN)
    \setlength\itemsep{-3pt}
    \item (Beifall beim BÜNDNIS 90/DIE GRÜNEN sowie des Abg. Niema Movassat [DIE LINKE])
    \setlength\itemsep{-3pt}
    \item (Beifall bei der CDU/CSU)
    \setlength\itemsep{-3pt}
    \item (Beifall beim BÜNDNIS 90/DIE GRÜNEN – Niema Movassat [DIE LINKE]: Gute Frage!)
    \setlength\itemsep{-3pt}
    \item (Beifall beim BÜNDNIS 90/DIE GRÜNEN und bei der LINKEN)
\end{itemize}
\subsection{Jung}
\noindent\textbf{Texts:} Herr Präsident! Meine Damen und Herren! Liebe Frau Kollegin Dr. Rottmann, ich glaube, das ging jetzt vielleicht doch ein bisschen übers Ziel hinaus.  Wir haben eine Richtlinie vorliegen – in der Tat –, die einen Rahmen vorgibt, und wir haben einen Entwurf aus dem Ministerium vorliegen, über den man diskutieren kann und bei dem es noch die eine oder andere Möglichkeit der Verbesserung gibt; aber grundsätzlich hält diese sich im Rahmen der Richtlinie – ob Sie das jetzt gut finden oder nicht.  – Sie hätten dann vielleicht auch mal ganz konkret vorschlagen können, an welcher Stelle Sie was anders machen wollen.  Sie haben hier einfach aufgezählt, was die Richtlinie vorgibt, und haben dann schlicht festgestellt, das sei alles nicht erfüllt. Da müssen wir schon ein bisschen genauer hinschauen, meine Damen und Herren; denn tatsächlich sieht es etwas anders aus. Worum geht es denn überhaupt bei diesem Gesetzentwurf? Vielleicht muss man das einmal festhalten: Es geht hier um den Schutz von Geschäftsgeheimnissen. Das sind technische Innovationen. Das ist Know-how. Das ist bestimmtes Wissen. Das sind bestimmte Informationen. Das sind Businesspläne und Ähnliches. Es geht also um alles, was einen legitimen Schutz im Rahmen des wirtschaftlichen Betriebes gewährleistet. Im Gegensatz zum klassischen geistigen Eigentum, das wir durch Patentrecht, Markenrecht, Designrecht, Urheberrecht und Ähnliches geschützt haben, gibt es in diesem Bereich bisher nur Richterrecht. Das ändert sich jetzt. Wir schaffen zum ersten Mal einen klar definierten einheitlichen Schutz von Geschäftsgeheimnissen, und zwar auf gesetzlicher Ebene, und schaffen damit Rechtssicherheit für alle, die tätig sind. Das ist insbesondere für innovative Unternehmen in Deutschland ein echter Fortschritt; das sollte man zunächst einmal festhalten, meine Damen und Herren.  Ich komme zum zweiten Punkt.  – Frau Dr. Rottmann, ich habe Ihnen eben doch auch zugehört. Dann lassen Sie mich doch auch darauf antworten. – Wir schaffen übrigens auch für Hinweisgeber mehr Rechtssicherheit und eine verbesserte Situation. Wenn Sie genau hinschauen, erkennen Sie das.  Wir haben richtlinienkonform umgesetzt. In diesen schwierigen Fällen, in denen jemand legitim ein Geheimnis offenbart – noch nicht mal die Fälle, in denen es sich um eine rechtswidrige Handlung handelt, sondern die schwierigen Zwischenfälle –, ist in dem Gesetzentwurf, den wir aus dem Ministerium bekommen haben, Straffreiheit vorgesehen. Das ist im Verhältnis zu dem Hilfsvehikel des rechtfertigenden Notstands, wie wir es im Moment haben, für die Hinweisgeber eine klare Verbesserung; das muss man an der Stelle auch einmal zur Kenntnis nehmen, meine Damen und Herren.  Lassen Sie mich vielleicht noch etwas in die tatsächliche Materie einsteigen. Wo wir noch ein Stück weit Verbesserungsbedarf sehen, das ist an der Stelle der Rechtfertigung, die Sie kurz angesprochen haben. Derzeit ist sie in Artikel 1 § 5 des Entwurfs als Rechtfertigung ausgestaltet. Da haben wir – so heißt es in einem Antrag – Gesinnungsprüfungen. Das ist eine sehr subjektivierte Sicht auf die Möglichkeit der Straffreiheit, der Rechtfertigung, was auch immer. Da gebe ich zu: Das geht auch uns ein bisschen zu weit an der Stelle.  Vielleicht betrachte ich das aus einer anderen Richtung als Sie als zu weit. Denn ich finde auch: Nein, wir sollten keine Gesinnungsprüfung vornehmen. Nein, wir haben auch kein Gesinnungsstrafrecht an der Stelle. Das heißt, wenn man Straffreiheit will, genügt es aus unserer Sicht nicht, wenn jemand nur den Schutz eines öffentlichen Interesses vorgibt. Wenn es sich um Gutgläubigkeit handelt, kann man darüber reden. Aber was für uns wichtig ist: Es darf sich nicht nur um ein vorgegebenes öffentliches Interesse handeln, wenn wir das Verraten von Geschäftsgeheimnissen erlauben, sondern es muss ein tatsächlich bestehendes öffentliches Interesse nach allgemeinem Rechtsempfinden sein. Ich glaube, an der Stelle müssen wir noch ein Stück weit nachbessern, meine Damen und Herren. Zu den Punkten, die Sie angesprochen haben, sage Ihnen ganz offen: Wenn wir dort eine klare Objektivierung hinbekommen, muss das im Sinne aller sein. Es kann am Ende nicht nur Gesinnungsprüfung, wie es in dem einen Antrag steht, sein. Dazu bin ich gerne bereit. Dann lassen Sie uns doch reden im Rahmen des parlamentarischen Verfahrens, in dem wir uns befinden. Lassen Sie uns doch darüber reden, ob wir vielleicht einen Tatbestandsausschluss statt einer Rechtfertigung machen. Das kann ich mir grundsätzlich vorstellen.  Lassen Sie uns darüber reden, ob wir die Rechte aus Arbeitsverhältnissen unberührt lassen – das will niemand ändern an der Stelle –, und lassen Sie uns darüber reden, wenn Sie bei den Journalisten solche Bedenken haben, ob wir dann vielleicht, wenn wir das andere hinkriegen, auch etwas bei der Beihilfestrafbarkeit machen. Ich könnte mir sogar vorstellen, dass man da eine klare Regelung findet. Das sind Dinge, über die man reden kann, wenn wir auf der anderen Seite auch eine klare Regelung finden.  Es wäre doch schön, wenn wir das im Rahmen von Änderungsanträgen machen. Wir befinden uns im parlamentarischen Verfahren. Dort kann man das alles behandeln. Stellen Sie doch die Anträge.  Lassen Sie uns über die Anträge reden, die die Koalition stellen wird, und dann, glaube ich, kommen wir zu einem sehr vernünftigen Gesetzentwurf. Dann muss dieser auch nicht mehr so heruntergeredet werden, wie wir das eben gehört haben. Herzlichen Dank.  

\noindent\textbf{Comment:}
\begin{itemize}
    \setlength\itemsep{-3pt}
    \item (Beifall bei der AfD)
    \setlength\itemsep{-3pt}
    \item (Beifall bei Abgeordneten der CDU/CSU und der SPD)
    \setlength\itemsep{-3pt}
    \item (Helin Evrim Sommer [DIE LINKE]: Wir haben doch Anträge gestellt!)
    \setlength\itemsep{-3pt}
    \item (Beifall bei der CDU/CSU und der SPD – Niema Movassat [DIE LINKE]: Stimmen Sie unseren Anträgen denn zu? War das so zu verstehen? – Gegenruf des Abg. Ingmar Jung [CDU/CSU]: Natürlich nicht!)
    \setlength\itemsep{-3pt}
    \item (Tabea Rößner [BÜNDNIS 90/DIE GRÜNEN]: Tut es nicht!
    \setlength\itemsep{-3pt}
    \item (Beifall bei der CDU/CSU und der SPD – Widerspruch beim BÜNDNIS 90/DIE GRÜNEN – Niema Movassat [DIE LINKE]: Ihr Gesetzentwurf geht gründlich übers Ziel hinaus; das stimmt!)
    \setlength\itemsep{-3pt}
    \item (Zuruf der Abg. Dr. Manuela Rottmann [BÜNDNIS 90/DIE GRÜNEN])
    \setlength\itemsep{-3pt}
    \item (Tabea Rößner [BÜNDNIS 90/DIE GRÜNEN]: Das stimmt nicht!)
    \setlength\itemsep{-3pt}
    \item (Dr. Manuela Rottmann [BÜNDNIS 90/DIE GRÜNEN]: Das steht im Antrag drin! Ich kann Ihnen nicht das Lesen abnehmen; tut mir leid!)
    \setlength\itemsep{-3pt}
    \item (Beifall der Abg. Dr. Manuela Rottmann [BÜNDNIS 90/DIE GRÜNEN])
    \setlength\itemsep{-3pt}
    \item (Helin Evrim Sommer [DIE LINKE]: Haben wir doch gemacht!)
\end{itemize}
\subsection{Jacobi}
\noindent\textbf{Texts:} Herr Präsident! Meine Damen und Herren! Der Gesetzentwurf der Bundesregierung zur Umsetzung der Richtlinie (EU) 2016\/943 liegt bereits im Rechtsausschuss. Die Grünen und Die Linke fordern mit ihren Anträgen die Bundesregierung auf, den Gesetzentwurf zu verändern.  Worum geht es? Gegenstand der EU-Richtlinie ist der Schutz von Geschäftsgeheimnissen. Das heißt, es geht um die Konfliktlage zwischen dem Inhaber eines solchen Geheimnisses und demjenigen, der ein fremdes Geheimnis für sich verwenden oder an Dritte weitergeben will. Das kann zum einen ein Wettbewerber sein, der sich fremde Betriebsgeheimnisse wirtschaftlich zunutze machen will; zum anderen können es aber auch Journalisten sein oder sogenannte Pfeifenbläser, im Englischen „Whistleblower“ genannt. Die Grünen nennen fünf Punkte, in denen sie den Gesetzentwurf abändern wollen. Die Linken wollen das wenigstens mengenmäßig übertreffen und zählen sechs Punkte auf. Ich gehe auf den Punkt 3 aus dem Linkenantrag ein; denn er lenkt den Blick auf Grundsätzliches. Er bezieht sich auf den Passus des Gesetzentwurfs, der regelt, wann denn das Ausspähen oder Weitergeben von fremden Geheimnissen erlaubt sein soll, nämlich dann, wenn jemand „in der Absicht handelt, das allgemeine öffentliche Interesse zu schützen“. Die Linke meint nun, das Wort „Absicht“ sei falsch gewählt, schließlich laute doch die englischsprachige Version an dieser Stelle: „for the purpose of protecting the general public interest“. Deshalb müsse es im Deutschen heißen „zum Zwecke“ anstatt „in der Absicht“. Nun könnte man darüber trefflich debattieren, zum einen darüber, ob eine der beiden Varianten im Rahmen eines Übersetzungsvorgangs wirklich zwingend ist – denn das englische Wort „purpose“ kann nun einmal beides bedeuten, „Absicht“ wie auch „Zweck“ –, zum anderen darüber, ob die Unterscheidung der beiden Worte im Deutschen wirklich die behauptete Bedeutung hat. Nun gibt es auch eine offizielle deutschsprachige Version der Richtlinie, die im Amtsblatt der EU veröffentlicht ist. Dort heißt es an der betreffenden Stelle überraschenderweise: „in der Absicht“; denn das hat die Bundesregierung in ihrem Gesetzentwurf getreulich abgeschrieben. Solche philologischen Betrachtungen sind ja durchaus interessant und anregend. Wichtiger, als diesen philologischen Ansatz weiterzuverfolgen, dürfte es aber sein, dass wir einmal innehalten und uns vergegenwärtigen, was wir hier eigentlich tun und in welche Rolle der Deutsche Bundestag mittlerweile geraten ist. Anstatt als Gesetzgeber darüber zu entscheiden, was in Deutschland als Recht gelten soll, unterhalten wir uns darüber, ob ein in Belgien entstandener englischsprachiger Text richtig übersetzt wurde und ob wir ihn auch richtig abgeschrieben haben. Man kann durchaus verstehen, dass sich gegen diesen Bedeutungsverlust des Parlaments hier bisher kein Widerspruch geregt hat. Es mag ja angenehm sein, wenn man weniger selbst entscheiden muss, wenn man sich im Zweifel hinterher darauf zurückziehen kann, zu sagen: „Wir waren es nicht; Brüssel ist es gewesen“, ob es nun um Exzesse beim Datenschutz geht oder um Dieselfahrverbote.  Ob das aber eines nationalen Gesetzgebungsorgans noch würdig ist, das muss am Ende das Volk bewerten.  Zurück zu den beiden konkreten Anträgen hier. Beide Anträge wollen den Bereich dessen, was ein Geschäftsgeheimnis sein soll, verengen.  Dazu soll der Definition des Geschäftsgeheimnisses in Artikel 1 § 2 des Gesetzentwurfs ein zusätzliches viertes Merkmal hinzugefügt werden, das des berechtigten Interesses. Die Anträge verweisen auf die Erwägungsgründe der Richtlinie. Im Erwägungsgrund 14 sei das „legitime Interesse“ erwähnt. Erwägungsgründe sind der eigentlichen Richtlinie vorangestellte Erläuterungen, in denen uns Brüssel mitteilt, was man sich so gedacht hat und warum man die Richtlinie so geschrieben hat, wie man sie geschrieben hat. Wenn der Normgeber in Brüssel in seinen Vorbemerkungen alles Mögliche anspricht und dann eine konkrete Norm formuliert, dann muss man als Adressat wohl davon ausgehen, dass er seine eigenen Vorbemerkungen in diesem Normtext berücksichtigt und eingearbeitet hat. Und wenn in den Erwägungsgründen ein „legitimes Interesse“ erwähnt wird und dann die konkrete Norm aus drei Definitionselementen besteht, dann muss man das wohl so verstehen, dass bei Vorliegen dieser drei Definitionselemente das „legitime Interesse“ eben gegeben sein soll. Die Definition des Geschäftsgeheimnisses findet sich in Artikel 2 der Richtlinie. Und die Bundesregierung hat in ihrem Gesetzentwurf diese Definition genau so abgeschrieben, wie sie halt in der Richtlinie steht. Das Ergebnis kann man nun in der Sache sinnvoll und wünschenswert finden oder auch nicht. Aber wenn man sich der Regelungsgewalt einer Organisation wie der EU unterwirft und diese etwas entscheidet, muss man das halt schlucken, auch wenn es einem nicht schmeckt. Dann sollte man sich halt vorher überlegen, ob man die eigenen Gesetzgebungsbefugnisse so leichthin aufgibt, wie das der Deutsche Bundestag gewohnheitsmäßig tut.  – Hören Sie zu, hören Sie zu! Bedenken Sie es wohl! Aber bevor jetzt jemand fragt: „Und wo bleibt das Positive?“ – gerne: Die beiden Anträge sind nicht per se und in toto unsinnig. Sie enthalten auch Ansätze, über die man durchaus noch einmal reden sollte.  Die verlangten Bereichsausnahmen sehen wir zwar im Medienbereich aufgrund dessen Unbestimmtheit eher weniger; für den Bereich des Arbeitsrechts, der im Hinblick auf die betroffenen Akteure und die infragekommenden Konstellationen klarer strukturiert ist, kann man dem aber womöglich durchaus nähertreten. Gleiches gilt für den Ansatz des Grünenantrags, den Geheimnisschutz im Zivilverfahren noch zu verbessern. Da gehen wir durchaus mit.  Der Überweisung der beiden Anträge in den Rechtsausschuss stimmen wir natürlich zu. Dort werden wir dann hoffentlich bei der Überarbeitung des Gesetzentwurfs der Bundesregierung zu einem vernünftigen Ergebnis kommen. Vielen Dank.  

\noindent\textbf{Comment:}
\begin{itemize}
    \setlength\itemsep{-3pt}
    \item (Beifall bei der AfD – Canan Bayram [BÜNDNIS 90/DIE GRÜNEN]: Ach Gott!)
    \setlength\itemsep{-3pt}
    \item (Beifall bei der SPD)
    \setlength\itemsep{-3pt}
    \item (Beifall bei der AfD)
    \setlength\itemsep{-3pt}
    \item (Helin Evrim Sommer [DIE LINKE]: Sehr großzügig!)
    \setlength\itemsep{-3pt}
    \item (Helin Evrim Sommer [DIE LINKE]: Genau!)
    \setlength\itemsep{-3pt}
    \item (Dr. Manuela Rottmann [BÜNDNIS 90/DIE GRÜNEN]: Nein!)
    \setlength\itemsep{-3pt}
    \item (Beifall bei Abgeordneten der AfD)
    \setlength\itemsep{-3pt}
    \item (Lachen bei Abgeordneten des BÜNDNISSES 90/DIE GRÜNEN)
\end{itemize}
\subsection{Scheer}
\noindent\textbf{Texts:} Sehr geehrter Herr Präsident! Liebe Kolleginnen und Kollegen! Die beiden vorliegenden Anträge bieten im Grunde genommen Anlass für eine zweite Lesung des bereits vorgelegten Gesetzentwurfs. Wir befinden uns im Gesetzgebungsverfahren; wir sind in guten Verhandlungen. Im Dezember haben wir eine sehr informative, lehrreiche Anhörung durchgeführt. Auch danach sind noch sehr viele Gespräche geführt worden. Wir sind ein gutes Stück vorangekommen. Ich möchte mich an dieser Stelle für die konstruktive Zusammenarbeit bedanken und, sofern meine Redezeit reicht, auf ein paar Dinge hinweisen – Herr Jung, gerne in unser beider Namen –, weil heute nicht der Eindruck entstehen sollte, dass man sich mit dem Gesetzentwurf bisher nicht auseinandergesetzt habe, zumal, wie wir alle wissen, in der Anhörung Aspekte aufgegriffen wurden – zum Beispiel ein möglicher Übersetzungsfehler, der in dem Richtlinienentwurf steckt –, mit denen wir uns auseinandersetzen sollten. Insofern empfinde ich diese Debatte über die Oppositionsanträge als so etwas wie eine eingeschobene zweite Lesung des Gesetzentwurfs. Man muss voranstellen, dass wir unterscheiden sollten – damit greife ich die harsche Kritik in den Oppositionsanträgen ein bisschen auf –; denn wir haben zwei unterschiedliche Richtlinien vor uns. Die eine ist jetzt umzusetzen; da sind wir in Verzug. Sie betrifft die Geschäftsgeheimnisse. Dann geht es um eine zweite Richtlinie, die bisher nur im Entwurf existiert. Diese ist als Whistleblower-Richtlinie in aller Munde. Die gilt es jetzt hier aber gerade nicht umzusetzen. Wir müssen darauf achten, dass wir keine unterschiedlichen Rechtsbegriffe in die Welt setzen; hier geht es jetzt tatsächlich um die Umsetzung einer Richtlinie zum Schutz von Geschäftsgeheimnissen, die in der Umsetzung als solche auch wiederzuerkennen sein muss. Was heißt das im Einzelnen? Zum einen geht es natürlich darum, dass das Geschäftsgeheimnis geschützt werden muss. Es ist aber klar, dass bei der Umsetzung der Richtlinie darauf geachtet werden muss, dass auch die Ausnahmevorschriften so umgesetzt werden, wie die Richtlinie das vorsieht. Das heißt, es muss tatsächlich ein Schutz für all die Bereiche vorgesehen werden, die ausgenommen werden sollen, für den Bereich der Pressefreiheit, aber auch für den Bereich der Vertretung von Arbeitnehmerinteressen. Diese Bereiche dürfen nicht durch einen radikalen Schnitt einfach vom Geschäftsgeheimnisschutz erfasst werden.  Insofern ist es wichtig, dass wir bei Artikel 1 § 5 des Entwurfs in Umsetzung des Artikels 5 der Richtlinie eine Konkretisierung hinbekommen. Ich bin froh, dass wir da eine Verständigung erreicht haben und der Meinung sind, dass das im Lichte der Richtlinie nur eine Ausnahme sein kann und nicht mit „Rechtfertigung“ betitelt werden kann.  Zum Zweiten ist eine Klarstellung erforderlich geworden – das haben Sie von der Opposition, wie Ihre Anträge zeigen, übrigens nicht erkannt –: Wir müssen klarstellen, wer nicht Rechtsverletzer ist. Das betrifft in Artikel 1 den § 8. Das ist zwar geregelt in § 5, betrifft aber § 8 und die dort geregelten Auskunftspflichten. Das ist, wie gesagt, uns in der Koalition aufgefallen. Auch da wollen wir eine Änderung vornehmen. Deswegen haben wir uns diesbezüglich verständigt. Ferner haben wir – das ist gerade schon von Herrn Jung angesprochen worden – die Objektivierung des Absichtsbegriffs in Angriff genommen. Der Absichtsbegriff, wie er jetzt gefasst ist, hat Schwächen. Da knüpfen wir an den Übersetzungsfehler an. In der Tat haben wir auch diesbezüglich eine Änderung im Blick. Auch der Anwendungsbereich des Gesetzes – Artikel 1 § 1 des Entwurfs – hat seine Tücken. Ich bin gleich am Ende meiner Redezeit angelangt, möchte aber noch sagen: Es muss klar sein, dass bei den Arbeitsverträgen und den spezialgesetzlichen Regelungen im Arbeitsrecht, im Bereich der Mitbestimmung, nicht reingegrätscht wird; so sage ich es jetzt einmal, weil meine Redezeit zu Ende ist.  In diesem Sinne: Ich freue mich auf die weiteren Gespräche, die wir dazu haben, und danke noch mal für die konstruktive Zusammenarbeit in alle Richtungen.  

\noindent\textbf{Comment:}
\begin{itemize}
    \setlength\itemsep{-3pt}
    \item (Beifall bei der SPD)
    \setlength\itemsep{-3pt}
    \item (Beifall bei der FDP)
    \setlength\itemsep{-3pt}
    \item (Beifall bei der SPD sowie bei Abgeordneten des BÜNDNISSES 90/DIE GRÜNEN und des Abg. Ingmar Jung [CDU/CSU])
    \setlength\itemsep{-3pt}
    \item (Beifall bei der SPD sowie des Abg. Ingmar Jung [CDU/CSU])
    \setlength\itemsep{-3pt}
    \item (Beifall bei Abgeordneten der SPD und des BÜNDNISSES 90/DIE GRÜNEN)
\end{itemize}
\subsection{Müller-Böhm}
\noindent\textbf{Texts:} Sehr geehrter Herr Präsident! Liebe Kolleginnen und Kollegen! Meine sehr geehrten Damen und Herren! Frau Scheer, ich glaube Ihnen wirklich, dass Sie sich darüber Gedanken gemacht haben und dass Sie dazu in Verhandlungen mit dem Koalitionspartner stehen. Ich nehme Ihnen auch die ernsthaften Bemühungen, die Sie gerade vorgetragen haben, ab. Eine Frage habe ich aber direkt zu Anfang: Warum kommt das so aus Ihrem Ministerium?  Es ist natürlich unsere Aufgabe als Oppositionsfraktion, auf Missstände hinzuweisen und kritisch mit Ihren Vorlagen umzugehen; ohne Zweifel. Wenn Sie sich jetzt so ausführlich Gedanken darüber machen, finde ich das ja gut. Aber ich frage mich schon: Warum kommt das so schlampig aus dem Ministerium? Das bleibt, ehrlich gesagt, offen.  Steigen wir also ein: Die Anträge von Grünen und Linken geben uns dankenswerterweise heute noch einmal die Chance, über den Gesetzentwurf zum Schutz von Geschäftsgeheimnissen zu sprechen. Ich habe es gerade schon gesagt: Es ist ein noch nicht zu Ende gedachter Gesetzentwurf. Ich möchte das ganz gerne an drei Beispielen verdeutlichen. Erstens. Es ist praktisch nicht hinnehmbar, dass in dem Gesetzentwurf bisher keine wirksamen Schranken für die Klassifizierung von Geschäftsgeheimnissen durch den Arbeitgeber bestehen. Ich möchte dafür ein Beispiel geben. Stellen Sie sich einen Apotheker vor, der wichtige Krebsmedikamente verkauft. Er streckt diese Medikamente so, dass sie quasi wirkungslos sind. Eine seiner Mitarbeiterinnen erfährt davon und sagt: Das finde ich eigentlich nicht so gut. – Dann stuft der Apotheker, der Arbeitgeber, das als Geschäftsgeheimnis ein, weil er damit ja schließlich sein Geld verdient. Man sollte eigentlich meinen, die Mitarbeiterin müsste das Ganze sinnvollerweise zur Meldung bringen und anzeigen. Aber sie würde – Stand jetzt – damit natürlich ihre wirtschaftliche Existenz enorm gefährden; denn das ist so nicht erlaubt. Das muss man auf jeden Fall ändern. Zweitens. Die Pressevertreter wurden bereits angesprochen. Auch denen wird dieser Gesetzentwurf weiterhin enorme Sorgen bereiten. Der Informantenschutz, so wie wir ihn bisher kennen und schätzen gelernt haben, wird im Grunde weitestgehend ausgehöhlt. Es ist aber nun einmal so, dass das ein elementarer Bestandteil der Pressefreiheit ist. Hinzu kommen auch noch unkalkulierbare Haftungsrisiken. Es kann ja eigentlich nicht sein, dass, wenn man über Missstände aufklären will, diese aber dann zu Geschäftsgeheimnissen erklärt werden, Aufklärung zu einem Verbrechen wird. Das ist doch unbegreiflich.  Drittens. Dass diese klaren, allgemeinen Schutzmechanismen nicht nur irgendwie gut für bestimmte Gruppen sind, sondern ganz allgemein auch für die Öffentlichkeit, das können wir auch an einem ganz konkreten Beispiel sehen. Es betrifft die SPD. Dazu müssen wir nach Hamburg schauen und uns einmal mit dem Rückkauf des Hamburger Wärmenetzes beschäftigen.  Dort ist Mitarbeitern aufgefallen, dass die Kalkulation zur Wertberechnung nicht korrekt ablief. Man könnte jetzt meinen, dass das Ganze irgendwie veröffentlicht werden konnte. Nein, das Ganze musste erst an die niederländische Behörde für Whistleblower gehen, damit man sich überhaupt mal mit diesen Informationen beschäftigt hat. Jetzt mag das vielleicht für Herrn Scholz, der damals Bürgermeister in Hamburg war, nur eine peinliche Unannehmlichkeit gewesen sein, aber den deutschen Steuerzahler hat das netto ungefähr 300 Millionen Euro gekostet. Das ist ein erheblicher Schaden für die Allgemeinheit und sicherlich nicht hinnehmbar.  Meine sehr geehrten Damen und Herren, Aufklärung von Missständen, die von gesellschaftlicher Bedeutung sind, darf kein Verbrechen sein. Wir brauchen klare Regeln, wie Menschen, die durch eine begründete Weitergabe von Informationen einen Dienst an der Allgemeinheit leisten, sinnvoll geschützt werden. Wir brauchen Ausnahmen für den Bereich der Presse, und wir brauchen klare Schutzregelungen für Whistleblower.  Ganz besonders kritisch zu sehen ist auch der vorgesehene Auskunftsanspruch des Geheimnisinhabers gegenüber dem Rechtsverletzer auf Offenlegung seiner Quellen. Das ist weder mit der Meinungsfreiheit deckungsgleich noch im Sinne der Richtlinie. Da muss dringend nachgearbeitet werden. Deswegen muss der Paragraf ersatzlos gestrichen werden.  Ich appelliere noch einmal an Sie: Nehmen Sie sich das bitte zu Herzen. Denken Sie an die Anhörung im Rechtsausschuss, und fangen Sie an, aus dem Gesetzentwurf etwas Besseres zu machen. Vielen Dank.  

\noindent\textbf{Comment:}
\begin{itemize}
    \setlength\itemsep{-3pt}
    \item (Beifall bei der FDP sowie bei Abgeordneten des BÜNDNISSES 90/DIE GRÜNEN und des Abg. Pascal Meiser [DIE LINKE])
    \setlength\itemsep{-3pt}
    \item (Beifall bei der FDP)
    \setlength\itemsep{-3pt}
    \item (Beifall bei Abgeordneten der FDP)
    \setlength\itemsep{-3pt}
    \item (Beifall bei der FDP sowie bei Abgeordneten des BÜNDNISSES 90/DIE GRÜNEN)
    \setlength\itemsep{-3pt}
    \item (Beifall bei der FDP sowie bei Abgeordneten des BÜNDNISSES 90/DIE GRÜNEN und des Abg. Niema Movassat [DIE LINKE])
    \setlength\itemsep{-3pt}
    \item (Benjamin Strasser [FDP]: Oh ja!)
    \setlength\itemsep{-3pt}
    \item (Beifall bei der FDP – Dr. Nina Scheer [SPD]: Ich habe es beantwortet!)
    \setlength\itemsep{-3pt}
    \item (Beifall bei der LINKEN)
\end{itemize}
\subsection{Movassat}
\noindent\textbf{Texts:} Herr Präsident! Meine Damen und Herren! Die Bundesregierung wollte ihr Geschäftsgeheimnisgesetz ursprünglich im Eiltempo durch den Bundestag bringen. Sie wollte, dass niemand mitbekommt, welche Einschränkungen für die Arbeit von Journalisten, Arbeitnehmervertretern und Whistleblowern geplant sind. Der Plan ging nicht auf, weil wir als Linke gemeinsam mit FDP und Grünen eine Anhörung im Rechtsausschuss durchgesetzt haben.  Das Urteil der Experten in dieser Anhörung war eindeutig. Es war vernichtend für den Entwurf des Geschäftsgeheimnisgesetzes, den die Bundesregierung vorgelegt hat. Deshalb beantragen wir als Linke und auch die Grünen heute eine Überarbeitung dieses Gesetzentwurfes. Liebe Kollegin Scheer, ich bin sehr froh, dass auch in Teilen der Koalition mittlerweile der Wunsch besteht, diesen Gesetzentwurf zu überarbeiten. Hätten Sie doch gleich einen ordentlichen Gesetzentwurf vorgelegt!  Ich will Ihnen noch einmal die drei zentralen Kritikpunkte nennen. Erstens. Laut Ihrem Gesetzentwurf ist im Prinzip alles ein Geschäftsgeheimnis, was einen wirtschaftlichen Wert hat. Was hat denn bitte keinen wirtschaftlichen Wert für ein Unternehmen am Markt? Nahezu alles. Der Begriff ist uferlos, und das hat Folgen, insbesondere für die Arbeit von Arbeitnehmervertretungen. Wie sollen Betriebsräte ihre Kollegen noch über irgendeinen Betriebsvorgang, zum Beispiel Personalabbaupläne, informieren können, wenn sie dafür mit einem Bein im Gefängnis stehen? Ihr Gesetzentwurf schützt völlig einseitig die Interessen der Arbeitgeber zulasten der Arbeitnehmer, und das lehnen wir ab.  Es gibt eine einfache Lösung, nämlich die Definition des Geschäftsgeheimnisses, wie sie die Rechtsprechung entwickelt hat. Danach ist ein Geschäftsgeheimnis eine Tatsache, an deren Geheimhaltung der Arbeitgeber ein berechtigtes Interesse hat. Diese Definition hat sich bewährt. Schreiben Sie sie doch ins Gesetz.  Zweitens. Die Anhörung machte deutlich, dass Ihr Gesetzentwurf investigativen Journalismus gefährdet. Schon heute wird etwa gegen den Chefredakteur von Correctiv, Oliver Schröm, ermittelt, weil er die Cum\/Ex-Geschäfte öffentlich machte. Solche Ermittlungen werden massiv zunehmen, wenn Ihr Gesetz so durchkommt; denn Sie führen mit § 23 einen neuen Straftatbestand ein, der die Arbeit von investigativen Journalisten unter Strafe stellt. Zwar kann das Handeln der Journalisten gerechtfertigt sein, aber das heißt eben, dass erst einmal ein Anfangsverdacht besteht und sich der Journalist dann möglicherweise vor Gericht verteidigen muss, um sozusagen gerechtfertigt aus dem Verfahren zu gehen. Das ist abschreckend für gute Pressearbeit, wie auch Journalistenverbände sagen. Mir ist rätselhaft, warum Sie diesen Weg gehen. Die EU-Richtlinie sieht ganz klar vor, dass sie keine Anwendung für den Bereich der Presse findet. Setzen Sie doch einfach die Richtlinie ordentlich um.  Drittens. Die Anhörung zeigte auch relativ deutlich, dass die Bundesregierung ein wenig Englischnachhilfe braucht. Ihr Gesetz sieht vor, dass Whistleblower, also Menschen, die Skandale an die Presse geben, in der Absicht handeln müssen, mit ihren Hinweisen dem öffentlichen Interesse zu dienen. Das ist eine Gesinnungsprüfung, die dem modernen Strafrecht zum Glück fremd ist. Die englischsprachige EU-Richtlinie spricht von „purpose“. Das meint – das haben die Experten in der Anhörung klar gesagt – in diesem Kontext „Zweck“. Das ist ein Übersetzungsfehler, der Ihnen unterlaufen ist. Das heißt, nach der Richtlinie handelt ein Whistleblower gerechtfertigt, wenn seine Veröffentlichung objektiv den Zweck erfüllt, dem öffentlichen Interesse zu dienen. Und so schreiben Sie es bitte auch ins Gesetz.  Meine Damen und Herren, ich will klar sagen: Geschäftsgeheimnisse gehören natürlich geschützt. Aber was die Bundesregierung vorgelegt hat, schießt weit über das Ziel hinaus und gefährdet die wichtige Arbeit von Journalisten, Arbeitnehmervertretern und Whistleblowern. Deshalb haben wir heute unsere Anträge vorgelegt, um Ihnen einen Denkanstoß zu geben und Ihnen zu zeigen, was Sie überarbeiten müssen. Folgen Sie den Anträgen, dann kriegen wir vielleicht einen vernünftigen Gesetzentwurf hin. Danke schön.  

\noindent\textbf{Comment:}
\begin{itemize}
    \setlength\itemsep{-3pt}
    \item (Beifall bei der LINKEN sowie des Abg. Roman Müller-Böhm [FDP])
    \setlength\itemsep{-3pt}
    \item (Beifall bei der LINKEN und dem BÜNDNIS 90/DIE GRÜNEN sowie des Abg. Roman Müller-Böhm [FDP])
    \setlength\itemsep{-3pt}
    \item (Beifall bei der LINKEN sowie bei Abgeordneten des BÜNDNISSES 90/DIE GRÜNEN)
    \setlength\itemsep{-3pt}
    \item (Beifall bei der CDU/CSU)
    \setlength\itemsep{-3pt}
    \item (Beifall bei der LINKEN)
\end{itemize}
\subsection{Hoffmann}
\noindent\textbf{Texts:} Danke. – Herr Präsident! Geschätzte Kolleginnen und Kollegen! Meine sehr verehrten Damen und Herren! Kollegin Rottmann, ich bin sehr dankbar, dass Sie uns darauf hingewiesen haben, dass wir Sie und Die Linke wegen Ihrer Gesetzentwürfe eigentlich zu einer Tasse Kaffee einladen müssten. Ich will mal sagen: Ich habe beide Gesetzentwürfe gelesen.  Für einen Kaffee reicht es nicht, maximal für ein Glas Wasser, und das haben Sie ja hier schon bekommen.  Liebe Kolleginnen und Kollegen, Spaß beiseite: Wir führen eine wichtige Debatte, die wir ja im Moment auch innerhalb der Großen Koalition führen. Tatsächlich geht es um zwei Zielrichtungen. Zielrichtung Nummer eins lautet: Wie können wir Geschäfts-, Betriebsgeheimnisse besser schützen? Das betrifft die Sicherung des Wirtschaftsstandortes Deutschland. Da geht es auch um den Schutz vor Wirtschaftsspionage. Zielrichtung Nummer zwei – mindestens genauso ebenbürtig – lautet: Wie können wir Hinweisgeber schützen, Menschen, die im Informationsinteresse der Öffentlichkeit Dinge zutage fördern, die strafbar sind und die unterbunden werden müssen? Ich will aber auch die Gelegenheit nutzen, um einmal die Schwäche Ihrer Anträge darzustellen.  Wenn man sie liest, dann stellt man fest, dass es eigentlich von der Schwerpunktsetzung im Wesentlichen immer nur um den Schutz von Hinweisgeberinnen und Hinweisgebern geht. Sie räumen diesem Aspekt von Anfang an ein überbordendes Gewicht ein.  Anders ist es nicht zu erklären, welches Gewicht Sie dem Schutzinteresse dieser Seite und dann auch den Journalisten von Anfang an einräumen wollen.  Dabei bekommen wir doch, liebe Kolleginnen und Kollegen, die verfassungsrechtliche Grenze nicht nur in Form von verschiedenen Beispielfällen durch die Rechtsprechung des BAG vorgelegt, sondern auch vom Bundesverfassungsgericht, wenn es uns aufzeigt, dass es am Ende des Tages immer eine Abwägungsentscheidung zwischen der Pressefreiheit zum Beispiel auf der einen Seite und dem Geheimhaltungsinteresse auf der anderen Seite ist. Dieses Verhältnis werden Sie durch gesetzliche Regelungen nicht anders gewichten bzw. verschieben können. Ein zweiter Punkt ist mir wichtig. Wir sollten aufpassen, dass wir bei der Debatte die Realität nicht ganz aus den Augen verlieren. Ich kann aus dem Bauch heraus nachvollziehen, wenn Sie in Ihren Anträgen fordern, dass wir Hinweisgeber vor Arbeitsplatzverlust, Zwangspensionierung, Karriereeinbußen und Mobbing schützen müssen. Ich will aber zwischen den Zeilen einmal darauf hinweisen, dass diese Konflikte heute schon, zumindest versuchsweise, gelöst werden. Schauen Sie sich einmal Urteile vom BAG an, bei denen am Ende eine Abfindung stand. Jetzt wird man natürlich sagen können: Eine Abfindung kann doch all das nicht ermöglichen, was wir hier fordern. – Aber, liebe Kolleginnen und Kollegen, ich will Ihnen schon sagen: Bestimmte Konstellationen der Lebenswirklichkeit werden Sie auch mit gesetzlichen Regelungen nicht einfangen können. Ich denke immer an den Fahrer, der den Gammelfleischskandal zutage gefördert hat. Sie können gesetzliche Regelungen so fassen, wie Sie wollen: Es wird am Ende des Tages so sein, dass es nicht möglich ist, eine Weiterbeschäftigungsmöglichkeit für den Fahrer in seinem Betrieb zu eröffnen, weil die sozialen Zerwürfnisse – egal ob gerechtfertigt oder nicht – im Betrieb so stark sind, dass es im Interesse des Mannes liegen muss, dort nicht mehr weiterzuarbeiten. Zum Schluss möchte ich noch auf einen Punkt eingehen, auf das Thema Geschäftsgeheimnis. Da heißt es, wir brauchen eine offizielle Legaldefinition. Dazu muss man sagen – Herr Kollege Movassat, Sie haben es gesagt –, es gibt von der Rechtsprechung diesbezüglich schon eine Definition.  – Hören Sie mir doch mal zu. – Wenn Sie den Referentenentwurf und die Begründung lesen, dann wird doch klar, dass alles zunächst auf dieser Definition basiert und dann in § 2 Nummer 1 noch zwei Aspekte ergänzt werden. Es ist eben nicht so, wie Sie und auch Sie, Herr Kollege Müller-Böhm, sagen, dass der Arbeitgeber einseitig festlegen kann, was ein Geschäftsgeheimnis ist und was nicht, weil in der Begründung – Kollege Movassat, vielleicht beantwortet das Ihre Frage – ausdrücklich steht, dass ein legitimes Interesse erforderlich sein muss. Ich glaube, dass die nächste Debatte, die wir in diesem Bereich führen, sehr fruchtbar sein wird. Wir kommen mit Sicherheit zu guten Ergebnissen. Vielen Dank für die Aufmerksamkeit.  

\noindent\textbf{Comment:}
\begin{itemize}
    \setlength\itemsep{-3pt}
    \item (Beifall bei der SPD)
    \setlength\itemsep{-3pt}
    \item (Helin Evrim Sommer [DIE LINKE]: Normale Anträge!)
    \setlength\itemsep{-3pt}
    \item (Beifall bei Abgeordneten der CDU/CSU)
    \setlength\itemsep{-3pt}
    \item (Helin Evrim Sommer [DIE LINKE]: So ist das! Ist ja auch richtig!)
    \setlength\itemsep{-3pt}
    \item (Niema Movassat [DIE LINKE]: Ja, vernünftigerweise!)
    \setlength\itemsep{-3pt}
    \item (Beifall bei der CDU/CSU)
    \setlength\itemsep{-3pt}
    \item (Dr. André Hahn [DIE LINKE]: Ja, dann schreiben Sie sie rein!)
    \setlength\itemsep{-3pt}
    \item (Dr. André Hahn [DIE LINKE]: Die gibt es gar nicht!)
\end{itemize}
\subsection{Rabanus}
\noindent\textbf{Texts:} Herr Präsident! Meine sehr verehrten Damen und Herren! Liebe Kolleginnen und Kollegen! Bevor wir diese parlamentarische Woche mit einer Aktuellen Stunde zum Sozialstaatskonzept der SPD beenden – herzlichen Dank an die FDP, dass Sie uns ermöglichen, dies in der Tagesordnung zu behandeln –,  beschäftigen wir uns noch mit zwei Anträgen zum Schutz des Geschäftsgeheimnisses von Grünen und Linken. Beide Anträge, meine Damen und Herren, sind von unterschiedlicher Qualität. Der Antrag der Linken hat eher Sammelsuriumcharakter mit mehr oder weniger substanziellen Punkten. Der Antrag der Grünen atmet ein bisschen mehr Beschäftigung mit der Sache, aber die nötige Qualität hat er auch nicht.  Für Qualität wäre es nötig gewesen, einen Änderungsantrag zu dem Gesetzentwurf einzubringen, der im Verfahren ist. Dafür hat es offensichtlich nicht gereicht. Aber okay, auch das nehmen wir so zur Kenntnis.  Natürlich ist es richtig: Bei der Anhörung im Rechtsausschuss Mitte Dezember zu dem Gesetzentwurf der Bundesregierung hat man sich kritisch mit einigen Punkten auseinandergesetzt,  insbesondere auch bei dem im Zentrum stehenden § 5 mit der Frage: Handelt es sich um eine Konstruktion über Rechtfertigungsgründe oder um eine Bereichsausnahme? Das ist von den Sachverständigen in unterschiedlicher Weise aufgegriffen worden. Verkürzt gesagt – vielleicht noch einmal für die Zuhörer auf der Tribüne –, lautet die Kernfrage: Muss ein Journalist gegebenenfalls erst vor Gericht erstreiten, straffrei zu bleiben, wenn er einen Skandal aufdeckt, oder ist von vornherein klar, dass es in seinen Möglichkeiten liegt, das straffrei zu machen? Da will ich, liebe Kolleginnen und Kollegen, auch gar kein Geheimnis aus meiner eigenen Position machen. Ich glaube, dass wir mit diesem Gesetz das Ziel verfolgen müssen, eindeutige und praxistaugliche Lösungen zu finden – so eben wie die Arbeit der Journalistinnen und Journalisten tatsächlich im wahren Leben ist. Wir brauchen ein Gesetz, das unterstützt und nicht erschwert. Es ist, glaube ich, das gemeinsame Ziel der Koalition, zu schauen: Wie können wir am Ende auf der Basis des Gesetzentwurfes des BMJV dieses Ziel erreichen? Auch ich glaube, dass wir die Bedenken in Bezug auf die Rechtfertigungslösung sehr ernst nehmen müssen. Ich glaube, dass wir mit der Bereichsausnahme für Journalistinnen und Journalisten einen klareren und praxistauglicheren Weg hätten,  übrigens inklusive der Frage des Schutzes von Informanten.  Ich bin den federführenden Berichterstattern Nina Scheer und Ingmar Jung – beide haben hier schon gesprochen – sehr dankbar, dass sie sich in sehr konstruktiven Gesprächen befinden. Es ist auch deutlich geworden, dass wir noch nicht in allen Punkten entscheidungsreif sind.  So ist das im normalen Leben. Auch die Frage nach dem berechtigten Interesse, das zur Begründung eines Geschäftsgeheimnisses notwendig ist, gehört dazu. Ich bin mir aber auch ganz sicher, dass es für uns als SPD ohnehin und für die Koalition insgesamt ein grundlegender Wert ist, Presse- und Medienfreiheit, insbesondere in Zeiten von Fake News, zu stärken und abzusichern. Unser Ziel muss es sein, die Informationsfreiheit zu schützen, damit Journalistinnen und Journalisten ihre Arbeit gut machen können. Es kommt auf Rahmenbedingungen an, die geeignet sind, und nicht auf die Frage, ob man eine gute Rechtschutzversicherung hat, um journalistisch arbeiten zu können.  Ich bin guter Dinge, dass die Koalition in der nächsten Zeit auch die letzten noch unklaren Fragen klären kann.  Dann gilt, vielleicht etwas abgewandelt, das Struck’sche Gesetz: Jedes Gesetz, auch dieses, verlässt den Bundestag noch ein bisschen besser, als es reingekommen ist. Herzlichen Dank.  

\noindent\textbf{Comment:}
\begin{itemize}
    \setlength\itemsep{-3pt}
    \item (Beifall bei der SPD sowie bei Abgeordneten der CDU/CSU)
    \setlength\itemsep{-3pt}
    \item (Beifall bei der SPD)
    \setlength\itemsep{-3pt}
    \item (Zuruf der Abg. Dr. Manuela Rottmann [BÜNDNIS 90/DIE GRÜNEN])
    \setlength\itemsep{-3pt}
    \item (Beate Müller-Gemmeke [BÜNDNIS 90/DIE GRÜNEN]: Sehr kritisch! – Dr. Manuela Rottmann [BÜNDNIS 90/DIE GRÜNEN]: Überaus kritisch!)
    \setlength\itemsep{-3pt}
    \item (Beifall der Abg. Dr. Nina Scheer [SPD])
    \setlength\itemsep{-3pt}
    \item (Beifall bei der CDU/CSU)
    \setlength\itemsep{-3pt}
    \item (Widerspruch beim BÜNDNIS 90/DIE GRÜNEN)
    \setlength\itemsep{-3pt}
    \item (Dr. André Hahn [DIE LINKE]: Gucken wir einmal, was herauskommt!)
    \setlength\itemsep{-3pt}
    \item (Beifall bei Abgeordneten der SPD)
    \setlength\itemsep{-3pt}
    \item (Beate Müller-Gemmeke [BÜNDNIS 90/DIE GRÜNEN]: Bis wann denn?)
    \setlength\itemsep{-3pt}
    \item (Beifall bei Abgeordneten der SPD – Roman Müller-Böhm [FDP]: Immer wieder gerne!)
\end{itemize}
\subsection{Hirte}
\noindent\textbf{Texts:} Herr Präsident! Liebe Kolleginnen und Kollegen! Wir reden hier eigentlich über etwas ganz anderes als über das, was die Anträge insinuieren; denn es sind keine Anträge, die im luftleeren Raum stehen, sondern – wir haben es schon mehrfach gehört – eigentlich handelt es sich um Änderungsanträge zu einem Gesetzentwurf, die außerhalb des eigentlichen Verfahrens eingebracht werden. Eigentlich geht es um noch etwas anderes. Beide Anträge beschäftigen sich ja im Wesentlichen mit dem Whistleblowing und gar nicht – auf einige Punkte komme ich gleich noch zu sprechen – so sehr mit den Fragen des Schutzes von Geschäftsgeheimnissen. Es geht darum, eine Richtlinie zu beeinflussen, die gerade in Brüssel verhandelt wird. Deshalb muss man die Anträge eigentlich so interpretieren, dass Sie eine Stellungnahme der Bundesregierung zu diesem zweiten Richtlinienpaket erzwingen wollen. Eine gute Idee – das muss ich ganz deutlich sagen –, aber das ist nicht das, was Sie in den Anträgen vorgelegt haben. Lassen Sie mich zu den einzelnen Punkten etwas sagen. Der erste Punkt betrifft Whistleblowing, etwas, was Ihnen wirklich am Herzen liegt. Die EU ist dabei, die zweite Richtlinie vorzubereiten. Wir haben darüber geredet. Alle Überlegungen, die wir hier haben, speisen wir über die Bundesregierung auch auf die europäische Ebene ein. Wenn in Ihrem Antrag, Frau Rottmann, steht, es fehle die zukunftsweisende Positionierung, so muss ich dazu sagen, dass das, was wir hier machen, nämlich auf der einen Seite Geschäftsgeheimnisse zu schützen und auf der anderen Seite Rechtssicherheit für Whistleblower zu schaffen und beides gegeneinander abzuwägen, ein wunderbares rechtspolitisches Ziel ist. Es geht um eine richtige dogmatische Strukturierung unseres Rechts, die wir in dieser Weise – bisher haben wir in diesem Bereich nur Richterrecht und verstreutes Recht – noch nicht haben. Das wäre ein großer Schritt nach vorne.  Wenn Sie dann sagen, der Whistleblower-Schutz müsse deutlich gestärkt werden – auch da geht es nicht um Geschäftsgeheimnisse –, muss man auch überlegen, worum es wirklich geht. Whistleblower-Schutz ist richtig und nötig. Die arbeitsgerichtliche Rechtsprechung hat das schon entwickelt. Aber es geht auch um das Interesse der Allgemeinheit bzw. der Öffentlichkeit. Bei Interna des Unternehmens – dazu gehören auch die Arbeitsbedingungen – muss man unter Umständen die Abwägung etwas anders vornehmen. Lassen Sie mich auf einige weitere Einzelheiten eingehen. Sie sagen, es müsste eigentlich eine allgemeine Regelung zum Whistleblower-Schutz nach der Whistle­blower-Richtlinie geben; der Fall, über den wir im Augenblick reden, sei doch eigentlich nur ein Spezialfall. – Ja, ich glaube, da haben Sie recht. Wir müssten eigentlich die später kommende Richtlinie als Grundlage der allgemeinen Regelung definieren. Die Ausnahmelösung, die wir hier im Augenblick haben, ist möglicherweise nur eine Interimslösung. Darüber müssen wir reden. In beiden Anträgen wird angesprochen, dass die Definition des Geschäftsgeheimnisses zu unbestimmt sei. Kollege Jung hat schon erläutert, worauf diese Begriffsbestimmung zurückzuführen ist, nämlich dass man sie im Zusammenhang mit der Begründung sehen muss.  Ich möchte nur darauf hinweisen: Diese Diskussion haben wir beim Geschäftsgeheimnisschutz an vielen Stellen, etwa, wenn es um das Kartellrecht geht, wenn es darum geht, was wirklich – sozusagen als Monopol – geschützt ist. Es geht auch darum, dass Geschäftsgeheimnisse wirklich faktisch geschützt werden. Genau das wird auch in der Richtlinie zum Ausdruck gebracht. Das berechtigte Interesse alleine, Kollege Movassat, reicht nicht. Es gibt auch andere Fälle, in denen Zahlen einer Geheimhaltung unterliegen und es nicht anschließend dem Richter überlassen sein kann, diese wirklich im Interesse des Unternehmens schützenswerten Zahlen der Öffentlichkeit preiszugeben. Einen letzten Punkt will ich ansprechen, weil er bisher nicht erwähnt wurde, nämlich die Frage des In-Camera-Verfahrens. Zu Recht weisen Sie im Grünenantrag auf das Verfahren hin. Was mich nur wundert: Ein In-Camera-Verfahren, das über das hinausgeht, was jetzt im Gesetzentwurf steht, entspricht genau dem, was normalerweise im Rahmen von Schiedsverfahren durchgeführt wird. Dass Sie hier etwas fordern, was Sie anderswo ablehnen, ist schon erstaunlich; aber wir gehen darauf noch weiter ein. Herzlichen Dank. 

\noindent\textbf{Comment:}
\begin{itemize}
    \setlength\itemsep{-3pt}
    \item (Zuruf der Abg. Beate Müller-Gemmeke [BÜNDNIS 90/DIE GRÜNEN])
    \setlength\itemsep{-3pt}
    \item (Beifall bei der CDU/CSU sowie bei Abgeordneten der SPD)
\end{itemize}
\section{Tagesordnungspunkt 20}
\subsection{Höferlin}
\noindent\textbf{Texts:} Sehr geehrter Herr Präsident! Liebe Kolleginnen und Kollegen! Nur 37 Prozent der Bevölkerung trauen der Bundesregierung zu, dass sie fähig ist, die Digitalisierung voranzutreiben; das sagt eine Studie des Vodafone-Instituts. – Gratulation, ich hätte nicht so viel erwartet. Eine andere Studie aus dem Jahr 2017 sagt voraus, dass bereits im kommenden Jahr mehr als 20 Milliarden Geräte im Internet der Dinge miteinander verbunden sein werden. 20 Milliarden Geräte, das ist dreimal so viel, wie wir Menschen auf der Welt haben. Ich weiß nicht, wie es Ihnen geht und was das bei Ihnen auslöst – mich macht diese Zahl sehr nervös; denn die Bundesregierung hängt bei der IT-Kompetenz streckenweise immer noch in der Vorschule fest. Dafür gibt es zahlreiche Beispiele. Fragen Sie doch mal die Bürger, wer eigentlich hier für die IT-Sicherheit zuständig ist.  Da kommt dann wahrscheinlich die Antwort: Irgendeiner der fünfzehn oder sechzehn Digitalminister wird es wohl sein. – Das ist desaströs; denn die IT-Sicherheit ist die Achillesferse des Informationszeitalters.  Die Bundesregierung vernachlässigt die IT-Sicherheit, aber nicht nur. Sie handelt auch teilweise unkoordiniert, in der Sache sogar schädlich. Die Bürger sind verunsichert. Die Bundesregierung setzt dem Ganzen die Krone auf, weil immer wieder die gleichen Reflexe ziehen, immer wieder das Gleiche wiederholt wird. Ich kann es nicht oft genug sagen: Der Versuch, das Internet vollumfänglich zu kontrollieren, wird scheitern. Der Staat muss sich auf die Kernaufgaben konzentrieren, nämlich auf den Schutz der Bürgerinnen und Bürger und auf den Schutz der wichtigen Infrastruktur.  Was brauchen wir dafür? In unserem Antrag haben wir das Wichtigste zusammengefasst. Liebe Bundesregierung, Sie können sich gerne bedienen – im Namen der Sicherheit ist Ihnen ja auch sonst jedes Mittel recht.  Worum geht es im Kern? Erstens. Wir sind überzeugt davon, dass Sie Sicherheitslücken schließen müssen, anstatt sie für eigene Zwecke offen zu halten, beispielsweise für den Staatstrojaner. Zweitens. Hören Sie endlich auf, Bürgerinnen und Bürger an vielen Stellen zu überwachen, beispielsweise im Rahmen der Vorratsdatenspeicherung oder mit den Staatstrojaner.  Fangen Sie an, die IT-Sicherheit zu stärken! Kommen wir zu den Sicherheitslücken. Die Sicherheitslücke selbst ist ein Sicherheitsrisiko; der Name sagt es ja schon. Die erste und letzte Linie der Verteidigung ist das Bundesamt für Sicherheit in der Informationstechnik, BSI. Das BSI ist dem Innenministerium nachgeordnet. Unser Heimatminister ist heute wieder nicht da; der geschätzte Kollege Krings nimmt die Position ein. Sie merken selber: Es gibt da einen gewissen Interessenkonflikt; denn beim Innenministerium sind auch das Bundeskriminalamt und das Bundesamt für Verfassungsschutz angesiedelt. Es geht darum, Sicherheitslücken zu schließen. Andere wollen sie offen halten und für eigene Zwecke nutzen – das kann nur schiefgehen.  Deshalb wollen wir Freie Demokraten, dass das BSI aus der Zuständigkeit des Innenministeriums herausgelöst wird. Zusätzlich muss das BSI eine zentrale Meldestelle werden – überhaupt eine zentrale Stelle – für die IT-Sicherheit in Deutschland. Mein Kollege Benjamin Strasser hat in der letzten Woche einen hervorragenden Vorschlag dazu gemacht: Lassen Sie uns doch bitte die Sicherheit – auch die IT-Sicherheit; die gehört dahin – im Rahmen einer Föderalismuskommission III intensiv besprechen; das würde die IT-Sicherheit deutlich stärken.  Nun noch ein Wort zur digitalen Selbstverteidigung. Stellen Sie sich vor, Sie werden gehackt und Sie können trotzdem ruhig schlafen, Sie können ganz in Ruhe Ihre Serie auf Netflix oder wo auch immer zu Ende schauen; denn Sie wissen: Es wird nichts Schlimmes passieren.  Viele von Ihnen\/uns haben zwischen den Jahren\/Anfang Januar eine solche Situation erlebt. Was ich beschrieben habe, wäre aber möglich, wenn es ein wirksames Recht auf Verschlüsselung gäbe, wenn die Daten und die E-Mails, die abgeflossen sind, eben verschlüsselt wären. Das Bundesverfassungsgericht hat vor über zehn Jahren das IT-Grundrecht entwickelt. Ich fordere die Bundesregierung auf, endlich nachzusteuern und dort auch wirklich Sicherheit für die Bürgerinnen und Bürger zu gewährleisten.  Sie finden in unserem Antrag noch viele weitere Ideen. Acht Seiten lassen sich etwas schwierig in vier Minuten quetschen. Der Präsident fängt auch schon an zu blinken  – zumindest hier vorne –; deswegen komme ich auch zum Ende. All diese Vorschläge müssen natürlich noch mit einem Satz enden: Es wäre deutlich einfacher, Sie würden die IT-Kompetenz und -Steuerung in ein federführendes und zugleich koordinierendes Digitalministerium einbinden.  Dann wären wir einen Schritt weiter, das würde die IT-Sicherheit voranbringen. Herzlichen Dank.    

\noindent\textbf{Comment:}
\begin{itemize}
    \setlength\itemsep{-3pt}
    \item (Beifall bei Abgeordneten der FDP und des BÜNDNISSES 90/DIE GRÜNEN)
    \setlength\itemsep{-3pt}
    \item (Beifall bei der FDP)
    \setlength\itemsep{-3pt}
    \item (Beifall bei Abgeordneten der CDU/CSU und der SPD)
    \setlength\itemsep{-3pt}
    \item (Saskia Esken [SPD]: Ach nee!)
    \setlength\itemsep{-3pt}
    \item (Saskia Esken [SPD]: Ho!)
    \setlength\itemsep{-3pt}
    \item (Beifall bei der FDP und dem BÜNDNIS 90/DIE GRÜNEN)
    \setlength\itemsep{-3pt}
    \item (Dr. Konstantin von Notz [BÜNDNIS 90/DIE GRÜNEN]: Die Opposition!)
    \setlength\itemsep{-3pt}
    \item (Heiterkeit)
    \setlength\itemsep{-3pt}
    \item (Michael Theurer [FDP]: Herr Präsident, die sind geistig präsent! – Katharina Dröge [BÜNDNIS 90/DIE GRÜNEN]: Wir waren eben noch zahlreich vertreten!)
    \setlength\itemsep{-3pt}
    \item (Beifall bei der CDU/CSU)
    \setlength\itemsep{-3pt}
    \item (Christoph Bernstiel [CDU/CSU]: Kommt auf die Seite an!)
    \setlength\itemsep{-3pt}
    \item (Beifall bei Abgeordneten der FDP)
    \setlength\itemsep{-3pt}
    \item (Beifall bei der FDP sowie bei Abgeordneten des BÜNDNISSES 90/DIE GRÜNEN)
\end{itemize}
\subsection{Bernstiel}
\noindent\textbf{Texts:} Sehr geehrter Herr Präsident! Werte Kolleginnen und Kollegen! Liebe Gäste! Zunächst einmal vielen Dank an die FDP, dass Sie dieses wichtige Thema der Cybersicherheit immer wieder auf die Tagesordnung setzen.  Dafür gebührt Ihnen wirklich Dank. Das muss man einfach einmal feststellen. Ich danke Ihnen vor allen Dingen dafür, dass wir als Koalitionsfraktionen die Möglichkeit bekommen, Ihnen noch einmal zu erklären, was wir schon alles in diesem Bereich tun.  Ich will zwei Punkte aufgreifen, die in Ihrem Antrag durchaus richtig sind. Der erste Punkt ist die Haushaltsausstattung. Sie sagen: Wir können nicht nur mehr Staat, mehr Sicherheitsbehörden fordern, sondern wir müssen auch dafür sorgen, dass sie mit den entsprechenden Haushaltsmitteln und personell ausgestattet werden. – Das haben wir in den letzten Haushaltsverhandlungen getan. Ich hoffe, dass Ihre Fraktion dann auch dabei ist, wenn es in diesem Jahr darum geht, weitere Maßnahmen zu finanzieren. Der zweite Punkt ist die Kooperation zwischen Bund und Ländern. Daran arbeiten wir schon. Das Leak im Januar hat gezeigt: Dies hätte vermieden werden können, wenn wir rechtzeitig Bescheid gewusst hätten, was in den Ländern passiert. Da gebe ich Ihnen explizit recht. Sie haben viele gute Punkte genannt. Einiges – das haben Sie selbst gesagt – ist aufgewärmt. Das hören wir hier regelmäßig. Ein Problem bleibt aber nach wie vor bestehen: Sie erkennen nicht, dass wir bereits handeln. Es hilft nun einmal nicht, wenn Sie immer nur schlaue Anträge stellen und auffordern, aber ignorieren, dass die Bundesregierung in diesem Bereich bereits alles tut.  Ich will Ihnen gerne zwei Punkte nennen. Gerade in dieser Woche gab es zwei Berichterstattergespräche zum neuen IT-Sicherheitsgesetz 2.0, das umfangreiche Maßnahmen enthält, die unsere IT-Infrastruktur verbessern werden. Eines fand gerade heute Morgen statt. Dieser Gesetzentwurf wird noch in diesem Jahr in das Plenum eingebracht. Wir werden damit einen entscheidenden Beitrag leisten, um unsere IT-Sicherheitsstruktur zu verbessern. Vor zwei Wochen haben Bundesminister Seehofer, der im Übrigen gerade bei der Sicherheitskonferenz in München ist – das für alle, die sich fragen, wo er gerade ist –, und Ursula von der Leyen eine neue Agentur ins Leben gerufen – die Agentur für Cybersicherheit –, die dankenswerterweise auch noch in meinem Wahlkreis angesiedelt werden wird.  Das BMI plant noch für dieses Jahr eine Informationskampagne für mehr Cybersicherheit, die deutschlandweit ausgerollt werden soll. Wir haben bereits im vergangenen Jahr mehr Stellen in nahezu allen Sicherheitsbehörden im Bereich der IT genehmigt und auch auf den Weg gebracht. Es ist also bei weitem nicht so, dass die Bundesregierung nichts tut, sondern wir sind dabei, umzusetzen. Liebe FDP, hätten Sie uns bei Jamaika nicht auf dem letzten Meter den Rücken gekehrt, dann wären Sie mit dabei, wüssten das alles und müssten jetzt nicht meinem Vortrag gespannt zuhören.  Zum Zweiten. Es gibt auch Punkte in Ihrem Antrag, die wir durchaus sehr kritisch sehen. Eines verwundert mich besonders: Sie fordern mehr Staat. Sie sagen: Der Staat ist zuständig für die ganze IT-Sicherheit. – Dabei sind Sie doch die liberale Partei und sagen immer: Der Staat soll sich möglichst wenig einmischen. – Was möchten Sie denn eigentlich? Wissen Sie, was wir in diesem Bereich wollen? Das haben Sie ausgelassen. Wir wollen den Nutzer in die Selbstverantwortung bringen und ihn stärken. Auch darüber müssen wir einmal reden, wenn wir über IT-Sicherheit reden. Jedes Netz ist nur so sicher wie der Anwender, der es verwendet.  Jetzt schauen wir uns einmal ein paar Zahlen an. 61 Prozent der Deutschen geben an, dass sie ein und dasselbe Passwort für mehrere Onlineaccounts benutzen. 6 Prozent geben sogar an, dass sie nur ein einziges Passwort für alle Zugänge im Internet haben.  Auch an dieser Stelle müssen wir ansetzen. Es hilft nichts, dass wir die beste Firewall haben, wenn der Schlüssel dafür direkt vor der Haustür liegt; das ist eine super Analogie. Die zehn häufigsten Passwörter sind: „Hallo“, „12345“ und – Achtung, jetzt seit neuem – „123456“.  Das Wort „Passwort“ ist nach wie vor unter den Top Ten der am meisten verwendeten Passwörter. Da müssen wir ansetzen. Genau das machen wir auch, indem wir im IT-Sicherheitsgesetz verankert haben, dass wir das BSI stärken wollen.  Das BSI soll in Zukunft die zentrale Anlaufstelle auch für Verbraucher sein. Das BSI ist es im Übrigen schon jetzt. Aber das Problem ist: Kaum jemand kennt die zentrale Hotline des BSI. Ich nenne die Nummer noch einmal: 0800 274 1000. Wer also Fragen hat, kann gerne beim BSI anrufen. Es berät schon heute. Wir wollen das in Zukunft weiter ausbauen und den Bürgern die Möglichkeit geben, zu unterscheiden. Ein ganz entscheidender Punkt, den ich nicht außen vor lassen möchte, ist das IT-Sicherheitskennzeichen. Das wird extra „Kennzeichen“ genannt, weil „Siegel“ etwas Statisches ist. Wir möchten den Bürgern in Zukunft die Möglichkeit geben, zu unterscheiden. Was ist billiger: IoT-Schrott aus dem Ausland, der überhaupt keine Sicherheitskennzeichen hat und keine Updateanforderungen erfüllt, oder Produkte, die zumindest in der Lage sind, Mindeststandards zu generieren? Das wollen wir einführen. Dieses IT-Sicherheitskennzeichen – Sie können sich darauf freuen – wird kommen.  Der Begriff „IT-Sicherheitskennzeichen“ ist natürlich etwas sperrig. Deswegen haben sich schlaue Leute im Ministerium einen tollen Begriff dafür einfallen lassen. Das ist der sogenannte elektronische Beipackzettel. Dieser wird maßgeblich dazu beitragen, unseren Bürgern in Zukunft mehr Orientierung zu geben, was sichere und unsichere Geräte sind. In Zukunft könnte es dann hoffentlich heißen: Bei Risiken und Nebenwirkungen im IT-Bereich fragen Sie das BSI oder Ihre Bundesregierung.  Herzlichen Dank.  

\noindent\textbf{Comment:}
\begin{itemize}
    \setlength\itemsep{-3pt}
    \item (Dr. Konstantin von Notz [BÜNDNIS 90/DIE GRÜNEN]: Hören Sie auf mit der Bürgerbeschimpfung! Das ist unerträglich!)
    \setlength\itemsep{-3pt}
    \item (Dr. Konstantin von Notz [BÜNDNIS 90/DIE GRÜNEN]: Ja, wir rufen die Hotline an!)
    \setlength\itemsep{-3pt}
    \item (Beifall bei der FDP)
    \setlength\itemsep{-3pt}
    \item (Beifall bei der AfD)
    \setlength\itemsep{-3pt}
    \item (Manuel Höferlin [FDP]: Nein, digitale Selbstverteidigung! – Dr. Konstantin von Notz [BÜNDNIS 90/DIE GRÜNEN]: Das erklären Sie mal den Leuten!)
    \setlength\itemsep{-3pt}
    \item (Beifall bei der CDU/CSU und der SPD)
    \setlength\itemsep{-3pt}
    \item (Beifall bei der CDU/CSU – Sebastian Hartmann [SPD]: Da hat er recht!)
    \setlength\itemsep{-3pt}
    \item (Manuel Höferlin [FDP]: 8 Prozent der Bundestagesabgeordneten benutzen nur ein Passwort!)
    \setlength\itemsep{-3pt}
    \item (Manuel Höferlin [FDP]: Das gönne ich Ihnen, Herr Kollege!)
    \setlength\itemsep{-3pt}
    \item (Manuel Höferlin [FDP]: Gerne! Ich bin gespannt!)
    \setlength\itemsep{-3pt}
    \item (Manuel Höferlin [FDP]: Sie verstecken das aber gut!)
    \setlength\itemsep{-3pt}
    \item (Dr. Konstantin von Notz [BÜNDNIS 90/DIE GRÜNEN]: Das BSI generiert jetzt Passwörter für Menschen, oder was?)
    \setlength\itemsep{-3pt}
    \item (Manuel Höferlin [FDP]: Das unterstützen wir!)
\end{itemize}
\subsection{Schulz}
\noindent\textbf{Texts:} Herr Präsident! Meine Damen und Herren! Dieser Tagesordnungspunkt widmet sich der IT-Sicherheit im Zeitalter der Digitalisierung, und das ist gut so. Wir als AfD legen mit unserem Antrag den Fokus auf das vieldiskutierte 5G-Netz, ein Netz, das für kurze Antwortzeiten im Datenverkehr vorgesehen ist und das nicht manipulierbar oder gar abschaltbar sein darf. Die Frequenzversteigerungen stehen unmittelbar bevor. Aufgabe der Politik ist es nun, einen sicheren Netzaufbau zu gewährleisten. Aber es gibt eine Schwachstelle im System. Die heißt wieder einmal: Dr. Peter Altmaier. Er ist nicht da. In der Energiewende wirft der Rechnungshof ihm Versagen vor. Seine sogenannte Gründungsoffensive floppte. Mit der Nationalen Industriestrategie 2030 zeichnet sich das Gleiche ab.  Dieser Minister soll nun 5G zu einem Erfolg führen. Aber hier geht es um mehr als nur darum, ein paar Strategiepapiere auf den Tisch zu legen. Es geht um die Sicherheit unserer Daten. Es geht darum, die nicht zählbaren Interaktionen in der neuen Datenwelt so durchzuführen, dass weder Bürger noch Industrie Schäden davontragen. Es geht darum, dass unsere privaten Daten nicht ausgespäht, dass die Produktionsgeheimnisse unserer Wirtschaft nicht nach Fernost abgeleitet werden. Manipulationsmöglichkeiten von vornherein zu minimieren, meine Damen und Herren, das ist das Ziel unseres Antrags. Für die AfD stellt 5G insgesamt eine kritische Infrastruktur da. Das muss auch gesetzlich so verankert werden: im Telekommunikationsgesetz und im Gesetz für Cybersicherheit.  Eine große Gefahr für die 5G-Sicherheit lässt sich schon heute ausmachen: Huawei. Dieses Unternehmen ist ein Netzwerklieferant mit engen Bindungen an den chinesischen Staat und somit schon per Gesetz zur Zusammenarbeit mit dem dortigen Geheimdienst verpflichtet. Viele Industrieländer sehen hier ein Risiko. Aber die Bundesregierung, die wieder ohne Minister hier sitzt, eiert da herum. Schon unter Präsident Obama wurden in den USA chinesische Staatskonzerne als Gefahr für die nationale Sicherheit ausgelistet. In Kanada, Neuseeland und Australien wird das ebenso gehandhabt. Auch Regierungen in Europa haben erkannt, um was es geht. In Großbritannien, in Norwegen, in Frankreich und auch in Polen schrillen die Alarmglocken wegen Huawei laut. Aber unsere Bundesregierung zögert und blockiert. Im Oktober erhielt die AfD von der Bundesregierung die Antwort, dass die generelle Abschottung öffentlicher Netzinfrastruktur gegen bestimmte Anbieter kein adäquater Schutzmechanismus sei. Dann ließ man verlauten, dass die Willensbildung der Regierung noch nicht abgeschlossen ist. Nicht viel schlauer wurden wir durch Bundeskanzlerin Merkel. Vor wenigen Tagen sagte sie in ihrer eigenen einfachen Sprache, konkret bezogen auf Huawei – ich zitiere –: Es geht darum, „dass eben nicht die Firma einfach die Daten an den Staat abgibt, die verwendet werden, sondern dass man da Sicherheiten bekommt“. „Dass man da Sicherheiten bekommt“ – ja, was meint sie denn, die Frau Merkel, damit? Was ich meine, ist klar: Mit einfacher Sprache und mit simplen Gedanken wird das Problem jedenfalls nicht gelöst.  Um es noch zu verdeutlichen: Sind die chinesischen Bauteile erst einmal drin in 5G, dann bleiben sie auch drin. Es gibt kein Zurück mehr, Frau Bundeskanzlerin. Was wir jetzt benötigen, ist aktives Handeln durch die Regierung. Im Koalitionsvertrag 2013 wurde formuliert: Wir brauchen Maßnahmen „zur Rückgewinnung der technologischen Souveränität“. – Und: Die Koalition – damals – unterstützt „die Entwicklung vertrauenswürdiger IT- und Netzinfrastruktur“. Aber offenbar haben Sie diese Ziele mittlerweile bereits aufgegeben. Sie hören auch nicht auf die Warnungen befreundeter Länder und nicht auf Experten wie den ehemaligen BND-Chef Schindler, der gerade aktiv vor Huawei warnt. Was Sie auf der Regierungsbank wollen, ist etwas anderes. Sie wollen sich elegant aus der Affäre ziehen und hoffen darauf, dass die deutschen „Netzbetreiber aufgrund des“ – Achtung – „bestehenden Sicherheitsrisikos freiwillig“ auf einen Netzausrüster Huawei verzichten werden. So steht es im „Handelsblatt“, meine Damen und Herren. Aber Hoffnung ist keine Strategie. Sie wollen sich die Finger nicht schmutzig machen, und die deutschen Netzbetreiber bekommen den Schwarzen Peter. Feige und verantwortungslos ist das. Vielen Dank.  

\noindent\textbf{Comment:}
\begin{itemize}
    \setlength\itemsep{-3pt}
    \item (Beifall bei der SPD)
    \setlength\itemsep{-3pt}
    \item (Beifall bei der AfD)
\end{itemize}
\subsection{Hartmann}
\noindent\textbf{Texts:} Sehr geehrter Herr Präsident! Meine sehr geehrten Damen und Herren! Liebe Kolleginnen und Kollegen! Wir wollen den digitalen Fortschritt. Die SPD will, dass alle von diesem digitalen Fortschritt und diesem digitalen Wandel profitieren. Darum muss im Mittelpunkt dieser Digitalisierung der Mensch stehen. Der Schutz seiner Souveränität, seiner Freiheit und seiner Sicherheit muss gestärkt werden. Die Technologie ist nicht nur Selbstzweck, sondern sie ist Mittel zum Zweck. Wir werden sie als Instrument brauchen, um die großen, neuen gesellschaftlichen Herausforderungen unserer Zeit zu bewältigen. Wir stehen vor einer Wahl: auf der einen Seite das chinesische Modell eines starken Staates, der – zensurgetrieben – keine Menschen- und Bürgerrechte schützt, auf der anderen Seite ein System stark privatwirtschaftlich geprägter Digitalisierung in den Vereinigten Staaten, mit starken Unternehmen, die Daten aber nicht preisgeben und den Datenschutz nicht achten.  Deshalb erhält der Datenschutz eine besondere Bedeutung bei der Gestaltung der Digitalisierung, die aus unserer Sicht nicht im Mittelpunkt einer reinen Digitalpolitik, sondern im Mittelpunkt der Gesellschaftspolitik steht. Meine Herren von der FDP – die wenigen, die anwesend sind, obwohl Sie den Antrag vorgelegt haben –, wir befinden uns auf dem Weg, einerseits das umzusetzen, was wir im Koalitionsvertrag dargelegt haben, und andererseits an das anzuknüpfen, was wir mit dem IT-Sicherheitsgesetz in der vergangenen Legislaturperiode auf den Weg gebracht haben. Wir rücken vor allen Dingen den Verbraucherschutz in den Mittelpunkt des zukünftigen Tuns eines starken BSI. Denn es ist der Mensch, den wir auf dem Weg der erfolgreichen Gestaltung der Digitalisierung stärken wollen. Ob neue Stellen, mehr Geld oder die Investition vor allen Dingen in Verschlüsselungstechnologien, wir als Bürgerinnen und Bürger erwarten, dass wir in die Lage versetzt werden, vom digitalen Wandel zu profitieren – und zwar unabhängig vom Staat –, und dass wir auf die Digitalisierung positiv schauen können. Meine Damen und Herren, ich finde, das kann sich sehen lassen, und es bedarf der Anträge der FDP nicht.  Dennoch haben Sie uns zu dieser Debatte eingeladen. Aber was Sie da anbieten, meine Herren von der FDP, ist in Wirklichkeit ein alter Hut, meine Herren von der FDP, die denn dann da sind.  Es ist in Wirklichkeit die Diskussion darüber, ob ein Digitalministerium geschaffen werden soll. Sie wollen uns außerdem eine weitere Föderalismusreform nicht ersparen; denn Sie stellen erneut die Kooperation auf der Bundesebene infrage. Ich sage Ihnen in aller Klarheit: Wenn wir ein Kooperations ver bot zum Ziel haben und vollkommen ausblenden, dass wir eigentlich ein Kooperations ge bot brauchen, das staatliche Ebenen dazu bringt, zu überlegen, wie sie Forschung, Technologie, aber auch die Sicherheitsbehörden neu vernetzen können, sodass es vor allen Dingen den Menschen nutzt – das Stichwort „Verbraucherschutz“ ist schon gefallen –, dann verschwenden wir eine Menge Zeit.  – Entschuldigen Sie mal, Herr Kollege! Was Sie von der FDP gemacht haben, ist doch Folgendes: Sie haben sich der Verantwortung für dieses Land bewusst verweigert.  Wenn Sie hinterher beklagen, dass Sie keine Verantwortung haben,  und am Spielfeldrand stehen, alles besser wissen wollen und schlaue Tipps geben, dann möchte ich Ihnen an dieser Stelle ein Angebot machen. Ich würdige doch gute Teile Ihres Antrages. Ich verdamme ihn doch nicht in Bausch und Bogen.  Ich habe doch nur kritisiert, dass Sie ein Digitalministerium schaffen und eine weitere Föderalismusreform durchführen wollen.  Ich wollte aber gerade würdigen, dass in Ihrem Antrag bestimmte Teile wirklich gut sind, vor allen Dingen diejenigen, die Sie aus dem Koalitionsvertrag von SPD und CDU und CSU abgeschrieben haben, und der Teil, in dem Sie würdigen, dass wir die Mittel für die entsprechenden Stellen schon eingestellt haben. Das ist doch ein guter Weg, meine Herren. Wir kümmern uns darum. Seien Sie versichert: Das Thema ist bei uns in guten Händen; denn Digitalpolitik ist Gesellschaftspolitik und darf nicht dem Klein-Klein staatlicher Souveränität oder einem Kooperationszwang unterliegen.  Herzlichen Dank.  

\noindent\textbf{Comment:}
\begin{itemize}
    \setlength\itemsep{-3pt}
    \item (Beifall bei der SPD sowie bei Abgeordneten der CDU/CSU)
    \setlength\itemsep{-3pt}
    \item (Beifall bei der SPD)
    \setlength\itemsep{-3pt}
    \item (Manuel Höferlin [FDP]: Funktioniert aber doch nicht!)
    \setlength\itemsep{-3pt}
    \item (Christian Dürr [FDP]: Waren Sie im November dabei, Herr Hartmann?)
    \setlength\itemsep{-3pt}
    \item (Manuel Höferlin [FDP]: Ja, aber das ist doch richtig! Das ist dringend notwendig!)
    \setlength\itemsep{-3pt}
    \item (Beifall bei Abgeordneten der SPD)
    \setlength\itemsep{-3pt}
    \item (Christian Dürr [FDP]: Das ist keine lustige Rede! Ist Ihnen das schon mal aufgefallen?)
    \setlength\itemsep{-3pt}
    \item (Manuel Höferlin [FDP]: Wir übernehmen mit den Ländern Verantwortung, wo es geht!)
    \setlength\itemsep{-3pt}
    \item (Christian Dürr [FDP]: Die Ebert-Stiftung hat wirklich schlechte Rhetorikseminare!)
    \setlength\itemsep{-3pt}
    \item (Manuel Höferlin [FDP]: Dann handeln Sie doch auch so!)
    \setlength\itemsep{-3pt}
    \item (Beifall bei der LINKEN)
\end{itemize}
\subsection{Hahn}
\noindent\textbf{Texts:} Herr Präsident! Meine Damen und Herren! Die Hackerangriffe auf das Außenministerium, auf diverse Bundestagsabgeordnete und nicht zuletzt der Datenklau Ende vergangenen Jahres haben die Lücken in unserer digitalen Sicherheitsstruktur einmal mehr deutlich gemacht.  Polizei, Verfassungsschutz und leider auch das Bundesamt für Sicherheit in der Informationstechnik waren nicht in der Lage, durchaus vorhandene Hinweise, die es zum Teil schon Monate vorher gab, zu einem Gesamtbild zusammenzufügen. Die Informationspolitik der Bundesregierung war indiskutabel. Als ein Betroffener des Datendiebstahls habe ich davon zuerst über die Medien erfahren, obwohl ich Mitglied im Innenausschuss und im Parlamentarischen Kontrollgremium für die Geheimdienste bin. Deutschland ist ganz offensichtlich nicht ausreichend gegen die Gefahren der digitalen Welt gerüstet. In dieser Situation brauchen wir aber keine wohlklingenden Wortschöpfungen wie das von Innenminister Seehofer angekündigte „Cyber-Abwehrzentrum plus“. Wir brauchen eine überzeugende Strategie, aus der hervorgeht, wie kritische Infrastrukturen und vor allem die Daten unserer Bürgerinnen und Bürger besser geschützt werden können.  Dazu muss insbesondere das Bundesamt für Sicherheit in der Informationstechnik zu einer unabhängigen Behörde entwickelt werden, deren Kernaufgabe darin besteht, die digitale Sicherheit im Land spürbar zu erhöhen. Im Klartext: Das BSI muss aus der Unterstellung des Innenministeriums befreit und eine eigenständige Institution werden.  Im Antrag der FDP stehen übrigens einige vernünftige Forderungen, über die wir gerne im Ausschuss diskutieren können. Dass ich unseren eigenen Antrag zur IT-Sicherheit deutlich besser finde, wird Sie nicht verwundern.  Ein ganz zentraler Punkt fehlt aber im Antrag der FDP, und zwar die Auflösung der sogenannten Zentralen Stelle für Informationstechnik im Sicherheitsbereich, ZITiS. ZITiS wurde ohne gesetzliche Grundlage gegründet und entwickelt im Regierungsauftrag selbst Hackermethoden, anstatt vor Ausspähung zu schützen. Damit muss endlich Schluss sein.  Die Linke wird dazu demnächst einen Antrag vorlegen. Natürlich müssen wir auch die Hersteller von IT-Produkten viel mehr in die Pflicht nehmen, als das bislang der Fall ist. Wir brauchen dort eine Produkthaftung, wie sie in anderen Bereichen, zum Beispiel bei der Medizintechnik oder bei Arzneimitteln, längst existiert. Der weltweite Markt für Informations- und Kommunikationstechnik wird von einigen wenigen Firmen aus den USA und China dominiert. Daraus resultierende Sicherheitsbedenken sind durchaus nachvollziehbar. Der Einsatz von Huawei-Technik beim Aufbau des 5G-Netzes in Deutschland birgt ohne Zweifel auch Risiken. Die Enthüllungen von Edward Snowden haben gezeigt, dass die NSA gezielt Manipulationsmethoden, -möglichkeiten und -hintertüren in US-Produkten nutzt, zum Beispiel bei Cisco-Routern, die im Übrigen – vielleicht weiß es der eine oder andere gar nicht – gerade in den Bürogebäuden des Bundestages zur WLAN-Versorgung eingebaut worden sind. Wir als Linke wollen nicht ausspioniert werden, auch nicht von China oder den USA.  Deshalb müssen wir diese Abhängigkeit mit einem europäischen Gegenmodell überwinden, das klar auf Open-Source-Technologie setzt. Hier haben Politik und Wirtschaft leider über zwei Jahrzehnte gepennt. Meine Damen und Herren – letzter Punkt –, der Kampf für mehr Sicherheit im digitalen Raum kann und darf nur mit rechtsstaatlichen Mitteln geführt werden. Deshalb sagen wir klar und deutlich Nein zu Staatstrojanern, Nein zum Handel mit Sicherheitslücken und Nein zu sogenannten Hackbacks.  Solche Maßnahmen, meine Damen und Herren, führen nicht zu mehr Sicherheit. Aber sie sind geeignet, das Vertrauen der Bürgerinnen und Bürger in die Integrität staatlicher Behörden zu zerstören. Das sollte das Parlament eigentlich verhindern. Herzlichen Dank.  

\noindent\textbf{Comment:}
\begin{itemize}
    \setlength\itemsep{-3pt}
    \item (Christoph Bernstiel [CDU/CSU]: Der Mensch ist die Lücke!)
    \setlength\itemsep{-3pt}
    \item (Beifall beim BÜNDNIS 90/DIE GRÜNEN)
    \setlength\itemsep{-3pt}
    \item (Beifall bei der LINKEN sowie bei Abgeordneten des BÜNDNISSES 90/DIE GRÜNEN)
    \setlength\itemsep{-3pt}
    \item (Christoph Bernstiel [CDU/CSU]: Da steht doch gar nichts drin!)
    \setlength\itemsep{-3pt}
    \item (Beifall bei der LINKEN – Tankred Schipanski [CDU/CSU]: Eine Ressortforschungseinrichtung!)
    \setlength\itemsep{-3pt}
    \item (Beifall bei der LINKEN)
\end{itemize}
\subsection{von Notz}
\noindent\textbf{Texts:} Herr Präsident! Meine sehr verehrten Damen und Herren! Im Bereich der IT-Sicherheit – so beginne ich die Reden zu diesem Thema seit vielen Jahren – brennt in Deutschland die Hütte lichterloh. Und daran hat sich nichts geändert: Stuxnet 2010, die Snowden-Veröffentlichungen 2013, der Angriff auf den Deutschen Bundestag 2015, WannaCry 2017, der sogenannte Regierungshack, die Politik-Leaks und nun die Collection \#1 bis \#5, seit Jahren maximale Probleme! Seit Jahren diskutieren wir hier darüber, und seit Jahren passiert gar nichts. Das ist absolut inakzeptabel.  Die Bilanz der Großen Koalition – sie hört es nicht gerne – ist verheerend. Ihre Politik gefährdet die IT-Sicherheit sehr viel mehr, als dass sie sie erhöhen würde. Die Themen sind hier angesprochen worden, und die Probleme sind offenkundig. Wenn Behörden wie ZITiS ohne irgendeine Rechtsgrundlage  Sicherheitslücken offenhalten und mit ihnen hehlen oder wenn gigantische Datenberge völlig unbescholtener Bürgerinnen und Bürger durch anlasslose Massenüberwachung wie PNR und Vorratsdatenspeicherung zusätzlich geschaffen werden, ist das hochproblematisch. Wenn Sie so agieren, dann sind Sie nicht Teil der Lösung, sondern Teil des Problems.  Dem Staat kommt eine direkte Verantwortung zum Schutz der digitalen Infrastruktur zu – das sagen nicht geringere Personen als Herr Papier, Herr Hoffmann-Riem und Herr Becker –, ob darauf nun private Kommunikation oder sensible Unternehmensdaten übertragen werden. Statt digitales Wettrüsten, verfassungsrechtlich hochproblematischer Hackbacks und irgendwelcher Cyberwar-Eskalationsträume brauchen wir eine besonnene, an realen Bedrohungslagen orientierte Politik in dem Bereich. Und diese Politik darf eben nicht nur auf irgendwelche Skandale und erfolgreiche Angriffe reagieren, sondern sie muss langfristig, strategisch und vor allen Dingen proaktiv vorangebracht werden.  Wie eine solche Politik aussehen kann, haben wir hier wiederholt, zuletzt mit einem sehr umfangreichen Antrag vor einem Jahr, aufgezeigt. Deswegen finde ich es ausgesprochen begrüßenswert, dass jetzt auch von der FDP und den Linken Anträge dazu vorliegen. Das zeigt die absolute Dringlichkeit und Wichtigkeit dieses Themas und gleichzeitig die Versäumnisse der Bundesregierung.  Man fragt sich: Was muss eigentlich noch passieren? Seit Jahren versprechen Sie das IT-Sicherheitsgesetz 2.0. Herr Krings, 2.0! Das sagt eigentlich alles.  Statt es endlich vorzulegen – gerne 4.0 –,  boykottieren Sie im Innenausschuss die von uns seit langem beantragte Anhörung zu diesem Thema. Dabei müssen wir endlich dem Grundrecht auf Vertraulichkeit informationstechnischer Systeme zum Durchbruch verhelfen; da ist der Gesetzgeber in der Pflicht. Wir brauchen klare Zuständigkeiten innerhalb der Bundesregierung, Herr Krings, gute Rechtsgrundlagen zum Beispiel für die Zusammenarbeit im Cyber-Abwehrzentrum,  neue Strukturen zur Erkennung hybrider Bedrohungslagen; Frau Merkel redet gerne darüber. Parlamentarisch passiert hier wegen der Verweigerungshaltung der GroKo praktisch nichts. Wir brauchen eine Meldepflicht für Sicherheitslücken.  Wir brauchen ein unabhängiges BSI – Kollege Hahn hat das gesagt –, durchgehende Ende-zu-Ende-Verschlüsselungen, neue Haftungsregelungen, verpflichtende Sicherheitsupdates, weniger Massenüberwachung und mehr freie und offene Software.  Darüber reden wir seit Jahren. Sie tun hier nichts. Diese Handlungen sind überfällig.  Ich komme zum Schluss. Mike Mohring – ich zitiere ihn nicht oft – hat gesagt: „Derzeit hinkt die Datensicherheit der Entwicklung weit hinterher“, man müsse jetzt rasch handeln. Recht hat der Mann. Wie hart einem das Nichtstun auf die Füße fallen kann, merken Sie im Augenblick bei Huawei. Das ist ein Riesenproblem, das Deutschland in innovativer Hinsicht massiv ausbremst, weil Sie seit Jahren die Antworten verweigern. Kommen Sie endlich aus der Box! Die Antworten liegen auf dem Tisch, durch Anträge der Opposition. Deswegen: Ganz herzlichen Dank für das Setzen des Themas! Uns allen einen guten Tag. Tschüss.  

\noindent\textbf{Comment:}
\begin{itemize}
    \setlength\itemsep{-3pt}
    \item (Manuel Höferlin [FDP]: Stimmt genau!)
    \setlength\itemsep{-3pt}
    \item (Beifall beim BÜNDNIS 90/DIE GRÜNEN)
    \setlength\itemsep{-3pt}
    \item (Beifall beim BÜNDNIS 90/DIE GRÜNEN und bei der FDP sowie der Abg. Dr. Michael Espendiller [AfD] und Dr. André Hahn [DIE LINKE])
    \setlength\itemsep{-3pt}
    \item (Christoph Bernstiel [CDU/CSU]: Kommt!)
    \setlength\itemsep{-3pt}
    \item (Beifall bei der CDU/CSU)
    \setlength\itemsep{-3pt}
    \item (Manuel Höferlin [FDP]: Verschlüsselt!)
    \setlength\itemsep{-3pt}
    \item (Christoph Bernstiel [CDU/CSU]: Dann wäre es ja autonom!)
    \setlength\itemsep{-3pt}
    \item (Beifall beim BÜNDNIS 90/DIE GRÜNEN und bei der LINKEN – Christoph Bernstiel [CDU/CSU]: Das stimmt doch gar nicht!)
    \setlength\itemsep{-3pt}
    \item (Beifall beim BÜNDNIS 90/DIE GRÜNEN – Christoph Bernstiel [CDU/CSU]: Dann können Sie sich schon einmal auf unser Gesetz freuen! Das wird schön!)
\end{itemize}
\subsection{Kuffer}
\noindent\textbf{Texts:} Herr Präsident! Kolleginnen und Kollegen! Das ist an sich ein wichtiges Thema. Da haben sich allerdings drei zusammengefunden. Ich kann Ihnen nur sagen: Sie wissen zu überraschen. Die Kollegen von der FDP werfen in ihrem Antrag Forderungen und Bundesbehörden so wild durcheinander, dass man geneigt ist, ihnen einen Organisationsplan der Bundesregierung zu reichen.  Die Präsenz zeigt im Übrigen, wie stark das Interesse an einer Lösung wirklich ist.  Die Linken verfangen sich in den altbekannten ideologischen Grabenkämpfen, und die AfD möchte anscheinend am liebsten den nächsten Handelskrieg anzetteln.  Kollege Schulz regt sich darüber auf, dass kein Minister anwesend ist. Ich weiß nicht, was er uns damit sagen will. Die Ministerin sitzt da auf der Bank.  Zum Kollegen Hartmann muss ich sagen: Ich habe mich als Mann von Ihnen fast diskriminiert gefühlt ob der als Stilmittel der Abfälligkeit zur Schau gestellten Verkrampfung über die Herren von der FDP, aber, wie gesagt, nur fast. Insofern tut der Debatte etwas Ernsthaftigkeit vielleicht ganz gut.  Es ist, glaube ich, gerade auch bei diesem Thema ganz gut, dass wir uns tatsächlich darum kümmern, dieses Land ordentlich und mit gesundem Menschenverstand zu regieren. Es ist unbestritten, dass die Digitalisierung eines der Megathemen unserer Zeit ist und dass der digitale Wandel die Art des Zusammenlebens, unsere Wirtschaft und Arbeitswelt bereits heute grundlegend beeinflusst. Es unterliegt ebenso keinem Zweifel, dass entscheidend für die Akzeptanz ist, dass die Nutzer den Anwendungen und auch den Chancen, die sich bieten, vertrauen können und dass der Schutz ihrer Daten gewährleistet ist. Die Bundesregierung hat hierfür in der Vergangenheit die richtigen Weichen gestellt, und wir haben uns auch für diese Legislatur ambitionierte Ziele gesetzt – die verschweigen Sie leider –, um die digitale Sicherheit weiter zu verbessern.  Es unterliegt aber überhaupt keinem Zweifel, dass wir hier immer wieder stark herausgefordert werden. Entscheidend ist nur, dass wir aus den jüngsten Vorfällen die richtigen Konsequenzen ziehen und dort nachsteuern, wo es nötig ist; das ist bereits gesagt worden. Wie von Innenminister Seehofer im Innenausschuss zugesagt, wird noch vor der Sommerpause ein erweitertes IT-Sicherheitsgesetz 2.0 in den Bundestag eingebracht. Lieber Kollege von Notz, man kann sich über die Semantik streiten. Mir ist es egal, ob Sie das Gesetz „2.0“  oder „Karl-Heinz“ nennen. Entscheidend ist für mich – das ist der Unterschied –, was drinsteht, und da werden Sie überrascht sein.  Wir haben die Gründung einer Cybersicherheitsagentur am vorgesehenen Standort in Halle gerade erst beschlossen Wir haben es mit dem aktuellen Haushalt ein weiteres Mal geschafft, das Bundesamt für Sicherheit in der Informationstechnik mit 350 neuen Stellen signifikant zu stärken, und wir wollen den Tatbestand des digitalen Hausfriedensbruchs im StGB einführen.  Ich glaube, die Vorfälle zwischen den Jahren zeigen, dass unsere Behörden eingespielt, professionell und verlässlich sind und an der Lösung des Problems gut gearbeitet haben und dass die richtigen Konsequenzen gezogen wurden. Die Regierungsnetze sind von Angriffen unbeschadet geblieben, und die digitale Sicherheitsarchitektur hat auch unter Volllast funktioniert. Mir bleibt zum Schluss, den vielen Mitarbeiterinnen und Mitarbeitern, die über die Feiertage in unzähligen Nachtschichten zur Sicherung unserer Infrastruktur tätig waren, unseren herzlichsten Dank auszusprechen.  Liebe Kolleginnen und Kollegen, mit der Schaufensterpolitik, die Sie mit Ihren Anträgen betreiben, kommen wir bei dem Thema jedenfalls nicht weiter. Es ist und bleibt so: Die Union ist die Partei der Sicherheit.  Wir sind es, die dafür sorgen, dass die Menschen in diesem Land sicher leben können, gerade im Bereich der IT-Sicherheit, und das wissen die Bürgerinnen und Bürger auch. Vielen Dank.  

\noindent\textbf{Comment:}
\begin{itemize}
    \setlength\itemsep{-3pt}
    \item (Manuel Höferlin [FDP]: Das würde mich interessieren!)
    \setlength\itemsep{-3pt}
    \item (Beifall bei der SPD)
    \setlength\itemsep{-3pt}
    \item (Manuel Höferlin [FDP]: An Zielen mangelt es Ihnen nicht, an Analysen auch nicht!)
    \setlength\itemsep{-3pt}
    \item (Beifall bei der CDU/CSU – Christian Dürr [FDP]: Sie sind in der falschen Woche gewesen! Karneval kommt noch!)
    \setlength\itemsep{-3pt}
    \item (Lachen bei der AfD und der FDP – Manuel Höferlin [FDP]: Der Worte vielleicht, aber nicht der Sicherheit!)
    \setlength\itemsep{-3pt}
    \item (Beifall bei der CDU/CSU – Dr. Konstantin von Notz [BÜNDNIS 90/DIE GRÜNEN]: Ja, was steht denn drin?)
    \setlength\itemsep{-3pt}
    \item (Beifall des Abg. Sebastian Hartmann [SPD])
    \setlength\itemsep{-3pt}
    \item (Manuel Höferlin [FDP]: Die Union wartet lieber ein bisschen ab, bis sich das Problem von alleine löst!)
    \setlength\itemsep{-3pt}
    \item (Christian Dürr [FDP]: Jetzt erzählen Sie doch mal, was Sie machen!)
    \setlength\itemsep{-3pt}
    \item (Sebastian Hartmann [SPD]: Ja, aber Ihnen auch!)
    \setlength\itemsep{-3pt}
    \item (Beifall bei der CDU/CSU)
    \setlength\itemsep{-3pt}
    \item (Beifall bei der CDU/CSU sowie bei Abgeordneten der SPD)
    \setlength\itemsep{-3pt}
    \item (Dr. Konstantin von Notz [BÜNDNIS 90/DIE GRÜNEN]: Sie nennen es 2.0!)
\end{itemize}
\subsection{Esken}
\noindent\textbf{Texts:} Herr Präsident! Meine sehr verehrten Damen und Herren! Liebe Kolleginnen und Kollegen! Wie fühlt es sich an, wenn Ihr E‑Mail-Account gehackt wurde, jemand Ihr Fotoalbum oder Ihre privaten Chats veröffentlicht hat? Ähnlich wie bei einem Einbruch ist der Schaden an der Wohnungstür vielleicht gar nicht so dramatisch. Aber schlimm ist das Gefühl, dass Ihre sichergeglaubte Privatsphäre verletzt ist und verletzt bleibt. Sie wissen von da an: Das kann jederzeit wieder passieren. Anfang 2019 wurden die Daten von Hunderten Politikern veröffentlicht. In unserer Aufregung darüber – und die war ja nicht klein – sollten wir nicht übersehen, dass jeden Tag Tausende von Bürgerinnen und Bürgern von Datenklau und anderen Vergehen im Internet betroffen sind. Wer gehackt oder im Onlinehandel betrogen wurde, darf Opferschutz und Beratung erwarten, aber auch Ermittlung und Strafverfolgung. Von einer klar definierten und abgestimmten Vorgehensweise der Behörden – das haben wir leider im Zusammenhang mit diesem Hack feststellen müssen – kann in solchen Fällen nicht wirklich die Rede sein. Zu oft erfahren Geschädigte bei den Behörden nicht viel mehr als ein Schulterzucken. Hier gilt wie beim Wohnungseinbruch, liebe Kollegen von der Union: Wer als Täter eine wirksame Strafverfolgung gar nicht befürchten muss, den schreckt auch ein verschärftes Strafmaß nicht ab.  – So ist es. Wir fordern die Bundesregierung deshalb auf, in Zusammenarbeit mit den Ländern zu definieren, welche Maßnahmen in welchem Fall von Cyberkriminalität zum Schutz der Opfer, zur Beweissicherung – zuhören, Kollege! – und zur Strafverfolgung ergriffen werden müssen und wie das Zusammenspiel der Behörden eigentlich funktionieren soll. Weil die Forensik bei solchen Vorkommnissen auch Erkenntnisse gewinnen kann, die zur Prävention geeignet sind, müssen diese auch zusammengeführt werden. Auch wenn ein Teil der Verantwortung bei den Nutzern liegt und – richtig – „12345“ noch kein sicheres Passwort ist, dürfen wir die Menschen mit dieser Verantwortung nicht alleine lassen. Deshalb wollen wir von der SPD die Diensteanbieter dazu verpflichten, Datensicherheit und Privatheit nach dem Stand der Technik zu gewährleisten und voreinzustellen. Wir wollen eine auf vielen Kanälen wirksame und an alle Zielgruppen gerichtete Informationskampagne zur IT‑Sicherheit auflegen. Erste Gespräche dazu haben wir geführt. Auch die Angriffe auf unsere Infrastruktur, auf Verwaltungen, Krankenhäuser oder die Stromversorgung, werden immer professioneller. Dass auch der Bundestag oder die Bundesbehörden dagegen nicht gewappnet sind, beunruhigt die Menschen, und zwar zu Recht. Es ist doch unsere staatliche Aufgabe, ein Höchstmaß an Sicherheit zu gewährleisten. Ich muss zustimmen: Wir dürfen nicht auf der anderen Seite die Sicherheit schwächen und dürfen insofern keine Schwachstellen offenhalten, um sie für eigene Zwecke zu nutzen.  Wir dürfen erwarten, dass der Staat bei sich selbst, bei den Institutionen der kritischen Infrastruktur insbesondere, für maximale Sicherheit sorgt: sichere IT‑Systeme, sichere Organisation und Mitarbeiter, die wissen, was sie tun, die wissen, wie man sich sicher verhält. Dazu entwickeln wir das IT‑Sicherheitsgesetz weiter, und die Kolleginnen und Kollegen, die hier Anträge einbringen, wissen das auch. Wir von der SPD plädieren dabei für eine strikt defensive Ausrichtung. Wir müssen uns für den Fall eines Cyberangriffs wappnen und in der Lage sein, das womöglich lahmgelegte öffentliche Leben so schnell wie möglich wiederherzustellen, und sollten nicht etwa gegen einen Täter zurückschlagen, den wir im Zweifel noch gar nicht kennen. Unser Katastrophenschutz muss sich auch mit solchen Vorkommnissen und Zusammenhängen mehr beschäftigen. Wichtigstes Vorhaben – das ist schon angesprochen worden – ist der Ausbau des Bundesamtes für Sicherheit in der Informationstechnik zur zentralen Behörde für Beratung und Verbraucherschutz, für Standardisierung und Zertifizierung in allen Fragen der IT‑Sicherheit. Wir setzen uns dafür ein, die Behörde dabei sehr weitgehend unabhängig zu stellen.  Auch die Hersteller von Informationstechnik werden wir in die Pflicht nehmen. IT‑Sicherheitskennzeichen werden bei der Auswahl sicherer Produkte sicher helfen. Aber erst klare Haftungsregeln machen eben auch deutlich, dass es bei der IT‑Sicherheit um ernsthafte, auch materielle Schäden geht. Die Menschen mit ihrer Verantwortung nicht alleine lassen, Anbieter und Hersteller in die Pflicht nehmen und Behörden für den Schutz der Bevölkerung stärken – das ist unser Weg für mehr IT‑Sicherheit in Deutschland. Vielen Dank.  

\noindent\textbf{Comment:}
\begin{itemize}
    \setlength\itemsep{-3pt}
    \item (Beifall bei der SPD)
    \setlength\itemsep{-3pt}
    \item (Beifall bei Abgeordneten der SPD und der FDP)
    \setlength\itemsep{-3pt}
    \item (Beifall bei der SPD und der FDP)
    \setlength\itemsep{-3pt}
    \item (Beifall bei der CDU/CSU)
    \setlength\itemsep{-3pt}
    \item (Manuel Höferlin [FDP]: Oder ein neues Gesetz! Hilft null dann!)
\end{itemize}
\subsection{Biadacz}
\noindent\textbf{Texts:} Sehr geehrter Herr Präsident! Liebe Kolleginnen und Kollegen! Wissen Sie, was mich eigentlich an diesen Anträgen, die wir heute beraten, massiv stört?  Obwohl ein paar gute Ideen dabei sind, schwingt eine negative Grundhaltung mit,  zum Beispiel wenn da vom „Cyber-Kundus“ die Rede ist. Das erinnert mich ein bisschen an eine Wortkreation, die bei der letzten Debatte zum Thema IT‑Sicherheit hier am Pult gefallen ist, nämlich das „cyberpolitische Bull­shit-Bingo“.  Sie unterstellen mit diesen flapsigen Begriffen, die Bundesregierung würde die IT‑Sicherheit in Deutschland gezielt schwächen,  statt sie zu stärken. Das können Sie doch nicht ernsthaft glauben, meine Damen und Herren.  Ich bin der Meinung, die Bundesregierung und wir als Gesetzgeber haben ein gemeinsames Interesse daran, den digitalen Raum so sicher wie möglich zu gestalten.  – Hören Sie mir zu! – Dazu gehört, gemeinsam dafür Sorge zu tragen, bestehende Risiken zu minimieren, damit die Digitalisierung am Ende allen Menschen in Deutschland mehr Chancen als Risiken bietet.  Ich halte es dagegen für grob fahrlässig, wenn die digitale Transformation als Schreckgespenst dargestellt wird, so wie es die drei Anträge suggerieren. Die digitale Transformation ist keine Gefahr,  nur weil sie mit Risiken im Bereich der IT‑Sicherheit verbunden wird. Es wird, erstens, immer Sicherheitslücken geben,  egal wie gut der Schutz vor Cyberangriffen ist. Der digitale Wandel ist nun mal dynamisch und agil. Deswegen werden sich Sicherheitslücken immer wieder auftun. Aus diesem Grund wäre es, zweitens, naiv zu glauben, wir könnten erst mal alle Gefahren abwehren und dann mit der digitalen Transformation beginnen.  Nein! Der digitale Wandel ist in vollem Gange. Wir brauchen daher eine dynamische und agile IT‑Sicherheitsstruktur.  Dafür, meine Damen und Herren, haben wir im Koalitionsvertrag wichtige Vereinbarungen getroffen: IT‑Sicherheitsgesetz 2.0 und zusätzliche Stellen im BSI. Das zeigt: Wir verbessern die IT‑Sicherheit in Deutschland Schritt für Schritt, meine Damen und Herren. Politik kann aber nur Rahmenbedingungen setzen. Um IT‑Sicherheit zu schaffen, reicht gesetzgeberisches Handeln nicht aus. Wir brauchen in Deutschland genauso das richtige Mindset für IT‑Sicherheit.  Das gilt auch für uns Abgeordnete. Wir müssen alle Bürger und Unternehmen dafür sensibilisieren, wie Daten und Geschäftsgeheimnisse vor Cyberangriffen besser geschützt werden können.  Das geht nicht, indem wir Angst verbreiten. Vielmehr müssen wir aufklären und vermitteln, also ein Bewusstsein für IT-Sicherheit schaffen. Um bei den Sicherheitslösungen vorne mit dabei zu sein und Standards zu setzen, brauchen wir in Deutschland innovative IT‑Sicherheitskonzepte. Ich sehe nicht nur den Staat in der Pflicht, diese zu entwickeln. Erstens. Es gibt viele kluge und kreative Köpfe in unserem Land, die ein Start-up im Bereich der digitalen Sicherheit gegründet haben.  Diese innovativen Sicherheitslösungen sollten nicht ungenutzt bleiben. Zweitens appelliere ich an alle, die sich vielleicht gerade überlegen, ein Start‑up hier in Deutschland neu zu gründen: Gründen Sie im Bereich IT‑Sicherheit! Wir, die Bundesregierung werden Sie dabei unterstützen.  Ich bin davon überzeugt: Mit diesen gesetzlichen Rahmenbedingungen, mit dem richtigen Mindset, mit den kreativen Sicherheitslösungen von Start‑ups im IT‑Sicherheitsbereich erhalten wir eine dynamische IT‑Sicherheitsarchitektur für unser Land. Das, meine Damen und Herren, ist eine klare Strategie und kein „cyberpolitisches Bullshit-Bingo“. Herzlichen Dank.  

\noindent\textbf{Comment:}
\begin{itemize}
    \setlength\itemsep{-3pt}
    \item (Manuel Höferlin [FDP]: Sehr richtig!)
    \setlength\itemsep{-3pt}
    \item (Dr. Gero Clemens Hocker [FDP]: Das ist Bullshit-Bingo!)
    \setlength\itemsep{-3pt}
    \item (Konstantin Kuhle [FDP]: Aber hallo! – Stephan Brandner [AfD]: Finden wir nicht!)
    \setlength\itemsep{-3pt}
    \item (Dr. Franziska Brantner [BÜNDNIS 90/DIE GRÜNEN]: Braucht man gar nichts machen, oder wie? – Manuel Höferlin [FDP]: Man muss sie aber nicht extra offenhalten!)
    \setlength\itemsep{-3pt}
    \item (Manuel Höferlin [FDP]: Das tun Sie doch auch!)
    \setlength\itemsep{-3pt}
    \item (Manuel Höferlin [FDP]: Nein! Aber wir müssen es besser organisieren!)
    \setlength\itemsep{-3pt}
    \item (Christoph Bernstiel [CDU/CSU]: Oh!)
    \setlength\itemsep{-3pt}
    \item (Dr. Konstantin von Notz [BÜNDNIS 90/DIE GRÜNEN]: Das ist mein Wort! Danke!)
    \setlength\itemsep{-3pt}
    \item (Beifall bei der CDU/CSU)
    \setlength\itemsep{-3pt}
    \item (Dr. André Hahn [DIE LINKE]: Ja, dann macht’s doch! – Dr. Konstantin von Notz [BÜNDNIS 90/DIE GRÜNEN]: Dann macht mal!)
    \setlength\itemsep{-3pt}
    \item (Beifall bei der CDU/CSU sowie bei Abgeordneten der SPD)
    \setlength\itemsep{-3pt}
    \item (Dr. André Hahn [DIE LINKE]: Nein! – Konstantin Kuhle [FDP]: Dass sie nicht von Ihnen sind!)
    \setlength\itemsep{-3pt}
    \item (Christian Dürr [FDP]: Nein! Die Gefahr sind Sie, ehrlicherweise!)
    \setlength\itemsep{-3pt}
    \item (Manuel Höferlin [FDP]: Durch Verschlüsselung beispielsweise!)
\end{itemize}
\end{document}